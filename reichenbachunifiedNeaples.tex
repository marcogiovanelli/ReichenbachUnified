% !TEX encoding = UTF-8 Unicode
\documentclass[draft]{article}
\usepackage{els}
\usepackage{i}
\usepackage{notate}
\usepackage{wrapfig}
\newcommand{\ap}{\ensuremath{\tau}\xspace}
\newcommand{\dap}{\ensuremath{d\ap}\xspace}
%\usepackage{geoA4}
\renewcommand{\oe}{;~o.e.{}}
\renewcommand{\me}{;~m.e.{}}
\newcommand{\oemph}[1]{\emph{#1}}
\newcommand{\memph}[1]{\emph{#1}}
\newcommand{\phin}{\ensuremath{\varphi_\nu}\xspace}
\newcommand{\manu}[1]{\citep[#1]{Reichenbach1928b}}
\newcommand{\nbein}{$n$-bein\xspace}
\newcommand{\vbein}{vierbein\xspace}
\newcommand{\hbein}{\ensuremath{h_{a}^{\nu}}\xspace}
\newcommand{\xdx}{\ensuremath{x_\nu} and \ensuremath{x_\nu + dx_\nu}\xspace}
\newcommand{\hbeinr}{\ensuremath{h_{\alpha}^{\nu}}\xspace}
\newcommand{\PRZL}{\citetitle{Reichenbach1928}\xspace}
\newcommand{\Reich}{Reichenbach\xspace}
\newcommand{\VZ}{\jt{Vossische Zeitung}\xspace}
\newcommand{\FP}{\german{Fernparallelismus}\xspace}
\newcommand{\DP}{distant parallelism\xspace}
\newcommand{\Gtmnbar}{\ensuremath{\bar{\Gamma}\tmn}\xspace}


\newcommand{\rhp}[2]{(\cite[#1]{Reichenbach1920a}; tr.\ \citeyear{Reichenbach1969} #2)\xspace}
\renewcommand{\rzlp}[2]{(\cite[#1]{Reichenbach1928}; tr.\ #2)\xspace}
\renewcommand{\rzlap}[2]{(\cite[#1]{Reichenbach1928}; tr.\ [#2])\xspace}
\newcommand{\vza}[1]{(\cite{Reichenbach1929c}; tr.\ \citeyear{Reichenbach1978}, 1:#1)\xspace}
\newcommand{\hpa}[2]{(\cite[#1]{Reichenbach1929}; tr.\ \citeyear{Reichenbach1978}, 1:#2)\xspace}

\makeatletter
\def\tagform@#1{\maketag@@@{[\ignorespaces#1\unskip\@@italiccorr]}}
\makeatother


\creflabelformat{equation}{#2[\textup{#1}]#3}

\newlist{W}{enumerate}{1}
\setlist[W]{label=W-\Roman*:}



%\newmdenv[linecolor=black]{infobox}
%%\renewcommand*{\multicitedelim}{\addcomma\space}
%\renewcommand*{\multicitedelim}{\addsemicolon\space}
\title{Coordination, Geometrization, Unification. An Overview of the Reichenbach--Einstein Debate on the Unified Field Theory Program}

%TODO check translaton of Reichenbach!!!
%TODO check **

\begin{document}
\maketitle

\begin{abstract}
\lipsum*[1-2]
\end{abstract}


\begin{keywords}
Reichenbach \sep Einstein \sep Weyl \sep Unified Field Theory \sep General Relativity \sep Geometrization \sep Unification \sep Coordination	
\end{keywords}

\section*{Introduction}

\lipsum[1]

\begin{description}
\item[Einstein-Reichenbach debate of Weyl's theory (1920-1922)]\label{reichenbachweyl} In his 1920 habilitation, Reichenbach, although rather in passing, accused Weyl of attempting to deduce physics from geometry, by reducing physical reality to \scare{geometrical necessity} \citep[73]{Reichenbach1920a}. On the contrary, the greatest achievement of \gr, Reichenbach claimed, was to have shifted the question of the truth of geometry from mathematics to physics \citep[73]{Reichenbach1920a}. That the separation . Einstein seemed to agree. Reichenbach  that After their correspondence, \citet[367--368]{Reichenbach1921}  accepted \citets{Weyl1921} counterargument that the geometry of \spti has nothing to do with behavior of \rac, but complained about the overly formal nature of the theory \citep[367]{Reichenbach1921}.



The ideea of coordinate, seems Einstein, Einstein seems also agree, that have become hard to dey that Einstein poisiton as  fundametanl, had become actually very different from what Kant had imagnined. It was Reichenabch that seems to indudec Einstin to take a philosopjical position


\item[Reichenbach-Einstein correspondence (1926-1927)]\label{reichenbacheinsteinI} 
In March 1926, after making some critical remarks on Einstein's newly published metric-affine theory \citep{Einstein1925a}, Reichenbach sent Einstein a 10-page \scare{note} \lettercpaep{Reichenbach}{Einstein}{24}{3}{1926}[15][224]. In it, he constructed a mock unification of the gravitational and electricity in a single geometrical framework, thereby showing that the \scare{geometrization} of a physical field was a mathematical trickery rather physical achievement. After a back and forth Einstein seemed to agree \citep{Lehmkuhl2014}. The note was later included as section \S49 in a long technical \Ap to the \PRZL \citep[\SS46-50]{Reichenbach1928} in which  \gr is presented as a \scare{physicalization of geometry} rather than a \scare{\emph{geometrizaton} of gravitation} \citep{Giovanelli2020}. 

\item[Reichenbach-Einstein correspondence (1928-1929)]\label{reichenbacheinsteinII}   A few months after the publication of the \PRZL \citep{Reichenbach1928}, \citet{Einstein19281,Einstein19282} launched yet another attempt at a \uft, the so-called \DP-field theory. Reichenbach, now back in Berlin, discussed the new theory in person with Einstein and sent him once again a manuscript with some comments. The unpublished manuscript is still extant \citep{Reichenbach1928b}. This exchange of letters marked the cooling of Einstein's and Reichenbach's personal friendship but also the end of their philosophical kinship. In the late 1920s, \citet{Reichenbach1929a,Reichenbach1929b,Reichenbach1929c} came to realize that in Einstein's mind, the actual goal of the \uftp was not the geometrization, but as the \emph{unification} of two different fields, an undertaking for the sake of which Einstein was ready to embrace a strongly speculative approach to physics \citep{Dongen2010}.  
\end{description}

The present paper does bring new material. It has the ambition to provide. The recognition of the importance of this episode has been an important result of the Reichenbach-scholarship of the last decades \citep{Ryckman1995,Ryckman1996}. The correspondence has been rediscovered and published \citep[\V{15}]{CPAE} only recently \citep{Giovanelli2016d}. The third episode is forthcoming \citep{Giovanelli2022}. However, we hope to this isolated episodes into a analysis. One the key theme of Reichenbach's philosophy is separation between mathematics and physics. Ultmiately, consider mathematical simplicty and the key two physical reality.

%They what was mathematically was most simple from a geometrical point of view, was also true from a physical point of view. \q{The general theory of relativity by no means turns physics into mathematics. Quite the opposite: it brings about the recognition of a physical problem of geometry}.  The defence to supprot \rt but also. What was the actual reason for the succes of \rt. Reichenach a fundamental position, it was Einstein who progressively was ready to philosophical compromise for the sake of physics. Even did not provide. That the geoemtrization was conidere the key, even more the. 

The electromagnetic field. On the background The in terms of test particles. Field equaitons in fourm. The makes then prediction on the values. General relativity \gmn with the gravitational fiedd, that are both and \rac. That the geometry of \spti. The field equatiosn could also be derived by variational principle. 

To impose on the structure of the universe the additional conditions that will lead to the electromagnetic field, we cannot stay in purely Riemannian space. The electromagnetic field cannot be described in it  by increasing the number of dimensions of a Riemannian space, in particular to five dimensions. Einstein explped on serveal. by going into a more general variety of spaces with any affine connection \Gtmn which will give us more latitude in the definition of the parallel displacement of a vector along an infinitely small closed contour. Indeed, to e.g.  that does not depend to \gmn at all, dropping the conditions of symmetry\etc. This second line of Weyl-Eddington-Schouten lineage and will attract our interest since, indeed also Einstein to this tradiation. The notion of parallel displacement of vectotrs is the the fundamneatl prtagonist of this history


\begin{itemize}
\item Weyl by notn ly on the \gmn but also on a four vector \phin Eddington to use the \Gtmn, Einstein \Gtmn and \gmn as non-symmetric. Einstein last attempt to use \Gtmn which is flat but non symmetric. The question of the physical interpreation concern; however, in later theory the question seems to have been abandoned.
\item This very convenient way of searching for the 'right' field-equations (you just have to search scalar density $\mathcal{H}$ has been widely adopted, and there are general reasons for believing that it is justified. demanding that the space-time integral of an invariant density , taken over any fixed region, be from the dynamical quantities  then finds the conditions needed demanding that the space-time integral of an invariant density $\int \mathcal{H}$, taken over any fixed region, be stationary. Variation with respect to som of the fundemntal variables or both. The trick (and a trick it is) is to find the right $\mathcal{H}$ and the right dynamical quantities to produce the desired equations, to reconver Einstein and Maxwell equations at least in approximate form, valid for weak fields.
\item Usually only after the field equations has been established, the field components which they connect (which are either the original basic geometrical field-quantities or, more often, vectors and tensors built up from them) have to be identified,  the components of the gravitantinal and electromagnetic tensors. In Eddington theory non-symmetric Ricci tensor if starts \gmn the it is natural to identiy with affine\etc.
\item since the fundamental field quantities have usually no physical meaning, the test electromagnetic field indirect way. Most  such a way that they possess only one or at most a small number of static spherically symmetric solutions and their laws of motion. E.g. \cop{would represent 1) uncharged point masses 2) at rest with respect to each other}, the theory would contradict experience.\cop{Thus it is clear that prior to the solution of this problem the behavior of a corpuscle in a field (its equations of motion) cannot be tackled And without the solution of this problem, the truth of the theory cannot be tested!}
\end{itemize}

Ultimately, the motivation wa not real 
That this skeme presented sveral eepsistemological deifficultis, the meaning of the fundamental field equantis, to derive ... was measru formal; the testing of the theory appead to be queite inderect. 

%Since the solutions that is the nor \rac are availabel from the outset intial definitions of the field quantitis was not available (Cartan, Besso).


%the electromagnetic field which has not had any geometrical interpretation and forms, with the gravitational field, a heterogenous ensemble. \cop{the sources of field which conserve a phenomenological interpretation even when one talks about uncharged particles}. \cop{The so-called unified theories attempt to remove the heterogenous features of the combined gravitation and electromagnetic fields}.  The equations which will describe the behavior of the gravitational and the electromagnetic field in empty space are, by definition, relative to a unified exterior'' case






% \cop{In this connection we are likely to think of the development of the theory of relativity into a world geometry, yet it would be quite erroneous to interpret this development as signifying a fusion of physics and mathematics}. \q{The general theory of relativity by no means turns physics into mathematics. Quite the opposite: it brings about the recognition of a physical problem of geometry}. .... \cop{Although both mathematics and physics are sciences, the difference between them is fundamental, and we must put it down as quite impossible that it will ever disappear.} Reichenbach considered as the fundamental key result of \rt. The theory that Euclidean geometry is a in itself \apr, that the choice was ultimately a physical question. However, will then exactly the different position. Einstein will claim the mathematical simplicity is itself is a key to reality and that the is ultimately to prove jus tkike it is necessary 4+4 is equal four. Berlin again will the end not only of a personal relatioship but also of a philosipilca


%Line Weyl-Eddington-Schouten. REichenbach as nothing also of this apparoach. A relativistic theory has to answer two questions: 1) What is the mathematical character of the field? 2) What equations hold for this field?. In Maxwll theory ...  In general relativity \gmn by the equivalence princopels, that also assuede the interpeation in terms of \rac. The was and physcally meanignuf. That was a geomtrical field, that als measure distances and time intervals, and determins geodeiscs on which travel AFter field equations solitons the couod be \rac, light raus test pargice.s This two systems was however, \gmn and \faraday has nothing to with one another. 
%
%That was continue was \Gmn and \gmn; new gefress f freedom; however, the \phin \Gtmn, or both\etc. Howver, the physical interpreation tuern to be. Itially it was measre by \rac. However, progessively more abstrac t ... To make it that, you have to impose on its basic geometrical field-quantities-the gir or the gik and pk or the Tmik, as the case may becertain restrictions, differential equations. little arbitrariness as possible. Now almost every kind of restriction contemplated hitherto has turned out to be equivalent to a Hamiltonian principle, demanding that the space-time integral of an in- ... modify Maxwell's equations of the electromagnetic field in such a way that they possess only one or at most a small number of static spherically symmetric solutions.
%
%

%In a filed htoery field quantitees, physoal meanig, serach for the field equaitisn, check. geometry one one ahd That Reichenbach procedd a physical meaning \apr to test the theory empirically, spearation of geometry and physics.  The choice of the geoemtrical strucutre must be justified ante facto. \gmn secojd  the \gmn must have a physocal meaing how to measure it.

%Thus one is guess wiich and try to find the right filed quatiosn first five definition, find the field equaitosn makes prediction about beheavior \rac masured; Thus, te only post facto of procuing that corresponde. Increasing there was no physca, mathematically, mathematiclaly ncessity
%
%The separation between mathematics and physics
%
%mathematical simplociti is the key for physical realit



%Reichenbach, Hans. 1920a. Die physikalischen Voraussetzungen der Wahrscheinlichkeitsrech- nung. Die Naturwissenschaften 8:46–55; Nachtrag, 349; English Trans. in Reichenbach 1978. vol. II. 293–311.
%Reichenbach, Hans. 1920b. Philosophische Kritik der Wahrscheinlichkeitsrechnung. Die Natur- wissenschaften 8:146–153; English Trans. in Reichenbach 1978. vol. II. 312–327.


\section{Coordination. The Weyl-Reichenbach Correspondence}
\label{Coordination}
%after I was transferred from active duty because of a severe illness

%tracted at the Russian front, I began to work as an engineer for a Berlin firm specializing in radio technology (from 1917 until 1920). During this period, and in my capacity as physicist, directed the loud-speaker laboratory of this firm. .. Soon thereafter, my father died and for the time being I could not give up my engineering position because I had to earn a salary in order to provide for my wife and myself. Nevertheless, in my spare time I studied the theory of relativity; I attended Einstein's lectures at the University of Berlin; at that time, his audience was very small because Einstein's name had not yet become known to a wider public. The theory of relativity impressed me immensely and led me into a conflict with Kant's philosophy. Einstein's critique of the space-time-problem made me realize that Kant's a priori concept was indeed untenable. I recorded the result of this profound inner change in a small book entitled Relativitätstheorie und Erkenntnis Apriori [1920]."


%Reichenbach were in friendly terms Einstein seems to have even tried to find Reichenbach a job \lettercpaep{Reichenbach}{Einstein}{16}{8}{1919}[9][89].

After serving in World~War~\rom{1}, from 1917 until 1920 Reichenbach worked in Berlin as an engineer specializing in radio technology to support himself after the dearth of his father. Nevertheless, in winter term 1917--1918 and in summer term 1919, he attended Einstein's Berlin lectures on special and general relativity. Einstein and  We posses three sets of Reichenbach's undated student notes \citep[028-01-04, 028-01-03, 028-01-01]{HR}. A set of notes \citep[028-01-01]{HR} seems to be very similar to Einstein's own lecture notes from 1919\citep{Einstein1919c}\footnote{Further information about Einstein as an academic teacher, see Vol. 3, the editorial note, "Einstein's Lecture Notes,"pp. 3-10, and for a survey of Einstein's academic courses, see Vol. 3, Appendix B.}. \cop{In presenting \gr, Einstein's lectures follow roughly the corresponding sections of his previous published presentations  of \rt \citepp{Einstein1916}{Einstein1914a}}. The mathematical apparatus of Riemannian geometry is introduced by starting from the metric $\gmn$ as the fundamental concept, that is from the formula to calculate the squared distance $ds^2=\gmn dx_\mu dx_\nu$ between two neighboring points \xdx in any coordinate system. From the \gmn the one calculate the so-called Christofell symbols \christoffel{\mu}{\nu}{\tau}, which enters in the geodesic equation, and Riemann tensor \rite which generalized the Gaussian notion of curvature. 

However, both Reichenbach's and Einstein's notes show in the lectures May-June 1919 Einstein used for the first time the interpretation of the curvature in terms of the parallel displacement of vectors, which was introduced by Tullio \citet{Levi-Civita1916} and applied to \rt by Hermann \citet{Weyl1918}. Both names are mentioned explicitly \citep[028-01-03, 33]{HR}. Instead of using the metric as a fundamental concept, it is more convenient to start by the the coordinate-independent condition that two vectors at neighboring points \xdx are parallel $dA^\mu=\Gtmn A^\nu x_\nu$ \citep[028-01-03, 33]{HR}\footnote{I have uniformized the notation used by \citet{Reichenbach1928} which in turn is based on \citet{Eddington1923,Eddington1925}}. The \Gtmn, which is supposed to be symmetrical in the lower indexes, is the so-called affine connection (\german{Zusammenhang} or displacement (\textit{Verschiebung})\footnote{The affine geometry is the study of parallel lines, \citet{Weyl1918b} hence the expression \scare{affine connection} (\german{affiner Zusammenhang}), where connection refers to the comparison at close points. However, because it is a relation of \scare{sameness} rather than parallelism that is relevant in this context, others, such as Reichenbach, prefer to speak of the operation of \scare{displacement} (\german{Verschiebung}), where the latter indicates the small coordinate difference $d\xn$ along which the vector is transferred. The word displacement also refers to the vector $dx_\nu$. To avoid confusion the world transfer \textit{Übetragung} was also used}. The metric could be introduced at further stage by defining the squared length of vectors $l^2=\gmn A^\mu A^\nu$ in a manner independent of the choice of the coordinates. By imposing the condition that the length of vectors does not change under parallel transport the $\Gtmn$ have the same numerical values of the Christoffel symbols (up to a sign). The structure of the Einstein-Riemann geometry is then completely determined without any reference to the metric \gmn. It differs from the Euclidean structure by the fact that when a vector is transported along a closed curve, it will acquire a rotation determined whose amount Riemann tensor \riteg.


%In the general case, if one parallel displaces a vector along different paths, one gets a different vector at a distant point \citep[028-01-03, 37]{HR}. 

%To overcome this \scare{mathematical injustice}, Weyl introduced \scare{metric connection} alongside the \scare{affine connection}. If a vector of length $l$ is displaced from $x_\nu$ to $x_\nu+dx_\nu$, it will in general have a new length $l+dl$, so that $dl/l=\phin dx_\nu$. In this way, in addition to the \scare{metric tensor} \gmn, a \scare{metric vector} $\phin$ is introduced, that could be identified with  electromagnetic four-potential. \cop{Weyl could then conclude that just like general relativity represented a geometrization of gravitational phenomena, Weyl's theory represented a unified geometrization of both gravitational and electromagnetic phenomena, which were, at that time, the only kind known}. Concluding the 1919 edition of the book, Weyl could declare that \q{physics and geometry coincide with each other}. The tendency of physicalizing geometry that have prevailed leading protagonists of the 19th century from Gauss to Helmholtz seemed to superseded have of geometrizing physics that run from Riemann to Einstein: \q{geometry has not been physics but physics has become geometry} \citep[263]{Weyl1919}.\footnote{Concluding the 1919 edition of the book, Weyl could declare that \q{physics and geometry coincide with each other}. }. 

%In this way, one could obtained a more general affine a connection expressed in terms of the \gmn and a four vector \phin 

This technical innovation in differential geometry played a fundamental role, not only in successive formulations of \gr, but more prominently in the development of the \uftp. If one starts with a symmetric \gmn the road is, so to speak, marked. The Christoffel symbols are the only possible destination. However, if one defines the displacement \Gtmn independently from the metric \gmn, the Riemannian connection $\Gtmn = \christoffel{\mu}{\nu}{\tau}$ appears only as a special case that has been achieved by introducing a series of conditions. In particular, \citet{Weyl1918a,Weyl1919a} was bothered by the asymmetry of the comparison of direction of vectors which is path-dependent could not be the comparison their lengths was distant-geometrical. To compensate for this  \scare{mathematical injustice}, Weyl could introduced a more general affine connection depending not only the metric/gravitational tensor \gmn but also the four-vector \phin. For formal reason, the latter could be identified with electromagnetic four-potential. \cop{Weyl could then conclude that just like general relativity represented a geometrization of gravitational phenomena, his theory represented a unified geometrization of both gravitational and electromagnetic phenomena, which were, at that time, the only kind known}. Weyl did not hesitate to declare that \q{Der Traum des Descartes von einer rein geometrischen Physik} had be fulfilled \citep[263]{Weyl1919}\footnote{Concluding the 1919 edition of the book, Weyl could declare that \q{physics and geometry coincide with each other}}. 


%It gives an idea of Reichenbach's intentions what he wrote to Einstein (June 15, 1920) when asking permission to dedicate him his book Relativitätstheorie und Erkenntnis Aprior®: "You know that with this work my intention was to frame the philosophical consequences of your theory and to expose what great discoveries your physical theory have brought to epistemology. I know very well that very few among tenured philosophers have the faintest idea that your theory is a philosophical feat and that your physical conceptions contain more philosophy than all the multivolume works by the epigones of the great Kant. Do, therefore, please allow me to express these thanks to you with this attempt to free the profound insights of Kantian philosophy from its contemporary trappings and to combine it with your discoveries within a single system." To this letter, Einstein replied (June 30, 1920)4: "The value of the th. of rel. for philosophy seems to me to be that it exposed the dubiousness of certain concepts that even in philosophy were recognized as small change. Concepts are simply empty when they stop being firmly linked to experience."

Einstein had repeatedly criticized Weyl's attempt \citep{Einstein1918b}. However, by the spring of 1919, after a correspondence with Theodore Kaluza he had started to show more interest for the the \uftp \citep{Wuensch2005}. The question fell into the background after the success of the eclipse expedition was announced in November 1919 \citep{Dyson1920}. By the end of the year, Einstein was turned into an international celebrity leaving him little time to work (\lettercpae{Einstein}{Fokker}{1}{12}{1919}[9][187], \lettercpae{Einstein}{Hopf}{2}{2}{1920}[9][295]). The German philosophical community started to show increasing interest in the theory \citep{Hentschel1990}. The question whether was compatible with Kant's philosophy became pressing. Reichenbach was aware of the fact he had acquired a technical knowledge of the new theory that was not comparable to that of any philosophers of his time. Thus, it was nearly inevitable that in February or March 1920 Reichenbach, who has just moved to Stuttgart, decided to write his habilitation on this topic \hide{Im Februar (oder März) 1920 beschloß ich, meine Habilitationsschrift zu schreiben}. \hide{Ich hatte in den Monaten vorher Relth. gearbeitet, auch nach Weyl; den Grund hatte ich schon 1917-1918}. In the preceding months, he had further worked on the theory \qt{also according to Weyl}{auch nach Weyl} \citep[044-06-23]{HR}. The Kapp-Pusch coup on \datemy{13}{3}{1920} gave Reichenbach a few days of leave from Huth radio industry were he was employed \citep[044-06-23]{HR}. Thus, he could work without interruptions and in ten days he completed an early draft. The manuscript was then typed and shown among others to Einstein. Thanks to the mediation of Arnold Berliner, the influential editor of the \jt{Naturwisseschaften}, Reichenbach obtained a publication agreement with Springer \citep[044-06-23]{HR}.



%11 Für weitere Ausführungen zur Entstehung von Relativitätstheorie und Erkenntnis apriori sei hier auf Kamlah 1979, Erläuterungen zu GW 3, 475-480; Maria Reichenbach, Vorwort zu Reichenbach 1965ü, The Theory of Relativity and A Priori Knowledge, xi-xliv oder Hentschel 1990, Interpretationen und Fehlinterpretationen….., 507-526 verwiesen.

Reichenbach's habilitation has recently attracted great attention \citep{Friedman2001}. Reichenbach borrowed from \citet{Schlick1918} that idea that physical knowledge is ultimately (\emph{Zuordnung}), the process of relating an axiomatically defined mathematical structure to concrete empirical reality. However, Reichenbach attempted  to give this insight a \scare{Kantian} twist. According to Reichenbach, in a physical theory beside the \scare{axioms of connections} (\german{Verknüpfungsaxiome}) encoding the mathematical structure of a theory, one needs special class of physical principles the \scare{axioms of coordination} (\german{Zuordnungsaxiome}) to ensure a univocal coordination of that structure to reality \citep{Padovani2009}. For the young Reichenbach, the latter axioms are \apr in the sense that they are \scare{constitutive} sense of the object of a physical theory. However, they are not apodeictic, or valid for all time. As it is well known, Reichenbach will soon abandon the project of a constitutive but relativized \apr. However, he firmly maintained \cop{the separation between the mathematical framework of a theory (the \scare{defined side}) and the way it related to empirical reality (the \scare{undefined side})} \rhp{**}{42} as an essential feature of his philosophy. 


% theory as a example of a neglect to lessons to that mathematic alone could be the key to physical reality



%\cop{it has been \gr revealed the physical character of geometry as a science of real space, imposed the division of labor between pure and applied geometry}.  

%There are no references to Weyl's theory, however is roughly Reichenbach discuss with Einstein would whom he was clearly in friendly term. It is probably that Einstein might his skepticism\footnote{???}.

%few days in the spring of 1920.40 The manuscript, entirely available at the Pittsburgh Archives, was ready for publication, with minor modifications, during the summer of 1920. This material makes it clear that the distinction between axioms in these terms was inserted only later in revisions. Interestingly, in the drafts of this work there are traces of another specification in a marginal note, added later but eventually omitted from the published version. With respect to the passage of p. 54 that I have quoted above, Sect. 3.2, in this note he writes:


%The distinction between connecting and coordinating axioms has been understandbly regarded as a very striking version of the relativized a priori.38 In The Theory of



\subsection{Reichenbach's Habilitation and his critique of Weyl Theory}

According to Reichenbach, \rt made this division of labor in the realm of geometry inevitable. The possibility of non-Euclidean geometries had already indicated that the \apr character of Euclidean geometry had no longer been taken for granted \rhp{}{3}. \Rt has now shown \apo that the theorems of Euclidean geometry do not apply to the physical space. It became necessary to \cop{distinguish between pure geometry as an uninterpreted formal system and applied geometry as an empirical theory of physical space} \rhp{}{76}. The propositions of pure geometry are neither true nor false. The question of the truth of geometry pertain to physics alone. However, rather in passing, Reichenbach indicated Weyl's theory as an example of how easy was to neglect this epistemological achievement. Weyl once again believed to have found a certain geometry that, for its intrinsic mathematical appeal, must be true for physical reality. \q{In this way the old mistake is repeated} \rhp{}{76}


%Euclid

%In this way does not challenge the validity of the Euclidean geometry but raised the question of its applicability to the physical space. It became inevitable to 

Reichenbach's brief outline of Weyl's theory is sufficient to grasp the gist of his argument. As Reichenbach's put it, \q{Weyl's generalization of the theory of relativity  \textelp{} abandons altogether the concept of a definite length for an infinitesimal measuring rod} \rhp{**}{**}. \cop{In Euclidean geometry a vector can be shifted parallel to itself along a closed curve so that upon its return to the point of departure it has the same direction and the same length. In the Einstein-Riemannian geometry it has merely the same length, no longer the original direction, after its return. In Weyl's theory it does not even retain the same length}. As we have seen, in this way, in addition to the \scare{metric tensor} \gmn, a \scare{metric vector} $\phin$ is introduced that, for formal reasons, could be identified with electromagnetic potential. Reichenbach conceded that Weyl's theory represented a possible generalization of Einstein's conception of \spti which, \q{although not yet confined empirically, is by no means impossible} \rhp{**}{**}. 

Reichenbach seemed to have been aware of Einstein's main objection to Weyl's proposal \citep[see]{Einstein1918b}. In general relativity, the length $ds$ of the time-like vector $dx_\nu$ is measured by a physical clock, e.g., by the crests of waves of radiation were emitted by an atom. If we maintain this interpretation, then Weyl's theories implies that \q{the frequency of a clock is dependent upon its previous history} \rhp{**}{**}. Two atomic clocks at one place, will in general not tick at the same rate when they are separated brought back together. This result appears to be contradicted by a vast amount of spectroscopic data shown that all atoms of the same type have the same systems of stripes in their characteristic spectra independently of their past history. Reichenbach conceded to Weyl that these effects might \q{compensate each other on the average} \rhp{**}{**}. Thus, the fact that \q{the frequency of a spectral line under otherwise equal conditions is the same on all celestial bodies} could he interpreted as an approximation, rather than being a consequence of the Riemannian nature of space-time \rhp{**}{**}. This, remark anticipates to a certain extent Weyl's line of defense. Reichenbach considered unacceptable was Weyl's justification for the choice of a more general geometry as the actual geometry of \spti.

According to Reichenbach, Weyl seems to imply that his non-Riemannian geometry must be true \emph{physically} because it is \emph{mathematically} superior to Riemannian geometry, being a true realization of the principle of locality. As we have seen, in Weyl geometry a vector moving close loop which would same length but different direction in Riemannian geometry, different length and different direction in Weyl's geometry. Thus, Weyl geometry eliminated the last distant-geometrical treatment of Riemannian geometry. Weyl geometry seems to be the most \scare{general geometry}, a purely infinitesimal geometry. Thus, there would be no reason to assume from the outset that a more special geometry like that of that of Riemann applies to reality. However, Reichenbach had already surmised that this generalization can be continued. In Weyl's geometry lengths can be compared at the same point in different directions, but not at distant points. \q{The next step in the generalization would be to assume that the vector changes its length upon turning around itself} \rhp{**}{**}. Probably, more complicated generalization could be thought of. \q{Nothing may prevent our grandchildren from being confronted some day by a physics that has made the transition to a line element of the fourth degree} \rhp{**}{**}\footnote{$d s^{4}=g_{\mu \sigma \sigma \tau} d x_{\mu} d x_{\nu} d x_{\sigma} d x_{\tau}$ instead of $d s^{2}=\gmn d x_{\mu} d x_{\nu}$ as in Riemannian geometry}. Thus, there is no \q{\scare{most general} geometry} that in and by itself must be physically true. No matter one pushes further the level of mathematical abstraction, \q{the difference between physics and mathematics} cannot be eliminated; geometry alone can never be sufficient to establish the reality of physical space \rhp{**}{**}. 

Weyl seemed to have forgotten the importance of the philosophical lesson of \gr, the unbridgeable \q{difference between physics and mathematics}. A mathematical axiom system is indifferent with regard to the applicability and \q{never leads to principles of an empirical theory} \rhp{**}{**}. \q{Only a physical theory could answer the question of the validity of a particular geometry} for physical space:

\q{[Thus] it is incorrect to conclude, like Weyl\footnote{\citep[227]{Weyl1918}. In the 1919 edition of the book, Weyl included a presentation of his theory. He concluded the book with an even more inspired rhetoric: \q{physics and geometry coincides with each other} \citep[263]{Weyl1919}. The tendency of physicalizing geometry that have prevailed leading protagonists of the 19th century from Gauss to Helmholtz seemed to superseded have of geometrizing physics that run from Riemann to Einstein: \q{geometry has not been physics but physics has become geometry} \citep[263]{Weyl1919}} and Haas\footnote{\citep{Haas1920}}, that mathematics and physics are but one discipline. The question concerning the validity of the axioms for the physical world must be distinguished from that concerning possible axiomatic systems. It is the merit of the theory of relativity that it renowned the question of the truth of geometry from mathematics and relegated it to physics. If now, from a general geometry, theorems are derived and asserted to be a necessary foundation of physics, the old mistake is repeated. This objection must be made to Weyl's generalization of the theory of relativity \textelp{} Such a generalization is possible, but whether it is compatible with reality \myemph{does not depend on its significance for a general local geometry}. Therefore, Weyl's generalization must be investigated from the viewpoint of a physical theory, and only experience can be used for a critical analysis. Physics is not a \scare{geometrical necessity}; whoever asserts this returns to the pre-Kantian point of view where it was a necessity given by reason \rhp{**}{**}}. 
%
To a certain extent, this objection contains the backbone of Reichenbach's criticism of the \uftp in the following decade. Weyl seems to have misunderstood the fundamental lesson of Einstein's theory. The question concerning the \q{validity of axioms for the physical world} must be distinguished from that concerning possible axiomatic systems is happened to be in reality. It is true that it is \q{a characteristic of modern physics to represent all processes in terms of mathematical equations}, and, one might add, progressively more abstract mathematics. Still, \q{the close connection between the two sciences must not blur their essential difference} \rhp{**}{**}. The truth of mathematical propositions depends upon internal relations among their terms; the truth of physical propositions, on the other hand, depends on relations to something external, on a connection with experience. \q{This distinction is due to the difference in the objects of knowledge of the two sciences} \rhp{**}{**}. The mathematical object of knowledge is uniquely determined by the axioms and definitions of mathematics. The definitions indicate how a term is related to that \citet{Schlick1918} had called \q{implicit definitions} \rhp{**}{**}. 

As it is well-known, Reichenbach will abandon the Kantian framework in which this distinction was initially presented. However, he never abandoned the idea that the separation between mathematics and physics was of paramount epistemological importance: mathematical necessity must be sharply distinguished form physical reality. The latter was the irreversible conceptual shift produced that \rt had forced upon philosophy. On \datef{24}{6}{1920}, Einstein praised Reichenbach's \german{Habilitationschrift} in a letter to Schlick \lettercpaep{Einstein}{Schlick}{19}{4}{1920}[9][378]. A few days later, Reichenbach asked Einstein to dedicate the book to him, insisting on the philosophical significance of \rt:  \qt{very few among tenured philosophers have the faintest idea that your theory performed philosophical act and that your physical conceptions contain more philosophy than all the multivolume works by the epigones of the great Kant}{Philosophen eine Ahnung davon haben, dass mit Ihrer Theorie eine philosophische Tat getan ist, und dass in Ihren physikalischen Begriffsbildungen mehr Philosophie enthalten ist, als in allen vielbändigen Werken der Epigonen des grossen Kant} \lettercpaep{Reichenbach}{Einstein}{13}{6}{1920}[10][57]. Einstein conceded, that the theory might have had philosophical relevance: \qt{The value of the th.\ of rel.\ for philosophy seems to me to be that it exposed the dubiousness of certain concepts that even in philosophy were recognized as small change \origins{Scheidemünzen}}{Der Wert der Rel.Th. für die Philosophie scheint mir der zu sein, dass sie die Zweifelhaftigkeit gewisser Begriffe dargethan hat, die auch in der Philosophie als Scheidemünzen anerkannt waren}  \lettercpaep{Einstein}{Reichenbach}{30}{6}{1920}[10][66]. Alleged \apr principles are like those parvenu that are ashamed of their humble origin and try to deny it: \qt{[c]oncepts are simply empty when they stop being firmly linked to experience}{Begriffe sind eben leer, wenn sie aufhören, mit Erlebnissen fest verkettet zu sein} \lettercpaep{Einstein}{Reichenbach}{30}{6}{1920}[10][66]\footnote{Einstein used a similar wording by commenting on the manuscript of Cassirer's \scare{Kantian} booklet on relativity. \q{Conceptual systems appear empty to me, if the manner in which they are to be referred to experience is not established} \lettercpaep{\Einstein}{\Cassirer}{6}{6}{1920}[10][44]. In particular, \q{[w]ith the interpretation of the $ds$ as a result of measurement, which is obtainable by means of measuring rods and clocks the general theory of relativity as a physical theory stands or falls} \lettercpaep{\Einstein}{\Cassirer}{6}{6}{1920}[10][44]. The gravitational redshift, can be taken as an empirical confirmation of general relativity only because different atoms of the same substance can be regarded as identically constructed clocks reproducing the identical unit of time. For this reason it is possible to \scare{normalize} the absolute value of $ds$ by counting the wave crests on atom. According, Weyl's theory deprived the $ds$ of any physical meaning. However, real \rac behave differently than predicted by Weyl theory forcing Weyl to assume an inconsistent position. According to Einstein, line \gr, Weyl's  \q{theory is based on a measuring rods geometry}, that is it presupposes the comparability of lengths. However, it entains only \q{thought measuring rods \origins{nur gedachte Massstäbe}} that behave differently from the real ones. \q{This is repugnant} \lettercpaep{\Einstein}{\Besso}{26}{8}{1920}[10][85\me]}. This remark, which Reichenbach would later quote in a published writing \citep[354]{Reichenbach1922a}, seals a sort of philosophical alliance between Reichenbach and Einstein. Against the Weyl's speculative style doing physics which reduced physical reality to geometrical necessity, \rt had has introduced a clear cut separation between between geometrical necessity and physical reality. As we shall see, this philosophical covenant will be broken less then a decade later.

%Sie gleichen Emporkömmlingen, die sich ihrer Abstammung schämen und sie verleugnen wollen

\subsection{The Reichenbach-Weyl Correspondence}
%EC an Hans Reichenbach, Hamburg 02. 07. 1920, 2 S., Pittsburgh. 285 EC an Hans Reichenbach, Hamburg 07. 07. 1920, 2 S., Pittsburgh.


Reichenbach's book was published a few months later just on that occasion the 86th Assembly of the \german{Versammlung der Gesellschaft Deutscher Naturforscher und Ärzte} in Bad Nauheim in September 1920. This meeting of fundamental importance for in the history of \rt, at least  the famous debate between Einstein and Philipp Lenard. Reichenbach met Weyl there for the first time, where the latter gave a talk on his theory \citep{Weyl1920}. Reichenbach might have assisted at the debate that followed in which Einstein rehearsed his objections against Weyl\footnote{Commenting on Weyl's talk, he pointed out once again that the \q{arrangement of \textins{his} conceptual system,} \q{it has become decisive \origins{massgebend} to bring elementary experiences into the language of signs \origins{Zeichensprache}} \citep[650]{Einstein1920c}. For Einstein, \q{temporal-spatial intervals are physically defined with the help of measuring rods and clocks}, under the assumption that \q{their equality is empirically independent of their prehistory} \citep[650]{Einstein1920c}. Einstein insisted that precisely upon this assumption rests \q{the possibility of coordinating \origins{zuzuordnen} a number $ds$ to two neighboring world points}; if this were impossible, general relativity would be robbed of \q{its most solid empirical support and possibilities of confirmation} \citep[650]{Einstein1920c}.
}, and the same time defended the possibility of a field theory of matter against Pauli\footnote{In is interesting to notice, that Einstein already showed a more flexible attitude replying to Pauli's remarks during the same discussion. \Pauli reiterated his objection based on his \scare{observability} criterion. Just as the field strength in the interior of the electron is meaningless because there is no smaller test particle than the electron, \q{one could claim something similar concerning spatial measurements, \myemph{since there are no infinitely small measuring-rods}} \citep[650]{Einstein1920c}. Einstein replied to \Pauli that \q{with the increasing refinement of the system of scientific concepts, the manner and procedure of associating the concepts with experiences becomes increasingly more complicated} \citep[650]{Einstein1920c}. In particular, he recognized that in cases such as that of the continuum theories, \q{one finds that a definite experience cannot be associated any longer with a concept} \citep[650]{Einstein1920c}. According to Einstein, there is an alternative: one can abandon \scare{continuum theories} for the sake of \Pauli's observability criterion, or replace such a \q{system of associating concepts \textins{with experiences} with a more complicated one} \citep[650]{Einstein1920c}.  Einstein's in his contributions to the the discussion which followed Max von \citets{Laue1920}'s Bad Nauheim paper. Einstein, however, in the very same sentence, did not hesitate to admit that \q{[it] is a logical shortcoming of the theory of relativity in its present form to be forced to introduce measuring rods and clocks \myemph{separately instead of being able to construct them as solutions to differential equations}} \citep[Einstein's reply to][662\me]{Laue1920}}.  Just after Bad Nauheim Moritz Schlick, who was at that time the leading philosophical authority in relativity theory wrote to Einstein about Reichenbach's book complaining about his critique of conventionalism \lettercpaep{Schlick}{Einstein}{23}{9}{1920}. The five letters that Reichenbach and Schlick exchanged between October and November of 1920\footnote{\letterhr{Schlick}{Reichenbach}{25}{9}{1920}[015-63-23]
\letterhr{Schlick}{Reichenbach}{26}{11}{1920}[015-63-22]
\letterhr{Schlick}{Reichenbach}{11}{12}{1920}[015-63-19]
\lettersa{Reichenbach}{Schlick}{29}{11}{1920}
\lettersa{Reichenbach}{Schlick}{10}{9}{1920}} turned out to be of fundamental importance in Reichenbach's intellectual biography. Reichenbach was confronted with Schlick's objection that his \scare{axioms of coordination} were nothing but \scare{conventions}. Reichenbach offered some resistance\footnote{In particular Reichenbach complained the notion of convention put too much on the arbitrariness of the principles, making even claims like the earth is round empirically not testable. The notion of simplicity seemed to Reichenbach to vague to allow for theory choice. In Poincaré conventionalism, Reichenbach missed  claim that \qt{the arbitrariness of the principles is constrained, if the principles are combined}{daß die Willkürlichkeit der Prinzipien eingeschränkt ist, sowie man Prinzipien KOMBINIERT} \letterhr{Schlick}{Reichenbach}{26}{11}{1920}[015-63-22]}, since it seems to make physical geometry empirically meaningless. However, Einstein's famous lecture on \scare{geometry and experience} of the end \datemy{27}{1}{1921} \citetitle{Einstein1921}\footnote{On \datef{14}{1}{1921} Einstein, while in Vienna, released the following declaration: \qt{A theoretical system can only claim completeness if the relationships of the concepts to the facts that can be experienced are clearly established. It is not enough, for example, to base the theory of relativity on a mathematical fundamental invariant [$ds$]. It must also be clear how this invariant is related to the observable facts as [that](2) [happened] for the fundamental concepts of Maxwell's theory by Heinrich Hertz. If one disregards this point of view, one can only arrive at unrealistic systems}{Wenn ich die gegenwärtige Lage der theoretischen Physik überschaue, so scheint mir ein Punkt von großer Wichtigkeit nicht hinlänglich beachtet zu werden. Ein theoretisches System kann erst dann Vollständigkeit beanspruchen, wenn die Beziehungen der Begriffe zu den erlebbaren Tatsachen eindeutig festgelegt sind. Es genügt zum Beispiel nicht, die Relativitätstheorie an eine mathematische auf eine mathematische Fundamentalinvariante zu gründen. Es muß auch klar sein, wie diese Invariante mit den beobachtbaren Tatsachen zusammenhängt wie [das](2] für die Fundamentalbegriffe der Maxwellschen Theorie durch Heinrich Hertz [geschehen ist.] Läßt man diesen Gesichtspunkt außer acht, SO kann man nur zu wirklichkeitsfremden Systemen gelangen} \CPAEp{7[13]}{50a}. A few days later \datef{27}{1}{1921} Einstein held \citetitle{Einstein1921} in Berlin. the lecture ultimately meant to address precisely this issue although in a popular form. (a) the invariant $ds$ is measured by ideal \rac, like the electric field strengths are measured by charged test particles (b) ideal \rac do no exist in nature (as pointed out by Poincaré). Conclusion: \emph{sub specie aeterni} geometry cannot be tested separately from the rest of physics (the famous $G+P$ formula). The choice of a particular geometry is ultimately justified by its success of delivering a good physical theory. It is assumed that solutions to the appropriate dynamical equations exist that can serve as \rac. In Weyl's theory, however, real \rac would behave differently differently from the ideal \rac in Weyl geometry. This incosistensy was the point that Einstein find unbearable. Thus, March 1921 Einstein preferred to suggest a theory in which there were no transportable ideal rods at all; only $ds=0$ would have physical meaning \citep{Einstein1921c}} seemed to have tipped the scale in Schlick's favor \citep{Reichenbach1921a}\footnote{Reichenbach seems to have ultimately translated the Einstein's $G+P$ formula into his $G + F$ formula, where $F$ is a \scare{universal force} affecting all bodies in the same way. By setting $F=0$ geometry becomes empirically testable. Reichenbach could embrace conventionalism, without having to accept that the proposition of geometry are empirical meaningless}.



%Indeed, there not only Einstein seems to defend conventionalism, but also offered the possibilities of avoiding the 

% Dingler



% Schlick position of the separation between mathematics and physics, which will be ulitately sanctioned by Einstein's himself in this famous lecture of Jarnaury 1021.

%That Einstein seemed to have then that the coordination could be directly with \rac. It is true that Einstein seemed to have a more geneours will be valid in principles. However, logical empiricists clearly interpreted as improved version of conventionalism, not simplicity by of geometry and physics. In this way, was bale to embrace conventionalism, that indeed to assure the univocality of coordination.

Reichenbach must have sent a copy of his \citetitle{Reichenbach1920a} \citep{Reichenbach1920a} also to Weyl despite the rather severe critiques he had expressed in the book. Weyl replied with some delay in February 1921. He did  not appear to be upset by Reichenbach's objections and replied rather amicably to some issues \q{which concern less the philosophical than the physical} \letterhrp{Weyl}{Reichenbach}{2}{2}{1921}[015-68-04]. In particular Weyl denied to have ever claimed that physics has been absorbed into mathematics:

\qt{It is certainly not true, as you say on p.\ 73, that, for me, mathematics (!!, e.g. theory of the $\zeta$-function?) and physics are growing together into a single discipline. I have claimed only that the \emph{concepts} in \emph{geometry} and field physics have come to coincide \textelp{} As for my extended theory of relativity, so I cannot admit that the epistemological situation is in any way different from that of Einstein. \textelp{} \emph{Experience} is in no way anticipated by the assumption of that general metric; that the laws of nature, to which the propagation of action in the ether is bound, can be of such a nature that they do not allow any curvature. \textelp{} What I stand for is simply this: The integrability of length transfer (if it exists, but I don't think so, because I don't see the slightest dubious reason for it) does not lie in the nature of the metric medium, but can only be based on a special law of action\footnote{That is on the field equations of the theory which in turned can be derived from an \scare{action principle}}. If the historical development had been different, it seems to me that no one would have thought of considering the Riemannian case from the outset. As far as the notorious \scare{dependence on previous history} is concerned, I probably expressed my opinion clearly enough in Nauheim \letterhrp{Weyl}{Reichenbach}{2}{2}{1921}[015-68-04]}{ Was nıeine erweiterte Relativitätstlıeorie betrifft. so kann ich nicht zugeben, daß da erkenntnislogisch die Sache irgendwie anders liegt wie bei Einstein. \textelp{}  Der Erfahrung wird durch die Annahme jener allgemeiner Metrik in keiner Weise vorgegriffen; dass die Naturgesetze, an welche die Wirkungsausbreituug in Äther gebunden ist, können ja von solcher Art sein, daß sie keine Streıfkenkrümmung zulassen. \textelp{} Wofür ich allein eintrete, ist dies: Die Integrabilität der Strekken\"ubertraguug (wenn sie besteht, ich glaubs uielıt. denn ich sehe nicht den geriugsteu zwiugeudeu Grund dafür) liegt uielıt im Wesen des metrischen Mediumsm, sondern kann nur auf einem besonderen Wirkungsgesetz berulıen. Ware die historische Entwieklung anders verlaufien, so seheint nıir wåire uienıand darauf verfallen. von vorn- herein gerade nur den Rieuıauuselıeu Fall iu Erwåígımg zu zielıeu. - *Nas die berüchtigte “Abhšingigkeit von der Vorgeschichte" betrifft, so habe ieh darüber wohl nıeine Ansicht deutlieh gemıg in Naulıeinı ausgesprochen. An der 4. Aufl. wird Sie wahrscheinlich vor al- lenı nıeine veränderte Stellımgnalnue zum Problem der Materie i}
%
In Bad Nauheim, Weyl outlined a now well-known speculative explanation for the discrepancy between the behavior of \scare{ideal} and \scare{real} rods. Roughly, Weyl suggested that the atoms we use as clocks might not \emph{preserve} their size if transported, but \emph{adjust} it every time to some constant field quantity, which he could identify with the constant radius of the spherical curvature of every three-dimensional slice of the world. The geometry read off from the behavior of material bodies would appear different from the actual geometry of \spti, because of the \scare{distortion} due to the mechanism of the adjustment. In 1921, the \scare{pivotal year} for unified field theories \citep[ch.\ 4]{Vizgin1994}, \Weyl (followed to some extent by \cite{Eddington1921,Eddington1921a}) reacted by expanding his strategy of \scare{doubling the geometry}, the real \scare{aether geometry} and the \scare{body geometry} distorted by the mechanism of adjustment, in three papers intended for different audiences, February \citep{Weyl1921a}, May \citep{Weyl1921d} and July \citep{Weyl1921e}. In the July paper Weyl also addresses Reichenbach's criticism publicly:

\q{From different sides\footnote{The reference is to \citealp{Reichenbach1920a} and \citealp{Freundlich1920} who however refers to \cite{Haas1920}} it has been argued against my theory, that it would attempt to demonstrate in a purely speculative way something \emph{a priori} about matters on which only experience can actually decide. This is a misunderstanding. Of course from the epistemological principle [aus dem erkenntnistheoretischen Prinzip] of the relativity of magnitude does not follow that the \textquotedblleft{}tract\textquotedblright{} displacement [Streckenübertragung] through \textquotedblleft{}congruent displacement\textquotedblright{} [durch kongruente Verpflanzung] is not integrable; from that principle that no \emph{fact} can be derived. The principle only teaches that the integrability \emph{per se} must not be retained, but, if it is realized, it must be understood as the \emph{outflow} [Ausfluß]\emph{of a law of nature }\citep[475; last emphasis mine]{Weyl1921b}}
%
Weyl never claimed that his geometry entails in its mathematical structure alone the \apr justification of its physical truth. On the very contrary, he questioned the alleged \apr status of the assumption that the comparison of lengths is path-independent. Weyl did not deny that \cop{empirical fact that two atoms of the same chemical substance placed identically in the same conditions, is independent of their prehistory}. However, the behavior of atoms does not have nothing to do with the abstract notion of parallel transport of vectors\footnote{\label{pauli}In September 1921, Pauli's encyclopedia article on relativity theory (which was finished in December, but underwent some improvements in April and May) was finally published, as part of the fifth volume of the \bt{Enzyklopädie der Mathematischen Wissenschaften}, and later as a book with an introduction by Pauli's mentor, Arnold Sommerfeld \citep{Pauli1921}. The article was unanimously considered a masterpiece, in particular by Einstein himself \citep{Einstein1921d}. Pauli introduced here the idea that Weyl provided two different versions of the theory. If Weyl's theory seeks to make predictions that are closely linked with the behavior of measuring rods and clocks, just like Einstein's theory, then the theory is clearly wrong. Not only should the effect of the electromagnetic field be noticeable in the spectral lines of a given substance, but, as \Pauli shows, \q{the differences would increase indefinitely in the course of time} \citeptra[763]{Pauli1921}[196]{Pauli1958}. If one renounces this interpretation, as Weyl later suggests, then the theory loses its physical meaning, becoming just a mathematical scheme that furnishes only \q{formal, and not physical, evidence for a connection between \textins{the} world metric and electricity} \citeptra[763]{Pauli1921}[196]{Pauli1958}. In this form the theory loess its \q{convincing power \origins{Uberzeugunggkraft}} \citeptra[763]{Pauli1921}[196]{Pauli1958}}.


%konstanten rein mathematisch berechnen lassen. Was ist z.B. ein Meter im Sinne einer solchen Theorie? Ein Raumgitter aus Pt-Atomen, deren jedes wieder aus Protonen und anderen Elementarteilchen besteht, die nach einem bestimmten Gesetz angeordnet sind; über alle Einzelheiten dieser Anordnung muss eine bestimmte Lösung der Feldgleichungen Auskunft geben. Und was ist eine Sekunde? Das so-und-so-vielfache der Schwingung in einem H-atom, der wiederum eine Lösung der Feldgleichungen entspricht. Also muss man auch aus der Feldtheorie ermitteln können, in wieviel Sekunden sich ein Lichtsignal vom einen zum

%Einstein sagt: Die Maßstablängen und die Frequenzen der Atomuhren folgen einer kongruenten Verpflanzung; mit ihrer Hilfe wird der Absolutwert des ds normiert (was nur möglich ist wegen der stillschweigend vorausgesetzten oder aus dem Verhalten der materiellen Körper abgelesenen Integrabilität der Streckenübertragung); bei solcher Normierung stellt sich der Krümmungsradius als konstant heraus. Ich sage: Maßstablängenund die Perioden der Atomuhren bestimmen sich durch Einstellungauf den Krümmungsradius; mit Hilfe des Krümmungsradius als Längeneinheit wird das ds normiert (diese Normierung ist stets möglich); als eine Folge der geltenden Naturgesetze kommt dann heraus, daß die kongruente Verpflanzung sich ebenso vollzieht, wie es die Einstellung bedingt und daher integrabel ist. Außerdem führt diese Theorie auf ein-

%Feld als ein Euklidisches ansehen kann. E: ist danach sicher, daß die „Körpergeometrie" welche in der geläufigen Weise das Maßver halten der materiellen Körper und ire Bewegung festlegt, nicht die ,Äthergeometrie" ist, sonderr diejenige Riemannsche Geometrie, in welche sie sich verwandelt, wenn man die kongruente Verpflanzung durch die Einstellung auf der Krümmungsradius ersetzt. In diesem Sinne ha® Einstein vollständig recht. Daß die der Naturgesetzen gemäß verlaufende Bewegung ernes Körpers und die Ubertragung der Uhr perioden nicht dem affinen Zusammenhang des Äthers folgt, geht übrigens schon rein forma

%rangieren. 

\subsection{The Weyl-Reichenbach Appeasement}

\hide{Im Sept. 1921 trug ich schon den ersten Bericht \"uber die Axiomatik auf dem Physikertag in .lena vor.3 Ich hatte damals gro§en Erfolg; aber niemand ist damals auf den Gedanken gekommen, mich in eine angemessene Stelle zu berufen. Ich blieb in Stuttgart sitzen. Niederschrift und Ausbau im Winter 1921 /22.}  \hide{Niederschrift und Ausbau im Winter 1921/22. Das alte MS wurde völlig umgestoßen. Genauere Behandlung der allg. Th. im Aug.-Spt. 1922. Vortrag darüber in Leipzig, Sept. 1922.}

Weyl's paper referencing Reichenbach appeared at beginning of September. A few week later Reichenbach and Weyl met again in Jena on occasion of the first  \german{Deutsche Physiker- und Mathematikertag} the first national scientific meeting held independently from the meetings of the \textit{Gesellschaft Deutscher Naturforscher und Ärzte}. Weyl gave a talk in which he tried to provide a mathematical justification for the quadratic nature of the metric\footnote{I assume that this talk would have again bee suspiction since the local is intrinscally mathematical reason. Indeed, in Reuchenbach view there was no reason}. Reichenbach presented a report of is work on the axiomatization of relativity. This report is the first testimony of the development of Reichenbach's philosophy after the Schlick-correspondence. Reichenbach suggested that in a physical theory one should distinguish the \emph{axioms} as empirical proposition about light rays, \rac\etc and the \emph{definitions} that establish the conceptual framework of the theory \lettercpaep{Reichenbach}{Einstein}{5}{12}{1921}[12][266]. After the conference, Reichenbach, sent Weyl a copy of the paper that came out in October \citep{Reichenbach1921d} possibly including a personal retraction of his criticisms, a few by writing to him a few months later from \datef{8}{1}{1922} keeping him up to date on the development of his axiomatics. This latter, which Weyl received only months later, is no longer extant. However, Reichenbach soon issued a public retraction. 

At about the same time, he started to work on a long review article on philosophical debate on relativity, that was finished in Spring 1922. On \datef{24}{3}{1922}. Erwin Freundlich sent to Einstein \q{die Druckbogen einer kritischen Untersuchung von Reichenbach auf dessen Wunsch}. \cop{Einstein expressed his general agreement with Reichenbach's analysis of the philosophical implications of relativity and praised its clarity (Docs. 119 and 366)}. The paper goes through from Schlick to Cassirer. However, the it also included a last section on Weyl's \uft: \cop{Man darf eine Darstellung der relativistischen Philosophie nicht abschließen, ohne der wichtigen Erweiterung zu gedenken, die vor 3 Jahren Weyl dem Raumproblem zuteil werden ließ} \citep{Reichenbach1922a}.

Reichenbach appears to be fully of his towards conventionalism. The choice between Euclidean and non-Euclidean geometries is conventional, that is depends on which rods are rigid\footnote{Reality does not unambiguously prescribe one geometry and that, in choosing the definition of congruence, we have it in our power to determine the nature of the geometry that will subsequently emerge. A deviation from Euclidean geometry, then the preceding argument means that we must interpret this deviation as action of a force that deforms the measuring rods; but to admit the existence of such universal forces in physics would be to introduce uncertainty into all practical measurements. By excluding this kind of forces from the outset one fixed the definition of geometry. This was for Reichenbach essentially the proposition that only physics and geometry taken together as a whole is subject to the test of experience. As we shall see this was probably not Einstein's intention. Indeed, atomic clocks there is littler room for decision}. However, however, all these \scare{rods} are assume to posses a common property. If  coincidence can be obtained in one place between a pair of points of one rod and a pair of points of the other, this coincidence will be possible at any other place and time, no matter how their prehistory might be. This might be calle the axiom of the Riemann class. The merit of Weyl is to have shown that even this axiom is not necessary: \q{Die grosse Entdeckung Weyls besteht darin, daß er einen allgemeineren Mannigfaltigkeitstypus aufdeckte, von dem auch der Riemannsche Raum nur ein Spezialfall ist} \citep[365]{Reichenbach1922a}. From this point of view, Weyl's theory is a purely mathematical discovery; it indicates a more general type of manifold that can be applied to reality when the Riemann class of axioms is not satisfied for natural rods. The fact he tried to follow this path, regardless of its empirical correctness, was a \q{genial advance \origins{genialer Vorstoß}} in the philosophical foundation of the relations between geometry and physics (\citealp[367f.]{Reichenbach1921}). Concerning the application of this mathematical apparatus to reality, Reichenbach embraces the two-theory interpretation\footnote{Reichenbach might have been inspired by \citet{Pauli1921}. However, his name is not mentioned}:

%{Another result of the restriction is that lengths at the same point but in different orientations become comparable without ambiguity. The ambiguity is limited to the comparison of lengths at different places.}

\begin{W}
\item\label{W1} In Weyl geometry, like in Riemannian geometry the length of vectors  $l^2=\gmn A^\nu A^\mu$ can be compared at the same point in different directions. Weyl dropped the assumption that $l$ remains unchanged under parallel transport at a distant point. If a vector of length $l$ is displaced from $x_\nu$ to $x_\nu+dx_\nu$, it will in general have a new length $l+dl$, so that $dl/l=\phin dx_\nu$. \q{The change in scale is measured by 4 quantities $\varphi_{\mu}$ forming a vector field}. As Reichenbach pointed out, \qt{this procedure is a purely \ls{mathematical} discoery}{Dieses Verfahren ist zunächst eine rein \ls{mathematische} Entdeckung} \citep[366]{Reichenbach1922a}, and as such is neither true nor false. It can applied to reality, if one coordinates the length $l$ as reading of some physical measuring instruments. As we have seen, in general relativity the length $ds$ of the time-like vector $dx_\nu$ is measured by a clock, e.g\. the spectra lines of an atomic clock. If this coordination is maintained so that \qt{still possible to measure also in this case}{auch in diesem Fall überhaupt noch zu »messen«} \citep[366]{Reichenbach1922a}. However, the existence of atoms with the same spectral lines shows that clocks behave differently than predicted by Weyl theory. If atomic clocks changed their periods as a function of their \spti paths, one would expect that atoms with different pasts would radiate different spectral lines. However, this is not the case\footnote{This is, of course, the celebrated objection against Weyl's theory \citets{Einstein1918c}}. Thus, it turned out that this axiom \qt{is quite well fulfilled in reality, so that the first way of generalization seems unsuitable. The latter was therefore rejected by Weyl}{Es zeigt sich allerdings, daß dieses Axiom in der Wirklichkeit recht gut erfüllt ist, so daß der erste Weg der Verallgemeinerung ungeeignet erscheint. Er wird deshalb von Weyl abgelehnt}  \citep[366]{Reichenbach1922a}

\item\label{W2} Weyl adopted a different strategy. He \qt{defines an ideal process of scale transfer, which however has nothing to do with the behavior of real scales}{definiert einen ideellen Prozeß der Maßstabs-Uebertragung, der jedoch mit dem Verhalten realer Maßstäbe nichts zu tun hat}  \citep[367]{Reichenbach1922a}. He needs this \q{Verpflanzungsprozeß} only because, he  \qt{he wants to identify the vector field \phin with the electromagnetic potential}{er das Vektorfeld \phin mit dem elektromagnetischen Potential identifizieren will}, like in \gr the \gmn where identified with the gravitational potentials \citep[367]{Reichenbach1922a}. Once one has individuated the basic geometrical field-quantities, the next step is to find the field equations \qt{then obvious forms for the most general physical equations arise}{dann naheliegende Formen für die allgemeinsten physikalischen Gleichungen entstehen} via \qt{the \scare{action principle} \origins{Wirkungsprinzip}}{(für das »Wirkungsprinzip«}  \citep[367]{Reichenbach1922a}---a variational principle applied to the invariant integral $\int \mathfrak{W} d x$ for a specific Lagrangian $\mathfrak{W}$. According to Reichenbach, however, in this way the \qt{theory loses its convincing character \origins{überzeugenden Charakter} and comes dangerously close to a mathematical formalism}{Theorie ihren überzeugenden Charakter und kommt einem mathematischen Formalismus bedenklich nahe, der um eleganter mathematischer Prinzipien willen die Physik unnötig kompliziert; und aus diesem Gedanken heraus wird die Weylsche Theorie von Physikern (besonders auch von Einstein) sehr zurückhaltend betrachtet.}\footnote{This choice of words is practically similar who claims the in the second form the theory lost his \emph{Uberzeugunggkraft} \citeptra[763]{Pauli1921}[196]{Pauli1958}}  \citep[367]{Reichenbach1922a}. For this reason, according to Reichenbach, \qt{Weyl's theory is viewed very cautiously by physicists (especially by Einstein)}{aus diesem Gedanken heraus wird die Weylsche Theorie von Physikern (besonders auch von Einstein) sehr zurückhaltend betrachtet}  \citep[367]{Reichenbach1922a}
\end{W}
% scheint sehr schwerwiegend. Wen jetzt auch kein direkter Widerspruch zur Erfahrung vorhanden ist, so scheint die Theorie dadurch doch vom physikalischen Standpunkt aus ihrer inneren Uberzeugunggkraft beraubt.387) So ist jetzt z. B. der Zusammenhang zwischen Elektro-
Ultimately, Reichenbach seems to imply that both strategies led to a dead end. From the point of view of \cref{W1} Weyl geometry is empirically inadequate from that of \cref{W2}, it does not have empirical content. Nevertheless, Reichenbach conceded that his original objections against Weyl's theory missed the point. Neither \cref{W1} nor \cref{W2} can be considered attempts to prove \apr that Weyl's non-Riemannian geometry must be true for reality for \apr reasons:

\qt{However, I have to retract my earlier objection [\citep[73]{Reichenbach1920}] that Weyl wants to \ls{deduce} physics from reason, after Weyl has cleared up this misunderstanding [\citep[475]{Weyl1921b}]. Weyl takes issue with the fact that Einstein simply accepts the unequivocal transferability of the standards. He does not wish to dispute the Riemann-class axiom for natural standards, but only to demand that the validity of this axiom, since it is not \ls{logically} necessary, should be understood as \scare{a consequence of a law of nature}. I can only agree with Weyl's demand; it is the importance of mathematics that they are. I can only agree with Weyl's demand; it is the importance of mathematics that, in uncovering more general possibilities, it marks the special facts of experience \ls{as special} and thus preserves physics from naivity \origins{Simplizität}. Admittedly, Weyl succeeds in explaining the unambiguous transferability of natural standards only very imperfectly. But the fact that Weyl tried to go this way, regardless of the empirical correctness of his theory, remains an ingenious advance towards the philosophical foundation of physics \citep[367f.]{Reichenbach1922a}}{Ich muß jedoch meinen früher erhobenen Einwand (47, S. 73), daß Weyl die Physik aus der Vernunft deduzieren will, zurücknehmen, nachdem Weyl dieses Mißverständnis aufgeklärt hat (72, S. 475). Weyl stößt sich daran, daß bei Einstein die eindeutige Uebertragbarkeit der Maßstäbe einfach hingenommen wird. Er will nicht das Axiom der Riemannklasse für natürliche Maßstäbe bestreiten, sondern nur fordern, daß die Geltung dieses Axioms, da es nicht logisch notwendig ist, als »Ausfluß eines Naturgesetzes verstanden werde«. Ich kann dieser Forderung Weyls nur zustimmen; es ist die Bedeutung der Mathematik, daß sie mit dem Aufdecken allgemeinerer Möglichkeiten die speziellen Tatbestände der Erfahrung als speziell kennzeich net und SO die Physik vor Simplizität bewahrt. Freilich gelingt Weyl die Erklärung der eindeutigen Uebertragbarkeit natürlicher Maßstäbe nur sehr unvollkommen 1). Aber daß Weyl diesen Weg zu gehen versucht, bleibt unabhängig von der empirischen Richtigkeit seiner Theorie ein genialer Vorstoß zur philosophischen Grundlegung der Physik.}

%1) Seine Erklärung, nach der die eindeutige Uebertragbarkeit durch Einstellung der Maßstäbe auf den Krümmungsradius der Welt« erfolgt, ist im wesentlichen nur ein andrer sprachlicher Ausdruck für den vorliegenden Tatbestand, keine Zurückführung auf in allgemeineres Gesetz. Insbesondere hat diese »Einstellung« nichts zu tun mit seiner kongruenten Verpflanzung, so dass diese physikalisch leer bleibt. An anderer Stelle (70, S. 133) findet sich eine

%Ultimately, Reichenbach argued, one would need to proper theory of matter that explains why all electron are of the certain size and charge and all atoms the same spectral line, while atoms in a rock-salt crystal are at the same reciprocal distances. 

Weyl's point was not that the axiom of Riemann class is necessarily false for \apr reasons, but that is not necessarily true: \cop{It cannot be a coincidence if two measuring rods placed next to each other are of the same length regardless of their location; this coincidence cries out  for an explanation}. Weyl's explanation of the apparent Riemannian behavior of \q{durch Einstellung der Maßstäbe auf den Krümmungsradius der Welt«}  \citep[368\fn{1}]{Reichenbach1922a} only means posing a problem rather than providing an answer. The problem could be solved only by developing a proper theory of matter. However, according to Weyl, it was legitimate to deduce the Riemannian behavior of real clocks from a theory based on the non-Riemannian behavior of geometrical vectors. In this way, however, the \qt{stellung« nichts zu tun mit seiner kongruenten Verpflanzung, so dass diese physikalisch leer bleibt}{congruent transplantation \textelp{} remains physically empty}  \citep[368\fn{1}]{Reichenbach1922a}. If the non-Riemannian congruent transplantation of vectors must be, then the real rods should better behave in a non-Riemannian way. 

Thus, Reichenbach concluded, the main achievement of Weyl was mathematical. As usual mathematics serve to enlarge the range of possibilities among which physicists can chose. This processes is however far from being concluded with Weyl's rather special affine connection:

\qt{The \ls{philosophical} significance of Weyl's discovery consists in the fact that it proved that the problem of space cannot be closed even with Riemann's concept of space. If the epistemology of today wanted to extend the assertion of Kant's transcendental aesthetics to the point that the geometry of experience must in any case at least have a Riemannian structure, it is held back by Weyl's theory. For that Weyl's space is at least \ls{possible} for reality cannot be denied. One must not even believe that Weyl's theory has reached the highest level of generality. Einstein has shown (14) that Weyl's requirement of the relativity of magnitude can also be satisfied without making use of Weyl's method of measurement. After that, Eddington (15) again developed a generalization of which Weyl's space class is only a special case, and Eddington's space class is again included as a special case in a more general one found by Schouten. The merit of Schouten's theory is that it gives the conditions under which the class of space developed is the most general; they are very general conditions, like differentiability and the like. But of course there is no absolutely most general space class; and the history of the mathematical problem of space may teach epistemology never to make general claims. There are no most general terms \citep[368\fn{1}]{Reichenbach1922a}}{Die \ls{philosophische}  Bedeutung der Weylschen Entdeckung besteht deshalb darin, daß sie bewiesen hat, daß ein Abschluß des Raumproblems auch mit dem Riemannschen Raumbegriff nicht gegeben ist. Wollte also die Erkenntnistheorie heutP- die Behauptung der transzendentalen Aesthetik Kants dahin erweitern, daß die Geometrie der Erfahrung auf jeden Fall wenigstens von Riemannscher Struktur sein muß, so wird sie durch die Weylsche Theorie daran zurückgehalten. Denn daß der Weylsche Raum wenigstens für die Wirklichkeit  \ls{möglich} ist, läßt sich nicht bestreiten. Man darf nicht einmal glauben, daß mit der Weylschen Theorie nun die höchste Stufe der Allgemeinheit erklommen sei. Einstein hat gezeigt (14), daß man die Weylsche Forderung der Relativität der Größe auch befriedigen kann, ohne von dem Weylschen Meßverfahren Gebrauch zu machen. Danach wurde von Eddington (15) wieder eine Verallgemeinerung entwickelt, von der die Weylsche Raumklasse nur ein Spezialfall ist, und die Eddingtonsche Raumklasse ist wieder als Spezialfall in eine allgemeinere eingegangen, die von Schouten (63) gefunden wurde. Der Vorzug der Schoutenschen Theorie besteht darin, daß hier die Bedingungen angegeben werden, unter welchen die entwickelte Raumklasse die allgemeinste ist; es sind sehr allgemeine Bedingungen, wie Differenzierbarkeit und ähnliches. Aber eine schlechthin allgemeinste Raumklasse gibt es natürlich nicht; und die Geschichte des mathematischen Raumproblems mag der Erkenntnistheorie eine Lehre sein, niemals schlechthin allgemeine Behauptungen aufzustellen. Es gibt keine allgemeinsten Begriffe}
%
This passage shows that Reichenbach was now familiar with some of the latest advances in differential geometry. Not only with Weyl, but also with \citets{Eddington1921} recent theory in which length cannot be compared even not at the same place in different directions. Moreover Reichenbach was familiar with \citets{Schouten1922} classification of connections. Thus, he was already aware that also the tacit assumption that the connection is symmetric could be dropped. In general by further relaxing the constraints on symmetry and on the relationship between the connection and the metric, a great number of ways to incorporate the electromagnetic field. In Reichenbach's view one should be free to chose among all these mathematical possibilities; however, mathematics alone cannot provide a criterion of choice for which possibility is realized in nature. Indeed, there was nothing special in Weyl geometry. \cop{Mathematics is the science of possibility, physics only is the science of reality}.

%Weyl's strategy \label{W2} appeared to him unconvincing since deprived geometry of every empirical content. Ultimately one might have used a metrical geometry which is contradicted by the actual behavior of \rac. Weyl's opinion was quite different. 

For Reichenbach there should have been no limit, in the choice of what mathematical structure one can chose. However, Reichenbach believed to be essential to provide geometrical structure it was essential to give it a physical interpretation in advance, before starting doing physics as in \cref{W1}. Indeed, in this form the makes predictions that could be confirmed and disconfirmed empirically. Embracing \cref{W2} Weyl have unwittingly a sort of Poincareain move introducing a universal distortion of all measuring instruments, thus depriving geometry of any empirical content\footnote{Indeed Weyl seem to introduce a sort of universal force that distorts all measuring instruments so that they behave in a Riemannian way, while the true geometry of \spti is non-Riemannian. A different approach was suggested by Eddington, which considered non-Riemannian geometries as mere \scare{graphical representation} that serve to organize different theories into a common mathematical framework. The \q{natural geometry} remains Riemannian \citep{Eddington1921}}. However, Weyl's stance was very different. In a latter written when the review article was already in press, Weyl confessed to Reichenbach that he never abandoned \cref{W1} in favor of \cref{W2} since he never adopted \cref{W1} in the first place: \q{Den Plan, starre Maßstäbe mit meiner Verpflanzımg zu identifizieren, habe ich aufgegeben. weil ich ihm nie gehabt habe}; on the contrary, \q{ich war überrascht, als ich salı, daß Physiker das in meine Worte hineininterpretiert hatten} \letterhr{Weyl}{Reichenbach}{20}{05}{1922}[015-68-02]. The atoms the we use as clocks are physical systems like any other that do not have in principle much to do with abstract mathematical structure one uses. It is the theory that decides whether we should is them as reliable clocks or not. In general, one starts from a certain mathematical structure; a physical interpretation in terms of \rac, test particles can only be provisional. Ultimately one has to find the field equations, and require that solutions to this equations exist exhibiting the postulated behavior of \rac. This reasoning applies to Einstein and Weyl's theory: \qt{Einstein has to show that from the dynamics of the rigid body it follows that the rod always has the same length, measured in his $ds$. Similarly, I have to show that the rod has always has the same length normalized $ds$ normalized by $R = const$}{Genau, wie  Einstein zeigen muß, daß aus der Dynamik des starren Körpers heraus ein solches Verhalten folgt, daß der Maßstab immer dieselbe Länge hat, gemessen in seinem $ds$, so muß ich zeigen, daß er immer das gleiche durch $R = const$ normierte $ds$ hat} \letterhrp{Weyl}{Reichenbach}{20}{05}{1922}[015-68-02]. In both cases the behavior of rods and clocks comes out as a byproduct of the theory. The unit of time should defined a certain number of spacing between the atoms of a cubic crystal system; each of atom in turn consists of electrons and protons arranged according to a specific law. A specific solution to the field equations must provide information about all the details of this arrangement.  The unit of time is a certain multiple of the vibration in an hydrogen atom, which in turn corresponds to a solution of the field equations. 



%Weyl managed to obtain both Maxwell's equations and new gravitational equations, different from Einstein's, and allowing a closed world introducing the cosmological term by hand.  Thus the letter could be used 



%




%Wenn wir zu Anfang dieses Kapitels mit Einstein die Maßbestimmung im Äther mit Hilfe von Maßstäben und Uhren definierten, so kann man das nur als eine vorläufige Anknüpfung an die Erfahrung gelten lassen, wie etwa auch die Definition der elektrischen Feldstärke als ponderomoto- rische Kraft auf die Einheitsladung. Es ist nötig, den Kreis zu schließen; nachdem einmal die physikalischen Wirkungsgesetze aufgestellt sind, muß man bewez"sen, daß hier die geladenen Körper unter dem Einfluß des elektromagnetischen Feldes, dort die Maßstäbe unter dem Einfluß des me- trischen Feldes zufolge der Wirkungsgesetze jenes Verhalten zeigen, das wir anfänglich zur physikalischen Definition der Feldgrößen benutzt haben. Es ist heute sicher, daß wir dazu der Ansätze der Quantentheorie be- dürfen. Die Bohrsehe Atomtheorie 39) zeigt, daß die Radien der Kreis- bahnen, welche die Elektronen im Atom beschreiben und die Frequenzen d e s a u s g e s e n d e t e n L i c h t s s i c h u n t e r B e r ü c k s i c h t i g u n g d e r Kon~titution des Atoms bestimmen aus dem Planckschen Wirkungsquantum, aus La- dung und Masse von Elektron und Atomkern. Ähnlich wie mit jenen Radien wird es sich mit den Gitterabständen in einem kristallinischen Medium verhalten und infolgedessen auch mit der Länge eines gegebenen starren Maßstabs. Die neueste Entwicklung der Atomphysik hat es wahr- scheinlich gemacht, daß die Urbestandteile aller Materie das Elektron und der Wasserstoffkern sind; alle Elektronen haben die gleiche Ladung und Masse, ebenso alle Wasserstoffkerne. Daraus geht mit aller Evidenz her- vor, daß sich dz"e Ato?nmassm, Uhrperioden und Maßstablängen 11icht durclt irgendeine Beharrungstendenz erhaltm; sondern es handelt sich da um einen durch die Konstitution des Gebildes bestimmten Gleichgewichts- zustand, auf den es sich sozusagen in jedem Augenblick neu einstellt. Das erklärt die folgende grundlegende Tatsache, von der wir bei der De- finition des metrischen Feldes ausgingen (ich spreche sie, statt für die geometrischen Radien der Atombahnen, lieber für die Atommassen aus, die offenbar etwas physikalisch Ursprünglicheres sind): Ein Wasserstoff- und ein Sauerstoffatom mögen jetzt, wo sie sich nebeneinander an der gleichen Feldstelle P befinden, ein bestimmtes Massenverhältnis I 1oo8: 161ooo be- sitzen; sie bewegen sich getrennt voneinander während langer Zeit in der Welt und treffen in einem viel späteren Weltpunkte P von neuem zu- sammen; wir finden daselbst genau das gleiche Massenverhältnis wie z"n P. Dies Massenverhältnis stellt sich nicht in P ein, weil es in P geherrscht hatte, sondern weil es durch die Konstitution des Wasserstoff- und des Sauer- stoffatoms erzwungen ist. Die Wiederkehr des gleichen Massenverhältnisses muß also darauf beruhen, daß sich jede Atommasse einzeln auf ein be- stimmtes Verhältnis einstellt zu dem an der betr. Feldstelle herrschenden Wert einer gewissen Feldgröße, welche die Dimension einer Länge (=Masse) besitzt.


\section{Geometrization: The Reichenbach-Einstein Correspondence 1926--1927}
\label{geometrization}

%Noting the very special nature of Weyl's generalization, Eddington started Dy assuming that there was no a priori connection between the metric and an initially arbitrary symmetric connection. He observed that the curvature or Riemann tensor from the affine connection without any metric. This Ricci tensor, however, will in general not be symmetric, even though the connection is. What has this to do with physics? Eddington noted that an affine connection enables us to compare tensors at neighboring points (in particular, to say when two neighboring vectors are parallel). He regarded the possibility of such a comparison between quantities at neighboring points in space-time as the minimum element necessary to do any physics: "For if there were no comparability of relations, even the most closely adjacent, the continuum would be divested of even the rudiments of structure and nothing in nature could resemble anything else."27


%According to Einstein \q{Begriff de Parallelverschiebung entstammt wie alle Begriffe der euklidischen Geometrie der Betrachtung der Lagerungs-Gesetze bezw. der Gesetze der relativen Verschiebung starrer Körper. Daher erhält die (Behauptung) Festsetzung ihre Evidenz, dass eine Strecke bei Parallelverschiebung ihren Betrag nicht ändere}.

%t\footnoteh{, that he even defined a form of \scare{Hegelry} \lettercpaep{Einstein}{Zangger}{1}{1}{1921} Überhaupt ist es bedenklich, in der Relativitätstheorie etwas im Formalen quasi a priori begründet zu sehen. Es handelt sich um (Konsequenzen) Anpassung der Theorie an ganz bestimmte Fakta (Konstanz d. L. Geschw.; Gleichheit der trägen und schweren Masse), nicht um im menschlichen Geist logisch Bedingtes und Präformiertes; anders beim Weylschen Versuch; fast möcht' ich es Hegelei nennen}


Up to this point, Reichenbach had good reasons to believe that these considerations were broadly in agreement with that of Einstein, who had continued to express skepticism towards Weyl's attempt (\lettercpae{Einstein}{Weyl}{6}{6}{1922}[13][219] \lettercpaep{Einstein}{Zangger}{18}{6}{22}[13][241]). The situation changed by the end of the 1922, when Einstein, during a trip to Japan, started to believe that Eddington's theory had potentialities that had not be exploited. On the shipboard he dotted down a five-page manuscript dated January 1923, from Singapore. Curiously, the third, fourth, and fifth pages \cop{were written on the back of the beginning of a typescript of Reichenbach’s Jena's talk} \citep{Reichenbach1921}. \citet{Eddington1921} had extended Weyl's approach, but starting with the solo affine connection \Gtmn without making any assumption about the relationship between the latter and the metric \gmn. In this context, vectors have no lengths, thus the theory avoided Weyl's inconsistency of having geometrical lengths behaving differently by real \rac\footnote{See \cref{conforma}}. Einstein was ready to embrace a more speculative strategy in the search for a \uft. He did not give any physical meaning to the symmetric affine connection \Gtmn and simply used it find a suitable Lagrangian $\mathcal{H}$. The comparison with experience would take place only \apo by integrating such equations and finding solutions corresponding to elementary particles.


%Although he does not have relations with the Notgemeinschaft, he is ready to support his request warmly, and will talk to Fritz Haber personally. TKS. [20 081]. 122. From Hans Reichenbach Stuttgart, 10 July 1923 Thanks AE for Abs. 89. Is sorry that the Academy did not agree to print his manuscript. Asks whether this was due to financial or other reasons. Springer cannot accept his suggestion that the Notgemeinschaft cover part of the printing cost. Verlag Witwer, on the other hand, has agreed to publish the work with support from the Notgemeinschaft. Enclosed sends the request to the Notgemeinschaft and asks to forward it to Fritz Haber and to put in a good word for him personally. Asks to mail back the manuscript. ALS. [20 082].

% asks again whether the manuscript of Reichenbach 1924 may be published in the Sitzungsberichte. Requests a prompt reply because he must give an answer to Springer.  50. Ilse Einstein to Hans Reichenbach Berlin, 12 May 1923 Informs him that AE has not seen Reichenbach's letter since he already left for the Netherlands. She no longer forwards AE's mail since he brings it all back unopened. She assures Reichenbach that she will give his letter to AE upon his return. AKS. [87 944]. 89. To Hans Reichenbach Berlin, 9 June 1923 In response to Abs. 34 and Abs. 44,

Einstein published this idea soon returning from Japan \citep{Einstein1923c,Einstein1923e}. On \datef{2}{5}{1923}, Reichenbach requested a copy of Einstein's paper for the Proceedings of the Prussian academy \letteraeap{Reichenbach}{Einstein}{2}{5}{1923}[20 080]. In the meantime, Reichenbach had concluded the manuscript of his book on the axiomatic and asked for Einstein's help in finding a publisher. Do to lack of founding, Reichenbach managed to publish the book only a year later on March 1924. With the \citetitle{Reichenbach1924} \citet{Reichenbach1924} Reichenbach's philosophy started to assume a more recognizable contour. In particular Reichenbach introduced the distinction between conceptual definitions used in mathematics and coordinate definitions used in physics which relate the concept of a theory to a \q{piece \originsg{Ding} of reality} \citeptra[5]{Reichenbach1924}[8]{Reichenbach1969}. There is little doubt that Reichenbach believed that this epistemological model was Einsteienan in spirit. However, at about that time Einstein explicitly confessed that he has changed in mind on the topic \citepp[1692]{Einstein1924}[see][]{Giovanelli2014a}. In particular, Einstein seemed that every concept of a theory, and in particular those pertaining to geometrical structure should receive a coordinative definition\footnote{\q{In particular, I would like to mention that criticism was rightly aimed against one statement by the reviewer: that a concept should only be permissible in physics when it can be established whether or not it applies in concrete cases of observation; [9] it is objected that, in general, it is not to an individual concept that possible experiences must correspond but to the system as a whole} \citep[1691]{Einstein1924}}. Ultimately only geometry and physics taken together could be compared with experience\footnote{In \gr that the \gmn had a physical meaning from the outset \emph{ex ante} that is the \rac that they can be measured with \rac with respect to given coordinate system. In his recent theory, Einstein started from the $\Gamma\tmn$ without giving to this quantity any physical meaning. If using the $\Gamma\tmn$ leads to the right set of field equations, then the use of $\Gamma\tmn$ as a fundamental variable is justified \emph{post facto}. In this sense only geometry (\gmn \Gtmn\etc) and physics (field equations) together could be compared with experience (finding appropriate exact solutions corresponding to electrons)}.

Weyl's negative review of \citep{Weyl1924} put an end to their previously amicable relationship. Reichenbach felt that Weyl---who showed some sympathies for phenomenology---had used his authority as mathematician to attack his philosophical reading of relativity \citep{Reichenbach1925}. What was worse, Reichenbach must have sensed that Weyl's speculative style of doing physics had become more prevalent. Einstein's last works seemed to show that he had also come under its spell. It is not surprising that Reichenbach might have felt necessary to make the case for an empiricist reading of \rt in a more popular form. At about the same time, Reichenbach, decided to write a book with the title \bt{Philosophie der exakten Naturerkenntnis}, which will then divided in two parts of which only the first one on space and time will be published. He wrote the first chapters in March 1925. During those same months, Reichenbach, despite the support of Max Planck, was struggling to obtain his \german{Umhabilitation}\footnote{The process of obtaining the \latin{venia legendi} at another university} from Stuttgart to Berlin in order to be appointed to a chair of natural philosophy that had been created there \citep{Hecht1982}. Reichenbach had been attacked for his pacifists positions during the war. After the situation seemed to turn for the best\todo{???}, in October 1925 he started to work more consistently on the book. He interrupted the drafting of the manuscript to follow the emerging quantum revolution at the turn 1926. As he further wrote: \qt{March-April 1926 Weyl's theory was worked on and the peculiar solution of \S49 was found. The entire \Ap was also written at that time. (Correspondence with Einstein)}{März-April 1926 wurde die Weylsche Theorie bearbeitet u. die eigentümliche Lösung des § 49 gefunden. Auch wurde damals der ganze Anhang geschrieben. (Korrespondenz mit Einstein)}. The correspondence with Einstein has been preserved. It testifies about Reichenbach's concern Einstein's style of doing physics becoming progressively more speculative\todo{better}. 

%that allowed him to evade the Scylla of positivism and the Charybdis of Kantianism

\subsection{Reichenbach's Geometrization of the Electromagnetic Field}
%
During a trip to South America in 1925 Einstein indeed became interested in the rationalistic and realistic reading of relativity proposed by \'Emile \citetp{Meyerson1925} \CPAE{14}{455, 6; \datedm{12}{3}{1925}} who could provide for the search of a \uft\footnote{Meyerson rationalistic realism that well supported the \uftp. In particular, to philosophical alternative to both empiricists and positivists. Against the positivists he defended the logical independence of concepts from sensory experiences; against the Kantians that uses non-empirical \scare{ideal} conceptual constructions were mere conventions. Against both Meyerson justified that the physicists are justified to assume that certain conceptual construction exist in reality, say electromagnetic field, electron independently of observation}. However he also realized that the Weyl-Eddington-Schouten line had dry up (\CPAE{14}{455, 9; \datedm{17}{3}{1925}})\footnote{These doubts became certainty when Einstein returned to Europe. \qit{On \datedm{1}{6}{1925}, I got back from South America}{Einstein wrote to Besso}{I am firmly convinced that the whole line of thought Weyl-Eddington-Schouten does not lead to anything useful from a physical point of view and I found a better trail that is physically more grounded}{Am 1. Juni bin ich von S\"udamerika wiedergekommen ... Ich bin fest \"uberzeugt, dass die ganze Gedanken-Reihe Weyl-Eddington-Schouten zu nichts physikalisch brauchbarem f\"uhrt und habe jetzt eine andere Spur gefunden, die mehr physikalisch fundiert ist} \lettercpaep{Einstein}{Besso}{5}{6}{1925}[15][2]}. By returning from South America he embraced what he considered to be a new approach. He introduced non-symmetric \Gtmn and the \gmn to be treated as independent fields in the variation. The antisymmetric part of the \gmn was the natural candidate for the representation of the electromagnetic field, at least for infinitely small fields. The physical test depended, as usual, on the construction of exact regular solutions corresponding to elementary particles. The paper was published in \datemy{1}{9}{1925} with the ambitious title \citetitle{Einstein1925}  \citep{Einstein1925}. However, by that time Einstein seemed to have already lost his confidence in that approach and moved on\footnote{In July he was still convinced that he had \qt{really found the relationship between gravitation and electricity}{Ich glaube nun, die Beziehung zwischen Gravitation und Elektrizit\"at wirklich gefunden zu haben} \lettercpaep{Einstein}{Millikan}{13}{7}{1925}[15][20]. However, during the summer, Einstein had already started to nurture some skepticism (\lettercpae{Einstein}{Ehrenfest}{18}{8}{1925}[15][49]; \lettercpae{Einstein}{Millikan}{13}{7}{1925}[15][20]; \lettercpae{Einstein}{Ehrenfest}{18}{9}{1925}[15][71]). The paper was published at the beginning of September, and by that time, Einstein probably already moved on (\lettercpae{Einstein}{Rainich}{13}{9}{1925}[15][106]; see \cite{Einstein1927c})}.

At the turn of the year, after working on the new quantum mechanics must have come to read Einstein's new paper. On \datef{16}{3}{1926}, Reichenbach sent a letter to Einstein in which, after discussing his academic misfortunes, he made some critical remarks \citep{Einstein1925a}. Reichenbach was quite skeptical of the viability of Einstein's current style of doing physics:

\qt{I have read your last work on the extended Rel.\ Th.\footnote{\cite{Einstein1925a}} more closely, but I still can't get rid of a sense of artificiality which characterizes all these attempts since Weyl. The idea, in itself very deep, to ground the affine connection independently of the metric on the \Gtmn alone, serves only as a calculation crutch here in order to obtain differential equations for the \gmn and the \phin and the modifications of the Maxwell equations which allow the electron as a solution. If it worked, it would of course be a great success; have you achieved something along these lines with Grommer? However, the whole thing does not have the beautiful convincing power \origins{Ueberzeugungskraft} of the connection between gravitation and the metric based on the equivalence principle of the previous theory}{Ich habe jetzt Ihre letzte Arbeit zur erweiterten Rel. Th. genauer gelesen, aber ich kann auch da das Gefühl des Künstlichen nicht los werden, das allen diesen Versuchen seit Weyl anhaftet. Die an sich doch sehr tiefe Idee, den affinen Zusammenhang unabhängig von dem metrischen zu begründen allein auf die $\Gamma^{i}_kl$ dient doch schließlich nur als Rechenknüppel, um Differentialgleichungen für die $g_{ik}$ und $\varphi_ik$ zu bekommen und solche Abänderungen der Maxwellschen Gleichungen zu bekommen, die das Elektron als Lösung zulassen. Wenn das geht, ist es natürlich ein großer Erfolg; haben Sie eigentlich mit Herrn Grommer etwas in dieser Richtung errichtet? Aber die ganze Sache hat doch nicht die schöne Ueberzeugungskraft, wie die auf das Aequivalenz-Prinzip gestützte Verknüpfung von Gravitation und Metrik in der früheren Theorie}[\letter{Reichenbach}{Einstein}{16}{3}{1926}][20-83][EA]
%
Reichenbach's objections are quite sensible. In general relativity the choice of the \gmn as fundamental variables anchored in the principle of equivalence. The latter justified the double meaning of the \gmn, as determining the behavior \rac, as well as the gravitational field. On the contrary, Einstein's new theory introduces the non-symmetric affine connection \Gtmn independently of the metric \gmn without giving to these field variables any physical motivation. The separate variation of the metric and connection was nothing more than a \scare{calculation device} to find the desired field equations. Only in hindsight, for formal reasons, the symmetric part was identified of \gmn was identified with the gravitational field and antisymmetric with the electromagnetic field. In this form the theory has little he \scare{convincing power} (\german{Überzeugungskraft})----the same expression that \citet[367]{Reichenbach1921} had used for characterizing Weyl's theory in his second form. Reichenbach would have been ready to retract his cricitism, if Einstein's theory delivered the \scare{electron}. This concession, however, barely hide his skepticism that his was a concrete possibility\footnoteh{wiesen hat ? Wevl versuchte diesen Gedanken durchzuführen, indem er für die Weltgeometrie einen allgerneineren Rahmen konstruierte, der noch über die Riemannsche Geometrie hinausgeht; ihm folgten Eddington und Einstein. Freilich kann dabei vorerst eine Lösung des eigentlichen Quantenrätsels nicht erwartet werden; es würde aber schon viel sein, wenn die atomistische Struktur der Materie, das körnerartige Auftreten der Elektrizität, verständlich würde. Diese anfangs viel Erfolg versprechenden Versuche scheinen sich jedoch nicht zu bewähren. Daneben aber besteht der Weg, nicht durch eine Revision der geometrischen Grundlagen, sondern durch Vorgehen in einer andern philosophischen Richtung die Lösung zu versuchen. Was die Sprünge im \citep[see][]{Reichenbach1926}} 

Einstein replied on \datedm{20}{3}{1926}  that he agreed with Reichenbach's \scare{$\Gamma$-Kritik}: \qt{I have absolutely lost hope of going any further using these formal ways}{Ich habe jetzt jede Hoffnung aufgegeben, auf diesem formalen Wege weiter zu kommen}; \q{without some real new thought} he continued, \qt{it simply does not work}{Ohne einen wirklich neuen Gedanken geht es nicht} \letteraeap{Einstein}{Reichenbach}{20}{3}{1926}[20-115]. Einstein's reaction reflects his disillusion with the attempts to achieve the sought-for unification of gravitational and electromagnetic field by searching form some combination of \Gtmn and \gmn. He would have probably been less ready to embrace the actual implications of Reichenbach's $\Gamma$-critique, the requirement that the operation of parallel displacement of vectors should received a \scare{coordinative definition} fro the outset. Reichenbach took the opportunity of Einstein's positive reaction and on \datef{31}{3}{1926} he sent him a note in which he developed the $\Gamma$-critique in details\todo{note}. In the note---which is extant---Reichenbach had developed his own \uft, a theory that upheld the $\Gamma$-requirement.



%In general relativity the motion of particles under the influence of both the gravitational field by the geodesic equations. By parallel-displacing a vector $u^\tau$ indicating the direction of a curve $\xn(\ap)$ at any of its points, one can define a special class of curves, the straightest lines. When the particle is not accelerating that is maintain the same velocity, the direction of the velocity vector does not vary. Thus, the motion of force-free particles can be used to define physically the straightest line between two \spti points.
%
%\begin{equation*}\label{eq:forceequation}
%\frac{d {u^\tau}}{d\ap} - \notateol{\Gtmn}{2}{\texts{Levi-Civita connection}} u^{\mu} u^\nu = 0
%\end{equation*}
%
%Planets describe a trajectory which itself is assigned because it is attracted by the force of the sun, but we say: the planet moves along the straightest line defined by the \Gtmn. In order to describe charged particles under the influence of both the gravitational and the electromagnetic a force term must be introduced. 
%
%\begin{equation*}\label{eq:forceequation}
%\frac{d {u^\tau}}{d\ap} - \notateol{\Gtmn}{2}{\texts{Levi-Civita connection}} u^{\mu} u^\nu = \notateur{\frac{\rho}{\mu} f_{\tau}^{\mu} u^\tau}{2}{\texts{force term}}
%\end{equation*}
%
%The force term indicates that the force experienced by charged particles is directly proportional to the charge and inversely proportional to the mass. The force term acceleration with respect to a given geodesic, i.e. the deviation from geodesic motion.  Thus gravitation appears to be geometrized, but  electromagnetism does not. 
%
%Reichenbach's idea was quite simple. In order to geometrize the electromagnetic field, one should transform the force equation into a geodesic equation. To this purpose Reichenbach introduced a more general affine connection \Gtmnbar in which the length of vectors is preserved under parallel transport but  the condition of symmetry is abandoned $\Gtmnbar \neq \bar{\Gamma}^\tau_{\nu\nu}$. The operation of displacement can receive a physical interpretation if one identifies the vectors in question with the velocity vector $u^\tau=dx_\nu/\dap$ of a charged mass point as the object of the displacement. A non-symmetric connection is always the sum of a symmetric connection \Gtmn and a non-symmetric three rank tensor with two lower indexes. Reichenbach introduced an affine connection with following definition:
%
%\begin{equation*}\label{eq:reichenbachconnection}
%\Gtmnbar = \notateor{\Gtmn}{2}{\texts{Levi-Civita connection}} + \notateul{\frac{\rho}{\mu} f_{\tau}^{\mu} u^\tau}{2}{\texts{skew-symmetric tensor}}
%\end{equation*}
%%
%Put in this form one can see by simple inspection the force term of the geodesic equation has been introduced in the definition of the of the tensorial part of the affine connection Reichenbach to rewrite the force equation in the form of geodesic equations in terms of \Gtmnbar. 
%
%\begin{equation*}
%\frac{d {u^\tau}}{d\ap} - \notateol{\Gtmnbar}{2}{\texts{Reichebach connection}} u^{\mu} u^\nu = 0
%\end{equation*}
%%
%In Riemannian geometry the straightest lines are identical with the shortest lines. If the connection is non symmetric, the straightest lines do not generally coincide with its geodesics, that is with the shortest lines. The electromagnetic field has been geometrized. However, the no new physical content has been added. Reichenbach's geodeisics eqautins dscirbe the same trajectiories as general relativistic force equation.
%
%Indeed, only after a coordinative definition has been chosen, can we apply the judgements \scare{true} or \scare{false}, since such judgements would concern only the question whether the objects which we have chosen for the displacement will satisfy the law, which we have formulated for the \Gtmn. 

%In general relativity the motion of charged particles under the influence of both the gravitational and the electromagnetic is described by a force equations. Planets describe a trajectory which itself is assigned because it is attracted by the force of the sun, but we say: the planet moves along the straightest line defined by the \Gtmn. The force term indicates that the force experienced by charged particles is directly proportional to the charge and inversely proportional to the mass. Thus gravitation appears to be geometrized, but  electromagnetism does not. In order to geometrize the electromagnetic field as well, one should transform the force equations into a geodesic equations. 
The content of the note has been presented in details elsewhere \citep{Giovanelli2016}, however the basic idea can be easily summarized. As it is well known, in \gr, uncharged particles under the influence of gravitational and electromagnetic field describe are not attracted by a force, but move along the straightest line defined by the \Gtmn. On the contrary, charged particles accelerate with respect to a given geodesic. Thus, one can say that gravitation has been \scare{geometrized}, but electromagnetism has been not. In order to geometrize the electromagnetic field as well, Reichenbach introduced a more general affine connection \Gtmnbar in which the length of vectors is preserved under parallel transport but the condition of symmetry is abandoned $\Gtmnbar \neq \bar{\Gamma}^\tau_{\nu\nu}$. The abstract notion of parallel-transported vectors were coordinated with the physical notion of the velocity vector $u^\tau=dx_\nu/\dap$ of a charged mass point. In Riemannian geometry the straightest lines are identical with the shortest lines. If the connection is non-symmetric, the straightest lines do not generally coincide with its geodesics, that is with the shortest lines. Reichenbach's connection \Gtmnbar was so defined that mass points of unit mass move (or their velocity four-vector is parallel-transported) along the straightest lines, and uncharged particles move on the straightest lines that are at the same time the shortest ones. Thus, both the gravitational and the electromagnetic field had been geometrized.


%In absence of the equivalence principle, the interpretation is ultimately abirtary.   In the presence of charge, the \Gtmn is non-Riemannian, charged particles move on the straightest lines, and uncharged particles on the shortest lines. In the absence of charge, the tensorial part of the connection vanishes, and the connection reduces to that of Riemannian geometry.

Einstein was not impressed \lettercpaep{Einstein}{Reichenbach}{31}{3}{1926}[15][239]. The definition of the tensorial part of the connection tensor, he argued, was arbitrary. Most of all, Reichenbach's equations of motion can be valid only for a certain charge-density-to-mass-density ratio $\rho/\mu$\footnote{In a given displacement, there is only one straightest line passing through a point in a given direction, but different test particles with different charge-to-mass ratios accelerate differently in the same electric field. Thus they cannot all travel on the same straightest line of the \emph{same} connection. In Reichenbach's theory particles with different charge-to-mass-ratio would travel on geodesic of a \emph{different} connections. This clearly make the theory anodyne} \lettercpaep{Einstein}{Reichenbach}{31}{3}{1926}[15][239]. Reichenbach rushed to point out that Einstein had misunderstood the spirit of the typescript. As Reichenbach explained, he was working on a philosophical presentation of the problem of space. \qt{Thereby I wondered what the geometrical presentation of electricity actually means}{Dabei überlegte ich mir, was die geometrische Darstellung der Elektrizität eigentlich bedeutet}[\letter{Reichenbach}{Einstein}{4}{4}{1926}][20-086][EA]\todo{Update letters}. Reichenbach wanted to challenge the idea that geometrizing a field is per se a useful heuristic strategy. Thus, Reichenbach decided to provide a toy model of a proper geometrization. 



%Recent attempts to geometrize the electromagnetic field seemed to have implmented now implemented Weyl's second strategy. One starts to find a suitable geometrical structure the \gmn and $\phi_\mu$, as in Weyl geometry, symmetric \Gtmn in Eddington's theory\etc, non-symmetric \gmn and \Gtmn in Einstein's metric-affine theory\etc. Those variables have adimittedly no physical meaning and there is no attempt to provide their physical interpretation, but are only tools to find the field Langrangian. Reichenbach claims that those theories are only \scare{graphical representations} (\german{graphische Darstellungen})---an expression he evidently borrowed from \citet[294ff.]{Eddington1925a}. \cop{The theory helps just like a graph provide useful economical device to organize physical already known knowledge}. Reichenbach adopted a different strategy, by attempting to get back to Weyl's original approach. Einstein's theory had given to the \gmn a physical meaning from the outset. Recent attempts to geometrize the electromagnetic field seemed to have implemented now implemented Weyl's second strategy, affine connection \Gtmn and to not give any physical meaning of displacement. Reichenbach claims that those theories are only \scare{graphical representations} (\german{graphische Darstellungen})---an expression he evidently borrowed from \citet[294ff.]{Eddington1925a}\todo{check letter}.

%After general relativity it was consider to geometrize the electromagnetic field Einstein's theory had given to the \gmn a physical meaning from the outset, thus Weyl attempted to do the same. His failure seemed to have Recent attempts to geometrize the electromagnetic field seemed to have implemented now implemented Weyl's second strategy, affine connection \Gtmn and to not give any physical meaning of displacement. 


The theory starts with a non-symmetric \Gtmn, a choice that is also arbitrary. In comparison with Eddington or Einstein's last theory, however, Reichenbach's approach had \qt{the advantage over other geometrical representations in that \emph{the operation of displacement possesses a physical realization \textins{Realisierung}}}{Diese Umschreibung hat aber vor anderen geometrischen Darstellungen den Vorteil, daß sie für die Verschiebungsoperation eine physikalische Realisierung besitzt}[\letter{Reichenbach}{Einstein}{4}{4}{1926}][20-086\me][EA], namely, the velocity-vector of charged mass particles. In this way, the geometrical predictions of the theory were suitable to be tested empirically. For this reason, in Reichenbach's view his geometrization was very similar to that provided by \gr in which the predicted \gmn are in principle comparable with the observed \gmn measured by \rac. Nevertheless, Reichenbach's theory, did not lead to any new physical prediction. Thus, Reichenbach concluded, a successful geometrization does not lead to a successful physical theory. Although Einstein probably continued to find the technical details of Reichenbach's attempt questionable, his philosophical point clearly resonated with Einstein:

\qt{You are completely right. It is incorrect to believe that \scare{geometrization} means something essential. It is instead a mnemonic device \origins{Eselsbrücke} to find numerical laws. If one combines geometrical representations \origins{Vorstellungen} with a theory, it is an inessential, private issue. What is essential in Weyl is that he subjected the formulas, beyond the invariance with respect to \textins{coordinate} transformation, to a new condition (\scare{gauge invariance})\footnote{That is, invariance by the substitution of $g_{ik}$ with $\lambda g_{ik}$ where $\lambda$ is an arbitrary smooth function of position \citep[c\f][468]{Weyl1918}. Weyl introduced the expression \scare{gauge invariance} (\german{Eichinvarianz}) in \cite[114]{Weyl1919a}}. However, this advantage is neutralized again, since one has to go to equations of the 4. order%
%
\footnote{Cf.~\cite[477]{Weyl1918}. Einstein regarded this as one of the major shortcomings of Weyl's theory; see \letter{Einstein}{Besso}{20}{8}{1918}[\VD{8b}{604}][CPAE], \letter{Einstein}{Hilbert}{9}{6}{1919}[\VD{9}{58}][CPAE]},%
%
which means a significant increase of arbitrariness}{Lieber Herr Reichenbach\\ Sie haben vollständig recht. Es ist verkehrt zu glauben, dass die \scare{Geometrisierung} etwas Wesentliches bedeutet Es ist mir eine Art Eselsbrücke zur Auffindung numerischer Gesetze. Ob man dann mit einer Theorie \scare{geometrische} Vorstellungen verbindet, ist unwesentliche Privatsache. Das Wesentliche bei Weyl liegt darin, dass er die Formeln neben der Invarianz bezüglich Transformationen einer neuen Bedingung (\scare{Eichinvarianz}) unterwirft. Dieser Vorteil wird aber wieder neutralisiert dadurch, dass man zu Gleichungen 4. Ordnung übergehen muss, was ein beträchtliches Wachsen der Willkür bedeutet.\\Mit besten Grüssen \\Ihr A. Einstein}[\letter{Reichenbach}{Einstein}{8}{4}{1926}][20-117][EA]

Einstein seem endorsed Reichenbach's claim that a \scare{geometrization} is not an essential achievement of general relativity. However, it is worth noticing, however, that Einstein goes further an claims that very notion of \scare{geometrization}, and for that matter the very notion of \scare{geometry} is meaningless \citep{Lehmkuhl2014}. The \gmn, \Gtmn\etc. are ultimately multi-components mathematical objects characterized by their transformation properties under coordinate transformation. There is nothing \scare{geometrical} about those quantities. Einstein's point is only apparently similar to that of Reichenbach. Einstein declared that the difference between geometry and rest of mathematics was inessential. On the contrary, as we shall see, Reichenbach intended to show that difference between geometry and physics was essential. Einstein's argument was meant to provide a support of the \uftp. Against those that believed that geometrization program could not be extended beyond the gravitational field, he could argue that geometrization has never been the issue in the first place\footnote{\citets{Pauli1926} review of the German translation of \citet{Eddington1925} is a typical example of this type of criticism: \qt{This natural geometry with the gravitational field and is based on the empirical fact of the equality of heavy and inert mass. An attempt at an anatogenic geometrical interpretation of the electromagnetic mass is now faced with the difficulty that there is no empirical fact corresponding to the equality of heavy and inert mass, which would make such an interpretation appear "natural". One has therefore helped oneself by taking a sufficiently general geometry as a basis, which initially makes no mention of a direct connection between the introduced geometric quantities and the observed behavior of the scales and clocks}{dieser natarlichen Geometrie mit dem Gravitationsfeld and beruht auf der Erfahrungstatsache der Gleichheit yon schwerer und trger Masse. Einem Versuch ether anatogen geometrischen Deutung des elektromagnetischen Fetdes steht nun die Schwierigkeit entgegen, dab bier keine tier Gleichheit yon schwerer und träger Masse entsprechende Erfahrungstatsache vorliegt, die eine solche Deutung als "natürlich" erscheinen lassen warde. Man hat sich daher so geholfen, dab man eine hinreichend allgemeine Geometrie zugrunde legte, bet der zunachst yon einem direkten Zusammenhang der eingeffihrten geometrischen Gr6Ben mit dem beobachteten Verhalten der Maßstäbe und Uhren keine Rede ist}}. Reichenbach's argument was on the opposite an attack on the \uftp since it was based on the idea the geometrization in itself would have led to physical results.

%As we shall see, Einstein's argument was an argument in favor of the \uftp; on the contrary Reichenbach had the very opposite goal to attack the program. Einstein that without the equivalence principle an geoemtrical interpretation of the electromangnetic field. That was not of geometrizal. There Reichenbach that the problem, \todo{check final}. It is interesting to notice have fully perceived the of the \uftp. The manurscript will become part of a longer appendix. 

\subsection{The \Ap to the \PRZL}

Strengthened by Einstein's apparent support in \datem{26}{5}{1926} Reichenbach presented the note in Stuttgart at the regional meeting of the German physical Society \citep{Reichenbach1926d}. In the following Reichenbach must have further work on the manuscript and by the end of the year, he could write to Schlick, keeping him up to date with his book project: \q{The first volume that deals with space and time,} he wrote, \qt{is finished}{der erste Band der Raum und Zeit behandelt, ist fertig} \letterp{Reichenbach}{Schlick}{6}{12}{1926}[][SN]\label{RZL1926}. Reichenbach hoped to publish the book in the forthcoming Springer series \scare{Schriften zur wissenschaftlichen Weltauffassung} directed by Schlick and Philipp Frank. However, Springer rejected the book as being too long. By July Reichenbach could announce to Schlick that he had reached a publication arrangement \letterp{Reichenbach}{Schlick}{2}{7}{1927}[][SN]. The publisher agreed to publish only the first volume under the title \citetitle{Reichenbach1928}. According to \Reich's later recollections, the manuscript of the first volume was not changed significantly after February 1927\hide{Das MS war seit Febr. 1927 nicht mehr nennenswert geändert worden}. The drafts were finished in September and the preface was dated October 1927. The note that Einstein had sent to Einstein became the \S49 of the Appendix of the book bearing the title. Most readers have insisted on Reichenbach's on the problem problem of the \emph{coordinative} in the philosophy of geometry. However, the book is also an attack on the \emph{geometrization} in physics.

%issue \emph{coordinatation} is main of the book, which implments coordinate definition, the difference between universal and differential forces; 

%This assumption was the new \emph{physical hypothesis} on which the strategy \cref{us} could be based \citep[see][\S50]{Reichenbach1928}. One searches for the most natural field structure, and the simplest field equations that such structure satisfies. After all, Einstein could claim, this is how physics has always been done: \ME are nothing but the simplest laws for antisymmetric tensor field \Fmn which is derived from a vector field; Einstein's equations were the simplest generally covariant laws that govern a Riemannian metric \gmn and so on. The only warranty of the success of this speculative groping in the chaos of mathematical possibilities was the unification power of the field equations obtained. The latter should have predicted some unknown coupling between the electromagnetic field and the gravitational field, which ultimately would have served as the basis of a theory of matter. This was indeed the case of the \FP-field theory.
 

The core of he general theory of relativity was the equivalence principle. The latter is based on the \emph{empirical fact} of the equality of inertial and gravitational mass implies that free-fall is locally indistinguishable from inertial motion. The equivalence principle is the \emph{physical hypothesis} that this indistinguishability can be extended to all non-mechanical phenomena \rzlp{264}{229f.}. Because of the equivalence principle, gravitation is a \emph{universal force} that cannot be neutralized or shielded. Thus, there is no way to separate the geometrical measuring instruments that are not affected by the field (\rac, light rays, force-free particles) from the dynamical ones that react to the field (charged particles). One does not speak of the deformation of our measuring instruments \q{produced by the gravitational field}, but we regard \q{the measuring instruments as \scare{free from deforming forces} in spite of the gravitational effects} \rzlp{294}{256}. \cop{The same measuring instruments as those used for geometry are at once indicators of the gravitational field}. \Rac are coordinated with $ds\pm1$

%However, one would be greatly mistaken to think thereby that Einstein held a kind of external view of (chrono)geometry, external in the sense that (chrono) geometry was considered to be constituted by rods and clocks. Thus in the sense that the (chrono-)geometrical structures of spacetime were presumed to exist independently of its being probed by rods and clocks, or even independently of the existence of rods and clocks. external in the sense that (chrono) geometry was considered to be constituted by rods and clocks. A net of \rac netwokr that we are to cover the body of the gravitatial field. That the network cover or does not cover the field.

In this respect, Reichenbach continues, \q{we may say that gravitation is \oemph{geometrized}} \rzlp{294\oe}{256}. However, Reichenbach wants to disabused the readers for taking this conclusion. The presence revealed by, and not defined by the geometry, it is revealed by their behavior of geometrical measuring instruments. Besides serving in their usual capacity of determining the geometry of space and time, they serve, therefore, also as indicators of the gravitational field. The geometrical interpretation of gravitation is consequently an expression of a real situation, the universal nature of gravitation. Measuring instruments made of whatever fields and particles can be used to explore the gravitational field, and the result of such measurements is independent of the device. \q{It is not the theory of gravitation that becomes geometry, but it is geometry that becomes an expression of the gravitational field}. In this sense, \q{The theory of relativity did not convert a part of physics into geometry. On the contrary, even more physics is involved in geometry}. The gravitational field has not become geometry as many have claimed; the gravitational field has a causal influence on the geometrical instruments used in geometry \rac, light rays, test particles\etc. The \Ap was nothing but the continuation of this line argument. \q{The geometrical interpretation of gravitation is merely the visual cloak in which the factual assertion} encoded by the equivalence principles. The cloak might be conceived a as an inextensible network of \rac, that have to be tailored to the body of the gravitational field. \q{It would be a mistake to confuse the cloak with the body which it covers; rather, we may infer the shape of the body from the shape of the cloak which it wears. After all, only the body is the object of interest in physics}. The fact that an Euclidean cloak, so to speak, does not fit the body of a real gravitational field allows to now something new about shape of the body, that is to make the new predictions.

Unfortunately, physicists have misunderstood the lesson of theory.  \q{The great success, which Einstein had attained with his geometrical interpretation of gravitation} led many \q{to believe that similar success might be obtained from a geometrical interpretation of electricity} \rzlap{352}{491}. After \gr was accepted by the physics community, the search for a suitable geometrical cloak that could cover the naked body of the electromagnetic field began. The separation of the \scare{operation of displacement of vectors} \Gtmn from the operation of comparison of length at distance \gmn gave physicists new mathematical degrees of freedoms that could be exploited to accommodate the electromagnetic field alongside the gravitational field. \q{However, the fundamental fact which would correspond to the principle of equivalence is lacking}. Thus, physicists needed proceed by trial and error in the search for suitable geometrical-field variables.

%\emph{graphical representation} and distinguish it in this manner from the geometrical interpretation of gravitation

%As we have, see Weyl, introduced a more general affine connection \Gtmn, which depended on both \gmn and \phin. For formal reason, the \phin was suitable to be identified with electromagnetic four-potential. Weyl geometry had the ambition to be the \scare{true} physical geometry of \spti, that can be true or false. Thus, Weyl initially considered to provide \emph{ante facto}, before searching for the field equations, a coordinate definition, that is a \q{realization of the process of displacement}. The choice is ultimately arbitrary since, without the equivalence principle, there was no physical reason for making the choices. However, in an analogy with \gr it was natural to assume that the lengths of vectors were supposed to be measured by \rac. \Rac became indicators of both the gravitational and the electromagnetic field. In this form, Weyl geometry makes assertions that can be true or false. In this form, \q{the Weylian space now constitutes the natural cloak for the field, which is composed of electricity and gravitation} \rzlap{354}{494}. However, it is precisely this behavior which does occur in reality. In the absence of the electromagnetic field, the \rac behave in a Riemannian way. \q{This means that we have found a cloak in which we can dress the new theory, but we do not have the body \hook{which} this new cloak would fit} 

Initially attempts were made to identify these geometrical structures with \scare{true} the geometry of \spti. The latter was supposed to be endowed with a more general general affine structure. To give this claim empirical content, Weyl initially provided a \q{realization of the process of displacement} in terms of the behavior of \rac. Weyl's project failed, since \rac did not behave as predicted by the theory.  \q{This means that we have found a cloak in which we can dress the new theory, but we do not have the body that this new cloak would fit} \rzlap{353}{493}. Nevertheless, physicists did not abandon the geometrization program. Rather they came to the conclusion that \cop{that \q{such \scare{tangible} realizations does not lead to the desired field equations}}. 

Thus, theories were proposed by \q{Weyl, Eddington and Einstein} which \q{renounced such a realization of the process of displacement}. The geometrical structure chosen did not have the ambition to the geometrical structure of \spti and does not have any physical meaning from outset. \q{Einstein, in particular, has devised several new formulations in which the geometrical interpretation is reduced to the role of a mathematical tool}.  \cop{The trick was to \scare{guess} the right action and the right dynamical quantities to produce the desired equations}, that is Maxwell and Einstein's field equations in first approximation. Practitioners of the \uftp seemed convinced that, \q{[\textelp{i}n this \scare{guessing} the geometrical interpretation of electricity is supposed to be the guide}. The fundamental geometrical field quantities have no physical meaning, thus the prediction of the theory cannot be directly compared with experience as in the case of \gr. The \uft can be confronted with reality indirectly only \emph{post facto} by delivering the electron: \q{The resulting differential equation would have to have a solution corresponding to the electron, and would have to show the discreet nature of the electron as a mathematical necessity} 



%8. The problem of identification. When a system of geometrical field-equations has tentatively been established in the way sketched above, the field components which they connect (which are either the original basic geometrical field-quantities or, more often, vectors and tensors built up from them) have to be identified, one by one, with the components of the known physical vectors and tensors, connected by more or less well-known physical field equations. Now, this business of identification

%According to Reichenbach, the reason for this failure was ultimately the lack of a proper analogon of a physical fact that plays the role of the \emph{equivalence principle}\footnote{Reichenbach (misleadingly) indeed claims that in his theory there is something comparable to the equivalence principle. However, he reports that this analogon is simply a reformulation of the well-known effects of the electromagnetic field on charge test particles and does allow to make any new predictions. See \anonymize{\cite{Giovanelli2020}} for more details}. As is well known, the \emph{empirical fact} of the equality of inertial and gravitational mass implies that free-fall is locally indistinguishable from inertial motion. The equivalence principle is the \emph{physical hypothesis} that this indistinguishability can be extended to all non-mechanical phenomena \rzlp{264}{229f.}. Because of the equivalence principle, gravitation is a \emph{universal force} that cannot be neutralized or shielded. Thus, there is no way to separate the geometrical measuring instruments that are not affected by the field (\rac, light rays, force-free particles) from the dynamical ones that react to the field (charged particles).


In Reichenbach's assessment, this post-\grc style of doing physics was ultimately motivated by the misunderstanding that the success of general relativity was due to the fact that it geometrized the gravitational field and that thus the geometrization of other fields would have lead to similar results. Reichenbach conceded that \gr was indeed a geometrical interpretation of the gravitational field. However, this interpretation was possible due to rather peculiar circumstances. The geometrical interpretation of \rt was (a) \emph{physically motivated} since the equivalence principles justified the use of geometrical indicators like \rac for the gravitational field (b) \emph{heuristically powerful}, it made new predictions, which were confirmed by measurements carried out with real physical systems. In absence of an anlogon equivalence principle the geometrical interpretation of electricity can indeed be carried through. This was what Reichenbach tried to show in section \S49 by providing a toy-geometrization of the electromagnetic field. A geometrization of a field is always possible, e.g. by introducing a suitable definition of the affine connection. One can even provide a geometrical interpretation in which the operation of parallel-transport of vectors has a physical meaning. However a successful geometrization does not necessarily lead to a successful physical theory. 

Reichenbach was quite skeptical that the geometrization program was very improbable that geometrization program was worth pursuing. \q{The many ruins along this road urgently suggest that solutions should be sought in an entirely different direction}. Why did physicists still persist? Reichenbach quite perspectively grasped their psychological motivation: \q{It is not the geometrical interpretation of electricity} but a deeper assumption which lies at the basis all these attempts; namely, \q{the assumption that the road to a simple conception, in the sense of a geometrical interpretation, is also the road to true relationships in nature}. The point of departure in this approach is \q{the (unwritten) assumption that whatever looks \emph{simple} and \emph{natural} from the viewpoint of the geometrical interpretation will lead to the desired changes in the equations of the field}. Indeed, by reading papers on the \uft one is struck by the fact that they are full of expressions like \scare{most natural assumption}, \scare{simplest invariant}\etc. \q{It is this assumption which constitutes the \emph{physical hypothesis} contained in these attempts}. \Gr was based on physical hypothesis based on the equivalence principle, which was hover based on an empirical fact, the identity of gravitational and inertial mass. The \uftp is is based a different physical hypothesis of more speculative nature that the world is geometrically simple. Needless, to say the idea that the \scare{simplicity} of mathematics could have have bearing for the truth of the theory was Reichenbach the consequence of a sever conceptual mistake \PRZL{**}{**}, in which again the separation between mathematics and physics was not taken into considerations.

%ut even this result is not essential, since in this case simplicity is not a criterion for truth. Simplicity certainly plays an important part in physics, even as a criterion for choosing between physical hypotheses. The significance of simplicity as a means to knowledge will ha\'e to be carefully examined in connection with the problem of induction, which does not fall within the scope of this book.

%However, The geometrical interpretation of electricity can be carried through in any case, \q{but it by no means follows that this added hypothesis must also be correct}. 

\q{The final decision regarding this new physical territory must be left to the physicist, whose physical instinct provides the sole illumination}. However, ultimately scientists' \q{physical instinct} pertains to the real of the logic of discovery and thus lies completely outside epistemology. However, Reichenbach made no mysteries that he hoped that his epistemological analysis could tie scientists to the mast of empiricism protecting them from \q{the sirens' enchantment \origins{Sirenenzauber} of a unified field theory} \citep[373]{Reichenbach1928}. \PRZL was not only a book about the problem of the \emph{coordination} between geometry and reality. It was an attack against the rhetoric of the \emph{geometrizaton} of physics that seemed to have started to dominate the relativistic community. If there is something we can learn from \gr, Reichenbach argues, it is that abstract geometry has been lowered to physics, and certainly not that physics has been absorbed into geometry\todo{1929}. 

\section{Unification: Reichenbach-Einstein Correspondence 1929--1930}
\label{unification}
%
In October 1927, Reichenbach moved back to Berlin where he took the position of as an \q{unofficial associate professor}  \citep{Hecht1982}.  At about the same time, Einstein read the manuscript of the \PRZL while on his way to Brussels to attend the fifth Solvay Congress  \citep{Bacciagaluppi2009} \lettercpaep{Einstein}{Elsa Einstein}{23}{10}{1927}[16][34]. Soon thereafter he wrote a short review. Einstein was quite perceptive in pointing out the two themes that Reichenbach had treated in the \Ap: \begin{inparaenum}[(1)] \item \qt{In the \Ap, the foundation of the Weyl-Eddington theory is treated in a clear way and in particular the delicate question of the \myemph{coordination} of these theories to reality}{In einem Anhang wird dann noch die Grundlage der Weyl-Eddingtonschen Theorien in klarer Weise dargestellt und insbesondere die heikle Frage der Zuordnung dieser Theorien zu der Wirklichkeit behandelt} \citep[20\me]{Einstein1928d}. As we have seen, Reichenbach had complained about the tendency of modern physicist claimed that, as in any other  theory, also in \uft, one should give physical meaning the operation of displacement from the outset, before starting to search for the field equations. Einstein did not comment further on this issue, probably because, over the years, he had come to realize that this requirement was too strict. However, Einstein seemed to be in full agreement with the second point made by Reichenbach: \item In the \Ap, \qt{in my opinion quite rightly---it is argued that the claim that general relativity is an attempt to \emph{reduce physics to geometry} is unfounded}{In diesem Kap., sowie in den vorangehende wird~--~meiner Ansicht nach mit vollem Recht~--~die Haltlosigkeit der These behauptet, nach welcher die Relativitätstheorie ein Versuch sei, die Physik auf Geometrie zurückzuführen} \citep[20\me]{Einstein1928d}. \end{inparaenum} As we have seen, Reichenbach and Einstein had already discussed this topic in a private correspondence less than two years earlier \anonymize{\citep{Giovanelli2016d}}. 

%For this reason, Einstein immediately perceived the importance of this theme in Reichenbach's book, a theme that later readers often overlooked \anonymize{\citep[]{Giovanelli2020}}. 

Simultaneously with Reichenbach's review, after a nearly a year-long correspondence, at the end of 1927, Einstein \letteraeap{Einstein}{Meyerson}{24}{12}{1927}[18-294] gave final authorization for the publication of another, more extensive review of \citetitle{Meyerson1925} written by the French philosopher \'Emile Meyerson \citep{Meyerson1925}. The review reveals how Einstein was had become on both issues. He embraced Meyerson's rationalist philosophy, insisting on the deductive-speculative nature of physics enterprise. He insiste again that geometry in this context is \qt{\myemph{devoid of meaning}}{le terme \scare{g\`eom\`etrique} employ\`e dans cet ordre d'id\`ees est enti\`erement vide de sens} \citep[165\me]{Einstein1928b}. However, he also clarified the motivations against the critique of the geometrization program: \q{The essential point of the theories of Weyl and Eddington}, was not to geometrize the electromagnetic field, but to \q{represent gravitation and electromagnetic under a unified point of view, whereas beforehand these fields entered the theory as logically independent structures} \citep[165]{Einstein1928b}. Einstein's further attempts at \uft testifies to his\todo{better} 

%\myemph{devoid of meaning}}{le terme \scare{g\`eom\`etrique} employ\`e dans cet ordre d'id\`ees est enti\`erement vide de sens} \citep[165\me]{Einstein1928b}

In spring 1928, during a period of rest, Einstein came up with a new proposal for a \uft. On \datef{7}{6}{1928} he presented a note to the Prussian Academy on a \scare{Riemannian Geometry, Maintaining the Concept of Distant Parallelism} \citep{Einstein19281}, a flat space-time that is nonetheless non-Euclidean since the connection is non-symmetrical. He introduced a new formalism, based on the concept of $n$-\german{Bein} (or $n$-legs), $n$ unit orthogonal vectors representing a local coordinate system attached at a point of $n$ dimensional continuum. Vectors at distance points considered as equal and parallel if they have the same local coordinates with respect to their \nbein. The \vbein-field \hbein defines both the metric tensor $g_{\mu \nu}$ and the electromagnetic four-potential $\fm$. Its sixteen components can be considered as the fundamental dynamical variables of the theory. The question arises as to the field equations that determine the \vbein-field. On \datef{14}{6}{1928} he submitted a second paper in which the field equations are derived from a variational principle \citep{Einstein19282}.

A few months later, Reichenbach managed to read this paper and prepared a type-scripted note \citep{Reichenbach1928a} with some comments that he send Einstein for feedback:
 
\qt{Dear Herr Einstein,\\ I did some serious thinking on your work on the field theory and I found that the geometrical construction can be presented better in a different form. I send you the ms. enclosed. Concerning the physical application of your work, frankly speaking, it did not convince me much. \myemph{If geometrical interpretation must be, then I found my approach simply more beautiful, in which the straightest line at least means something.} Or do you have further expectations for your new work?}{Lieber Herr Einstein, \\ Ich habe mir Ihre neuen Arbeiten zur Feldtheorie durch Kopf gehen lassen und gefunden, dass man die geometrische Aufbau besser in anderer Form darstelle kann. Ich schicke Ihnen einliegend des Ms. Was die physikalische Anwendung betrifft, so hat mich Ihre Arbeit, offen gesagt, wenig überzeugt. \myemph{Wenn es nun einmal geometrische Deutung sein muss, so finde ich schlechthin meinen Ansatz schöner, bei dem die geradesten Linien wenigstens etwas bedeuten}. Oder sollten Sie doch noch Aussichten in Ihren neuen Ansatz sehen?}[\letter{Reichenbach}{Einstein}{17}{10}{1928}][20-92\me][EA]

In this passage, Reichenbach makes two apparently unrelated points, which are however part of single two-step argument. In the note sent to Einstein Reichenbach had shown that, if one lets aside from the \nbein formalism, Einstein's new geometrical settings could be easily inserted into the Weyl--Eddington--Schouten lineage as a special case, that is is a particular case of a flat space with a non-symmetric affine connection\footnote{One starts from a general non-symmetric affine \Gtmn connection, and imposes the condition that length of vectors does not change under parallel transport. Then Einstein and Riemann space could be obtained  \q{exchangeability of the specializations} \manu{5}. If one imposes that the Riemann tensor vanish one obtain Einstein space; if one impose that the connection is symmetric one obtains Riemannian space}. If this is the case, then the Reichenbach could rase the same objection he had raised against most \uft field theory.  According to Reichenbach, \q{real physical achievement is obtained only if, moreover, the operation of displacement is filled with physical content} \manu{7}. Einstein's geometry, being flat, implies the existence of a straight line, a line of which all elements are parallel to each other, which is not identical with a geodesic \citep[224]{Einstein19282}. However, as Reichenbach reported, the latter has no physical meaning in Einstein's theory.  \q{If geometrical interpretation must be}, Reichenbach concluded, then his \S49-theory was preferable since the straightest lines and shortest line correspond to the motion of charged and uncharged test particles under the influence of the combined gravitational-electromagnetic field. Once again, Einstein's goal was to use this geometrical apparatus a starting point find the right action. However, Reichenbach commented, nothing new came out of it: \q{[T]he derivation of the Maxwellian and gravitational equation from a variational principle was already achieved by other approaches} \manu{6}, like, say, Einstein--Eddington purely affine theory. 



%Indeed, Reichenbach had used a geometrical setting cognate to that of Einstein's---a non-flat metric space with a non-symmetric connection. As we have seen, in such a geometry, the straightest and the shortest lines were different. However, they were both physically \scare{realized}, in the paths of charged and uncharged test particles under the influence of electromagnetic and gravitational fields. 



In the subsequent letter, Einstein defended his classification of geometries, but did not comment on Reichenbach's objection. However, he invited Reichenbach and his first wife Elisabeth for a cup of tea on \datef{5}{11}{1928}. On that occasion, Einstein might have informed Reichenbach about his plan to abandon the variational strategy to find the field equations. However, it is quite probable that Einstein might have explained to Reichenbach that his goal was \emph{not} to provide a \emph{geometrization} of the electromagnetic field, but to provide their \emph{unification} of both fields. Thus, Einstein's choice of the field-structure was not motivated by geometrical considerations, nor had a geometrical meaning, but exclusively the goal to find the right field equations. The field equations yield classical equations of gravitation and of electromagnetism only to first order. That this they should predict \emph{new} effects in the case of strong fields. To obtain this result, Einstein was ready to adopt a whatever-it-takes strategy. Not only  he was ready to forgo any physical interpretation of the fundamental variables was not physically motivated; he was even ready to abandon the variational approach as in the paper he was working on. 

It is hard to imagine that their philosophical disagreement did not emerge at this point. In a semi-popular paper Einstein had submitted a few week later. \german{Festschrift} on the occasion of the seventieth birthday of Aurel Stodola\label{stodola}, Einstein insisted on the speculative nature of the new theory. Indeed, for \FP, no attempt was made to give a direct physical meaning to the fundamental field variables \hbein. One starts from this mathematical structure and then searches for the simplest and most natural field equations that the \vbein-field can satisfy \citep[131]{Einstein1929}. The physical soundness of the field equations thus found can be confirmed only by integrating them, which was usually a very difficult task. Einstein warned his readers of the dangers of proceeding \q{along this speculative road} \citep[127]{Einstein1929}. In a footnote, Einstein even endorsed \qt{Meyerson's comparison with Hegel's program \origins{Zielsetzung}}, which \qt{illuminates clearly the danger that one here has to fear}{Meyersons Vergleich mit Hegels Zielsetzung hat sicher eine gewisse Berechtigung; er beleuchtet hell, die hier zu fürchtende Gefahr} \citep[127]{Einstein1929}.

%Reichenbach was very different that the one he had imagined. A few weeks after he wrote to Reichenbach, Einstein had submitted \german{Festschrift} on the occasion of the seventieth birthday of Aurel Stodola\label{stodola}, Professor of Mechanical Engineering at the ETH (\lettercpae{Honegger}{Einstein}{02}{11}{1928}[16][abs.\ 732]; \lettercpae{Einstein}{Honegger}{14}{11}{1928}[16][abs.\ 750]; cf.\ \cite{Einstein1929d}), in which Einstein's philosophical point more much more, speculative of physics endorsing again Einstein insisted on the speculative nature of the new theory, which, however, he presented as a continuation of the same strategy that was successful in his search for the field theory of gravitation: individuate a suitable field structure, the \gmn, and search for simplest differential generally covariant equations that can be obeyed by the \gmn. For \gr, the choice of the \gmn was suggested by a physical fact, the equivalence principle. Indeed, for \FP, no attempt was made to give a direct physical meaning to the fundamental field variables \hbein. One starts from this mathematical structure and then searches for the simplest and most natural field equations that the \vbein-field can satisfy \citep[131]{Einstein1929}. The physical soundness of the field equations thus found can be confirmed only by integrating them, which was usually a very difficult task.  Einstein warned his readers of the dangers of proceeding \q{along this speculative road} \citep[127]{Einstein1929}. In a footnote, Einstein even endorsed \qt{Meyerson's comparison with Hegel's program \origins{Zielsetzung}}, which \qt{illuminates clearly the danger that one here has to fear}{Meyersons Vergleich mit Hegels Zielsetzung hat sicher eine gewisse Berechtigung; er beleuchtet hell, die hier zu fürchtende Gefahr} \citep[127]{Einstein1929}.


%However, in the search for a more general mathematical structure that would include the electromagnetic field, Einstein continued, \qt{the experience does not give---so it seems---any starting point}{für die L?sung dieses Problem gibt uns die Erfahrung---wie es scheint---keinen Anhaitapunkt} \citep[128]{Einstein1929}.  Thus, the only hope is to develop a theory \qt{in a speculative way}{auf spekulativem Wege gewonnenen Theorie} \citep[128]{Einstein1929}. To solve this problem, the physicist must venture along \qt{a purely intellectual path}{auf rein gedankHchem Wege} having as only motivation the deep conviction of the \qt{formal simplicity of the structure of reality}{berzeugung der formalen Einfachheit der Struktur der Wirklichkeit} \citep[127]{Einstein1929}. The belief in the fundamental simplicity of the real is \qt{so to speak, the religious basis of the scientific endeavor}{sozusagen die religiöse Basis des wissenschaftHchen Bemühens} \citep[127]{Einstein1929}.  Indeed, for \FP, no attempt was made to give a direct physical meaning to the fundamental field variables \hbein. One starts from this mathematical structure and then searches for the simplest and most natural field equations that the \vbein-field can satisfy \citep[131]{Einstein1929}. The physical soundness of the field equations thus found can be confirmed only by integrating them, which was usually a very difficult task. 



\subsection{Reichenbach's Articles}

%In the meantime, on \datef{4}{11}{1928}, an article by Paul Miller appeared in \jt{The New York Times} with the sensational title \enquote{Einstein on Verge of Great Discovery; Resents Intrusion}. %Reichenbach conceded that Einstein's theory provided a unification of gravitation and electricity which had more than just formal significance, since it made \q{new assertions concerning the relation between gravitation and electricity in relatively complicated fields}[][][Reichenbach1929c]. However, he maintained his skepticism by claiming that the theory was \q{only a first draft, lacking the persuasive powers of the original relativity theory because of the \myemph{very formal method by which it is} established}[][\me][Reichenbach1929c]. 

In the late 1920s Reichenbach was a regular contributor to the \jt{Vossische Zeitung}, at that time Germany's most prestigious newspaper; not surprisingly he was asked for a comment on Einstein's theory, which had started attracting irrational attention in the daily press \citep[see][346]{Pais1982}. With the advantage of having personally discussed the topic with Einstein a few weeks earlier, Reichenbach published a brief didactic paper on Einstein's theory on \datef{25}{1}{1929} \citep{Reichenbach1929c}. Reichenbach seems to have indeed profeted from the conversation with Reichenbach Reichenbach did not present anymore the \uftp as a geometrization program. Reichenbach reported that the novelty of \FP consisted in the fact that it no longer seeks to establish a formal synthesis between already established theories; instead, it produces new laws, of which gravitational and electromagnetic field equations are only a first approximation\footnote{It might be indeed argued that this is true for previous theories. However, Reichenbach seems to share \citet[84]{Eddington1923}'s analysis that most of those theories were primarily \scare{graphical representations} of the relations between certain quantities \citep[\S15 and \S50]{Reichenbach1928a}. \citet[281]{Eddington1929} considered Einstein's \FP-field theory as a mere graphical representation: the graph of a moving particle with time and space as coordinates is no better than one using velocity and curvature as coordinates. However, Reichenbach seems to considered it as the a proper non-geometrical unification. See also \cite[121f.]{Goldstein2003}}. For strong fields, there would be a much closer interdependence between electromagnetism and gravitation. In principle, the theory could receive experimental proof if the effects predicted did not remain beyond the threshold of experimental detection. However, the problem of the constitution of matter or the quantum problem were far from being satisfactorily addressed. Thus, Reichenbach concluded that \q{for the time being, no pronouncement can be made concerning the physical significance of the theory} \vza{262}. 

Einstein was very upset for Reichenbach's decision to leak a private conversion to the press. The exchange of the letters that ensued put a serious strain in their personal relationships.However, also their philosophical views that have become irreconciliable. By the time of the publication of the article for the \VZ, Reichenbach had already written two papers on the \FP. The first article of the order of publication was entitled \citetitle{Reichenbach1929b} \citep{Reichenbach1929b} and would appear in February in the \citejournal{Reichenbach1929b}.  These articles represent Reichenbach's last important contribution to issues related to \rt and \spti theories. On the one hand, Reichenbach attempted to make his previous reflections about the \uftp in the \Ap to the \PRZL to bear fruit \citep[\S46]{Reichenbach1928}. On the other hand, he added new elements of clarification by clearly distinguishing the \scare{geometrization program} and the \scare{unification program}. 

In the first paper for the \citejournal{Reichenbach1929b}, Reichenbach introduced the history of the \uft in an entirely different manner than before. In the \Ap to the \PRZL the history of the \uftp program was ultimately as linear history of the geometrization program, which had progressively became more abstract and speculative for the sake of the geometrizing the field. Now Reichenbach describes the history of the \uft as the progressive \emph{downfall} of the geometrization program and the concurrent \emph{rise} of the unification one. Most physicists, including Einstein \citeyearp{Einstein1923d,Einstein1925a} considered  preferable to sacrifice the geometrical interpretation---i.e., to relinquish the coordination of geometrical notion of parallel transport of vectors with the behavior \rac---and then to use the geometrical variables (\Gtmn, $\varphi_\nu$ and so on) as \scare{calculation device} for the greater good of finding the field equations. From the field variables, one has to attempt to establish the simplest differential invariants that can be used as an action function. The test of the theory can happen only in hindsight, by finding the solutions and equations of motions corresponding to elementary particles. 

Reichenbach had come to understand that, in Einstein's view, the aim of the \uftp was not the geometrization of the electromagnetic field alongside the gravitational field; it was the unification of the electromagnetic and gravitational field. Thus, Reichenbach's concern became to explain what \scare{unification} means in this context. The problem was addressed in detail in the more technical paper, which grew out of the manuscript that Reichenbach had sent to Einstein \citep{Reichenbach1929a}, which was submitted on \datedm{22}{1}{1929} bearing the same title \citetitle{Reichenbach1929a} as the manuscript \citep{Reichenbach1928b}. The first part of the paper reproduces the manuscript he sent to Einstein, with minor changes. In the second, part Reichenbach introduced the distinction between a formal formal and inductive unification, which mimics Reichenbach's more famous distinction between two types of simplicity \citepp[9]{Reichenbach1924}[\S11]{Reichenbach1929}. The former is an application of the latter to the case of unified field theories. In this way Reichenbach could describe to opposite approaches to the \uftp. In this setting, his \S49-theory came handy. The theory uses a similar geometrical setting of Einstein theory. Both use a non-Symmetric affine connection, but Einstein impose the further conditions that the geometry is flat, that is allow for distant parallelism.

According to Reichenbach, his \S49-theory was able to provide a \emph{proper} geometrical interpretation of the combined gravitational/electromagnetic field. However, the theory could achieve only a \emph{formal} unification \cref{fu} because no new testable predictions were made:

\qt{The author \textins{Reichenbach} has shown that the first way can be realized in the sense of a combination of gravitation and electricity to one field, which determines the geometry of an extended Riemannian space; it is remarkable that thereby \myemph{the operation of displacement receives an immediate geometrical interpretation, via the law of motion of electrically charged mass-points}. The straightest line is identified with the path of electrically charged mass-points, whereas the shortest line remains that of uncharged mass points. In this way one achieves \myemph{a certain parallelism to Einstein's equivalence principle}. By the way [the theory introduces] a space which is
cognate to the one used by Einstein, i.e., a metrical space with non-symmetrical \Gtmn. The aim was to show that the geometrical interpretation of electricity does not mean a physical value of knowledge per se}{Daß der erste Weg durchführbar ist im Sinne einer Zusammenfassung yon Gravitation und Elektrizität zu einem Feld, welches die Geometrie in einem erweiterten Riemannschen Raum bestimmt, ist vom Verfasser gezeigt worden; es ist bemerkenswert, daß dabei die Verschiebungsoperation eine unmittelbare geometrische Deutung finden kann, nämlich durch das Bewegungsgesetz elektrisch geladener Massenpunkte. Es wird dort die geradeste Linie mit der Bahn des elektrisch geladenen Massenpunkts identifiziert, während die kürzeste Linie die des ungeladenen Massenpunkts bleibt. Hierdurch wird eine gewisse ParallelRat zu dem Einsteinschen-Aquivalenzprinzip erreicht. ?brigens wird dort ein dem Einstein'schen Raum verwandter Raum, nämlich ein metrischer Raum mit unsymmetrischen \Gtmn zugrunde gelegt. Absicht nämlich, zu zeigen, daß geometrische Deutung der Elektrizität an sich noch keinen physikalischen Erkenntniswert bedeutet}[][688\me][Reichenbach1929a]

Notice that, according to Reichenbach, the advantage of his own approach consists in the fact that it provides a physical realization of the displacement operation.  Assume one wants to give a geometrical interpretation of a combined gravitational/electromagnetic field using the affine connection as a fundamental variable; in that case, one should at least provide a coordinate definition of the operation of parallel displacement of vectors before starting to search for the field equations. Otherwise, it is hard to understand in which sense one could test whether the latter made correct predictions. Reichenbach's point can be to read between the lines of his letter. Reichenbach's theory was precisely meant to show that a successful geometrical interpretation alone is not sufficient to achieve a substantive unification. For Reichenbach, this should have been a warning that the very hope that the geometrical interpretation of a physical field itself was the key to new physical insights was misplaced. 

Einstein \FP-field theory is an instance of this second approach, which claims to achieve \cref{iu}, an inductive unification, by renouncing to the geometrical interpretation:

\qt{On the contrary, Einstein's approach of course uses the second way, since it is a matter of increasing physical knowledge; it is the goal of Einstein's new theory to find such a concatenation of gravitation and electricity, that only in first approximation it is split in the different equations of the present theory, while is in higher approximation reveals a reciprocal influence of both fields, which could possibly lead to the understanding of unsolved questions, like the quantum puzzle. However, it seems that this goal can be achieved only \myemph{if one dispences with an immediate interpretation of the displacement, and even of the field quantities themselves}. From a geometrical point of view this approach looks very unsatisfying. Its justification lies only on the fact that the above mentioned concatenation implies more physical facts that those that were needed to establish it }{Der Einsteinsche Ansatz benutzt dagegen natürlich den zweiten Weg, denn ihm ist es ja um Vermehrung des physikalischen Wissens zu tun; es ist als Ziel der neuen Theorie Einsteins, eine derartige Verkettung yon Gravitation und Elektrizität zu finden, daß sie nur in erster Näherung in die getrennten Gleichungen der bisherigen Theorie zerspaltet, während sie in höherer Näherung einen gegenseitigen Einfluss beider Felder lehrt, der möglicherweise zum Verständnis bisher ungelöster Fragen, wie der Quantenrätsel, führt. Aber dieses Ziel scheint nur erreichbar zu sein unter Verzieht auf eine unmittelbare physikalische Interpretation der Verschiebungsoperation, ja sogar der eigentlichen Feldgrössen selbst. Vom geometrischen Standpunkt als deshalb ein solcher Weg sehr unbefriedigend erscheinen; seine Rechtfertigung wird allein dadurch gegeben werden können, daß er durch die genannte Verkettung mehr physikalische Tatsachen umschließt, als zu seiner Aufstellung in ihn hineingelegt wurden}[][688\me][Reichenbach1929a]

Einstein's theory claimed to be an \emph{inductive unification} of the dynamics of two physical fields, i.e., a unification of the fundamental interactions described by a single, non-decomposable set of field equations. In Reichenbach's view, \FP appeared not only as a formally satisfying unification but as a real advance over the available theories. It entails some coupling between the two fields that was not present in the given individual field theories. However, Reichenbach argues that Einstein could only achieve this result at the expense of a physical interpretation of the fundamental geometrical variables, the \hbein. This approach, however, made the theory impossible to be confirmed or disproved experimentally by observing the behavior of suitable indicators. Indeed, Einstein had always insisted that the physical test of the field equations ultimately depends on the construction of exact solutions that reflect the behavior of known elementary particles \citep[24]{Einstein1930i}. One cannot define the field quantities in advance in terms of the behavior of test particles, as in other field theories. The laws governing the latter are unknown before integrating the field equations \letterp{Einstein}{Cartan}{7}{1}{1930}[A-XVI][Debever1979]\todo{selfplagiarism}.

In Reichenbach's diagnoses, the stagnation of the \uftp depended on the presence of a sort of \emph{trade-off between geometrization and unification} of which physicists were only partially aware. \Gr was the only theory that was able to combine both virtues: \begin{inparaenum}[(a)] \item the theory provided a proper \emph{geometrical interpretation} of the gravitational field because it introduced a coordinative definition of the field variables \gmn, in terms of the behavior of those that were traditionally considered geometrical measuring instruments, such as \rac, light rays, free falling particles \item the theory provided a \emph{proper unification} by predicting that the gravitational field had certain effects on such measuring instruments that were not implied by previous theories of gravitation---such as gravitational time dilation \end{inparaenum} \citep[350\hide{**}]{Reichenbach1928}. Successive attempts to include the electromagnetic field in the frame of \gr failed to uphold this standard. According to Reichenbach, the reason for this failure was ultimately the lack of a proper analogon of a physical fact that could play the role of the \emph{equivalence principle}. 


The effective interplay between geometrization and unification did not seem reproducible without the equivalence principle. Thus, to replicate the success of \gr, physicists were forced to make a choice. Two strategies seem to have been available, which ultimately depended on physicists' interpretation of Einstein's theory of gravitation. (a) Many, e.g. Weyl, considered \gr a successful theory \emph{because} it had provided a geometrical interpretation of the gravitational field; then, one could hope to obtain the same success by geometrization the electromagnetic field as well. (b) For others, in particular, Einstein \gr was a successful theory \emph{because} it had achieved the unification of two different fields, gravitational and inertial field. In this way, however, the gravitational/inertial field was provisionally isolated from a more general field of unknown mathematical structure, encoding quantities corresponding to the electromagnetic field. Like many others, Reichenbach believed that without a new \emph{physical hypothesis}---that is a physical fact that played the role of the strict equality of inertial and gravitational mass---, both strategies, \cref{gs} and \cref{us} had little hope of success.

Einstein was caught between a rock and a hard place. Without the equivalence principle further geometrization of electromagnetic fields was should not be pursed, Einstein could counter this objections by claiming that geometrization was never goal, even not in the case of \rt. The achievement \gr was have combined inertial and gravitational just like \sr has combined magnetic and electric field. However, without an analogon of the equivalence principle, there seem to be no further justification for searching for a further unification of the electromagnetic and gravitational field. Einstein however, considered the separation as theoretically unbearable. We do not have any physical fact that gives any clue as to what may be the more comprehensive mathematical structure in which electromagnetic and gravitational field will appear as two components. Einstein ultimately \emph{physical hypothesis} that that nature was mathematically simple. One searches for the most natural field structure, and the simplest field equations that such structure satisfies. The only warranty of the success of this speculative groping in the chaos of mathematical possibilities was the unification power of the field equations obtained. 

It is not suprising that this philosophical outlook was for Reichenbach anathema. The core of Reichenbach's philosophy was the \emph{separation of mathematical necessity and physical reality}. Reichenbach had always perceived this separation as nothing more than a philosophical distillation of Einstein's scientific practice: \q{Mathematics teaches what is permissible and what is forbidden, but never what is physically correct}. In the search of a \uft, Einstein had come implicitly to question this distinction between physics and mathematics, ultimately pleading for a \emph{reduction of physical reality to mathematical necessity}. Einstein put it candidly in his Stodola-\german{Festschrift}'s contribution---that he sent for publication toward the end of January \lettercpaeabsp{Einstein}{Honegger}{30}{1}{1929}[864]. The ultimate goal of understanding reality is achieved when one could prove that \qt{even God could not have established these connections otherwise than they actually are, just as little as it would have been in his power to make the number 4 a prime number}{selbst Gott jene Zusammenhänge nicht anders hätte festlegen können, als sie tatsächHch sind, ebensowenig, als es in seiner Macht gelegen wäre, die Zahl 4 zu einer Primzahl zu machen} \citep[127]{Einstein1929}.

%\footnote{See \lettercpae{Weyl}{Einstein}{18}{5}{23}[13][30] and \letter{Weyl}{Seelig}{19}{5}{1952}, cit.\ in \cite[274f.]{Seelig1960}}. Einstein's philosophical volte-face might have appeared to Reichenbach as a sort of \french{trahison des clercs}, an unacceptable intellectual compromise. \begin{inparaenum}[(a)] \item 


% In this sense, Einstein's God indeed resembles Spinoza's God \citep{Einstein1929e}, for whom the laws of nature are necessary, and rather than, say, Leibniz's God for whom the laws of nature are contingent. 

\section{Conclusion}

Just after the publication of the new expected derivation of the \FP-field equations, \citep{Einstein1929b}, Einstein wrote a popular account of the theory on the first page of their Sunday supplement of the \jt{New York Times} on \datedm{3}{2}{1929} and in \jt{The Times} of London in two installments on \datedm{4}{2}{1929} and \dated{5}{2}{1929} \citepp{Einstein1929-2-3}{Einstein1929-2-4}[also published as][]{Einstein1930h}. Einstein insisted on the highly speculative nature of \uftp, without being afraid endorsing even his somewhat outrageous comparison with Hegel. The fact Einstein chose to mention Meyerson rather than Reichenbach as a philosophical reference in a popular presentation of his last theory for a major newspaper has of course a quiete symbolic significance. After a decade of personal friendship and intellectual exchange that had shaped the history of 20th-century philosophy of science and Einstein seems to have put into question the very core of his philosophical alliance with Reichenbach. Whereas Reichenbach considered the separation between mathematics and physics the great achievement of the theory, Einstein sa in mathematical simplicity itself the key to the unification\todo{selplagiarism}.

Although Reichenbach's approach attracted the attention of mathematicians,  Reichenbach's skepticism towards this approach seemed to have been shared by the physics community\footnote{Weyl, who had always been scolded by Einstein for his speculative style of doing physics could relaunch the accusation in a paper \citep{Weyl1929c} in which he had uncovered the gauge symmetry of the Dirac theory of the electron \citepp{Dirac1928}{Dirac1928b}. \q{The hour of your revenge has come}, Pauli wrote to Weyl in August: \qt{Einstein has dropped the ball of distant parallelism, which is also pure mathematics and has nothing to do with physics and \emph{you} can scold him}{jetzt hat Einstein den Bock des Fernparallelismus geschossenf , der auch nur reine Mathematik ist und nichts mit Physik zu tun hat, und Sie konnen schimpfen} \letterpaulip{Pauli}{Weyl}{26}{8}{1929}[235]. As Pauli complained, writing to Einstein's close friend Paul \Ehr, \q{God seems to have left Einstein completely!} \letterpaulip{Pauli}{\Ehr}{29}{9}{1929}[237]}. Einstein was fully aware of the marginality of his position, but, throughout 1929, continued express his confidence in \FP program. He continued to defend the theory in public (in talks given in October and December) \citep{Einstein1930,Einstein1930a,Einstein1930b}, as well as in as well in private correspondence\footnote{Pauli did not hesitate to describe Einstein's presentation at the Berlin Colloquium as a \qt{terrible rubbish}{schrecklichen Quatsch} \letterpaulip{Pauli}{Jordan}{30}{11}{1929}[238]. When he received the drafts of Einstein's \jt{Annalen} paper, he wrote only slightly more politely. Pauli wrote that he did not find the derivation of the field equations convincing; they show \qt{no similarities with the usual facts confirmed by experience}{kaum eine Ahnlichkeit mit den gewohnlichen durch die Erfahrung gesicherten physikalischen Sachverhalten zu haben scheinen} \letterpaulip{Pauli}{Einstein}{19}{12}{1929}[239]. In particular, Pauli missed the validity of the classical tests of general relativity, perihelion motion and gravitational light bending: \qt{These results seem to be lost in your sweeping dismantling of the general theory of relativity. However, I hold on to this beautiful theory, even if it is betrayed by you!}{Die scheint doch bei Ihrem weitgehenden Abbau der allgemeinen Relativitatstheorie verloren zu gehen. Ich halte jedoch an dieser schonen Theorie fest, selbst wenn sie von Ihnen verraten wird!} \letterpaulip{Pauli}{Einstein}{19}{12}{1929}[239]. When Einstein expressed caution towards the definitive validity of his equations, he, \qt{so to speak, took the words right out of my mouth of criticism-loving physicists}{haben Sie den Kritik libenden Physikern sozusagen das Wort abgeschnitten} \letterpaulip{Pauli}{Einstein}{19}{12}{1929}[239]. Pauli knew that Einstein would not have changed his mind, but he was ready to \q{make any bet} that \q{after a year at the latest you will have given up all the distant parallelism, just as you had given up the affine theory before} \letterpaulip{Pauli}{Einstein}{19}{12}{1929}[239]. Einstein complained that Pauli's remarks were superficial and asked him to return on the issue after some months \letterpaulip{Einstein}{Pauli}{19}{12}{1929}[140]. Although the \uftp was disavowed by its own initiators \citep{Weyl1931}, Einstein insisted in the pursuit of \FP discussing with Mayer two solutions of his last field equations \citep{Einstein1930g}}. However, only a few months later Einstein and Walther Mayer presented a new approach \citep{Einstein1931} that, by generalizing the \nbein formalism to five dimensions. The optimism once again faded away quickly, since the theory was unable to solve the problem of matter. In a popular talk given in Vienna towards mid-\datemy{14}{10}{1931}, Einstein could only describe his field-theoretical work since \gr as a \qt{cemetery of buried hopes}{Friedhof von Begrabener Hoffnungen} \citep[441]{Einstein1932b}.

Einstein's philosophical outlook\footnote{However, if many readers might have easily recognized someone like Pauli in Lanczos's \scare{positivist}, other were baffled to find out Einstein located among the \scare{metaphysicians}. At the beginning of 1932 the introduction of Lanczos's 1931 paper was published at Berliner's suggestion as a \latin{seperatum} in the \citejournal{Lanczos1932} \q{to make it available to a larger public} \citep[113\fn{1}]{Lanczos1932}} appeared as quite quite scandalous to his philosophical allies like Frank\footnote{It is probably this article of Lanczos that Frank read with some bewilderment, as he reports in his Einstein's biography \citep{Frank1947}. Frank was \q{quite astonished} to find the theory of relativity characterized as the expression of a realist program \q{since I had been accustomed to regarding it as a realization of \Mach's program} \citep[215]{Frank1947}. However, when Frank met Einstein in Berlin at around the same time, he found out that Lanczos had indeed well characterized Einstein's point of view \citep[215f.]{Frank1947}. According to his recollection, Einstein complained that \q{\textins{a} new fashion} had arisen in physics according to which quantities that in principle cannot be measured do not exist, and that to \q{to speak about them is pure metaphysics} \citep[216]{Frank1947}. Frank objected that this was the very same philosophical attitude that led to relativity theory. By contrast, Einstein insisted, the essential point of relativity theory is to \q{regard an electromagnetic or gravitational field as a physical reality, in the same sense that matter had formerly been considered so}  \citep[216]{Frank1947}. The theory of relativity teaches us the connection between different descriptions of one and the same reality. Was not a theory about the behavior of \rac, but a unification of two fields}. In 1933 Einstein gave the his famous Oxford address: \q{Nature is the realization of the most simple mathematical ideas} \citep{Einstein1933}. Experience remains the sole criterion of the physical adequateness of a mathematical construction, but the creative role belongs to mathematics: \q{I hold it to be true that pure thought is competent to comprehend the real, as the ancients dreamed} \citep[167]{Einstein1933}. After all the search field theories has always followed the same heuristic pattern: \q{the theorist's hope of grasping the real in all its depth} lies \q{in the limited number of the mathematically existent simple field types, and the simple equations possible between them} \citep[168]{Einstein1933}. Maxwell's equations are the simplest laws for an anti-symmetric tensor field which is derived from a vector, Einstein's equations are the simplest equations for the metric tensor\etc. This strategy applies to Einstein's last attempt at a \uft on a theory based on semi-vectors. After ordinary vectors, the simplest mathematical fields that are possible in four dimensions, and seems to describe certain properties of elementary particles. One has to search for the the simplest laws these semi-vectors satisfy \citep[168]{Einstein1933}.

%These semi vectors are, after ordinary vectors, the simplest mathematical fields that are possible in a metrical continuum of four dimensions, and it looks as if they describe, in a natural way, certain essential properties of electrical particles. [16]



%\cop{To further justify his methodological conviction, Einstein gave two more examples. The first of these was the set of Maxwell's equations; they are the simplest laws for an anti-symmetric tensor field which is derived from a vector}. At that time he was again in another attempt, in which again. \cop{The semivector was thus an outstanding example to support his view that "in the limited number of the mathematically existent simple field types, and the simple equations possible between them, lies the theorist's hope of grasping the real in all its depth.}

%\cop{The semivector was thus an outstanding example to support his view that "in the limited number of the mathematically existent simple field types, and the simple equations possible between them, lies the theorist's hope of grasping the real in all its depth."19 In the end, it was this conviction that gave Einstein the strength to maintain for some thirty odd years that his program in classical field theory provided a viable alternative to the quantum theory}

%; the law that describes the dynamics of electromagnetically charged particles, the Dirac equation}. \cop{To the audience in Oxford, he announced his latest, most appealing result: the simplest laws these semivectors satisfy elucidate the dual existence} of "two sorts of elementary particles, of different ponderable mass and equal but opposite electrical charge.". That mathematical necessity is the key two reality. There is non separation between mathematics and physics. Clearly Reichenbach's battle against to convince Einstein's himself, who in Reichenbach's eyes was the very origin. 

%The semivector was thus an outstanding example to support his view that "in the limited number of the mathematically existent simple field types, and the simple equations possible between them, lies the theorist's hope of grasping the real in all its depth."19 In the end, it was this conviction that gave Einstein the strength to maintain for some thirty odd years that his program in classical field theory provided a viable alternative to the quantum theory}.  ..  However, Reichenbach feared Weyl's opposition: \q{He is my adversary since a long time,} he wrote to the American philosopher Charles W.\ Morris, a supporter of a form a \q{mathematical mysticism} that was \q{very much opposed to my empiricistic interpretation of relativity} \lettehrp{Reichenbach}{Morris}{12}{4}{1936}[013-50-78]. Thus, in April 1936, Reichenbach turned to Einstein to ask his support: \qt{I surmise that Weyl's opposition persists to these days and therefore I'd be grateful if you could put a word in my favor}{Ich vermute, daB Herrn Weyls Gegnerschaft noch heute fortdauert, und darum ware ich Ihnen sehr dankbar, wenn Sie da zu meinen Günsten eintreten kGnnten} \letteraeap{Einstein}{Reichenbach}{2}{5}{1936}[20-118].  However, Einstein might have been academically more favorable to Reichenbach, he was certainly far away to him philosophically. As he once famously pointed out he had become \qt{a believing rationalist}{das Gravitationsproblem raich zu einem glaubigen Rationalifaten gemacbt, d.h zu einen,der die einzige zuverlassige Quelle der Wabrbeit in der matbematischen Einfacbbeit Bucht} \letteraeap{Einstein}{Lanczos}{24}{1}{1938}[15-268], convinced that physical truth lies in mathematical simplicity \citep{Ryckman2014}. 

Einstein left for soon thereafter for Princeton, and Reichenbach for Istambul. After the initial enthusiasm, Reichenbach later tried to obtain a position in Princeton as well \citep{Verhaegh2020a}. However, Reichenbach feared Weyl's opposition: \q{He is my adversary since a long time,} he wrote to the American philosopher Charles W.\ Morris, a supporter of a form a \q{mathematical mysticism} that was \q{very much opposed to my empiricistic interpretation of relativity} \letterhrp{Reichenbach}{Morris}{12}{4}{1936}[013-50-78].  Thus, in April 1936, Reichenbach turned to Einstein to ask his support: \qt{I surmise that Weyl's opposition persists to these days and therefore I'd be grateful if you could put a word in my favor}{Ich vermute, daB Herrn Weyls Gegnerschaft noch heute fortdauert, und darum ware ich Ihnen sehr dankbar, wenn Sie da zu meinen Günsten eintreten kGnnten} \letteraeap{Einstein}{Reichenbach}{2}{5}{1936}[20-118]. By this time, it was ironically Einstein the one indulging in the sort of mathematical mysticism that Reichenbach attributed to Weyl. Einstein answered that he had heard from Rudolf Carnap that Princeton did not want to hire more Jews: \qt{also up here not all that glitters is gold,}{Es ist oben hier auch nicht alles Gold was glänzt} he remarked bitterly (\letter{Einstein}{Reichenbach}{2}{5}{1936}[20-118][EA]).\todo{selfplagiarim}

%n letters to Reichenbach dated August 14 1940 and again on the August 22, soon after Reichenbach's arrival at UCLA from Turkey, Einstein, writing from Knollwood, Saranac Lake, New York, quotes his statement in behalf of ..

In 1938 Reichenbach managed to move to the United States \citep{Verhaegh2020a}.  \cop{Reichenbach and Einstein entered into contact again to support Bertrand Russell}\todo{??}. Later both contributed to the Russell to the volume in Russell's honor for the series \textit{Library of Living Philosophers} edited by Paul \citet{Schilpp1944}. Reichenbach was asked collaborate for a similar volume in honor of Einstein a few years later \citep{Schilpp1949}. In some unpublished notes about Reichenbach's contribution, \citet{Einstein1949f} recognized the merits: \qt{Hans Reichenbach is so famously distinguished by many of his colleagues by the fact that he never seek for the universality of knowledge by sacrificing clarity}{Hans Richenbach von so viclen sei.ner hollogen auszeichnet, 1st der Umstand, dass er Allgomeinheit der Erenntnis niemals erkauft durch Opferung der Klerheit. br sieht in der logischen Kritik der Lehren und liethoden der FinzelWissenschaften die Hauptaufgabe der Philosophie}. However, Einstein disagreed that the conceptual basis of \gr was the recognition of definitional nature if congruence. The latter result serves only to create the necessary freedom in the choice of the fundamental concepts. The definition of congruence in terms of rigid bodies could not be considered at most as psychologically necessary. Indeed, there are no truly rigid body in nature. If so Reichenbach's claim \qt{\scare{the meaning of a statement is reducible to its verifiability}\footnote{In English in the text}}{Dann erschetnt ueberhaupt die These problematisch "the meaning of a statement is reducible to its verifiability"s} appears to be problematic; in particular Einstein found \qt{dubious whether this conception of \scare{meaning}\footnote{In English in the text} can be applied to the single statement\footnote{In English in the text}}{ss erscheint naemlich zweifelhaft, ob man an dieser Auffassung von "meaning" fuer das einzelne statement festhalten karn}. 

As is well-known, Einstein reformulate this line of argument Reichenbach in the Schilpp volume by staging a dialogue between Reichenbach-Helmholtz (rigid rods exists), Poincaré (rigid rods do not exist), and an anonymous non positivists (only geometry and physics can be compared with experience). The question at stake, Einstein put it jokingly, was Pilates famous question \scare{What is truth}? However, the importance of this issue cannot be fully understood if one does not appreciate it goes to the roots of Einstein's work on the \uft. At that time, Einstein had returned to his 1925 metric-affine approach introducing non-symmetric \gmn and \Gtmn as fundamental variables. It is maybe not surprising that Besso raised against Einstein a similar objection that Reichenbach had raised over twenty years earlier. The symmetric part of the \gmn should define a geodesics. Do these geodesics represent the trajectories of particles What is their meaning? Einstein's reply reveals his fundamental philosophical concern.

\qt{Your questions are fully legitimate, but not answerable for the time being \textelp{} This is because there is no real definition of the field in a consistent field theory. It is true that this puts you in a Don Quixotic situation, in that you have absolutely no guarantee whether it ever possible to know if the theory is \scare{true}. \textit{A priori} there is no bridge to empiricism. For example, there isn't a \scare{particle} in the strict sense of the word because, the existence of particles doesn't fit the program of representing reality by everywhere continuous, even analytic functions. For example, in theory there is a symmetric \gmn and then a geodesic line. But from the outset one has no clue that these lines have any physical meaning, not even approximately \textelp{} It boils down to the fact that a comparison with what is empirically known can only be expected from the fact that strict solutions of the system of equations can be expected found, that reproduce the behavior of empirically \scare{known} structures and their interactions. Since this is extremely difficult, the skeptical attitude of contemporary physicists is probably is completely understandable. In order to really grasp this conviction of mine, you must read my answer in the anthology [\cite{Einstein1949a}]}{Deine in dem Brief vom 11.IV gestellten Fragen sind durchaus natürlich, aber his auf Weiteres nicht beantwortbar. Dies liegt daran. dass es in enner Dies liegt daran, dass es in einer konsequenten Theorie des Feldes keine Realdefinition für das Feld gibt. Es ist wahr, dass man dadurch in eine Don Ouixotische Situation kommt, indem man durchaus keine Gewähr dafür hat. dass esjemals möglich ist zu wissen, ob die Theorie ,,wahr“ ist. Es ist a priori keine Brücke zur Empirie gegeben. Es gibt z.B. nicht ein „Teilchen im strengen Sinne des Wortes, weil dies nicht zu dem Programm passt; die Realität durch überall konti­nuierliche, ja sogar analytische Funktione 4zu repräsentieren. In der Theorie gibt es z.B. ein symmetrisches \gmn geodätische Linie. Aber man hat von vorneherein gar keinen Anhaltspunk dafür, dass diesen Linien irgend eine physikalische Bedeutung zukommt, auch nicht approximativ  \textelp{} Es kommt darauf hinaus, dass ein Vergleich mit empirisch Bekanntem nur davon erwartet werden kann, dass man strenge Lösungen des Glei­ chungssystems findet, in denen sich empirisch „bekannte“ Gebilde und ihre Wechselwirkungen „ spielgeln D a dies ungeheuer schwer ist, ist die skep­ tische Haltung der zeitgenössischen Physiker wohl zu verstehen nicht näher kommt. Um diese meine Ueberzeugung wirklich zu begreifen, musst Du meine Antwort in dem Sammelband}

This passage many of the themes between Reichenbach and Einstein; why Einstein did not believe anymore that of coordinates while, the geometrical meaning of the fundamental variables of the theory was inessential; ultimately was only the mathematical simplicity that could serve as a guide. In popular paper written at about the same time, Einstein described himself as a \emph{tamed metaphysician}  \citep[13]{Einstein1950c} that mathematical simplicity as a key to physical reality, which existen independently of the subject.  


%In his fundanetll his that mathematical coordinate. Howcer, non pistuve. That this, ... this was not what is truth was for the. This was indeed the fundamental probmle that in which the Reichenabc's philsophy does nto fit. Its proble. Besso was ultimately concerend, that \gmn then they define a straigth line, and this whold be line on whoch particles move. The one can define physcally the field using charge partice. However, charged particles are solition, by integratig the field equaits:


%However, from the direct does not really emerge the real of the debate. I think a more is in which he that ultimately that geodesics line, with test particles. \q{Ist es nicht etwa so, das die Entwicklung der Physik so, wie Du sie siehst, mit ihrem freien Aufbau von Gedanken, gewissermassen „, Möglichkeiten", und der Kontrolle durch die Erfahrung, die erst die „wirklichen" ergibt,}

%\q{ Von too aus die Weiterentwicklung, mit neuem Gedankenaufbau, neuen Möglichkeiten einsetzt dast diese Entwicklung, förmlich Symbol,}.


%Thi was rogupnyy the point amde by Reecihenach. Howver, Einstein could not agree: ...\q{Es ist aber auch wertvoll, hinterher die ganse Sache logisch-formal zu analysieren. Dem solange man den empirischen Gehalt der Theorie wegen vorläufig unübennndHcher mathematischer Schwierigkeiten nicht feststellen kann, ist die logische Einfachheit das einzige, wem auch natürlich unzureichende Kriterium des Wertet der}

%%-Dass ich nicht iceiss, ob diese Theorie physikalisch toutrifft liegt einzig
%%daran, dass es nicht gelingt, etwas über die Existenz »und den Bau überall  singularitätsfreier Losungen solche*. mehtUnearer Gkicktmgnyttem
%s a g e n . ■. • . :





%questions has some decades before. Einstein was {non-positivist}, \citep[678]{Einstein1949a}. For the objects used as tools for measurement do not lead an independent existence alongside of the objects implicated by the field-equations. \cop{far as to imagine that intervals are physical entities of a special type, intrinsically different from other physical variables ("reducing physics to geometry, etc.).},  For this reason Einstein connitued that reduction of physocs to geoemtry was meanigless geometrical and non gemetrical field \citep{Einstein1949}. Search for the filed structure, and search for the equations of such complexity as are the equations of the gravitational field can be found only through the discovery of a logically simple mathematical condition which determines the equations completely or [at least] almost completely. In popular paper written at about the same time, Einstein described himself as a \emph{tamed metaphysician}  \citep[13]{Einstein1950c} that mathematical simplicity as a key to physical reality, which existen independly of the subject. Einstein had siply got back to his 1925 paper metri theory. Thus were not have change., and the same, indeed Besso


%Besso was ultimately concerend, that \gmn then they define a straigth line, and this whold be line on whoch particles move. The one can define physcally the field.   Einstien's awner was preciselty the reason whel the are gron abot. \q{Es ist aber auch wertvoll, hinterher die ganse Sache logisch-formal zu analysieren. Dem solange man den empirischen Gehalt der Theorie wegen vorläufig unübennndHcher mathematischer Schwierigkeiten nicht feststellen kann, ist die logische Einfachheit das einzige, wem auch natürlich unzureichende Kriterium des Wertet der}

%Reichenbach's reply to Einstein that could have been more distant. As Einstein wrote to neither the coordination, the role of geometry had been irrelevant for Einstein, and the had unificaition was the only gola, was mathematical simplicity\footnote{Ciao }. Einstein did not any coordination with the behavior of \rac was not necessary; could again isnsit that the separation between was meaningless; that mathematical simpplcity was for realty. Altyout i the real of dialogue more in that in \rt. At that time metric-affine theory. That \gmn then geodeisc like, waht is a lectromagent fiel, can chareged or a diple move under the influece, of the fiel. Besso, how the dfeitniotn of fields in sich theory. E.g. feodesci line interepted as on which particle, charged partid deflect fro

%[Princeton,] le 15 avril 1950


%\q{Im gus - Feld gibt es geradeste Linien, die zugleich kürzeste sind (das stimmt doch?). Ein, ,darauf gesetxtes" Probekorperchen, mit gegebener Impulsenergietensordichte - Lauft darauf entlang (?). ,Hat es Mihe" nicht dabei au, ,zerlaufen"? Welche inneren Krafte, was fiir eine $A r t$ inneres Feld, mïssen dabei cor ausgesetat werden? Was ein elektrisches oder magnetisches Feld, ,ist" erfahrt man doch erst eben durch ein geladenes, bezw. dipolar tig in sich fixiertes Probekörperchen (ich denke dabei zundchst nicht an die erforderlichen inneren Krafte). Oder gehbren solcha Fragen nicht mehr sum in sich geschlossenen, mathematischen Diamant"? beato. sind es nicht notwoendige Brücken aur Ertrobung?}



%As is well known, Einstein raised this precise objection against Reichenbach in the so-called \enquote{Reply to Criticisms} of the Schilpp-Volume \citep{Einstein1949a}. At the end of a fictional dialogue between Reichenbach and Poincar\'{e} \citep[677--678]{Einstein1949a}, Einstein entrusted his epistemological views to the persona of an \emph{anonymous non-positivist} (or, as he put it elsewhere, a \scare{tamed metaphysician}; \cite[3]{Einstein1950}): a theory has a \scare{meaning}, a physical content, only as a whole, even if its parts, in isolation, do not find a direct physical interpretation \citep[678]{Einstein1949a}


%\footnote{\q{}}. 



%%gur Feld gibt es geradeste Linien, die zugleich kürzeste sind (das Im stimmt doch?). Ein „darauf gesetztes** Probekörperchen, mit gegebener Impulsenergietensordichte läuft darauf entlang (?). „Hat es Mühe" nicht dabei au „zerlaufen"? Welche inneren Kräfte, was für eine Art inneres Feld, missen dabei vorausgesetat werden? War ein elektrisches oder magnetisches Feld Mist' erfährt man dock erst eben durch ein geladenes, bezw. dipolartie in sich fixiertes Probekörperchen (ich denke dabei zundchst nicht an die en for derlichen inneren Kräfte). Oder gehören solcha Fragen nicht mehr gum in sich reschlossenen-mathematischen_Diamant"? bezw. rind et micht notwendige Brücken aur Erbrobung l


Replying to Reichenbach paper Einstein described himself as \scare{non-positivist}, that only  geometry cannot separately from the rest physics, Einstein did not that the coordination of the geometrical with reality was the necessary of theory. Thus, no real definition of such quantities seems to be possible: \q{To really understand my point of view you must read my answer in the \textins{Schilpp}-volume \textins{Sammelband}} (\letter{Einstein}{Besso}{15}{4}{1950}[438--439][Einstein1972]).  Neverhtesl \q{Don Quixotian situation} in which one finds oneself in the search for a unified, non-dualistic, field theory. 



%When in 1953 Schilpp asked Einstein for contributing to the volume of the same series in honor of Carnap \letteraeap{Schilpp}{Einstein}{11}{5}{1953}[80-539], he famously declined. After \q{Reichenbach's death (a few weeks ago),}\footnote{Reichenbach died on \datef{9}{4}{1953}} Schilpp wrote, Carnap was the most important exponent of logical empiricism \letteraeap{Schilpp}{Einstein}{11}{5}{1953}[42-534]. Although Einstein agreed with this assessment, he expressed disenchantment toward that type of philosophy that Schlick, Reichenbach, and Carnap represented: \q{the old positivistic horse, which originally appeared so fresh and frisky, has become a pitiful skeleton} (\letteraea{Einstein}{Schlipp}{19}{5}{1953}[42-534]; quot.\ and tr.\ in \cite[374]{Howard1990}). 



%The of \gr was also a about \uft field theory. Indeed, that skeleton was ultimatly been. Einstein only partially conceed that the problem of coordination of mathematical strutucs with reality; the apparent geometrization . The third issue of unification was by Reichenbach intudictive; was mathemtical unificaiton in which of mathematical simplicity. A complex interppaly in 

%That with the same that was in Einstein's lecture of 1921. However, this separation was ultimately that had challenged starting at least, never agreed coordinationa, that provisiona, which the all theory geometry plus physics. This but much more proram; that between geometry and physsc, geometry and mathematics was intessentia. On many in that unificatiom that mathematilca simplicity was the to such unficaiton. 




\printshorthands
\printbibliography

\end{document}