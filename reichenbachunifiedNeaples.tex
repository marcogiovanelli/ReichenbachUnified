% !TEX encoding = UTF-8 Unicode
\documentclass[draft]{article}
\usepackage{els}
\usepackage{i}
\usepackage{notate}
\usepackage{wrapfig}
\newcommand{\ap}{\ensuremath{\lambda}\xspace}
\newcommand{\dap}{\ensuremath{d\lambda}\xspace}
%\usepackage{geoA4}

\newcommand{\phin}{\ensuremath{\varphi_\nu}\xspace}
\newcommand{\manu}[1]{\citep[#1]{Reichenbach1928b}}
\newcommand{\nbein}{$n$-bein\xspace}
\newcommand{\vbein}{vierbein\xspace}
\newcommand{\hbein}{\ensuremath{h_{a}^{\nu}}\xspace}
\newcommand{\hbeinr}{\ensuremath{h_{\alpha}^{\nu}}\xspace}
\newcommand{\PRZL}{\citetitle{Reichenbach1928}\xspace}
\newcommand{\Reich}{Reichenbach\xspace}
\newcommand{\VZ}{\jt{Vossische Zeitung}\xspace}
\newcommand{\FP}{\german{Fernparallelismus}\xspace}
\newcommand{\DP}{distant parallelism\xspace}
\newcommand{\Gtmnbar}{\ensuremath{\bar{\Gamma}\tmn}\xspace}

\newcommand{\rhp}[2]{(\cite[#1]{Reichenbach1920a}; tr.\ \citeyear{Reichenbach1969} #2)\xspace}
\renewcommand{\rzlp}[2]{(\cite[#1]{Reichenbach1928}; tr.\ #2)\xspace}
\renewcommand{\rzlap}[2]{(\cite[#1]{Reichenbach1928}; tr.\ [#2])\xspace}
\newcommand{\vza}[1]{(\cite{Reichenbach1929c}; tr.\ \citeyear{Reichenbach1978}, 1:#1)\xspace}
\newcommand{\hpa}[2]{(\cite[#1]{Reichenbach1929}; tr.\ \citeyear{Reichenbach1978}, 1:#2)\xspace}

\makeatletter
\def\tagform@#1{\maketag@@@{[\ignorespaces#1\unskip\@@italiccorr]}}
\makeatother


\creflabelformat{equation}{#2[\textup{#1}]#3}



%\newmdenv[linecolor=black]{infobox}
%%\renewcommand*{\multicitedelim}{\addcomma\space}
%\renewcommand*{\multicitedelim}{\addsemicolon\space}
\title{Coordination, Geometrization, Unification. The Reichenbach--Einstein Debate on the Unified Field Theory Program}

%TODO check translaton of Reichenbach!!!
%TODO check **

\begin{document}
\maketitle

\begin{abstract}
\lipsum*[1-2]
\end{abstract}


\begin{keywords}
Reichenbach \sep Unified Field Theory \sep General Relativity \sep Geometrization \sep Unification \sep Coordination	
\end{keywords}

\section*{Introduction}

\lipsum[1]

\begin{description}
\item[Einstein-Reichenbach debate of Weyl's theory (1920-1922)]\label{reichenbachweyl} In his 1920 habilitation, Reichenbach, although rather in passing, accused Weyl of attempting to deduce physics from geometry, by reducing physical reality to \scare{geometrical necessity} \citep[73]{Reichenbach1920a}. On the contrary, the greatest achievement of \gr, Reichenbach claimed, was to have shifted the question of the truth of geometry from mathematics to physics \citep[73]{Reichenbach1920a}. That the separation . Einstein seemed to agree. Reichenbach  that After their correspondence, \citet[367--368]{Reichenbach1921}  accepted \citets{Weyl1921} counterargument that the geometry of \spti has nothing to do with behavior of \rac, but complained about the overly formal nature of the theory \citep[367]{Reichenbach1921}.

%The recognition of the importance of this episode has been an important result of the Reichenbach-scholarship of the last decades \citep{Ryckman1995,Ryckman1996}. In the early 1920s, Weyl and Reichenbach could be seen as Einstein's \scare{agonists}---that is champions of two different \scare{Einstein} \citep{Ryckman2005}---debating over the role of the \emph{coordination} of geometrical structures and measuring devices, whether the latter must be describable within the framework of \gr or not \citep{Giovanelli2013a}.

%complained that in this way the theory loses \qt{convincing character}{überzeugenden Charakter} . 

The ideea of coordinate, seems Einstein, Einstein seems also agree, that have become hard to dey that Einstein poisiton as  fundametanl, had become actually very different from what Kant had imagnined. It was Reichenabch that seems to indudec Einstin to take a philosopjical position


\item[Reichenbach-Einstein correspondence (1926-1927)]\label{reichenbacheinsteinI} 
The correspondence has been rediscovered and published \citep[\V{15}]{CPAE} only recently \citep{Giovanelli2016d}. In March 1926, after making some critical remarks on Einstein's newly published metric-affine theory \citep{Einstein1925a}, Reichenbach sent Einstein a 10-page \scare{note} \lettercpaep{Reichenbach}{Einstein}{24}{3}{1926}[15][224]. In it, he constructed a mock unification of the gravitational and electricity in a single geometrical framework, thereby showing that the \scare{geometrization} of a physical field was a mathematical trickery rather physical achievement. After a back and forth Einstein seemed to agree \citep{Lehmkuhl2014}. The note was later included as section \S49 in a long technical \Ap to the \PRZL \citep[\SS46-50]{Reichenbach1928} in which  \gr is presented as a \scare{physicalization of geometry} rather than a \scare{\emph{geometrizaton} of gravitation} \citep{Giovanelli2020}. 

\item[Reichenbach-Einstein correspondence (1928-1929)]\label{reichenbacheinsteinII}   A few months after the publication of the \PRZL \citep{Reichenbach1928}, \citet{Einstein19281,Einstein19282} launched yet another attempt at a \uft, the so-called \DP-field theory. Reichenbach, now back in Berlin, discussed the new theory in person with Einstein and sent him once again a manuscript with some comments. The unpublished manuscript is still extant \citep{Reichenbach1928b}. This exchange of letters marked the cooling of Einstein's and Reichenbach's personal friendship but also the end of their philosophical kinship. In the late 1920s, \citet{Reichenbach1929a,Reichenbach1929b,Reichenbach1929c} came to realize that in Einstein's mind, the actual goal of the \uftp was not the geometrization, but as the \emph{unification} of two different fields, an undertaking for the sake of which Einstein was ready to embrace a strongly speculative approach to physics \citep{Dongen2010}.  
\end{description}

This episodes has been analyzed more extensively. the goal of this paper is two the result a complex overvieew. I think that a clarification emerge on what the matter of cntention. One the key theme of Reichenbach's philosophy is separation between mathematics and physics. In 1920 axiom and axiom. Th elatter as \apr principles. Weyl's theory that physically true because it was mathematically necessary.   After a discussion with Schlick, Reichenbach abandoned this approach. However, he never abandoned the of the separation between mathematics and physics. The further developmen of the \uftp appead Eddington ahd Einstein himslef. They what was mathematically was most simple from a geometrical point of view, was also true from a physical point of view. \q{The general theory of relativity by no means turns physics into mathematics. Quite the opposite: it brings about the recognition of a physical problem of geometry}.  The defence to supprot \rt but also. What was the actual reason for the succes of \rt. ERichenach a fundamental position, it was EInstein who progressively was ready to philosophical compromise for the sake of physocs. Einstein was ready to abanonde that geometry tested speartey from the rest of physics, was ready to the very idea was essential to, that mathematical simplicty was in tisef the key to realtiy. REichennach was that ultimatly more plalta taht  tha of stin

% \cop{In this connection we are likely to think of the development of the theory of relativity into a world geometry, yet it would be quite erroneous to interpret this development as signifying a fusion of physics and mathematics}. \q{The general theory of relativity by no means turns physics into mathematics. Quite the opposite: it brings about the recognition of a physical problem of geometry}. .... \cop{Although both mathematics and physics are sciences, the difference between them is fundamental, and we must put it down as quite impossible that it will ever disappear.} Reichenbach considered as the fundamental key result of \rt. The theory that Euclidean geometry is a in itself \apr, that the choice was ultimately a physical question. However, will then exactly the different position. Einstein will claim the mathematical simplicity is itself is a key to reality and that the is ultimately to prove jus tkike it is necessary 4+4 is equal four. Berlin again will the end not only of a personal relatioship but also of a philosipilca






\section{Coordination. Einstein, Reichenbach and Weyl Theory}
\label{Coordination}
After serving in World~War~\rom{1}, Reichenbach attended Einstein's lectures on special and general relativity in Berlin. We posses three sets of Reichenbachs of undated notes (HR-028-01-04, HR-028-01-03, HR-028-01-01). A set notes seems to corresponds are very similar to Einstein's own notes to the Einstein's lecture on spring term 1919 \citep{Einstein1919c}\footnote{Further information about Einstein as an academic teacher, see Vol. 3, the editorial note, "Einstein's Lecture Notes,"pp. 3-10, and for a survey of Einstein's academic courses, see Vol. 3, Appendix B.}. \cop{In the lectures follow the corresponding sections Einstein's previous published presentations  of \rt \citepp{Einstein1916}{Einstein1914a}.} However, both Reichenbach's and Einstein's lecture notes show in the 1919 lectures Einstein also used for the first time new interpretation of the curvature in terms of the parallel displacement introduced by Tullio \citet{Levi-Civita1916} and applied to \rt by Hermann \citet{Weyl1918}. Both names are mentioned explicitly. 

In the original presentation of \rt, Einstein started the metric $\gmn$ the distance $ds$ of two nearby points $dx_\nu$ from its coordinates $x_\nu$ indepnednly of the choce coordinate systems. In the lectures showed how. $dx_\nu$ are the componennt of vectros $A^\tau$. One can start with so called affine connection \Gtmn for the conditions for a coordinate independent condition, that two vectors of equal and parallel two vectors at neighboring points are equal and parallel. As the size of each displacement goes to zero, this broken line becomes a continuous curve. If we use a parameter $\ap$ to designate points along the curve a vector $dx\nu$, then $dx_\nu/d\ap$ is a vector that indicate the direction of the curve at a certain point. Thus parallel-transporting parallel to its self one can determine the straightest line among to points. The connection is curved if one parallel displaces along different paths, one gets, in general, a different vector at a distant point \citep[028-01-03, 37]{HR}. Einstein was very impressed The fundamental notions of could be recovered without any reference to the metric. The metric could be inserted separately. The length of vectors $l^2=\gmn A^\mu A^\nu$, can be introduced separately to compare the length of non-parallel vectors. Indeed, by assuming that the length does not change under parallel transport one can recover Riemannian geometry.  


From discussions with Einstein, Reichenbach might have become immediately aware that he was skeptical possible generalization as the premise of \uftp. Weyl was bothered by asymmetry comparison of direction of vectors which is path-dependent could not be the comparison their lengths was distant-gemetrical. To overcome this mathematical injustice, beside the affine connection Weyl also a introduced \scare{metric connection}. If a vector of length $l$ is displaced from $x_\nu$ to $x_\nu+dx_\nu$, it will in general have a new length $l+dl$, so that $dl/l=\phin dx_\nu$. In this way, in addition to the \scare{metric tensor} \gmn, a \scare{metric vector} $\phin$ of the same importance is introduced. That the \gmn are identified with the potentials of the gravitational field because of a \emph{physical fact} the equivalence principles. The \phin could be identified with the potentials electromagnetic field  because of the \emph{mathematical fact} that the $F\mn$ is the curl of the \phin, like in the first two Maxwell equations. 

\q{Der Traum des Descartes von einer rein geometrischen Physik}. Just like general relativity represented a geometrization of gravitational phenomena, Weyl's theory represented a unified geometrization of both gravitational and electromagnetic phenomena, which were, at that point, the only kind known. With some rethoric Weyl hoped to achieve a purely field-theoreticl representation of matter, Matter had seemingly become an epiphenomenon of the \scare{world metrics}. With some rethoric Weyl did not heistoate to speak of the dream of Descartes \... \q{physics and geometry coincide with each other},  \q{geometry has not been physicalized but physics has been geometrized} \citep{Weyl1919}. 


%Ihre Gesetze werden ebensowenig in der Wirklichkeit jemals verletzt, wie es Wahrheiten gibt, die mit der Logik nicht im Einklang sind; aber über das inhaltlich-Wesen- hafte dieser Wirklichkeit machen sie nichts aus, der Grund der Wirklich- keit wird in ihnen nicht erfaßt

%ir hatten erkannt, daß Physik und Geometrie schließlich zusammenfallen, daß die Weltmetrik eine, ja viel- mehr die physikalische Realität ist. Aber letzten Endes erscheint so diese ganze physikalische Realität doch als eine bloße Form ; nicht die Geo- metrie ist zur Physik, sondern die Physik zur Geometrie geworden. Wir haben nicht mehr wie nach alter Anschauung einen leeren Raum als die Form, in deren Rahmen sich eine Materie von gediegener Wirklichkeit konstituiert, und als den Schauplatz, auf welchem sich die wirklichen Geschehnisse, das sind dieser Materie Veränderungen abspielen; sondern die gesamte physische Welt ist zur Form geworden, der aus ganz andern Bezirken als denen der Physis ihr Inhalt zuwächst. 

%In his eyes, physics seemed to be transformed to a purely formal status and was absorbed by geometry. Matter had seemingly become an epiphenomenon of the "world metrics" which started to acquire a slightly mystical flavour


\subsection{Reichenbach's Habilitation and his critique of Weyl Theory}

Between 1918 and 1919, Einstein criticized repeatedly the theory. The most famous, but by no means only objection. Since we use atomic clocks with to measure the length $ds$ of the time-like displacement vector $dx_\nu$, the theory should have predicted that the rate of ticking of atomic clocks should depend on the electromagnetic field they have ben through in the past. However, atomic spectroscopic overwhelmingly show that spectral lines of atoms are well-defined. In general, Einstein was initially rather cautious about the unification project that he seemed to have thought to be premature. In spring 1919, however, Einstein reoriented his views, probably after a correspondence with Theodore Kaluza, who suggested a \uft based on Riemannian geometry with five dimensions \citep{Wuensch2005}. Einstein ultimately decided not publish Kaluza's paper. However, at about the same he submitted a paper what can be considered his first attempt to find a connection between \gr and the structure of matter \citep{Einstein1919}\todo{better}.  

These attempts were interrupted by the confirmation of the Dyson-Eddington expedition. By the end of 1919, Einstein was turned into an international celebrity. Philosophers also to work on the theory. In February or March 1920 Reichenbach decided to write his habilitation \hide{Im Februar (oder März) 1920 beschloß ich, meine Habilitationsschrift zu schreiben}. \hide{Ich hatte in den Monaten vorher Relth. gearbeitet, auch nach Weyl; den Grund hatte ich schon 1917-1918\todo{check} in Vorlesungen bei Einstein gelegt, aus welchen meine Kenntnis der Th. herrührt.}. Reichenbach felt that he was in had detailed technical knowledge that was not comparable to that of any philosophers, possibly including Schlick. As he later recalled \q{Vorlesungen bei Einstein gelegt, aus welchen meine Kenntnis der Th. herrührt}, and he the previous \q{in den Monaten vorher Relth. gearbeitet, auch nach Weyl}. Moreover, he was clearly already following early attempts at \uftp. E.g. was aware that \q{neuen Einsteinschen Auffansung} in which  \q{bei der innerhbalb des Elcktroms wicder die nicht-Euklid. Geometric gilt} \citep[028-01-04, Randbemerkung zu Blatt 18]{HR}\todo{check}. 


%There remains the peculiarity that the defined side does not carry its justification within itself ; its structure is determined from outside. Although there is a coordination to undefined elements, it is restricted, not arbitrary. This restriction is called "the determination of knowledge by experience." We notice the strange fact that it is the defined side that determines the individual things of the undefined side, and that, vice versa, it is the undefined side that prescribes the order of the defined side. The existence of reality is expressed in this mutuality of coordination.

%trikingly this idea of the mutuality of coordinating principles is presented in a very straightforward way in Reichenbach (1920). Here Reichenbach describes how the mathematical framework of a theory (the “defined side”) is related to empirical reality (the “undefined side”) :

The Kapp-Pusch coup on 13 March 1920, gave Reichenbach a few days of leave from Huth radio industry were he was employed. Thus, he could work without interruptions and in ten days he completed the manuscript, which was submitted in April\todo{check} The book was an attempt to \scare{save Kant from Kantians} that Reichenbach carried out in his habilitation has recently attracted renewed attention. Kantian theory in a limited way with the theory of relativity by distinguishing two senses of the \apr. provisionally adopted axiom of coordinations. While a priori in the constitutive sense, the coordination principles are contingent, process of a construction of a representation is hidden behind the idea of \scare{coordination}. \todo{??}. However, we will consider Reichenbach's early work only in as much as it includes his first critique to the \uftp. Reichenbach abandoned his \scare{Kantinanism} but will remain faithful to this line of criticism in the following years. The separation was the an essential part of the book. And it is precisely in ths context that Weyl cites Weyl theory as a counter exaple.

Reichenbach alludes briefly to \q{Weyl's generalization of the theory of relativity which abandons altogether the concept of a definite length for an infinitesimal measuring rod} \rhp{**}{**}. Reichenbach conceded that Weyl's theory represents a possible generalization of Einstein's conception of \spti which, \q{although not yet confined empirically, is by no means impossible} \rhp{**}{**}. Reichenbach reiterated Einstein's main objection that the theory would imply that \q{the frequency of a clock is dependent upon its previous history} \rhp{**}{**}. This result appear to be contracted by experience. However, Reichenbach conceded to Weyl, that  \q{these influences compensate each other on the average} \rhp{**}{**}. Thus, the fact that \q{the frequency of a spectral line under otherwise equal conditions is the same on all celestial bodies} could he interpreted as an approximation, rather than being a consequence of the Riemannian nature of space-time \rhp{**}{**}. What Reichenbach considered unacceptable was Weyl's justification of his purely infinitesimal geometry. 

According to Reichenbach, Weyl seems to imply that his non-Riemannian geometry must be true \emph{physically} because it is  \emph{mathematically} superior to Riemannian geometry, being a true realization of the principle of locality. As we have seen in Weyl geometry vector moving close loop which would same length but different direction in Riemannian geometry, different length and different direction in Weyl's geometry. Thus, in the new Weyl geometry last remanent of distance of Riemannian geometry had been eliminated. Weyl geometry seems to be the most \scare{general geometry}. There would be no reason to assume that a more general geometry applies to reality. However, Reichenbach, had already surmised that this generalization can be continued. In Weyl's geometry the length can be compared at the same point in different directions, but not at distant points. \q{The next step in the generalization would be to assume that the vector changes its length upon turning around itself} \rhp{**}{**}. Probably, more complicated generalization could be thought of. Thus, there is no \q{\scare{most general} geometry} that must be physically true. No matter one pushes further the level of mathematical abstraction, \q{the difference between physics and mathematics} cannot be eliminated; geometry alone can never be sufficient to establish the reality of physical space \rhp{**}{**}. 

A mathematical axiom system is indifferent with regard to the applicability and \q{never leads to principles of an empirical theory} \rhp{**}{**}. The axioms of Euclidean geometry are neither true nor false and only physics can decide whether Euclidean geometry applies to reality. This applies to more complicated geometrical systems as well:

\q{[Thus] it is incorrect to conclude, like Weyl\footnote{\todoi{check}} and Haas\footnote{\citep{Haas1920}}, that mathematics and physics are but one discipline. The question concerning the validity of the axioms for the physical world must be distinguished from that concerning possible axiomatic systems. It is the merit of the theory of relativity that it renowned the question of the truth of geometry from mathematics and relegated it to physics. If now, from a general geometry, theorems are derived and asserted to be a necessary foundation of physics, the old mistake is repeated. This objection must be made to Weyl's generalization of the theory of relativity  \textelp{} Such a generalization is possible, but whether it is compatible with reality \myemph{does not depend on its significance for a general local geometry}. Therefore, Weyl's generalization must be investigated from the viewpoint of a physical theory, and only experience can be used for a critical analysis. Physics is not a \scare{geometrical necessity}; whoever asserts this returns to the pre-Kantian point of view where it was a necessity given by reason \rhp{**}{**}}. 
%
This objection contains the backbone of Reichenbach's criticism of the \uftp over the years. The question concerning the \q{validity of axioms for the physical world} must be distinguished from that concerning possible axiomatic systems. Weyl seems to have unlearned precisely the fundamental of Einstein's theory who had showed the question whether the geometry of \spti is Euclidean or not is a physical question.

It is true that it is \q{a characteristic of modern physics to represent all processes in terms of mathematical equations}, and, one might add, progressively more abstract mathematics. Still, \q{the close connection between the two sciences must not blur their essential difference} \rhp{**}{**}. The truth of mathematical propositions depends upon internal relations among their terms; the truth of physical propositions, on the other hand, depends on relations to something external, on a connection with experience. \q{This distinction is due to the difference in the objects of knowledge of the two sciences} \rhp{**}{**}. The mathematical object of knowledge is uniquely determined by the axioms and definitions of mathematics. The definitions indicate how a term is related to that \citet{Schlick1918} had called \q{implicit definitions} \rhp{**}{**}. \cop{The nature of physical space cannot be determined by such definitions. It is a thing of the real world, not an object of the logical world of mathematics. In such a system is a statement regarding the significance of physics, the assertion that the system of equations is true for reality. This relation is totally different from the internal simplicity of mathematics}. 

%\cop{The \emph{axioms of connection} are the empirical laws of physics, the fundamental equations of a theory. The \emph{axioms of coordination} determine the rules of the application of the axioms of connection to reality, that is, they determine the rules of the connection.

%Reichenbach will soon abandoned his Kantianism under the influence of Schlick. 

However never abandoned the separation between mathematics and physics that will become one of central claims of the logical empiricism. What is more he believed that precisely this separation was most important results of \rt. On \datef{24}{5}{1920}, \todo{sent} The copy in Einstein's library contains some marginal annotations by hand, for example \q{sehr gut} on p.\ 74 to \cop{Reichenbach's contention that it is impossible to infer \textit{a priori} principles}. Einstein praised Reichenbach's \german{Habilitationschrift} in a letter to Schlick \lettercpaep{Einstein}{Schlick}{19}{4}{1920}[9][378]. A few days later Reichenbach asked Einstein to dedicate the book to him, insisting on the philosophical significance of \rt:  \q{Philosophen eine Ahnung davon haben, dass mit Ihrer Theorie eine philosophische Tat getan ist, und dass in Ihren physikalischen Begriffsbildungen mehr Philosophie enthalten ist, als in allen vielbändigen Werken der Epigonen des grossen Kant}. Indeed the old \apr for which Euclidean geometry is itself the geometry of reality  had become untenable. Einstein's seemed to agree on this point. \cop{Der Wert der Rel.Th. für die Philosophie scheint mir der zu sein, dass sie die Zweifelhaftigkeit gewisser Begriffe dargethan hat, die auch in der Philosophie als Scheidemünzen anerkannt waren}. In this sense, Einstein found Kant's apriorism untenable. Einstein insisted that \q{Begriffe sind eben leer, wenn sie aufhören, mit Erlebnissen fest verkettet zu sein}. 

%The subsequent months will be of fundamental importance in Reichenbach's biography. A correspondence with Schlick in October 1920 seems to have convinced Reichenbach that us axioms of coordination were simply \scare{convetions}. Reichenbach seemed to have been concerned that this solution would have made deprived on any physical content. \cop{Reichenbach explains that the principle of simplicity – used by Poincaré and Schlick to choose the \scare{right} geometry – does not seem “univoque” to him. As a consequence, as far as the principle of simplicity is not involved, the “coordination principles” cannot be conventions, because there must be a “synthetic” criterion to decide which}. ust after the passage Reichenbach adds : “the alternative between the two conventions appears as synthetic knowledge”. It is possible that Einstein's endorsme moght have convied. That to resctrit the coordiantes, ultimately found tin the $G+P$ formula. However, that one has to seprate mathematics from physics will as constantn of Reichenbach's philosopjy. It is this point that not only his critisim of Weyl but also of the \uftp attempts\todp{.


%\cop{that the arbitrariness of the principles is limited as well as principles are combined}



%Einstein used a similar wording by commenting on the manuscript of Cassirer's \scare{Kantian} booklet on relativity. \q{Conceptual systems appear empty to me, if the manner in which they are to be referred to experience is not established} \lettercpaep{\Einstein}{\Cassirer}{6}{6}{1920}[10][44]. In particular, \q{[w]ith the interpretation of the $ds$ as a result of measurement, which is obtainable by means of measuring rods and clocks the general theory of relativity as a physical theory stands or falls} \lettercpaep{\Einstein}{\Cassirer}{6}{6}{1920}[10][44]. The gravitational redshift, can be taken as an empirical confirmation of general relativity only because different atoms of the same substance can be regarded as identically constructed clocks reproducing the identical unit of time. For this reason it is possible to \scare{normalize} the absolute value of $ds$ by counting the wave crests on atom. According, Weyl's theory deprived the $ds$ of any physical meaning. However, real \rac behave differently than predicted by Weyl theory forcing Weyl to assume an inconsistent position. According to Einstein, line \gr, Weyl's  \q{theory is based on a measuring rods geometry}, that is it presupposes the comparability of lengths. However, it entains only \q{thought measuring rods \origins{nur gedachte Massstäbe}} that behave differently from the real ones. \q{This is repugnant} \lettercpaep{\Einstein}{\Besso}{26}{8}{1920}[10][85\me]

%As we have seen, the gravitational redshift, just like the transverse Doppler effect in special relativity, can be taken as an empirical confirmation of general relativity only because different atoms of the same substance can be regarded as identically constructed clocks reproducing the identical unit of time. 


%\cop{Ich freue mich wirklich sehr darüber, dass Sie mir Ihre ausgezeichnete Broschüre widmen wollen, noch mehr aber darüber, dass Sie mir als Dozent und Grübler ein so gutes Zeugnis ausstellen. Der Wert der Rel. Th. für die Philosophie scheint mir der zu sein, das sie die Zweifelhaftigkeit gewisser Begriffe dargethan hat, die auch in der Philosophie als Scheidemünze anerkannt waren. Begriffe sind eben leer, wenn sie aufhören, mit Erlebnissen fest verkettet zu sein. Sie gleichen Emporkömmlingen, die sich ihrer Abstammung schämen und sie verleugnen wollen}. Einstein made a similar claims by writing to Cassirer in the very same days.



%Reichenbach attended Einstein's Ieerures in Berlin (see Doc. 57, note 2).

%The thesis was published after 15 June 1920 as Reichenbach 1920 (see Doc. 57). Stuttgart, Wiederholdstr. 13. d. 15. Juni 1920. 
%
%
%Widmung .That most of the discussion of the book, relativized constitutive \apr, however, one first by a philosophers of Weyl theory, and in particular was that is itself superior. 


\subsection{The Reichenbach-Weyl Correspondence}

Reichenbach met Weyl for the first time at the 86th Assembly of the \german{Versammlung der Gesellschaft Deutscher Naturforscher und Ärzte} in Bad Nauheim in September 1920. Reichenbach where he might have assisted ad the discussion between Einstein, Pauli and Weyl following the latter talk \citep{Weyl1920}. Immediately, thereafter, Reichenbach must have sent a copy of his \citetitle{Reichenbach1920a} \citep{Reichenbach1920a}. Weyl replied with some delay in February 1921 since he was in Barcelona in the meantime. Weyl was not upset by Reichenbach's criticisms and replied rather amicably to some issues \q{which concern less the philosophical than the physical} \letterp{Weyl}{Reichenbach}{2}{2}{1921}[**][HR]. In particular for our goals it is interesting to consider the following point made by Weyl.

\qt{It is certainly not true, as you say on p.\ 73, that, for me, mathematics (!!, e.g. theory of the $\zeta$-function?) and physics are growing together into a single discipline. I have claimed only that the concepts in geometry and field physics have come to coincide \textelp{} As for my extended theory of relativity, so I cannot admit that the epistemological situation is in any way different from that of Einstein. \textelp{} Experience is in no way anticipated by the assumption of that general metric; that the laws of nature, to which the propagation of action in the ether is bound, can be of such a nature that they do not allow any curvature. \textelp{} What I stand for alone is this: The integrability of length transfer (if it exists, I think uielıt. because I don't see the slightest dubious reason for it) does not lie in the nature of the metric medium, but can only be based on a special law of action\todo{That is on the field equations of the theory which in turned can be derived from an action principles}. If the historical development had been different, it seems to me that no one would have thought of considering the Riemannian case from the outset. As far as the notorious \scare{dependence on previous history} is concerned, I probably expressed my opinion clearly enough in Nauheim}{ Was nıeine erweiterte Relativitätstlıeorie betrifft. so kann ich nicht zugeben, daß da erkenntnislogisch die Sache irgendwie anders liegt wie bei Einstein. \textelp{}  Der Erfahrung wird durch die Annahme jener allgemeiner Metrik in keiner Weise vorgegriffen; dass die Naturgesetze, an welche die Wirkungsausbreituug in Äther gebunden ist, können ja von solcher Art sein, daß sie keine Streıfkenkrümmung zulassen. \textelp{} Wofür ich allein eintrete, ist dies: Die Integrabilität der Strekken\"ubertraguug (wenn sie besteht, ich glaubs uielıt. denn ich sehe nicht den geriugsteu zwiugeudeu Grund dafür) liegt uielıt im Wesen des metrischen Mediumsm, sondern kann nur auf einem besonderen Wirkungsgesetz berulıen. Ware die historische Entwieklung anders verlaufien, so seheint nıir wåire uienıand darauf verfallen. von vorn- herein gerade nur den Rieuıauuselıeu Fall iu Erwåígımg zu zielıeu. - *Nas die berüchtigte “Abhšingigkeit von der Vorgeschichte" betrifft, so habe ieh darüber wohl nıeine Ansicht deutlieh gemıg in Naulıeinı ausgesprochen. An der 4. Aufl. wird Sie wahrscheinlich vor al- lenı nıeine veränderte Stellımgnalnue zum Problem der Materie i}. 
%
In his talk in Bad Nauheim \citet{Weyl1920a} introduced the distinction between \german{Einstellung} and \german{Beharrung} to explain away the discrepancy between the non-Riemannian behavior of the \scare{ideal} time-like vectors implied by his theory and the Riemannian behavior of the \scare{real} clocks that are actually observed. That the geometry of \spti is non-Riemannian, in spite of the fact that the \rac behave in a Riemannian way. Thus in spite a non-Riemannian behavior it might came out at the end the \rac will show a Riemannian one. He suggested that atomic clocks might not \emph{preserve} their Bohr\todo{Laue} radius if transported, but \emph{adjust} it every time to some constant field quantity, which he could identify with the constant radius of the spherical curvature of every three-dimensional slice of the world, furnishing a natural unit of length. Weyl had developed this idea of doubling the geometry, in several papers published in 1921. In the July paper Weyl also address in public Reichenbach's criticism

\q{From different sides\footnote{The reference is to \citealp{Reichenbach1920a} and \citealp{Freundlich1920} who however refers to \cite{Haas1920}} it has been argued against my theory, that it would attempt to demonstrate in a purely speculative way something \emph{a priori} about matters on which only experience can actually decide. This is a misunderstanding. Of course from the epistemological principle [aus dem erkenntnistheoretischen Prinzip] of the relativity of magnitude does not follow that the \textquotedblleft{}tract\textquotedblright{} displacement [Streckenübertragung] through \textquotedblleft{}congruent displacement\textquotedblright{} [durch kongruente Verpflanzung] is not integrable; from that principle that no \emph{fact} can be derived. The principle only teaches that the integrability \emph{per se} must not be retained, but, if it is realized, it must be understood as the \emph{outflow} [Ausfluß]\emph{of a law of nature }\citep[475; last emphasis mine]{Weyl1921b}}

\cop{Weyl's theory would have its reason for being only if one could, by some means, deduce from this theory that the equality of two objects of nature, for example the rigorous equality of two atoms of the same chemical substance placed identically in the same conditions, is independent of their prehistory. This independence is one of the most solidly established experimental facts. To in the possibility (of arriving) of demonstrating this independence by logical or mathematical means, starting from the theory of Weyl in its present form. If this achieved, that a theory in which the behavior of vectors is Riemannian, but the behavior appears not to be.} However, the same problem should have emerged in Einstein theory, which also ultimately should provide a theory of matter from which the Riemannian behavior of clocks should be derived. 

%At the end of the theory, e.g. electrons with certain size, and that atoms stable atoms that we use as clocks. If the theory predict that this physical system behave so that the spectra lines is always constant. However, this has nothing to do with the geometrical structure on which that theory is based. Still this solution was not considered satisfying. The we could have a the

\subsection{The Weyl-Reichenbach Appeasement}

\hide{Im Sept. 1921 trug ich schon den ersten Bericht \"uber die Axiomatik auf dem Physikertag in .lena vor.3 Ich hatte damals gro§en Erfolg; aber niemand ist damals auf den Gedanken gekommen, mich in eine angemessene Stelle zu berufen. Ich blieb in Stuttgart sitzen. Niederschrift und Ausbau im Winter 1921 /22.}  \hide{Niederschrift und Ausbau im Winter 1921/22. Das alte MS wurde völlig umgestoßen. Genauere Behandlung der allg. Th. im Aug.-Spt. 1922. Vortrag darüber in Leipzig, Sept. 1922.}

In September 1921, Reichenbach presented the first report on axiomatics of \sr at the Physicists' Day in Lena \lettercpaep{Reichenbach}{Einstein}{5}{12}{1921}[12][266], where he met Weyl again. He worked further on the project on a mono Septer 1922. The correspondence with Weyl that ensued turned out to be less amicable, since the latter raised sever criticism\todo{??}. \cop{Reichenbach might ahve also sent Weyl a personal retraction, possibly in lost latter from 8.1.1922. The letter had been forwarded from Zurich to Barcelona, where Weyl was giving his Catalonian Lectures (Weyl 1923)}.  However, Reichenbach di issue a public retraction. At about the same time, he started to work on a long review articles on relativity, that was finished in Spring 1922. On \datef{24}{3}{1922}. Erwin Freundluch sent to Einstein \q{die Druckbogen einer kritischen Untersuchung von Reichenbach auf dessen Wunsch}. The presentation entails also a long, a more balanced review of Weyl theory\footnote{which was surprisingly excluded from the translations of this writing in the 1970s}. By the time. Reichenbach has now abandoned his Kantianism, and to develop a form conventionalism to translate in a form of geometrical empiricism.  Geometry and experience became for Reichenbach of the separation between geometry and physocs. Reichenach that this separation was in the wrok as he have leard from personal conservations.

%Man darf eine Darstellung der relativistischen Philosophie nicht abschließen, ohne der wichtigen Erweiterung zu gedenken, die vor 3 Jahren Weyl dem Raumproblem zuteil werden ließ. Denn obgleich es sich hier zunächst um die Aufstellung einer mathemati- schen Theorie handelt, ist sie von ähnlicher philosophischer Bedeu- tung wie die Riemannsche Verallgemeinerung der euklidischen Geometrie, und darum unabhängig von aller Anwendung auf die Physik eine Erweiterung unseres philosophischen Wissens vom Raume. Die grosse Entdeckung Weyls besteht darin, daß er einen allgemeineren Mannigfaltigkeitstypus aufdeckte, von dem auch der Riemannsche Raum nur ein Spezialfall is

The class (a) of Riemannian geometries is fixed by the axiom that lengths do not depend on their prehistory. \q{zwei natürliche Maßstäbe, die sich einmal zur Deckung bringen lassen, lassen sich auch nach dem Transport auf verschiedenen Wegen wieder zur Deckung bringen}. The (b) choice among geometries is ultimately conventional, that we consider rigid. The convention can be fixed by eliminating what we later will called forces of type $X$. After that the question whether is euclidean or not is empirical question, that can be answered by carefully by shielding from differential forces. Thus the sum, $G+P$ ultimately fixes the convention of (b). it may well also have Euclidean relations with the field $X = 0$, but this is a point which we can never know a priori. The merit of Weyl to have shown that the assumption (a) was not necessary. The first way to show it, in two ways Reichenbach seems to espouses that idea of the two versions of Weyl theory:

\begin{itemize}
\item This procedure is first of all a purely mathematical discovery; it indicates a more general type of manifold that can be applied to reality when the Riemann class axiom is not satisfied for natural scales. hat length of limes using a measuring isntumes, and this would give different results if would a different length a different prehisory. If a vector of length $l$ is displaced from $x_\nu$ to $x_\nu+dx_\nu$, it will in general have a new length $l+dl$, so that $dl/l=\phin dx_\nu$. \q{The change in scale is measured by 4 quantities $\varphi_{\mu}$ forming a vector field}. This is a mathematical discovery that it is neither true nor false. It cab applied to reality, if one coordinates the length $l$ as reading of some physical measuring instruments. In this case makes predictions about the behavior of \rac. If one interpretes the $\varphi_{\mu}$ as the electromangetic field potnetials. Then would have that would depend on their prehistory.  however, that this axiom is quite well fulfilled in reality, so that the first way of generalization seems unsuitable. Indeed, the existence of atoms with the same spectral lines shows that clocks behave differently that predicted by Weyl, theory, whereas Einstein's prediction about the behavior of clocks has been confirmed.

\item Way reintroduced as a mathematical aid; it defines an ideal process of transfer of length which, however, has nothing to do with the behavior of real \rac. He needs this engraftment process because he wants to identify the vector field \phin with the electromagnetic potential, and then obvious forms for the most general physical equations arise (for the \scare{action}). the second step is find the field equations, via the \scare{action principle}. \cop{One constructs a scalar quantity (the action) from the dynamical quantities \gmn and \phin then finds the conditions needed to restrict the scalar to an extremum (a maximum or minimum) with respect to variations in those dynamical quantities. The problem the right action and the right dynamical quantities to produce the desired equations}, that is to recover Einstein and Maxwell field equations.  To test theory its capacity do deliver such explaining the existence of electrons of the same charge and size. With this, however, the \q{theory loses its convincing character and comes dangerously close to a mathematical formalism}, that unnecessarily complicates physics for the sake of elegant mathematical principles; and because of this thought, \q{Weyl's theory is viewed very cautiously by physicists (especially by Einstein)}.  
\end{itemize}

Thus Weyl1 that \rac and empriical content; since he recongiszx that \rac simpy complciated. That the behavior of \rac cannot be from. Unfortunately, however, the theory does not agree with the physical facts. Even if the electromagnetic field is introduced, the behavior of \rac is still integrable. This is confirmed by a large amount of experimental knowledge about spectral lines of atoms that are typically employed as clocks. Those spectral lines are always sharp, well-defined spectral lines. If atomic clocks changed their periods as a function of their \st paths, one would expect that atoms with different pasts would radiate different spectral lines \rzlap{355}{494}\footnote{This is, of course, the celebrated objection against Weyl's theory \citets{Einstein1918c}}. However, Weyl's had then to explain Weyl's explanation, according to which the unequivocal transferability takes place by adjusting the standards to the radius of curvature of the world, \q{is essentially just another linguistic expression for the facts at hand, not a reduction to a more general law}. Most of all the theory would, a succesful expation. In particular, this \scare{setting} has nothing to do with its \q{congruent transplantation, so this remains physically empty}.  

Still the objection was intrinsically mathematical was not fair, wanted to derive physical reality from pure mathematics:

\q{However, I have to retract my earlier objection (47, p. 73) that Weyl wants to deduce physics from reason, after Weyl has cleared up this misunderstanding (72, p. 475). Weyl takes issue with the fact that Einstein simply accepts the unequivocal transferability of the standards. He does not wish to dispute the Riemann-class axiom for natural standards, but only to demand that the validity of this axiom, since it is not logically necessary, be understood as "the emanation of a law of nature." I can only agree with Weyl's demand; it is the importance of mathematics that they are with the. gesetzes verstanden werde«. Ich kann dieser Forderung Weyls nur zustimmen; es ist die Bedeutung der Mathematik, daß sie mit dem Aufdecken allgemeinerer Möglichkeiten die speziellen Tatbestände der Erfahrung als speziell kennzeichnet und SO die Physik vor Simplizität bewahrt}

But the only fact that he had tried to follow this path, regardless of its empirical correctness, was a genial advance [genialer Vorstoß] in the philosophical foundation of physics (\citealp[367f.]{Reichenbach1921}). However, Reichenbach that he clearly he initially attributed to Weyl. Weyl had not showen being truly infinitesimal and myst be true \apr for reality, it that that given a geometry there always a more general one. Still Weyl's theory has a profound philosophical meaning. Eiclidea geometry was not ovious the discovery of non-Riemannian gemetory 

%Reichenbach recognized that Weyl wanted to exception to the fact, that Einstein \q{has simply condoned [einfach hingenommen] the univocal transportability of natural measuring-rods [eindeutigen Uebertragbarkeit natürlicher Maßstäbe]}. He did not want to \q{dispute the axiom of the Riemannian class for natural measuring rods; he wants only to urge that the validity of this axiom, being not logically necessary, \emph{``is understood as an outflow }[Ausfluß] \emph{of a law of nature}}. 

\qt{The \ls{philosophical} significance of Weyl's discovery consists in the fact that it proved that the problem of space cannot be closed even with Riemann's concept of space. If the epistemology of today wanted to extend the assertion of Kant's transcendental aesthetics to the point that the geometry of experience must in any case at least have a Riemannian structure, it is held back by Weyl's theory. For that Weyl's space is at least \ls{possible} for reality cannot be denied. One must not even believe that Weyl's theory has reached the highest level of generality. Einstein has shown (14) that Weyl's requirement of the relativity of magnitude can also be satisfied without making use of Weyl's method of measurement. After that, Eddington (15) again developed a generalization of which Weyl's space class is only a special case, and Eddington's space class is again included as a special case in a more general one found by Schouten (63). The merit of Schouten's theory is that it gives the conditions under which the class of space developed is the most general; they are very general conditions, like differentiability and the like. But of course there is no absolutely most general space class; and the history of the mathematical problem of space may teach epistemology never to make general claims. There are no most general terms}{Die \ls{philosophische}  Bedeutung der Weylschen Entdeckung besteht deshalb darin, daß sie bewiesen hat, daß ein Abschluß des Raumproblems auch mit dem Riemannschen Raumbegriff nicht gegeben ist. Wollte also die Erkenntnistheorie heutP- die Behauptung der transzendentalen Aesthetik Kants dahin erweitern, daß die Geometrie der Erfahrung auf jeden Fall wenigstens von Riemannscher Struktur sein muß, so wird sie durch die Weylsche Theorie daran zurückgehalten. Denn daß der Weylsche Raum wenigstens für die Wirklichkeit  \ls{möglich} ist, läßt sich nicht bestreiten. Man darf nicht einmal glauben, daß mit der Weylschen Theorie nun die höchste Stufe der Allgemeinheit erklommen sei. Einstein hat gezeigt (14), daß man die Weylsche Forderung der Relativität der Größe auch befriedigen kann, ohne von dem Weylschen Meßverfahren Gebrauch zu machen. Danach wurde von Eddington (15) wieder eine Verallgemeinerung entwickelt, von der die Weylsche Raumklasse nur ein Spezialfall ist, und die Eddingtonsche Raumklasse ist wieder als Spezialfall in eine allgemeinere eingegangen, die von Schouten (63) gefunden wurde. Der Vorzug der Schoutenschen Theorie besteht darin, daß hier die Bedingungen angegeben werden, unter welchen die entwickelte Raumklasse die allgemeinste ist; es sind sehr allgemeine Bedingungen, wie Differenzierbarkeit und ähnliches. Aber eine schlechthin allgemeinste Raumklasse gibt es natürlich nicht; und die Geschichte des mathematischen Raumproblems mag der Erkenntnistheorie eine Lehre sein, niemals schlechthin allgemeine Behauptungen aufzustellen. Es gibt keine allgemeinsten Begriffe}.

This passage that the physicssts as the largest freedom in the choice of the mathematical strucure. At this point was aware fo Schouten and suggested the possibility of abandoning that the connection was symmetric. This procedure howver, Reichenbach felt that a coordinate definition was necessary; whatever choice the first step to coordinate this with a piece of realty. Or to follow Weyl 2 and procede As it turned out, Weyl did not actually agree with the the double approach interpretation: \q{Den Plan, starre Maßstäbe mit meiner Verpflanzımg zu identifizieren, habe ich aufgegeben. weil ich ihm nie gehabt habe}, sondern \q{ich war überrascht, als ich salı, daß Physiker das in meine Worte hineininterpretiert hatten} \letterp{Weyl}{Reichenbach}{20}{05}{1922}[015-68-02][HR]. However, will that this was for him exactly the problem the second version of Weyl's theory will open the gate for increasingly more. Indeed, Reichenach reaslisd that Einstein had precisley taken this path. 

%The epistemological model Reichenbach had in mind was however, the very opposite. We have complete freedom that the choice was, must be given a physical menaing, that can be interprted as electromagnetic field. \gmn measured with rac.Weyl later abandoned this approach,\gmn and \phin. The first the comparsion of theng at one point, the latter for the change of lenght at distance. Only in hightsidet after finding the field equations. Certain solutions, that could be the tested by chechign wither the \rac behave as predicted.



\section{Geometrization: Reichenbach's Correspondence with Einstein}
\label{geometrization}
By the end of the 1922, Einstein had started to find Eddington's theory appealing. Semi-metrical theory, that is was the reason why Einstein. The manuscript consists of five pages. Ironically, the third, fourth, and fifth pages were written on the back of the beginning of a typescript of Reichenbach’s contribution to the meeting of German physicists in Jena. Einstein, why he was skeptical of Weyl's approach. The reason was only superficially similar to that of Einstein. Weyl's theory was semi-metrical. It assumed the assumption of the transportability of geometrical lengths, and only forgo that this is path independent. Since actual \rac that we are transportable, real rods behave differently from geometrical lengths. Einstein1923d This inconsistency was for Einstein unacceptable. He considered preferable to forgo the very idea of comparability of lengths. Following Eddington a very general affine connection in which the length of vectors is not defined. That to start from the a general affine connection \Gtmn. Any physical interpretation. From this one Differently the tensor is not further if is symmetric. From one can obtained the Riemann and Ricci tensor, that can be split it into two parts that, could be  The latter can be split into parts den Komponenten dr. des Linienelementes die Invariante $R dx.dx$, liefert.. that behave as elementary particles. Einstein published thereafter, into two papers. EInstien conceded that the choice of \Gtmn as fundamental variable is not physically motivated, and \Gtmn has no physical meaning tat all \q{an dem zentral-symmetrischen Falle ihre Vereinbarkeit mit der Erfahrung zu prüfen. (121 Ich glaube aber nicht, dass man auf diesem Wege zu einer u},

%is semi-metrical and indeed, it was in general more effective to go in where the affine connection has not physical meaning at all 1923. Weyl's theory assume that there transportable lenghts, but simply denies that is path independent. Howevr, tha such lengths are measured with \rac. Weyl ulitmately must however, that do not behavte. Eddington's theory since the theory, statrs from a generla affine connection of the form



%The choice of   \cop{The procedure is, in any case, arbitrary, because we have taken the I's to have l physical meaning, and then we take the simple expression as a tensor to deduce the laws of gravitation and electromagnetism by variation. Nevertheless, we avoid Weyl's weak point. Up to now, the calculations I have made with respect to gravitational and electromagnetic fields have given results already known. As to the structure of electrons, the calculations I are so complicated that I have not been able to obtain anything definitive up to the present.} Ultimately, sicne ultimatley onlyt combination geometry and physics and testable, that is of gemetry and the field equations. As Einstein's eplianed in Madrid \cop{The procedure is, in any case, arbitrary, because we have taken the I's to have l physical meaning, and then we take the simple expression as a tensor to deduce the laws of gravitation and electromagnetism by variation. Nevertheless, we avoid Weyl's weak point. Up to now, the calculations I have made with respect to gravitational and electromagnetic fields have given results already known. As to the structure of electrons, the calculations I are so complicated that I have not been able to obtain anything definitive up to the present.}, Einstein delivered his lectures in French. 

Einstein published two papers on this theory \citep{Einstein1923c,Einstein1923e} Reichenbach asked requested copies of his papers (Abs. 44) on  also searching form help to publish his axiomatization book \citep{Reichenbach1924}. Reichenbach, as others probably found the entire procedure questionable. As Pauli's requirement that an abstract concept, like the $\Gamma\tmn$ and the to deduce the \Fmn from it that are observable. Ultimately, it should only be permissible in physics when it can be established whether it applies in concrete cases of observation. This requirement does not seem far from the view that Einstein often defended in the past. However, Einstein realized that this requirement was too severe. In several writings of those years Einstein had repeatedly insisted geometry cannot be tested separately from physics. The choice was not even physically motivated \citep{Einstein1923}. Possible experiences, he claimed, must correspond not to an individual concept but to the system as a whole (\cite[1692]{Einstein1924}; \cf\cite{Giovanelli2014a}). If starting from $\Gamma\tmn$ leads to a promising set of field equations, then the use of $\Gamma\tmn$ as a fundamental variable is justified that lead to new results this choice can be justified \emph{post facto}


%\qit{The mathematics is enormously difficult}{he wrote to Besso}{the link with what can be experienced is unfortunately becoming increasingly indirect}{Das Mathematische ist enorm schwierig, der Zusammenhang mit dem Erfahrbaren wird leider immer indirekter} \lettercpaep{Einstein}{Besso}{5}{1}{1924}[14][190]. However, Einstein was still convinced that a field theory that might offer the solution to the quantum problem \citep{Einstein1923f} was at least \qt{a logical \emph{possibility}, to do justice to reality without \emph{sacrificium intellectus}}{\latin{sacrificium intellectus} der Wirklichkeit gerecht zu werden} 	\lettercpaep{Einstein}{Besso}{5}{1}{1924}[14][190], that is, without retreating to a positivist-phenomenalist agnosticism.  Einstein distanced himself from the logical positivists' insistence on the need to coordinate every fundamental concept of a theory to a \q{piece \originsg{Ding} of reality} \citeptra[5]{Reichenbach1924}[8]{Reichenbach1969}, although they did not seem to have taken notice. However, the more rationalistic attitude of neo-Kantian philosophers like Ernst \citet{Cassirer1921} and his followers \citepp{Winternitz1923}{Elsbach1924} did not seem to offer a valid alternative \citepp{Einstein1924a}{Einstein1924}. According to Einstein, science extensively uses non-empirical \scare{ideal} conceptual constructions (say the \gmn, the $\Gamma\tmn$, etc.) \citep[1691]{Einstein1924}. But non wasy there were \apr by their success. 

It is against this background that Einstein became interested in in the rationalistic reading of relativity suggested by \citetp{Meyerson1925}. Einstein still found Meyerson's account \qt{unfair}{ungerecht} \qt{as the escapades by Weyl and \Eddington are considered to be essential parts of the theory of relativity}{als die Eskapaden von Weyl und \Eddington zum Wesen der Relat.~Theorie gerechnet} (\CPAE{14}{455, 6; \datedm{12}{3}{1925}}). However, he will soon embrance ... The interest soon fadered. \qt{Night, sweating properly \textelp{} the conviction of the impossibility of the field theory in the current sense becomes stronger}{Nachts geh\"{o}rig schwitzen \"uberzeugung von Unm\'ogl. von Feldtheorie im bisherigen Sinne verst\"arkt sich} (\CPAE{14}{455, 9; \datedm{17}{3}{1925}}). 

These doubts became certainties when Einstein returned to Europe. \qit{On \datedm{1}{6}{1926}, I got back from South America}{Einstein wrote to Besso}{I am firmly convinced that the whole line of thought Weyl-Eddington-Schouten%
%
\footnote{\citet{Schouten1924} claimed that it was possible to overcome a shortcoming of Einstein-Eddington's affine theory (in which no electromagnetic field can exist in a place with vanishing electric current density) by dropping the assumption of the symmetry of the affine connection}% 
%
does not lead to anything useful from a physical point of view and I found a better trail that is physically more grounded}{Am 1. Juni bin ich von S\"udamerika wiedergekommen ... Ich bin fest \"uberzeugt, dass die ganze Gedanken-Reihe Weyl-Eddington-Schouten zu nichts physikalisch brauchbarem f\"uhrt und habe jetzt eine andere Spur gefunden, die mehr physikalisch fundiert ist} \lettercpaep{Einstein}{Besso}{5}{6}{1925}[15][2]. As he explaiend to besso to use both \Gtmn and \gmn but were not symemtric\todo{metric affine theory}. At the beginning of \datemy{9}{7}{1925}, Einstein presented at the Academy of Science the new trail he anticipated to Besso, a further attempt at a \uft \citep{Einstein1925a}, in which both the affine connection and the metric were considered as fundamental variables without assuming their symmetry. Einstein commented to Millikan with the usual initial enthusiasm: \qt{I now think I have really found the relationship between gravitation and electricity}{Ich glaube nun, die Beziehung zwischen Gravitation und Elektrizit\"at wirklich gefunden zu haben} \lettercpaep{Einstein}{Millikan}{13}{7}{1925}[15][20]. However, during the summer, Einstein had already started to nurture some skepticism (\lettercpae{Einstein}{Ehrenfest}{18}{8}{1925}[15][49]; \lettercpae{Einstein}{Millikan}{13}{7}{1925}[15][20]; \lettercpae{Einstein}{Ehrenfest}{18}{9}{1925}[15][71]). The paper was published at the beginning of September, and by that time, Einstein probably already moved on (\lettercpae{Einstein}{Rainich}{13}{9}{1925}[15][106]; see \cite{Einstein1927c}). 


\subsection{The Einstein-Reichenbach Correspondence}

Probably, to write Entsprechend sollte das Buch heißen: "Philosophie der exakten Naturerkenntnis" u. das Gesamtgebiet behandeln. Später (1926) entstand dann der Plan, es in 2 Bände zu teilen, u. den 1. Bd. mit dem Untertitel "Raum und Zeit" schon für sich erst mal herauszubringen. Der Verzicht auf die Zusammenfassung in Werk und die Wahl des Titels "Phil. d. R.-Zt.L." geschah erst im Mai 1927 auf Wunsch Dr.\ Lübkes im Verlag de Gruyter, der es nicht für ratsam hielt, zwei zeitlich so getrennte Bände in 1 Werk zusammenzufassen.. Im März 1925 wurden nur die ersten §§ niedergeschrieben.  
During those same months, Reichenbach, despite the support of Max Planck, was struggling to obtain his \german{Umhabilitation}\footnote{The process of obtaining the \latin{venia legendi} at another university} from Stuttgart to Berlin in order to be appointed to a chair of natural philosophy that had been created there \citep{Hecht1982}. Im Okt. 1925 begann dann die erste große Arbeit an dem Buch; der Abschnitt Raum wurde hier im wesentlichen geschrieben, after having work on the new quantum theory. März-April 1926 wurde die Weylsche Theorie bearbeitet u. die eigentümliche Lösung des § 49 gefunden. Auch wurde damals der ganze Anhang geschrieben. (Korrespondenz mit Einstein). On \datef{16}{3}{1926}, Reichenbach sent a letter to Einstein in which, after discussing his academic misadventures, he remarked on the new \scare{metric-affine} theory \citep{Einstein1925a}:

\qt{I have read your last work on the extended Rel.\ Th\footnote{\cite{Einstein1925a}} more closely, but I still can't get rid of a sense of artificiality which characterizes all these attempts since Weyl. The idea, in itself very deep, to ground the affine connection independently of the metric on the $\Gamma^{i}_{kl}$ alone, serves only as a calculation crutch here in order to obtain differential equations for the $g_{ik}$ and the $\varphi_{ik}$ and the modifications of the Maxwell equations which allow the electron as a solution. If it worked, it would of course be a great success; have you achieved something along these lines with Grommer? However, the whole thing does not have the beautiful convincing power \origins{Ueberzeugungskraft} of the connection between gravitation and the metric based on the equivalence principle of the previous theory}{Ich habe jetzt Ihre letzte Arbeit zur erweiterten Rel. Th. genauer gelesen, aber ich kann auch da das Gefühl des Künstlichen nicht los werden, das allen diesen Versuchen seit Weyl anhaftet. Die an sich doch sehr tiefe Idee, den affinen Zusammenhang unabhängig von dem metrischen zu begründen allein auf die $\Gamma^{i}_kl$ dient doch schließlich nur als Rechenknüppel, um Differentialgleichungen für die $g_{ik}$ und $\varphi_ik$ zu bekommen und solche Abänderungen der Maxwellschen Gleichungen zu bekommen, die das Elektron als Lösung zulassen. Wenn das geht, ist es natürlich ein großer Erfolg; haben Sie eigentlich mit Herrn Grommer etwas in dieser Richtung errichtet? Aber die ganze Sache hat doch nicht die schöne Ueberzeugungskraft, wie die auf das Aequivalenz-Prinzip gestützte Verknüpfung von Gravitation und Metrik in der früheren Theorie}[\letter{Reichenbach}{Einstein}{16}{3}{1926}][20-83][EA]
%

The \scare{convincing power} (\german{Überzeugungskraft}) of general relativity \citep[367]{Reichenbach1921}, in which the identification of the $g_{ik}$ with the gravitational potentials was anchored in the principle of equivalence, that justified the connection and the behavior of \rac. Einstein's theory introduces the affine connection independently of the metric. However, it does not attribute any physical meaning to neither of the; the separate variation of the metric and connection was nothing more than a \scare{calculation device} to find the desired field equations. Only in hindisight the symmetric part was indetified of the \gmn and antisymmetic with the electromagnetic field, and merely for formal reasons. Reichenbach, however, was ready to revise his negative judgment if Einstein's theory delivered the \scare{electron}, althogith he was probably very skeptical that this was possible \citep{Reichenbach1926}

On \datef{20}{3}{1926} Einstein replied that he agreed with Reichenbach's \scare{$\Gamma$-Kritik}: \qt{I have absolutely lost hope of going any further using these formal ways}{Ich habe jetzt jede Hoffnung aufgegeben, auf diesem formalen Wege weiter zu kommen}; \q{without some real new thought} he continued, \qt{it simply does not work}{Ohne einen wirklich neuen Gedanken geht es nicht} \letterp{Einstein}{Reichenbach}{20}{3}{1926}][20-115][EA] Einstein's reaction reflects his disillusion with the attempts to achieve the sought-for unification of gravitational and electromagnetic field via some generalization of Riemannian geometry. He would have probably been less ready to embrace Reichenbach's critique if he had knew what the latter exactly had in mind, that is the requirement that \Gtmn had to be given a physical meaning in advanced. Reichenbach took the opportunity of Einstein's positive reation and on \datef{31}{3}{1926} that se sent him a note in which he has elaborated this criticism in details.

%He wrote to Einstein that \q{geometrical interpretation of electricity is only a visualization}, that does not have in itself any empirical content. \q{I have attached the note and would be grateful if you could give it a look}. Reichenbach attached in which he presented a sort of \uft.

The metric \gmn is measured not only by rigid rods but also by ideal clocks, which give physical meaning to the length of the four-dimensional vector $l$. A similar coordinative definition should be provided for the displacement \Gtmn. Since we have to maintain the direction of a four-dimensional vector, Reichenbach suggested that one can tentatively adopt the velocity four-vector $u^\tau$ of charged mass points as the physical realization of the displacement \citep[\todo{page}]{Eddington1923}. The vector with  physical meaning that vector is the four-velocity $u^\tau=d\xn/d\ap$. By parallel-displacing a vector $u^\tau$ indicating the direction of a curve $\xn(\ap)$ at any of its points, one can define a special class of curves, the straightest lines.  When the particle is not accelerating (that is, it moves inertially), the direction of the velocity vector does not vary. Thus, the motion of force-free particles can be used to define physically the straightest line between two \spti points. The motion of charged particles under the influence of electromagnetic field deviate from the geodesics. In \gr the motion of particles under the combined action of the gravitational and electromagnetic field is the following force equation:

\begin{equation*}\label{eq:forceequation}
\frac{d {u^\tau}}{d\ap} - \notateol{\Gtmn}{2}{\texts{Levi-Civita connection}} u^{\mu} u^\nu = \notateur{\frac{\rho}{\mu} f_{\tau}^{\mu} u^\tau}{2}{\texts{force term}}
\end{equation*}
%
Since the gravitational charge is the same to all particles, the mass coupling factor can be eliminated from the geodesic equation\todo{check}. Planet describes a trajectory which itself is assigned because it is attracted by the force of the sun, but we say: the planet moves along the straightest line defined by the \Gtmn. The force term indicates that the force experienced by charged particles is directly proportional to the charge and inversely proportional to the mass. Thus gravitation appears to be geometrized, but not electromagnetism does not. In order to geometrize the latter as well to absorbed the force term into the definition of the affine connection. To this purpose Reichenbach introduced a non-symmetric affine connection \Gtmnbar in which however the lengths of vectors is maintained. Indeed, the velocity vector of particles is per definition always equal $1$ and only the direction indicate the change of velocity. Every is the sum of and a non-symmetric tensor. With some change in notation, the definition looks as follow:

  
\begin{equation*}\label{eq:reichenbachconnection}
\Gtmnbar = \notateor{\Gtmn}{2}{\texts{Levi-Civita connection}} + \notateul{\frac{\rho}{\mu} f_{\tau}^{\mu} u^\tau}{2}{\texts{skew-symmetric tensor}}
\end{equation*}

In this form one can see by simple inspection the force term in the definition of the of the affine connection uncharged particles travel and charged particles on the shortest line, that a different affine connection for every particle. By substuting \Gtmnbar into \cref{forceequation} one eliminate the force term equations and turn the force equation into geodesic equations. The stragirest of this continuum do not generally coincide with its geodesics, that is with the shortest lines. In the presence of charge, from every point, in every direction, an autoparallel and a geodesic emerge, which generally diverge. This divergence is an index of the presence of charge..

Einstein was not impressed. The definition of the tensorial part of the connection tensor was clearly arbitrary. Most of all, Reichenbach's equations of motion can be valid only for a certain charge-density-to-mass-density ratio $\rho/\mu$\footnote{In a given displacement, there is only one straightest line passing through a point in a given direction, but different test particles with different charge-to-mass ratios accelerate differently in the same electric field. Thus they cannot all travel on the same straightest line of the same connection. Every particle would travel have with different curvature. This clearly make the theory anodyne}. However, Reichenbach rushed to point out Einstein had misunderstood the spirit of the typescript. As Reichenbach explained He was working on a philosophical presentation of the problem of space. \qt{Thereby I wondered what the geometrical presentation of electricity actually means}{Dabei überlegte ich mir, was die geometrische Darstellung der Elektrizität eigentlich bedeutet}[\letter{Reichenbach}{Einstein}{4}{4}{1926}][20-086][EA]. 

After Weyl also Eddington and Einstein had attempted to geoemtrized the electromagnetic field. In all \uft has now implemented Weyl's second strategy. One starts to find a suitable geometrical structure the \gmn and $\phi_\mu$, as in Weyl geometry, symmetric \Gtmn in Eddington's theory\etc, non-symmetric \gmn and \Gtmn in Einstein's theory\etc. Those variables have however no physical meaning and there is no attempt to provide a physical interpretation. By borrowing an expression of Eddington graphical only \scare{graphical representations} (\german{graphische Darstellungen})---an expression he evidently borrowed from \citet[294ff.]{Eddington1925a}. \cop{The theory helps just like a graph provide useful economical device to organize physical already known knowledge}, however non of those theories has brought anything new Reichenbach adopted a different strategy, by attempting to get back to Weyl's original approach. The theory starts with a non-symmetric \Gtmn. It has \qt{the advantage over other geometrical representations in that \emph{the operation of displacement possesses a physical realization \textins{Realisierung}}}{Diese Umschreibung hat aber vor anderen geometrischen Darstellungen den Vorteil, daß sie für die Verschiebungsoperation eine physikalische Realisierung besitzt}[\letter{Reichenbach}{Einstein}{4}{4}{1926}][20-086\me][EA], namely, the velocity-vector of charged mass particles. In Eddington's parlance is a \scare{natural geometry}, that is it is a proper geometrical interpretation interpretation of the electromagnetic field.  However, also this strategy does no bring anything new, physically. 

Thus Reichenbach was able to provide a proper geometrical interpretation of the electromagnetic field in itself does not lead to any new physical results. Thus geometrization is not itself a However, Reichenbach's theory and nevertheless did not bring any new result. Reichenbach's philosophical point clearly resonated with Einstein:

\qt{You are completely right. It is incorrect to believe that \scare{geometrization} means something essential. It is instead a mnemonic device \origins{Eselsbrücke} to find numerical laws. If one combines geometrical representations \origins{Vorstellungen} with a theory, it is an inessential, private issue. What is essential in Weyl is that he subjected the formulas, beyond the invariance with respect to \textins{coordinate} transformation, to a new condition (\scare{gauge invariance})\footnote{That is, invariance by the substitution of $g_{ik}$ with $\lambda g_{ik}$ where $\lambda$ is an arbitrary smooth function of position \citep[c\f][468]{Weyl1918}. Weyl introduced the expression \scare{gauge invariance} (\german{Eichinvarianz}) in \cite[114]{Weyl1919a}}. However, this advantage is neutralized again, since one has to go to equations of the 4. order%
%
\footnote{Cf.~\cite[477]{Weyl1918}. Einstein regarded this as one of the major shortcomings of Weyl's theory; see \letter{Einstein}{Besso}{20}{8}{1918}[\VD{8b}{604}][CPAE], \letter{Einstein}{Hilbert}{9}{6}{1919}[\VD{9}{58}][CPAE]},%
%
which means a significant increase of arbitrariness}{Lieber Herr Reichenbach\\ Sie haben vollständig recht. Es ist verkehrt zu glauben, dass die \scare{Geometrisierung} etwas Wesentliches bedeutet Es ist mir eine Art Eselsbrücke zur Auffindung numerischer Gesetze. Ob man dann mit einer Theorie \scare{geometrische} Vorstellungen verbindet, ist unwesentliche Privatsache. Das Wesentliche bei Weyl liegt darin, dass er die Formeln neben der Invarianz bezüglich Transformationen einer neuen Bedingung (\scare{Eichinvarianz}) unterwirft. Dieser Vorteil wird aber wieder neutralisiert dadurch, dass man zu Gleichungen 4. Ordnung übergehen muss, was ein beträchtliches Wachsen der Willkür bedeutet.\\Mit besten Grüssen \\Ihr A. Einstein}[\letter{Reichenbach}{Einstein}{8}{4}{1926}][20-117][EA]

Einstein not only endorsed Reichenbach's claim that a \scare{geometrization} is not an essential achievement of general relativity, but also questioned the meaning of the notion of \scare{geometrization}, and for that matter the very notion of \scare{geometry} \citep{Lehmkuhl2014}. The declare that the difference geometry and rest of mathematics was irrelevant. Reichenbach that difference between geometry was essential. As we shall see, Einstein's argument was an argument in favor of the \uftp; on the contrary Reichenbach had the very opposite goal to attack the program. Reichenbach offered to send Einstein the corresponding epistemological sections of the text on which he was working (possibly \S50 of the \Ap). There Reichenbach that the problem, \todo{check final}. It is interesting to notice have fully perceived the of the \uftp. The manurscript will become part of a longer appendix. That the message not absoerd into geoemtry, but lowered geometry into the to physocs. In this way, hoed to save \q{the sirens' enchantment \origins{Sirenenzauber} of a unified field theory} \citep[373]{Reichenbach1928}.


%This was a battle on what was \gr and what was continue the key insight of general relativity. The argument was mean to counter who declared the \uftp as mislieading without the equivalence that garaneed between a connection with gravitationa field and \rac lighr rays, that is geometrical measuring instruments. 

%and the theory did not turn physics into geometry; Einstien to show that difference between geomery and rest of mathematics; the unificaiton program

%Objections against the very project of geometrization were raised my some at that time. Pauli e.g. has clearly a similar point to Reichenbach only graphical representations, however there was no motivation for geometrizing the electromangetic field wihout the equivalence principle. Einstein, in his willingels, that geometrization was not point. As we shall see for Einstein the point was the unification of the two fields. 


%\q{The general theory of relativity by no means turns physics into mathematics. Quite the opposite: it brings about the recognition of a physical problem of geometry} \q{development of the theory of relativity into a world geometry}. 

%The abstract of this presentation was published under the title \citetitle{Reichenbach1926d} \citep{Reichenbach1926d}. Thus, he concluded, providing a geometrical interpretation of a physical field is not in itself a physical achievement: \q{the geometrical interpretation is only a different parlance, which does not entail anything new physically}. \citep[25; my emphasis]{Reichenbach1926d}. As one can infer that at this point the pages were already in of developmen. The reason for the success is that theory had indeed effects on gravitational field, not that has been reduced to geometry. That the was uneceasry, there was not reason to pursue the proect. .That difference between geometry and physics, that on the very oppsite, the gola that there was non difference between geometry and the rest mathematics. Thus, could be further pursed, of unification and not of geometrya,







%The pages of this, that geometrization was but has an effect on geometry geometry is like cloak. The main goal. Geometry was different from physics, and maintial the difference. Indeed, that has effect on geometry, but it is not geometry. Einstein wanted to that geometry was not different from the rest of mathematics. The goal of introducing was to unify the two fields, into a single Lagrangian. Neither the of these variables was not essential, and only at the end after having integrated the field equations.

%eses, and since the theory of relativity has revealed the physical character of geometry as a science of real space, we can no longer doubt that these, too, are basically divisions of physics and that only the division of labor justifies the existing separation of specialties. We may, then, say that there are only


%Mathematics is the intellectual tool of physics; it teaches what is permissible and what is forbidden, but never what is physically co"ect.


%
%
%
%Reichenbach revealed that what he wanted to achieve was a geometrical interpretation of a physical field \scare{in \myemph{the same sense as gravitation}} in Einstein's theory, i.e., one that was \emph{just as good} as that attained by general relativity. The geometrical operation of displacement has a physical interpretation in Reichenbach's theory, just like the $ds$ does in general relativity Thus, Reichenbach claims to have provided not just a successful \scare{geometrical interpretation} of the electromagnetic field, but an interpretation that was of the same \scare{quality} as the one \gr provided for the gravitational field. However, this was Reichenbach's point: the theory was not a successful physical theory like general relativity. Thus, he concluded, providing a geometrical interpretation of a physical field is not in itself a physical achievement: \q{the geometrical interpretation is only a different parlance, which does not entail anything new physically}. [][25; my emphasis][Reichenbach1926d]. 



%https://arxiv.org/pdf/1802.00492.pdf
%http://www.physics.ntua.gr/ModifiedGravity2018/Talks/Iosifidis.pdf
%http://www.weylmann.com/weyltheory.pdf
%https://inis.iaea.org/collection/NCLCollectionStore/_Public/18/010/18010695.pdf?r=1&r=1


\section{Unification: Reichenbach and Einstein in Berlin}
\label{unification}
On \datef{26}{5}{1926} Reichenbach presented the note in Stuttgart at the regional meeting of the German physical sosciety. In August 1926, Reichenbach was granted teaching privileges as an \q{unofficial associate professor} (\german{nichtbeamteter au\ss{}erordentlicher Professor}) at the University of Berlin \citep{Hecht1982}. The discussion seminar that he started to hold in October became soon the basis of the so-called \scare{Berlin group}, which, together with Schlick's cognate \scare{Vienna circle} has marked the history of 20th century \scare{scientific philosophy} \citepp{Danneberg1994}{Milkov2013}. By the end of the year, Reichenbach wrote to Schlick, keeping him up to date with the progress of a two-volume book he was writing, which was supposed to bear the title \bt{Philosophie der exakten Naturerkenntnis}. \q{The first volume that deals with space and time,} he wrote, \qt{is finished}{der erste Band der Raum und Zeit behandelt, ist fertig} \letterp{Reichenbach}{Schlick}{6}{12}{1926}[][SN]\label{RZL1926}. Reichenbach hoped to publish the book in the forthcoming Springer series \scare{Schriften zur wissenschaftlichen Weltauffassung} directed by Schlick and Philipp Frank. However, Springer rejected the book as being too long. According to \Reich's later recollections, the manuscript of the first volume was not changed significantly after February 1927\hide{Das MS war seit Febr. 1927 nicht mehr nennenswert geändert worden}. By July Reichenbach could announce to Schlick that he had reached a publication arrangement \letterp{Reichenbach}{Schlick}{2}{7}{1927}[][SN]. The publisher agreed to publish only the first volume under the title \citetitle{Reichenbach1928}. The drafts were finished in September and the preface was dated October 1927. 

If one reads that that reduciton to geoemtry, but on the very opposite. The was not absored binto. The exaple of precisely meant to be shown that was like the 

So far, we have always that a simple physical realization must be given for the process of displacement. In our example we ourselves gave such a realization and we obtained, in this way, an actual geometrical interpretation of electricity. Attempts which were made by Weyl, Eddington and Einstein, on the other hand, renounced such a realization of the process of displacement. It is generally believed that such \scare{tangible} realizations does not lead to the desired field equations. 




%that is gravitationa as efect on geometrical measureing insturmes \rac, light rays\etc, that could be isolated. On the opposite, to use geometrica withou, mathematical simplicity, itsefl Reichenbach hoped that his critical epistemological reflections could have served, so to speak, to tie physicists to the the mast of empiricism, so that they could resist to \q{the sirens' enchantment \origins{Sirenenzauber} of a unified field theory} \citep[373]{Reichenbach1928}.


%Some weeks later, in \datem{01}{12}{1927}, Reichenbach wrote to Einstein that Paul Hinneberg, the editor of the \citejournal{Einstein1928d} had told him that Einstein intended to write a review of his forthcoming book, \citetitle{Reichenbach1928}. Reichenbach sent him the galley proofs and also added that he would send an \Ap in the coming days \letteraeap{Einstein}{Reichenbach}{1}{12}{1927}[20-090]\todo{abs 295?}. Einstein's review appeared in the first 1928 issue of the \citejournal{Einstein1928d} \citep{Einstein1928d}. 


Einstein read the manuscript of the \PRZL while on his way to Brussels to attend the fifth Solvay Congress  \citep{Bacciagaluppi2009} \lettercpaep{Einstein}{Elsa Einstein}{23}{10}{1927}[16][34], and wrote a review. The review was little more than a short summary. However, Einstein emphasized two philosophically significant issues, both concerning the \Ap of the book, that is precisely that we have so far. \begin{inparaenum}[(1)] \item \qt{In the \Ap, the foundation of the Weyl-Eddington theory is treated in a clear way and in particular the delicate question of the \myemph{coordination} of these theories to reality}{In einem Anhang wird dann noch die Grundlage der Weyl-Eddingtonschen Theorien in klarer Weise dargestellt und insbesondere die heikle Frage der Zuordnung dieser Theorien zu der Wirklichkeit behandelt} \citep[20\me]{Einstein1928d}. Reichenbach had claimed that, as in any other  theory, also in \uft, one should give physical meaning to the variables used (\gmn, \Gtmn\etc.) from the outset, before starting to search for the field equations. Einstein did not comment further on this issue, probably because, over the years, he had come to realize that this requirement was too strict. However, Einstein was in full agreement with the second point made by Reichenbach: \item In the \Ap, \qt{in my opinion quite rightly---it is argued that the claim that general relativity is an attempt to \emph{reduce physics to geometry} is unfounded}{In diesem Kap., sowie in den vorangehende wird~--~meiner Ansicht nach mit vollem Recht~--~die Haltlosigkeit der These behauptet, nach welcher die Relativitätstheorie ein Versuch sei, die Physik auf Geometrie zurückzuführen} \citep[20\me]{Einstein1928d}. \end{inparaenum} As we have mentioned, Reichenbach and Einstein had already discussed this topic in a private correspondence less than two years earlier \anonymize{\citep{Giovanelli2016d}}. For this reason, Einstein immediately perceived the importance of this theme in Reichenbach's book, a theme that later readers often overlooked \anonymize{\citep[]{Giovanelli2020}}. 


Simultaneously with Reichenbach's review, after nearly a year-long correspondence, at the end of 1927, Einstein \letteraeap{Einstein}{Meyerson}{24}{12}{1927}[18-294] gave final authorization for the publication of another, more extensive review of \citetitle{Meyerson1925} written by the French philosopher \'Emile Meyerson \citep{Meyerson1925}. The review was published in Spring 1928 in French \citep{Einstein1928b}. As we have seen, Einstein's disagreed with first issue, indeed that a physical meaning. (1) Geoemtry and physics as a whole, the Zuordning, and ineed it will be ultimately the mathematical simplicity that would to right field equations Without the equivalence principle, mathematical simplicity of the thoery as whole become the only guide in the quest for the fundamental field structure. For this reason, however, Einstein very much appreciated Meyerson's insistence on \scare{the deductive-constructive character} of relativity theory (on this episode see \anonymize{\cite{Giovanelli2018a}}). The geoemtrical nature of those concepts was not signifciatn. However, in the context of an otherwise laudatory review, Einstein strongly disagreed. (2) According to Einstein, \qt{the term \scare{geometrical} used in this context is entirely \myemph{devoid of meaning}}{le terme \scare{g\`eom\`etrique} employ\`e dans cet ordre d'id\`ees est enti\`erement vide de sens} \citep[165\me]{Einstein1928b}. The goal of the \uftp was not to \scare{geometrize} both fields but to \scare{unify} them, to show that they are nothing but two aspects of a unique \scare{total} field of unknown structure. This was then sense of Einstein's argument, that geoemtry was inessetial for this project; objections that geometrization has become, could retort that geoemtrization have never been the point.


% Historical reasons aside, there was no real ground to define \gmn, the gravitational field, as a geometrical field, and, say, the \Fmn, the electromagnetic field, as a non-geometrical field. 

This will soon become evident. In spring 1928, during a period of rest after a circulatory collapse, Einstein, as he wrote to Besso, \qt{laid a wonderful egg in the area of general relativity}{ein wundervolles Ei gelegt auf dem Gebiete der allgemeinen Relativität}[][40-69][EA]. On \datef{7}{6}{1928} he presented a note to the Prussian Academy on a \scare{Riemannian Geometry, Maintaining the Concept of Distant Parallelism} \citep{Einstein19281}, a flat space-time that is nonetheless non-Euclidean since the connection is non-symmetrical. In a note presented at the Academy on \datedm{7}{6}{1928} \citep{Einstein19281}, he introduced a new formalism, based on the concept of $n$-\german{Bein} (or $n$-legs), $n$ unit orthogonal vectors representing a local coordinate system attached at a point of $n$ dimensional continuum. Two vectors $A$ and $B$ at distant points can be considered as equal and parallel if they have the same local coordinates with respect to their \nbein. The metric and the affine connection would be written in terms of the \nbein. Because the \nbein determines the metric, but not the other way around, it provides more degrees of freedom---16 components of the \vbein compared to the 10 of the metric. Einstein expected that the former could be exploited to incorporate the electromagnetic field alongside the gravitational field. Thus, the \vbein-field \hbein defines both the metric tensor $g_{\mu \nu}$ and the electromagnetic four-potential $\fm$. Its sixteen components can be considered as the fundamental dynamical variables of the theory. The question arises as to the field equations that determine the \vbein-field. On \datef{14}{6}{1928} he submitted a second paper in which the field equations are derived from a variational principle \citep{Einstein19282}.

Reichenbach again managed to read also this paper and wrote to Einstein with some comments on the theory on \datef{17}{10}{1928}:
 
\qt{Dear Herr Einstein,\\ I did some serious thinking on your work on the field theory and I found that the geometrical construction can be presented better in a different form. I send you the ms. enclosed. Concerning the physical application of your work, frankly speaking, it did not convince me much. \myemph{If geometrical interpretation must be, then I found my approach simply more beautiful, in which the straightest line at least means something.} Or do you have further expectations for your new work?}{Lieber Herr Einstein, \\ Ich habe mir Ihre neuen Arbeiten zur Feldtheorie durch Kopf gehen lassen und gefunden, dass man die geometrische Aufbau besser in anderer Form darstelle kann. Ich schicke Ihnen einliegend des Ms. Was die physikalische Anwendung betrifft, so hat mich Ihre Arbeit, offen gesagt, wenig überzeugt. \myemph{Wenn es nun einmal geometrische Deutung sein muss, so finde ich schlechthin meinen Ansatz schöner, bei dem die geradesten Linien wenigstens etwas bedeuten}. Oder sollten Sie doch noch Aussichten in Ihren neuen Ansatz sehen?}[\letter{Reichenbach}{Einstein}{17}{10}{1928}][20-92\me][EA]

There are two aspects of this passage that only apparently unrelated. In the manuscript to which Reichenbach refers has been preserved, however we cannot of the manuscript here:

\begin{itemize}
\item Reichenbach in fact defines a metrical space by imposing the condition $d(l^2)=0$ to the displacement space \Gtmn, which in general is non-symmetrical; he then obtains Einstein space by requiring that the Riemann tensor $R^\tau_{\mu\nu\sigma}(\Gamma)$ vanishes\footnotep{The $\Gamma$ alludes to the fact that this condition can be defined without reference to the $\gmn$}. One starts from a general affine connection had can introduced two different specializaiton. One can impose the is falt, or impose and rich Euclidean geometry. 

\item This, however, was only a minor point. He could extened the same critigue, that he had raised previous theories Reichenbach's further remark concerning the physical application of Einstein's geometrical setting is, from a philosophical standpoint, more interesting, even if Einstein did not comment on it, that is in Reichenbach's theory the straight liens have no physical meaing, Reichenbach claims that, if one really wants to provide a geometrical interpretation of gravitation and electricity, then his own approach was better after all.  While had used a affine connection, an non-flat, that Einstein did not have provide a coordinate definition, that was without any meaning.  
\end{itemize}


% However, from Einstein's reply on \datef{19}{10}{1928} one can easily infer that Reichenbach must have sent him the classification of geometries which would appear in an article Reichenbach submitted in \datemy{1}{2}{1929} \citep[][see below in this section]{Reichenbach1929a}. Einstein agreed that in principle it was possible to proceed as Reichenbach suggested, \qt{starting with displacement law, and to specialize it on the one hand with the introduction of a metric on the other side with the introduction of integrability properties}{Von einem Verschiebungsgesetzt ausgehen und einerseits durch Einführung einer Metrik anderseits durch Einführung Integrabilitätseigenschaften spezializieren}[\letter{Einstein}{Reichenbach}{19}{10}{28}][20-094][EA].



%Reichenbach uses his own toy-theory as a benchmark for a good \scare{geometrical interpretation} (but of course not for a good physical theory). Reichenbach's theory provides a physical meaning to the displacement operation and thus a physical definition of a straightest line. 

%On the contrary, Einstein's theory did not attempt to provide a physical interpretation of the notion of displacement, nor even the field quantities; if the theory has nothing more to offer, Reichenbach claims, (i.e., if the theory does not solve the problem of the electron) it is merely a \scare{graphical representation} (cf.\ also \cite{Eddington1929} for a similar judgment). 

In the subsequent letter, Einstein defended his classification of geometries, but did not comment of Reichenbach's objections. In a note added by hand at the bottom of the typewritten letter, Einstein invited Reichenbach and his first wife Elisabeth for a cup of tea on \datef{5}{11}{1928}, mentioning that Erwin Schrödinger\footnote{Schrödinger succeeded Max Planck at the Friedrich Wilhelm University in Berlin in 1927. He held his inaugural lecture on \datef{4}{7}{1929} \citep{Schroedinger1929a}} would also be present (\letter{Reichenbach}{Einstein}{17}{10}{1928}[20-92][EA]). Einstein might have with some techinical details. However, that a phylosophical disctussion mtigh have ensued. Reichenbach was very different that the one he had imagined. A few weeks after he wrote to Reichenbach, Einstein had submitted \german{Festschrift} on the occasion of the seventieth birthday of Aurel Stodola\label{stodola}, Professor of Mechanical Engineering at the ETH (\lettercpae{Honegger}{Einstein}{02}{11}{1928}[16][abs.\ 732]; \lettercpae{Einstein}{Honegger}{14}{11}{1928}[16][abs.\ 750]; cf.\ \cite{Einstein1929d}), in which Einstein's philosophical point more very velear




%It was probably on that occasion that Einstein told Reichenbach about the physical consequences of the theory he was working on. Einstein might have explained to Reichenbach that the theory did not simply, that the problem was indeed not geoemtrization; thus a proper in term of natural geometry, was not essential for Einstein; once one choces, the \vbein as fundamental variables, to find the field equations do not need to have any physical meanign, was the unification of the two fields that the theor between.


%Einstein agreed to contribute with a semi-popular review article on his new theory, \citetp{Einstein1929}. The manuscript was submitted on \datedm{10}{12}{1928} \citep[see][]{Sauer2006}. Einstein's philosophical stance took a turn that Reichenbach probably did not predict. 


Einstein insisted on the speculative nature of the new theory, which, however, he presented as a continuation of the same strategy that was successful in his search for the field theory of gravitation: individuate a suitable field structure, the \gmn, and search for simplest differential generally covariant equations that can be obeyed by the \gmn. For \gr, the choice of the \gmn was suggested by a physical fact, the equivalence principle. However, in the search for a more general mathematical structure that would include the electromagnetic field, Einstein continued, \qt{the experience does not give---so it seems---any starting point}{für die L?sung dieses Problem gibt uns die Erfahrung---wie es scheint---keinen Anhaitapunkt} \citep[128]{Einstein1929}.  Thus, the only hope is to develop a theory \qt{in a speculative way}{auf spekulativem Wege gewonnenen Theorie} \citep[128]{Einstein1929}. To solve this problem, the physicist must venture along \qt{a purely intellectual path}{auf rein gedankHchem Wege} having as only motivation the deep conviction of the \qt{formal simplicity of the structure of reality}{berzeugung der formalen Einfachheit der Struktur der Wirklichkeit} \citep[127]{Einstein1929}. The belief in the fundamental simplicity of the real is \qt{so to speak, the religious basis of the scientific endeavor}{sozusagen die religiöse Basis des wissenschaftHchen Bemühens} \citep[127]{Einstein1929}. 

Indeed, for \FP, no attempt was made to give a direct physical meaning to the fundamental field variables \hbein. One starts from this mathematical structure and then searches for the simplest and most natural field equations that the \vbein-field can satisfy \citep[131]{Einstein1929}. The physical soundness of the field equations thus found can be confirmed only by integrating them, which was usually a very difficult task. Einstein warned his readers of the dangers of proceeding \q{along this speculative road} \citep[127]{Einstein1929}. In a footnote, Einstein even endorsed \qt{Meyerson's comparison with Hegel's program \origins{Zielsetzung}}, which \qt{illuminates clearly the danger that one here has to fear}{Meyersons Vergleich mit Hegels Zielsetzung hat sicher eine gewisse Berechtigung; er beleuchtet hell, die hier zu fürchtende Gefahr} \citep[127]{Einstein1929}.

\subsection{Reichenbach's Articles}

%In the meantime, on \datef{4}{11}{1928}, an article by Paul Miller appeared in \jt{The New York Times} with the sensational title \enquote{Einstein on Verge of Great Discovery; Resents Intrusion}. %Reichenbach conceded that Einstein's theory provided a unification of gravitation and electricity which had more than just formal significance, since it made \q{new assertions concerning the relation between gravitation and electricity in relatively complicated fields}[][][Reichenbach1929c]. However, he maintained his skepticism by claiming that the theory was \q{only a first draft, lacking the persuasive powers of the original relativity theory because of the \myemph{very formal method by which it is} established}[][\me][Reichenbach1929c]. 

At about the same time, Einstein's theory had started attracting irrational attention in the daily press. In the late 1920s Reichenbach was a regular contributor to the \jt{Vossische Zeitung}, at that time Germany's most prestigious newspaper; not surprisingly he was asked for a comment on Einstein's theory. With the advantage of having personally discussed the topic with Einstein, Reichenbach published a brief didactic paper on Einstein's theory on \datef{25}{1}{1929} \citep{Reichenbach1929c}.   Reichenbach reported that the novelty of \FP consisted in the fact that it no longer seeks to establish a formal synthesis between already established theories; instead, it produces new laws, of which gravitational and electromagnetic field equations are only a first approximation\footnote{It might be indeed argued that this is true for previous theories. However, Reichenbach seems to share \citet[84]{Eddington1923}'s analysis that most of those theories were primarily \scare{graphical representations} of the relations between certain quantities \citep[\S15 and \S50]{Reichenbach1928a}. \citet[281]{Eddington1929} considered Einstein's \FP-field theory as a mere graphical representation: the graph of a moving particle with time and space as coordinates is no better than one using velocity and curvature as coordinates. However, Reichenbach seems to considered it as the a proper non-geometrical unification. See also \cite[121f.]{Goldstein2003}}. For strong fields, there would be a much closer interdependence between electromagnetism and gravitation. In principle, the theory could receive experimental proof if the effects predicted did not remain beyond the threshold of experimental detection. However, the problem of the constitution of matter or the quantum problem were far from being satisfactorly addressed. Thus, Reichenbach concluded that \q{for the time being, no pronouncement can be made concerning the physical significance of the theory} \vza{262}. 

However, this in personal relationship. However, was that philosophical outset has become progressively more complicated. The first article of the order of publication was entitled \citetitle{Reichenbach1929b} \citep{Reichenbach1929b} and would appear in February in the \citejournal{Reichenbach1929b}. The second article was an extended version of the manuscript that Reichenbach had sent to Einstein in October and bore the same title \citetitle{Reichenbach1929a} \citep{Reichenbach1929a}. It was published only in September in the \citejournal{Reichenbach1929a}. These articles represent Reichenbach's last important contribution to issues related to \rt and \spti theories. On the one hand, Reichenbach attempted to make his previous reflections about the \uftp in the \Ap to the \PRZL to bear fruit \citep[\S46]{Reichenbach1928}. On the other hand, he added new elements of clarification by clearly distinguishing the \scare{geometrization program} and the \scare{unification program}.

In the first paper for the \citejournal{Reichenbach1929b}, Reichenbach introduced the history of the \uft in an entirely different manner than before. The brief history of the \uftp appeared to him as the progressive \emph{downfall} of the geometrization program and the concurrent \emph{rise} of the unification one.  In this manner, however, Reichenbach concluded, the \scare{geometrization program} was implicitly abandoned and substituted by a new, different \scare{unification program}. Most physicists, including Einstein \citeyearp{Einstein1923d,Einstein1925a} considered this strategy legitimate. It was preferable to sacrifice the geometrical interpretation---i.e., to relinquish the coordination of geometrical notion of parallel transport of vectors with the behavior \rac---and then to use the geometrical variables (\Gtmn, $\varphi_\nu$ and so on) as \scare{calculation device} for the greater good of finding the field equations. From the field variables, one has to attempt to establish the simplest differential invariants that can be used as an action function. The test of the theory can happen only in hindsight, by finding the solutions and equations of motions corresponding to elementary particles.

Reichenbach will return to this different programs, in the following article a proper geometrizaiton withut unificaiton. These theories, provide a but this does to any phsical reaslt. The very difference between and geometrization program. The first approach was the one used by Reichenbach himself in his own \scare{unified field theory}:

\qt{The author \textins{Reichenbach} has shown that the first way can be realized in the sense of a combination of gravitation and electricity to one field, which determines the geometry of an extended Riemannian space; it is remarkable that thereby \myemph{the operation of displacement receives an immediate geometrical interpretation, via the law of motion of electrically charged mass-points}. The straightest line is identified with the path of electrically charged mass-points, whereas the shortest line remains that of uncharged mass points. In this way one achieves \myemph{a certain parallelism to Einstein's equivalence principle}. By the way [the theory introduces] a space which is
cognate to the one used by Einstein, i.e., a metrical space with non-symmetrical \Gtmn. The aim was to show that the geometrical interpretation of electricity does not mean a physical value of knowledge per se}{Daß der erste Weg durchführbar ist im Sinne einer Zusammenfassung yon Gravitation und Elektrizität zu einem Feld, welches die Geometrie in einem erweiterten Riemannschen Raum bestimmt, ist vom Verfasser gezeigt worden; es ist bemerkenswert, daß dabei die Verschiebungsoperation eine unmittelbare geometrische Deutung finden kann, nämlich durch das Bewegungsgesetz elektrisch geladener Massenpunkte. Es wird dort die geradeste Linie mit der Bahn des elektrisch geladenen Massenpunkts identifiziert, während die kürzeste Linie die des ungeladenen Massenpunkts bleibt. Hierdurch wird eine gewisse ParallelRat zu dem Einsteinschen-Aquivalenzprinzip erreicht. ?brigens wird dort ein dem Einstein'schen Raum verwandter Raum, nämlich ein metrischer Raum mit unsymmetrischen \Gtmn zugrunde gelegt. Absicht nämlich, zu zeigen, daß geometrische Deutung der Elektrizität an sich noch keinen physikalischen Erkenntniswert bedeutet}[][688\me][Reichenbach1929a]

Notice that, according to Reichenbach, the advantage of his own approach consists in the fact that it provides a physical realization of the displacement operation.  The disadvantage is that it is only a \emph{unification of the representations} of two physical fields in a common geometrical setting, which as mearely an economical meaning. 

The second class of theoriesm in which there is unficaiton without geometrizatio, The second approach is the one used by Einstein, and it presented  the opposite characteristics:

\qt{On the contrary Einstein's approach of course uses the second way, since it is a matter of increasing physical knowledge; it is the goal of Einstein's new theory to find such a concatenation of gravitation and electricity, that only in first approximation it is split in the different equations of the present theory, while is in higher approximation reveals a reciprocal influence of both fields, which could possibly lead to the understanding of unsolved questions, like the quantum puzzle. However, it seems that this goal can be achieved only \myemph{if one dispences with an immediate interpretation of the displacement, and even of the field quantities themselves}. From a geometrical point of view this approach looks very unsatisfying. Its justification lies only on the fact that the above mentioned concatenation implies more physical facts that those that were needed to establish it }{Der Einsteinsche Ansatz benutzt dagegen natürlich den zweiten Weg, denn ihm ist es ja um Vermehrung des physikalischen Wissens zu tun; es ist als Ziel der neuen Theorie Einsteins, eine derartige Verkettung yon Gravitation und Elektrizität zu finden, daß sie nur in erster Näherung in die getrennten Gleichungen der bisherigen Theorie zerspaltet, während sie in höherer Näherung einen gegenseitigen Einfluss beider Felder lehrt, der möglicherweise zum Verständnis bisher ungelöster Fragen, wie der Quantenrätsel, führt. Aber dieses Ziel scheint nur erreichbar zu sein unter Verzieht auf eine unmittelbare physikalische Interpretation der Verschiebungsoperation, ja sogar der eigentlichen Feldgrössen selbst. Vom geometrischen Standpunkt als deshalb ein solcher Weg sehr unbefriedigend erscheinen; seine Rechtfertigung wird allein dadurch gegeben werden können, daß er durch die genannte Verkettung mehr physikalische Tatsachen umschließt, als zu seiner Aufstellung in ihn hineingelegt wurden}[][688\me][Reichenbach1929a]

Einstein's theory was claimed to be a \emph{unification of the dynamics} of two physical fields, i.e., a unification of the fundamental interactions. However, Reichenbach argues that Einstein could achieve this result only at the cost of dispensing with a physical interpretation of the fundamental quantities. 




However, in Reichenbach's reconstruction, after Weyl's failure of pursuing \cref{gs}, most physicists, and in particular Einstein, opted for \cref{us}. Einstein seemed to believe that \cref{us} could be justified based on a different ground, assuming that nature satisfies the simplest imaginable mathematical laws. This assumption was the new \emph{physical hypothesis} on which the strategy \cref{us} could be based \citep[see][\S50]{Reichenbach1928}. One searches for the most natural field structure, and the simplest field equations that such structure satisfies. After all, Einstein could claim, this is how physics has always been done: \ME are nothing but the simplest laws for antisymmetric tensor field \Fmn which is derived from a vector field; Einstein's equations were the simplest generally covariant laws that govern a Riemannian metric \gmn and so on. The only warranty of the success of this speculative groping in the chaos of mathematical possibilities was the unification power of the field equations obtained. The latter should have predicted some unknown coupling between the electromagnetic field and the gravitational field, which ultimately would have served as the basis of a theory of matter. This was indeed the case of the \FP-field theory. 


%Thus, according to Reichenbach, his own theory had the ambition of being a \scare{\emph{proper geometrical interpretation}} (or, one might say, to provide a \scare{natural geometry}), but it was physically sterile; Einstein's theory sought to be physically fruitful, but it was merely a \scare{\emph{graphical representation}} (see also \cite{Eddington1929}). Clearly, for Reichenbach, only general relativity was able to combine both virtues: it was a proper geometrical interpretation (the $ds$, and thus the \gmn are measured using rods and clocks) that leads to new physical results. Reichenbach did not seem to realize (or at least does not explicitly point out) that this epistemological standard had become hard to comply with in precisely the context of the field-theoretical explanation of the electron that he was calling for. 


To Reichenbach's dismay, Einstein had abandoned the \emph{physical heuristic} that leads him to \gr in the name of a \emph{mathematical heuristic} that was not different from Weyl's speculative approach that he had dismissed a decade earlier\footnote{See \lettercpae{Weyl}{Einstein}{18}{5}{23}[13][30] and \letter{Weyl}{Seelig}{19}{5}{1952}, cit.\ in \cite[274f.]{Seelig1960}}. Einstein's philosophical volte-face might have appeared to Reichenbach as a sort of \french{trahison des clercs}, an unacceptable intellectual compromise. \begin{inparaenum}[(a)] \item The core of Reichenbach's philosophy was the \emph{separation of mathematical necessity and physical reality}. Reichenbach had always perceived this separation as nothing more than a philosophical distillation of Einstein's scientific practice: \q{Mathematics teaches what is permissible and what is forbidden, but never what is physically correct}. Are necessary but are not.

\item In the search of a \uft, Einstein had come implicitly to question this distinction between geometry and mathematics, ultimately pleading for a \emph{reduction of physical reality to mathematical necessity}\footnote{Already in his habilitation, Reichenbach, although rather in passing, accused Weyl of attempting to deduce physics from geometry, by reducing physical reality to \scare{geometrical necessity} \citep[73]{Reichenbach1920a}. However, the greatest achievement of \gr, Reichenbach claimed, was to have shifted the question of the truth of geometry from mathematics to physics \citep[73]{Reichenbach1920a}. Einstein was now committing the very same \q{old mistake} again \citep[73]{Reichenbach1920a}. On Reichenbach's habilitation, see \citet{Padovani2009}} \end{inparaenum}. Einstein put it bluntly in his Stodola-\german{Festschrift}'s contribution---that he sent for publication toward the end of January \lettercpaeabsp{Einstein}{Honegger}{30}{1}{1929}[864]. The ultimate goal of understanding reality is achieved when one could prove that \qt{even God could not have established these connections otherwise than they actually are, just as little as it would have been in his power to make the number 4 a prime number}{selbst Gott jene Zusammenhänge nicht anders hätte festlegen können, als sie tatsächHch sind, ebensowenig, als es in seiner Macht gelegen wäre, die Zahl 4 zu einer Primzahl zu machen} \citep[127]{Einstein1929}. In this sense, Einstein's God indeed resembles Spinoza's God \citep{Einstein1929e}, for whom the laws of nature are necessary, and rather than, say, Leibniz's God for whom the laws of nature are contingent. 

\subsection{A Parting of The Ways. Positivists and Metaphysicians}
\label{positivistsmetaphysicians}

On \datef{30}{1}{1929}, Einstein's rumored new derivation of the \FP-field equations was published in the Proceedings of the Berlin Academy with the ambitious tile \citetitle{Einstein1929b} \citep{Einstein1929b}. Despite his anger toward Reichenbach's \scare{leaks}, Einstein did not hesitate to feed the hopes of the general public by popularizing his new theory in the daily press. On \datef{2}{2}{1929}, in its section \citet{Nature1929}, \emph{Nature} reported an interview of Einstein published in the \jt{Daily Chronicle}, on \datef{26}{1}{1929}, a day after the publication of Reichenbach's infamous article in the \VZ. Einstein's quarrel with Reichenbach had deeper philosophical roots that went way beyond questions of academic etiquette. A few days later, Einstein wrote a popular account of the new theory \citep{Einstein1929-02-03}. Its English translation was published on the first page of their Sunday supplement of the \jt{New York Times} on \datedm{3}{2}{1929} and in \jt{The Times} of London in two installments on \datedm{4}{2}{1929} and \dated{5}{2}{1929} \citepp{Einstein1929-2-3}{Einstein1929-2-4}[also published as][]{Einstein1930h}. 

%In the paper after providing an overview of his work on relativity, he described the method that guided him to his last field theory. 
Einstein insisted on \qt{the degree of formal speculation, the slender empirical basis, the boldness in theoretical construction, and finally the fundamental reliance on the uniformity of the secrets of natural law and their accessibility to the speculative intellect}{spekulativ-formalistische Zug, die Schmalheit der Erfahrungsbasis[,] die Kühnheit der theoretischen Konstruktion, das ihr zugrunde liegende Vertrauen in die Einheitlichkeit und die Durchdringbarkeit der Geheimnisse der Naturgesetzlichkeit durch die spekulative Vernunft} \citep[114]{Einstein1930h}. This \q{speculative method}, Einstein claimed, was the same that lead to to success of \gr: \qt{Which are the simplest formal structures that can be attributed to a four-dimensional continuum, and which are the simplest laws that may be conceived to govern these structures?}{Welches sind die einfachsten und natürlichsten Bedingungen, welchen ein Kontinuum der skizzierten Art unterworfen werden kann? Die Beantwortung dieser Frage, welche ich in einer neuen Arbeit [7] versucht habe, liefert einheitliche Feldgesetze für Gravitation und Elektromagnetismus} \citep[115]{Einstein1930h}. In trying to defend this epistemological stance, Einstein was not afraid to side with \qt{Meyerson in his brilliant studies on the theory of knowledge}, who had emphasized the \scare{Hegelian} nature of such enterprise, \qt{without thereby implying the censure which a physicist would read into this}{geistreiche Erkenntnisstheoretiker Meyerson die geistige Einstellung der Relativitats-Theoretiker mit derjenigen Descartes und sogar Hegels verglichen ohne indes mit jenem Vergleich jenen Tadel zu verbinden, den das Ohr eines Physikers haturgemass heraushoren wird} \citep[115]{Einstein1930h}. 

The fact the Einstein chose to mention Meyerson rather than Reichenbach as a philosophical reference in a popular presentation of his last theory for a major newspaper cannot be underestimated. Of course, Einstein was well aware of Reichenbach's technically informed work on this very subject, having discussed it with him in the previous months. Nevertheless, as he did in the contribution for the Stodola-\german{Festschrift} (see above \cref{stodola}), Einstein preferred to side with Meyerson's less detailed, but, in his view, a more profound philosophical outlook---endorsing even his somewhat outrageous comparison with Hegel \citep{Giovanelli2018a}. After a decade of personal friendship and intellectual exchange that had shaped the history of 20th-century philosophy of science and, to a certain extent of 20th-century physics, a minor squabble had unwittingly revealed a nearly unbridgeable philosophical divide. Einstein seems to have put into question the very core of philosophical their alliance. Reichenbach was to avoid to reduce physics to geometry. For Einstein was to geometry was different from the rest of mathematics. It is worth noticing that this devide was part of a larger cultural devide.

After 1919 Einstein benefited from a universal acclaim among the general public; however his positions among the physics community became progressively more isolated. Till 1925-1926 the \uftp was pursued by scholars of the stature by Weyl and Eddington, but was also regarded as a viable options by leading quantum theoreticians \citep[209]{Vizgin1994}. Even, after the 1925--1927 rapid advance in \qm, were made of relate unified theories to quantum theory \citep{Klein1926a}. However, most leading physicists  soon started to perceive the program as obsolete. Einstein was fully aware of the marginality of his position, but, throughout 1929, continued express his confidence in \FP program. In the second paper of this year finished in \datem{19}{8}{1929}---the fourth in the series in the Berlin Academy---which reflects the priority dispute with Élie Cartan \citep{Debever1979}, Einstein returned to the Hamiltonian principle after objections raised by his collaborators Lanczos and Müntz \citep{Einstein1930c}.  In spite of the many doubts, Einstein was finally convinced that he had \q{found the simplest legitimate characterization of a Riemannian metric with distant parallelism that can occur in physics} \letterp{Einstein}{Cartan}{25}{8}{1929}[\D{V}][Debever1979].

However, like Reichenbach, fellow physicists were not impressed, in particular given the growing success of \qm-program. Weyl, who had always been scolded by Einstein for his speculative style of doing physics could relaunch the accusation in a paper \citep{Weyl1929c} in which he had uncovered the gauge symmetry of the Dirac theory of the electron \citepp{Dirac1928}{Dirac1928b}. \q{The hour of your revenge has come}, Pauli wrote to Weyl in August: \qt{Einstein has dropped the ball of distant parallelism, which is also pure mathematics and has nothing to do with physics and \emph{you} can scold him}{jetzt hat Einstein den Bock des Fernparallelismus geschossenf , der auch nur reine Mathematik ist und nichts mit Physik zu tun hat, und Sie konnen schimpfen} \letterpaulip{Pauli}{Weyl}{26}{8}{1929}[235].  \cop{Although Einstein's papers had been discussed widely especially among mathematicians, Einstein was aware of the poor reception that his work had especially among the colleagues that he probably felt has his peers} \citep{Goldstein2003}. As Pauli complained, writing to Einstein's close friend Paul \Ehr, \q{God seems to have left Einstein completely!} \letterpaulip{Pauli}{\Ehr}{29}{9}{1929}[237].

%An invitation to the 1930 Rouse Ball lecture at Cambridge gave Weyl the opportunity to review the Whole development of matter concepts which had taken place during the long decade just coming to an end.





%\footnoteh{Einstein agreed and gave a talk on the Problem of Space, Field, and Ether in Physics on December 11, 1929, Essentially the same talk was delivered to a large audience on the opening day of the Second World Power Conference which took place in Berlin from 16–25 June, 1930.  The text of this lecture was then published in the conference’s Transactions [Einstein 1930d]. A similar popular account of Space, Ether and the Field in Physics was published in Forum Philosophicum [Einstein 1930c] together with an English translation. Indeed, the text of the two penultimate paragraphs of this version and [Einstein 1930d] that characterize the distant parallelism are identical. A two-page abbreviated version of [Einstein 1930c] also mentions the distant parallelism approach [Einstein 1930e]}

Nevertheless, Einstein continued to defend the theory in public (in talks given in October and December) \citep{Einstein1930,Einstein1930a,Einstein1930b}, as well as in as well in private correspondence. However, Pauli did not hesitate to describe Einstein's presentation at the Berlin Colloquium as a \qt{terrible rubbish}{schrecklichen Quatsch} \letterpaulip{Pauli}{Jordan}{30}{11}{1929}[238]. When he received the drafts of Einstein's \jt{Annalen} paper, he wrote only slightly more politely \cop{that he no longer believed that the quantum theory might be an argument for the distant parallelism after Weyl's work on Dirac theory had shown that Dirac’s electron theory could be incorporated into a relativistic gravitation theory if the \vbein are introduced but the equations remain invariant if the \vbein at distant points are rotated in arbitrary manner}. Pauli also wrote that he did not find the derivation of the field equations convincing; they show \qt{no similarities with the usual facts confirmed by experience}{kaum eine Ahnlichkeit mit den gewohnlichen durch die Erfahrung gesicherten physikalischen Sachverhalten zu haben scheinen} \letterpaulip{Pauli}{Einstein}{19}{12}{1929}[239]. In particular, Pauli missed the validity of the classical tests of general relativity, perihelion motion and gravitational light bending: \qt{These results seem to be lost in your sweeping dismantling of the general theory of relativity. However, I hold on to this beautiful theory, even if it is betrayed by you!}{Die scheint doch bei Ihrem weitgehenden Abbau der allgemeinen Relativitatstheorie verloren zu gehen. Ich halte jedoch an dieser schonen Theorie fest, selbst wenn sie von Ihnen verraten wird!} \letterpaulip{Pauli}{Einstein}{19}{12}{1929}[239]. When Einstein expressed caution towards the definitive validity of his equations, he, \qt{so to speak, took the words right out of my mouth of criticism-loving physicists}{haben Sie den Kritik libenden Physikern sozusagen das Wort abgeschnitten} \letterpaulip{Pauli}{Einstein}{19}{12}{1929}[239]. Pauli knew that Einstein would not have changed his mind, but he was ready to \q{make any bet} that \q{after a year at the latest you will have given up all the distant parallelism, just as you had given up the affine theory before} \letterpaulip{Pauli}{Einstein}{19}{12}{1929}[239].

%\q{dann sagen Sie erst etwas dariiber, wenn mindestens ein Vierteljahr vergangen ist} 

Einstein complained that Pauli's remarks were superficial and asked him to return on the issue after some months \letterpaulip{Einstein}{Pauli}{19}{12}{1929}[140]. Although the \uftp was disavowed by its own initiators \citep{Weyl1931}, Einstein insisted in the pursuit of \FP discussing with Mayer two solutions of his last field equations \citep{Einstein1930g}\todo{field equations admitted at least one unphysical solution, namely, a static configuration of uncharged, gravitating bodies.}. However, Pauli would have clearly won the bet. Only a few months later Einstein and Walther Mayer presented a new approach \citep{Einstein1931} that, by generalizing the \nbein formalism to five dimensions, may have appeared more promising. This approach was ideally connected with that of Kaluza, but the shortcoming of that theory \qt{by sticking to the four-dimensional continuum, but with vectors with five components}{werden bei der im folgenden dargelegten Theorie dadurch vermieden, daß man zwar bei dem vierdimensionalen Kontinuum bleibt, aber in diesem Vektoren mit fünf Komponenten \textelp{} einführt} at each point of four-dimensional space-time \citep[377]{Einstein1931}. The optimism once again faded away quickly, since the theory was unable to solve the problem of matter. In a popular talk given in Vienna towards mid-\datemy{14}{10}{1931}, Einstein could only describe his field-theoretical work since \gr as a \qt{cemetery of buried hopes}{Friedhof von Begrabener Hoffnungen} \citep[441]{Einstein1932b}.

A few days later, Lanczos wrote to Einstein from the United States \letteraeap{Lanczos}{Einstein}{20}{10}{1931}[15-243] where he had just taken a position at Purdue University. Lanczos told Einstein that, at Arnold Berliner's suggestion, the influential editor of the \jt{Die Naturwissenschaften}, he had prepared a semi-popular presentation of \FP approach for the \jt{Ergebnisse der Exakten Wissenschaften}, a series sponsored by Berliner's journal \citep{Lanczos1931}. Lanczos had worked on the topic during his time as Einstein's assistant. The Lanczos/Einstein relation had become somehow strained \citep{Stachel1994}, and Lanczos was not fully convinced by Einstein's approach. However, he was confident to have found \q{a tone that should correspond to your conviction as well. I think that, deep down, we have something in common} \letteraeap{Lanczos}{Einstein}{20}{10}{1931}[15-243]. Lanczos presented \FP as a completion rather than an generalization of Riemannian geometry; nevertheless he also recognized the correctness of Reichenbach's approach \citep[118]{Lanczos1931}. What is more important, he opened the paper with some general considerations which give a glimpse in the philosophical atmosphere which pervaded the physics community. Lanczos distinguished between two \qt{spiritual attitudes}{geistige Haltung} towards relativity: 

\begin{enumerate}
\item\label{p} a \emph{positivist-subjectivist} insistence that physics has to do with observable quantities, and what cannot be observed is not part of physics. This \q{rigorous and therefore more intolerant form of positivism} \citep[104]{Lanczos1931}, defended in particular by quantum theoreticians, lead to the rejection of the \uft program as such.\ Since a field is nothing but a tool to describe the behavior of test particles, \rac and so, it is vain to search for solutions of the field equations that correspond to protons and electrons. In fact, the fields inside of elementary particles \q{could never in their details become the object of observation} \citep[104]{Lanczos1931}, since there are no test particles or measuring scales smaller than the electron itself. 

\item\label{m} A \emph{metaphysical-realistic} perspective, based on the conviction that physical reality exists independently of the possibility of measuring or observing it. If \sr seemed to be close to the positivistic/operationalistic ideals, with \Mink the theory underwent a \qt{\scare{metaphysical} turn}{metaphysische Wandlung} in favor of a \qt{logical-constructive understanding \origins{Verstehen}}{zugunsten eines logisch-konstruktiven Verstehen} \citep[103]{Lanczos1931}. \Gr had finally brought \qt{the logical-deductive exploration into the depths of nature, under the presupposition of its universality and understandability, and with faith in the laws of mathematics}{das logisch-deduktive Eindringen in die Natur, unter Voraussetzung ihrer Universalitiit und Verstehbarkeit, und im Verlrauen au/ das mathematische Gesetz} \citep[102]{Lanczos1931}. 
\end{enumerate} 
%
The positivist described by Lanczos could be easily identified with Pauli, who had indeed raised similar objections against Weyl's theory early on \citepp{Pauli1919}[see][13]{Hendry1984}. However, Pauli,  by reviewing Lanczos article, did not fully recognize himself in the portrait of the \scare{positivist} \citep{Pauli1932-3-11}. Such labels, he argued, \q{are highly subjective and arbitrary}; it is obvious that in order to gain new scientific insights one does not only requires inductive generalizations, but also logical-constructive imagination. Pauli mocked the \emph{Naturwissenschaften} for having published the paper in series entitled \scare{Results in exact sciences} (\german{Ergebnisse der Exakten Wissenschaften}). Indeed, Einstein published this sort of theories at rhythm of one each year and in every case he claims that it is the definitive solution: \qt{Einstein's new field theory is dead, long live Einstein's new theory!}{Die neue Feldtheorie ~EINSTEINS ist tot. ES Iebe die neue Feldtheorie EINSTEINS !}. 


%However, it was undeniable that many supporters of \qm had used a a positivistic rhetorics, and it was against this rhetorics that Einstein, somehow tongue in cheek, was not ashamed to define himself as a metaphysician \citep{Einstein1932a}. It was this attitude that caught many of Einstein's philosophical allies by surprise.\todo{improve}


%Lanczos was meant probably the Pauli-Einstein debate, a more general the reaction of Einstein's philosophical\todo{improve}. 

However, if many readers might have easily recognized someone like Pauli in Lanczos's \scare{positivist}, other were baffled to find out Einstein located among the \scare{metaphysicians}. At the beginning of 1932 the introduction of Lanczos's 1931 paper was published at Berliner's suggestion as a \latin{seperatum} in the \citejournal{Lanczos1932} \q{to make it available to a larger public} \citep[113\fn{1}]{Lanczos1932}. It is probably this article of Lanczos that Frank read with some bewilderment, as he reports in his Einstein's biography \citep{Frank1947}. Frank was \q{quite astonished} to find the theory of relativity characterized as the expression of a realist program \q{since I had been accustomed to regarding it as a realization of \Mach's program} \citep[215]{Frank1947}. However, when Frank met Einstein in Berlin at around the same time, he found out that Lanczos had indeed well characterized Einstein's point of view \citep[215f.]{Frank1947}. According to his recollection, Einstein complained that \q{\textins{a} new fashion} had arisen in physics according to which quantities that in principle cannot be measured do not exist, and that to \q{to speak about them is pure metaphysics} \citep[216]{Frank1947}. Frank objected that this was the very same philosophical attitude that led to relativity theory. By contrast, Einstein insisted, the essential point of relativity theory is to \q{regard an electromagnetic or gravitational field as a physical reality, in the same sense that matter had formerly been considered so}  \citep[216]{Frank1947}. The theory of relativity teaches us the connection between different descriptions of one and the same reality. Was not a theory about the behavior of \rac, but a unification of two fields.


%http://philsci-archive.pitt.edu/18196/1/Coming%20to%20America-%20Carnap%2C%20Reichenbach%2C%20and%20the%20Great%20Intellectual%20Migration.%20Part%20II%2C%20Hans%20Reichenbach.pdf

%


\section{Conclusion}



%\footnoteh{\label{frank}A feeling of growing philosophical distance was sensed by the members of Viennese group. I mention here only an example. As he recounts in his Einstein's of biography, at the beginning of 1932, Philipp Frank read a paper by \citet{Lanczos1932} on \FP with some bewilderment. Lanczos claimed that \rt was based on a metaphysical interpretation of science, while \qth was based on a positivistic interpretation. Frank was \q{quite astonished}, since he had been accustomed to regarding relativity \q{as a realization of Mach's program} \citep[215]{Frank1947}. However, when Frank met Einstein in Berlin at around the same time, he found out that Lanczos had indeed well characterized Einstein's point of view correctly \citep[215f.]{Frank1947}. Frank described the same sense of surprise in his interview with Thomas Kuhn \citep{Frank1962-07-16}}

%As we have seen, besides the deterioration of their personal relationship, Einstein's extreme rationalism in those years \citep{Einstein1933} could not be more distant from Reichenbach's inductivism \citepp{Reichenbach1931}[see][]{Galavotti2009}. The  philosophical squabbles between Reichenbach's Berlin Group and the Vienna Circle lead by Schlick started to emerge and the turn of 1930s and soon grew to an open conflict; The  philosophical squabbles between Reichenbach's Berlin Group and the Vienna Circle lead by Schlick started to emerge and the turn of 1930s and soon grew to an open conflict; 

Lanczos' reconstruction is too broadly stroked to be fully accurate; nevertheless it undeniably grasps something of the intellectual mood (\german{geistige Einstellung}) of the time. Reichenbach, like Pauli, would not have been entirely pleased of having been cast among the \scare{positivists}, with whom he was in conflict for quite some time. however, like Frank, he would have been puzzled, if not appalled, by seeing Einstein categorized among the \scare{metaphysicians}. However, Einstein's lecture from 1933 leaves no doubt that Lanczos reconstruction of accurate. The most simple realization of mathematical ideals. \cop{In Oxford, he instead proclaimed that all one needed to do in order to arrive at the general theory of relativity, was to "ask what are the simplest laws which a [Riemannian] metric can satisfy}. \cop{To further justify his methodological conviction, Einstein gave two more examples. The first of these was the set of Maxwell's equations; they are the simplest laws for an anti-symmetric tensor field which is derived from a vector. Then, he turned to the subject that he was deeply involved in at the time of the Oxford lecture; the law that describes the dynamics of electromagnetically charged particles, the Dirac equation}. \cop{To the audience in Oxford, he announced his latest, most appealing result: the simplest laws these semivectors satisfy elucidate the dual existence of "two sorts of elementary particles, of different ponderable mass and equal but opposite electrical charge." The semivector was thus an outstanding example to support his view that "in the limited number of the mathematically existent simple field types, and the simple equations possible between them, lies the theorist's hope of grasping the real in all its depth."19 In the end, it was this conviction that gave Einstein the strength to maintain for some thirty odd years that his program in classical field theory provided a viable alternative to the quantum theory}. That mathematical necessity is the key two reality. There is non separation between mathematics and physics. Clearly Reichenbach's battle against to convince Einstein's himself, who in Reichenbach's eyes was the very origin.


% Reichenbach invited Einstein to contribute to the newly founded journal \jt{Erkenntnis} published by Felix Meiner and edited with Carnap \letteraeap{Reichenbach}{Einstein}{25}{4}{1930}[73-226].  However, to no avail. That 1933 lecture in can be considered of this 


%Nevertheless, when Hugo \citet{Dingler1933}, a few years later, launched a political attack against the journal, he mocked Reichenbach as \qt{Einstein's self-proclaimed personal philosopher \origins{Leibphilosoph}}{Einsteins nominierter Leibphilosoph} who replaced logic with the authority of a great physicist \citep[VI]{Dingler1933}.  But Dingler did not mean to open a scholarly dispute \citep{Howard2003}. Reichenbach replied from his Turkish exile, insisting on the political independence of journal \citep{Reichenbach1934}. However, the situation rapidly deteriorated, and the seventh volume of \jt{Erkenntnis} (1937-1938) was edited by Carnap alone.


Einstein left for soon therefart, Reichenach for Turkey. Reichenbach's initial enthusiasm  soon waned and he tried to obtain a position in Princeton  \citep{Verhaegh2020a}. However, Reichenbach feared Weyl's opposition: \q{He is my adversary since a long time,} he wrote to the American philosopher Charles W.\ Morris, a supporter of a form a \q{mathematical mysticism} that was \q{very much opposed to my empiricistic interpretation of relativity} \letterp{Reichenbach}{Morris}{12}{4}{1936}[013-50-78][HR]. Thus, in April 1936, Reichenbach turned to Einstein to ask his support: \qt{I surmise that Weyl's opposition persists to these days and therefore I'd be grateful if you could put a word in my favor}{Ich vermute, daB Herrn Weyls Gegnerschaft noch heute fortdauert, und darum ware ich Ihnen sehr dankbar, wenn Sie da zu meinen Günsten eintreten kGnnten} \letteraeap{Einstein}{Reichenbach}{2}{5}{1936}[20-118]. By this time, it was ironically Einstein the one indulging in the sort of mathematical mysticism that Reichenbach attributed to Weyl. As Einstein famously confessed to Lanczos, his work on \gr had made him \qt{a believing rationalist}{das Gravitationsproblem raich zu einem glaubigen Rationalifaten gemacbt, d.h zu einen,der die einzige zuverlassige Quelle der Wabrbeit in der matbematischen Einfacbbeit Bucht} \letteraeap{Einstein}{Lanczos}{24}{1}{1938}[15-268], convinced that physical truth lies in mathematical simplicity \citep{Ryckman2014}. However, he continued, the mathematical formulation of the laws of nature need not to be of \q{\emph{geometrical} nature} \letteraeap{Einstein}{Lanczos}{24}{1}{1938}[15-268]. 

In 1938 Reichenbach managed to move to the United States \citep{Verhaegh2020a}. The American years did nothing to bridge the philosophical cleavage that had emerged during their late Berlin time. Even the point on which Einstein and Reichenbach seemed to have agreed in 1926, the fact that that the mathematical formulation of the laws of nature need not to be of \q{\emph{geometrical} nature} \letteraeap{Einstein}{Lanczos}{24}{1}{1938}[15-268], was based on a fundamental misunderstanding. While Reichenbach had used the argument against the \uft program, Einstein's relied on it to defend it. While Reichenbach feared the absorption of physics into geometry, Einstein questioned the separation geometry and mathematics (\letteraea{Einstein}{Lanczos}{21}{3}{1942}[15-294]; \letteraea{Einstein}{Barrett}{19}{6}{1948}[6-58]).  

Einstein (\citeyear*{Einstein1949f,Einstein1949a}) praised \citets{Reichenbach1949}’s contribution to the volume in his honor of the series \textit{Library of Living Philosophers} edited by Paul \citet{Schilpp1949}. On those occasion, famously Einstein reestablished a dialogue with Reichenbach (Helmholtz) and Poincaré. Indeed, that $ds$ was As we have, here introduced a third figure non-positivist, that the separation. Howver, that he did not agree that single mathematical. 

It was mainly the \uftp.

That only the whole, the issue of coordination was then in a quite different way. The choice of $\gmn$, the choice is not justified, by the direct physical meanig. The question of identification, was only after the integration of the field equaitons. these would have ... The only guardeed was indeed mathematicla, that only semplicity, was the garantiee of realty. Einstein was not only a non-positivist the coordination and reality. He was a non ... However, the self-described \q{tamed metaphysician}  \citep[13]{Einstein1950c}. However, an even was. 

%The theory as whole; Reichenbach replied in 1953 althoguth probably in a more popular book, that possible that after was an empirical fract, the testablility of geometry separately from the rest of pjysics. 

%, Einstein believed that mathemtical semplicity was the only key to reality. Had grown increasingly impatient toward any philosophy that smelled of \scare{positivism}. 

Reichenbach the separation of mathematics and physics Reichenbach replied in 1953 
When in 1953 Schilpp asked Einstein for contributing to the volume of the same series in honor of Carnap \letteraeap{Schilpp}{Einstein}{11}{5}{1953}[80-539], he famously declined. After \q{Reichenbach's death (a few weeks ago),}\footnote{Reichenbach died on \datef{9}{4}{1953}} Schilpp wrote, Carnap was the most important exponent of logical empiricism \letteraeap{Schilpp}{Einstein}{11}{5}{1953}[42-534]. Although Einstein agreed with this assessment, he expressed disenchantment toward that type of philosophy that Schlick, Reichenbach, and Carnap represented: \q{the old positivistic horse, which originally appeared so fresh and frisky, has become a pitiful skeleton} (\letteraea{Einstein}{Schlipp}{19}{5}{1953}[42-534]; quot.\ and tr.\ in \cite[374]{Howard1990}).

The of \gr was also a about \uft field theory. Indeed, that skeleton was ultimatly been. Einstein only partially conceed that the problem of coordination of mathematical strutucs with reality; the apparent geometrization . The third issue of unification was by Reichenbach intudictive; was mathemtical unificaiton in which of mathematical simplicity. A complex interppaly in 

%That with the same that was in Einstein's lecture of 1921. However, this separation was ultimately that had challenged starting at least, never agreed coordinationa, that provisiona, which the all theory geometry plus physics. This but much more proram; that between geometry and physsc, geometry and mathematics was intessentia. On many in that unificatiom that mathematilca simplicity was the to such unficaiton. 




\printshorthands
\printbibliography

\end{document}