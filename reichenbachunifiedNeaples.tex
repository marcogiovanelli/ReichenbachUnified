% !TEX encoding = UTF-8 Unicode
\documentclass[draft]{article}
\usepackage{els}
\usepackage{notate}
\usepackage{wrapfig}

%\usepackage{geoA4}

\title{-}
\author{-}

\begin{document}
\maketitle

\begin{abstract}
\lipsum*[1-2]
\end{abstract}

\begin{keywords}
Reichenbach \sep Unified Field Theory \sep General Relativity \sep Geometrization	
\end{keywords}


coordination ... geometrization... unification.


%fter these mathematical preliminaries, Einstein-borrowing elements from Wey~1918b, bute~sentiallyfollowing Einstein 1916o (Vol. 6, Doe. 41)-uses variational methods to den~efi~ld ~uatlO~s and the law of energy-momentum conservation (see [pp. 13-17]). For the a?pro~ImatiVe mtegrat10n of the field equations, Einstein follows (with some small impr~vements)El~stem191Sa (Doe. 1), pp. 17-19. The derivation of the exact solution of the field equatiOns for a pomt mass and the perihelion advance of an orbit is taken from Weyl1918b (see (p. 21] and, for somewh~t more detail, [p. 24]). The final pages of these notes, (pp. 21-24], deal with cosmology and combme ele- ments from Einstein 1917b (Vol. 6 Doe. 43) and Einstein 1919a (Doe. 17). . . These lectures also cover the g~neralizationof electrodynamics from special to general relatiVIty ([p.I2]), the equations of motion for frictionless fluids ([p. 14]), and the behavior of rods and clocks Weak gravitational fields ([p. 20]).

\section{Coordination. Reichenbach - Einstein Encounter and Weyl-Theory}

\subsection{Reichnebach's as Einstein's Lecture and his Early Critique of Field theory}

Nach seinem Einsatz im Ersten Weltkrieg hörte Reichenbach in Berlin& Einsteins Vorlesungen zur Speziellen und Allgemeinen Relativitätstheorie. Seine Mitschriften dieser Veranstaltungen sind erhalten geblieben: Klassische und Statistische Mechanik (SS 1918, HR-028-01-02), Allgemeine Relativitätstheorie (Teil 1 HR-018-0104, Teil 2 HR-028-01-03, Teil 3 HR-028-01-01, alle Tile undatiert). In Teil 1 der Mitschrift zur Allgemeinen Relativitätstheorie, behan- 1917--1918 and 1918-1919

Reichenbach first encounter with relativity happened in Winter term 1918-1919 when he followed Einstein's lectures in Berlin. The notes taken by Reichenbach are extant and are very similar to Einstein's own notes and in some points complementary\footnote{Further information about Einstein as an academic teacher, see Vol. 3, the editorial note, "Einstein's Lecture Notes,"pp. 3-10, and for a survey of Einstein's academic courses, see Vol. 3, Appendix B.}. \cop{These lectures essentially follow the corresponding sections of Einstein 1914o (Vol. 6, Doe. 9) and Einstein 1916e (Vol. 6, Doe. 30). The introduction of the Riemann curvature tensor given in these two papers can be found in these lecture notes as well (see [p. 11]), but it is clear from entries on [p. 10] and [p. 25] and from Reichenbach's notes (the last part of the second notebook) that. From the metric, Christoffel symbos, and Riemann tensor is flat.

After this standard presentation, in his lecture notes, Reichenbach also presented the new interpretation of the curvature in terms of the parallel displacement given by Tullio Levi-Civita and Hermann Weyl (who are mentioned explicitly on [p.IO]) and in Reichenbach's note on page. \cop{In Reichenbach's notes, the curvature tensor is introduced at a later point, just before the discussion of gravitational field equations (see [p. 14]).} At that point, one finds the derivation that Einstein began but did not finish at the foot of [p. 10], along with a similar derivation that can be found on [p. 25]. On [p. 10],  Two vectors are parallel if they have the same components. However, this is not the case in the general case, using non-linear coordinates. In order to to reinstate the connection, one has to 

In Euclidean geometry, it is always possible to introduce a Cartesian coordinate system in which two vectors are equal and parallel when they have the same components. However, this relation does not hold if we introduce curvilinear coordinates, \eg polar coordinates. Consequently, vectors at different points can no longer be directly compared. If one displaces a vector to a neighboring point $dx_\nu$, one does not know whether the vector has remained the \scare{same} by simply examining its components. The \scare{connection} (\german{Zusammenhang}) from a point to another is lost. Because the affine geometry is the study of parallel lines, \citet{Weyl1918b} used to speak of the necessity of establishing an \scare{affine connection} (\german{affiner Zusammenhang}). However, because it is a relation of \scare{sameness} rather than parallelism that is relevant in this context, others, such as Reichenbach, prefer to speak of the operation of \scare{displacement} (\german{Verschiebung}), where the latter indicates the small coordinate difference $d\xn$ along which the vector is transferred.  

To reinstate the \scare{connection} one requires to introduce a rule for comparing vectors at infinitesimally separated points. Given a vector $A^\tau$ at \xn in an arbitrary coordinate system, we need to determine the components of the vector $A^{\ast\tau}$ at $\xn+d\xn$ that is to be considered the \scare{same vector} as the given vector $A^\tau$. The vector $A^{\tau}$ at the point $P\left(x^{\nu}\right)$ and the vector $A^{\tau}+d A^{\tau}$ at the point $P^{\ast}\left(x^{\nu}+d x^{\nu}\right)$ are the \scare{same vector}, if they satisfy the condition:

\begin{equation}\label{eq:affine}
dA^\tau = \Gamma^\tau_{\mu\nu}A^{\mu} dx_\nu\,.
\end{equation}
%
The quantity $\Gamma^\tau_{\mu\nu}$ is known as the affine connection or displacement. It has three indices, i.e., entails $\tau$ possible combinations of $\mu \times \nu$ coefficients, which can vary arbitrarily from point to point, i.e., in the general case, are functions of $x_\nu$. Because in general $\Gtmn \neq \Gtnm$, the \Gtmn has $n \times n^2$ coefficients. If a vector $A^\tau$ is given at the point $P$ with coordinates \xn, \cref{eq:affine} yields the unknown components of the vector $A^{\ast\tau}$ at $P^*$ with coordinates $\xn+dx_\nu$. Continuing this process $
\xn+d\xn+d^{\ast} \xn+d^{\ast \ast} \xn \ldots$, we can parallel displace a vector from any given point to any other distant point. As is well known, the most characteristic feature of the operation of displacement is that if one parallel displaces $A^\tau$ along different paths, one gets, in general, a different vector $A^{\ast \tau}$ at a distant point. The transfer of a vector $A^{k}$ from $P$ to the distant point $Q$ along a given curve $C$ is, of course, unique and reversible; that is, on transferring back along the same curve you get back the initial vector at $P$. But the transfer along some other curve $C^{\prime}$ would lead to another vector at $Q$. Altermatively, if you transfer $A^{k}$ from $P$ to $Q$ along $C$, then back to $P$ transfer $A^{k}$ from $P$ to $Q$ along $C$, then back to $P$ along $C^{\prime}$, you obtain at $P$ a vector different in direction and length from the original $A^{k}$. you associate with any contravariant vector $A^{k}$ its invariant $g_{i k} A^{i} A^{k}$ ('square of the length'). The required correspondence between the vectors $A^{k}$ at $P$ and $A^{\prime k}\left(=A^{k}+\delta A^{k}\right)$ at the neighbouring point $Q$ can then be set up by adopting the following four rules. The \gmn are determine the connection.

However, soon If one starts with a symmetric metric \gmn, the Christoffell symbols are, so to speak, the only possible choice, that is one is lead to. However, if one defines the operation of displacement independently from the metric, the Riemannian connection appears only as a special case that is achieved by introducing a series of conditions. Therefore, Weyl realized formalism opened a vast array of possibilities that physicists hoped to exploit to accommodate the electromagnetic field in the geometrical structure of \spti. Weyl considered natural to maintain that the is symmetric, and exploited the possibility of considering to abandon the condition, so se it to the particular value. Indeed, Riemannian geometry is near-geometrical however that fact that we compare the at distance. 

%One one can assume that the a more general affine connection, in which the change by a factor. The metrical connection of the space depends not only on the quadratic form (2) (which is determined only up to a proportionality factor) but on the linear form (7). If, without changing coordinates; more general affine connection. However, that the same length, by dropping this assumption, that a new with could be identify with electromagnetic four potential. That a more general affine connection \Gtmn. The length of a vector $l$ can be introduced separately. That one has to impose as a condition that the length is transported without being changed, that is that the covariant derivatives do not change: That the length of time-like is measure by an atom, that thus, a more general affine connection, the sum of the Christoffel. Thus Weyl


Weyl that is symmetric, but abandoned the assumption that was covariant however, that only the ratio of the $\gmn$ by parallel transport, that is the length of vectors. The underlying idea was that one ought not to admit the physical possibility of comparing 'lengths' at distant world points, as Einstein's theory did. Weyl's modification automatically introduces, in addition to the 'metrical tensor' $g_{i k}$, a 'metrical vector' $\varphi_{k}$ of equally fundamental standing with $g_{i k}$, and capable of being interpreted as the electromagnetic potential. For the $\Gamma^{\prime \prime}$ you now no longer get just the $\left\{\begin{array}{c}k \\ m l\end{array}\right\}$, but

\begin{equation*}
\christoffel{\mu}{\nu}{\tau}
\end{equation*}
%
The restrictive Hamiltonian principle. A connected manifold of any type is not yet a field theory. To make it that, you have to impose on its basic geometrical field-quantities-the $g_{i k}$ or the $g_{i k}$ and $\varphi k$ or the $\Gamma_{m l^{k}}$, as the case may be-certain restrictions, differential equations. One wishes them to flow from some general principle which leaves as little arbitrariness as possible. Now almost every kind of restriction contemplated hitherto has turned out to be equivalent to a Hamiltonian principle, variant density $\mathcal{E}$, taken over any fixed region, be stationary:
$\delta \int \mathcal{L} d x_{1} d x_{2} d x_{3} d x_{4}=0$. This very convenient way of searching for the 'right' field-equations (you just have to search for 'the right $f$ ') has been widely adopted, and there are general reasons for believing that it is justified. (If the field equations amount to a Hamiltonian principle, the conservation laws are an automatic consequence of the general invariance.)




%On pp. 124-128 of Weyl 1919c, the revised version of the manuscript enclosed with this document, and in sec. 35 of Weyl 1919d, the author showed that from the conservation laws in the new theory, a conveniently chosen gauge, and the basic assumption that the new theory is compatible with the existence of material particles, one can derive equations of motion for the new theory that are the same as those in ordinary general relativity, which means that in WeyPs theory particle trajectories are not geodesics. %As Pauli had pointed pout at about the same time this was indeed problematic, since that physical meaning of rods and clocks inside of the electron becomes meaningless. Thus the very meaning of the claim that Euclidean geometry cannot be that space geometry around the sun is non-Euclidean. see also Einstein. Thus to measure with \rac $ds$ inside of the electron, was meaingless. This problem would have immediately raise on the meaing of the coordination of the $ds$ as the result of \rac. \todo{Pauli}

\subsection{Reichenbach's Habilitation and his critique of Weyl Theory}

By 1919, Einstein  \cop{argued repeatedly that Weyl’s theory runs into difficulties if one assumes, as Einstein thought one should, first, that clocks and rods directly measure the line element $ds^2$ (see, for instance, Doc. 512). This famous obectios. that vectors have length, are measured by clocks. Iinded, bu. Howver, Einstein aslo poitend the trajectories of free uncharged particles are not geodesics (see, for instance, Doc. 579) in the theory becolse of the presence of in the definition of the connecto. If one makes the first assumption in Weyl’s theory, Einstein argued, the rate of a clock will be dependent on its prehistory; if one makes the second, uncharged particles will be affected by the electromagnetic four-vector potential. In Doc. 661, Einstein reiterated both charges, that of the fourth-order and not of the second order}.  Einstein held the opinion that the world-lines of an uncharged particle should be described by a geodesic equation that does not explicitly depend on the electromagnetic four-potential, as it does in Weyl's theory. 

Nevertheless, Einstein had started to interest in the unified field theory program, after his correspondence with Kaluza. Reichenbach discussed Einstein's first attempt to develop a relativistic theory of matter, \q{Auch in der ncucn Einstr~insclwn Auffansung, bei der innerhbalb des Elcktroms wicder die nicllt-cuklid. Geometric gilt} (HR-028-01-04, Randbemerkung zu Blatt 18.)\todo{check}. Thus, at that time, Reichenbach was not only proficiency as no other philosopher, but also was up to date with first attempts at \uft. With this background, in Februar or April 1920 Reichenbach decided to write his habilitation. \scare{Im Februar (oder März) 1920 beschloß ich, meine Habilitationsschrift zu schreiben. Ich hatte in den Monaten vorher Relth. gearbeitet, auch nach Weyl; den Grund hatte ich schon 1917-1918\todo{check} in Vorlesungen bei Einstein gelegt, aus welchen meine Kenntnis der Th. herrührt. Der Kapp-Putsch gab mir 8 Tage Freiheit. Ich brauchte nicht in die Fabrik Huth und konnte un- unterbrochen schreiben. Nachts stand ich auf u. schrieb weiter. Wegeu des Streiks versagte die elektr. Beleuelıtuugç ich schrieb deshalb bei Azetylen (Falırı'ad1ateı'1ıe). Die Schrift ist in etwa 10 Tagen niedergesclırieben}. Since he mentions it in his habilitation in Stuttgart, which probably finished around March 1920. 

In the book Reichenbach alludes briefly how \q{Weyl's generalization of the theory of relativity which abandons altogether the concept of a definite length for an infinitesimal measuring rod}. Weyl's theory represents a possible generalization of Einstein's conception of space which, although not yet confined empirically, is by no means impossible. However, seems be empirically false. \q{According to Weyl the frequency of a clock is dependent upon its previous history}; which seems to be false. However, one can also that the influences that an \q{these influences compensate each other on the average, then the experiences until now}  \q{the frequency of a spectral line under otherwise equal conditions is the same on all celestial bodies, can he interpreted as approximations, rather that an expression of a true geometry of space-time. The apparent would be non exact, and the real could be non-Riemannian}. 

The justification of the choice of such geometry was according to Reichenbach ultimately. In Weyl geometry vector moving close loop which would same length but different direction in Riemannian geometry, different length and different direction in Weyl's geometry, epistemological, was a purely infinitesimal geometry eliminated the last remanet of distance of Riemannian geometry. Reichenbach, indeed he already surmised that  this generalization can be continued. would be to assume that the vector changes its length upon turning around itself. \q{The next step in the generalization would be to assume that the vector changes its length upon turning around itself. There is no \scare{most general} geometry}\todo{check which geometry would be this?}. Thus, to compare even not at the same point, that to consider as intrinsically more satisfying that Riemannian geometry.  The claim that there is serve against  the ambition \scare{epistemological superiority} that Weyl attributed to his purely infinitesimal geometry, which made necessarily the best candidate to be the geometry valid in reality. This critique against Weyl's theory emerges in another passage:

\q{Yet it is incorrect to conclude, like Weyl and Haas, that mathematics and physics are but one discipline. The question concerning the validity of axiotns for the physical world n1ust be distinguished from that concerning possible axiomatic systen1s. It is the merit of the theory of relativity that it renowed the question of the truth of geon1etry from mathematics and relegated it to physics. If now, from a general geometry, theorems are derived and asserted to be a necessary foundation of physics, the old mistake is repeated. This objection n1ust be 1nade to Weyl's generalization of the theory of relativity ... Such a generalization is possible, but whether it is compatible with reality does not depend on its significance for a general local geometry. Therefore, Weyl's generalization must be investigated from the viewpoint of a physical theory, and only experience can be used for a critical analysis. Physics is not a \scare{geometrical necessity}; whoever asserts this returns to the pre-Kantian point of view where it was a necessity given by reason}

This critique contains the backboene of Reichebach's criticism of the \ufp int years. In spite of his Kantian framework, Reichenbach \cop{It is characteristic of modern physics to represent all processes in terms of mathematical equations}. \q{But the close connection between the two sciences must not blur their essential difference}. The truth of mathematical propositions depends upon internal relations among their terms; the truth of physical propositions, on the other hand, depends on relations to something external, on a connection with experience. \q{This distinction is due to the difference in the objects of knowledge of the two sciences}. Weyl's attempt had to blur the between the sciences. In the context of his etherodox Kantianism, Reichenbach defended an empirical stance towards the question of which geometry is valid in reality. Similar critique was also formulated by Erwin Freundlich. In this way, not only pre-Kantian rationalism, it is also the very spirit of \rt theory. Indeed, that even geometry, that was to be ultimately Eucliean, was inded to be left to physics. 

%The axioms of connection are the empirical laws of physics, the fundamental equations of a theory. The axioms of coordination determine the rules of the application of the axioms of connection to reality, that is, they determine the rules of the connection. However, 

Between implicit definitions and the which coordinating principles, could plany the first the second would never be abannoded by Reichenbach. However, as we shall see it will be abanoded by Einstien. Einstein received possibly 24. May 1920 Einstein praised Reichenbach's \german{Habilitationschrift} in Einstein to Moritz Schlick, 19 April 1920 (Vol. 9, Doc. 378). The copy in Einstein's library contains some marginal notes, e.g., \scare{sehr gut} on p. 74 to Reichenbach's contention that it is impossible to infer \textit{a priori} principles. Indeed, were free conditions of h. 

The thesis was published after 15 June 1920 as Reichenbach 1920 (see Doc. 57). Stuttgart, Wiederholdstr. 13. d. 15. Juni 1920. Widmung Reichenbach attended Einstein's Ieerures in Berlin (see Doc. 57, note 2). That most of the discussion of the book, relativized constitutive \apr, however, one first by a philosophers of Weyl theory, and in particular was that is itself superior. \q{Philosophen eine Ahnung davon haben, dass mit Ihrer Theorie eine philosophische Tat getan ist, und dass in Ihren physikalischen Begriffsbildungen mehr Philosophie enthalten ist, als in allen vielbändigen Werken der Epigonen des grossen Kant.} \q{die tiefe Einsicht der Kanti- schen Philosophie von ihrem zeitgenössischen Beiwerk zu befreien} \cop{Der Wert der Rel.Th. für die Philosophie scheint mir der zu sein, dass sie die Zweifelhaftigkeit gewisser Begriffe dargethan hat, die auch in der Philosophie als Scheidemünzen anerkannt waren}. 


\subsection{Bad Nauheim}

Hermann Weyl kanute Reichenbaclı persönlich bereits seit 1920, deuu beide nahnıen au der Naturfosclıerversammlung, that might the opportunity. But this was precisely the defence strategy that Weyl had sketched in 1919 of his theory and defended in public  at the 86th Assembly of the Association of German Scientists and Physicians (Versammlung der Gesellschaft Deutscher Naturforscher und Ärzte) in Bad Nauheim in September 1920. At Bad Nauhe Weyl's contribution of \scare{guiding field,}  affine connection to defined Einstein's theory against Lenard's attacks, his own theory against Einstein by introducing the opposition \scare{Einstellung and Beharrung}. \citet{Weyl1920a} introduced the distinction between \german{Einstellung} and \german{Beharrung} to explain away the discrepancy between the non-Riemannian behavior of the \scare{ideal} time-like vectors implied by his theory and the Riemannian behavior of the \scare{real} clocks that are actually observed\footnote{He suggested that atomic clocks might not \emph{preserve} their Bohr radius if transported, but \emph{adjust} it every time to some constant field quantity. Weyl suggested that the atoms we use as clocks might not preserve their size if transported, but adjust it every time to some constant field quantity, which he could identify with the constant radius of the spherical curvature of every three-dimensional slice of the world, furnishing a natural unit of length. The geometry read off from the behavior of material bodies would appear different from the actual geometry of space-time, because of the \scare{distortion} due to the mechanism of the adjustment. Two identical \scare{classic} atomic systems with different prehistories would probably differ in some small detail due to their interaction with the environment, and their spectral lines would be slightly shifted, so that classically, a spiraling charge should emit light of all colors. Emerging quantum theory had already made clear that the spectral identity of atoms revealed by experience cannot be explained in this framework. The fact that all atoms of the same type are exactly identical clearly cannot depend on an initial agreement established in the past, which has been \scare{preserved} since then, even though the atoms had encountered very different physical circumstances. It was more plausible to argue that they \scare{adjust} anew each time to a certain equilibrium value}. In the discussion following the debate Einstein that rehitherard, of the need of a coordinate $ds$. However, opposite raised by Pauli. That if the theory a theory of matter it would have been completely missing the point. Indeed, Einstein was that in front abandon the theory. It is interesting to notice Einstien reaction of that geometry of \spti was non-Euclidean. In this case, according to mode, the one should ahve check if this prediciton is correct. This was of course in principle impossible since there is non rod smaller than $x$. Indeed, Einstein that the model in which there is clear cut $ds$ was coordinated, had to be. Indeed, as ... ultimately the introduction of \rac as separte was a provisional claim. 

Just after Bad Nauheim Moritz Schlick, who was at that time the leading philosophical authority in relativity theory wrote to Einstein about Reichenbach's book complaining about his critique of conventionalism. \cop{Reichenbach scheint mir der Konventionslehre von Poincaré gegenüber nicht gerecht zu sein; was er apriorische Zuordnungsprinzipien nennt und mit Recht von den empirischen Verknüpfungs- prinzipien unterscheidet, scheint mir vollkommen identisch mit Poincarés „Konventionen“ zu sein und keine darüber hinausgehende Bedeutung zu haben.[2] R’s Anlehnung an Kant scheint mir genau betrachtet nur rein terminologisch zu sein}. Tha the very notion of coordination was challanged by at Bad Nauhem. The same point was made by Einstein in his lecutre of January, 

%The exchange of letters that insuded was essential to convert Reichenbach to conventionalism. n a leller to Schlick, Born defended this claim, expressing skepticism regarding contemporary allempts to explain matter a~ '"nothing but an accumulation of energy" ("nichts als eine Anhäufung von Energie"). He conceded that in future field-theoretical accounts, in which matter would be conceived of as singularities of the field, a '"spcc1al kmd of reality" ("eine besondere Art Realität") could be ascribed to those singularities (sec Max Bom to Moritz Schlick, 8 September 1920, NeHR, Vienna Circle Archive, Schlick Papers, lnv. 93 ). It was this series of problem that Einstein to his famous $G+P$ formula. That are by the motion of that Lorentz force law. However, that could not be uphold. Provisional in which rods and clocks and true one in which they play a fundamental role. The very notion of coordination that was fundamental for Reichenbach and Schlick was for Einstein at most provisional. Both in the infinitely small and in the cosmology. That the theory could then introduced concepts that were directly measurable, but that could as whole, if the theory to make proper predictions. E.g. the concept external pressure, to that of internal pressure is manomenter inside of a body. However, one the entire field of idrodynamics.  That the very role of Zuordnungsprinzipine was lost. 

\todo{Zurodnungsprinzipien}

%

%Reichenbach scheint mir der Konventionslehre von Poincaré gegenüber nicht gerecht zu sein; was er apriorische Zuordnungsprinzipien nennt und mit Recht von den empirischen Verknüpfungs- prinzipien unterscheidet, scheint mir vollkommen identisch mit Poincarés „Konventionen“ zu sein und keine darüber hinausgehende Bedeutung zu haben.[2] R’s Anlehnung an Kant scheint mir genau betrachtet nur rein terminologisch zu sein.
%


% , a full account of which was published in 1924 but was ?begun in the fall of 1920 and essentially completed in March 1923? (?im Herbst 1920 begonnen und im März 1923 im wesentlichen abgeschlossen?); see Reichenbach 1924, p. vii.  -->

\subsection{The Reichenbach Weyl's Correspondence}


%Hermann "Weyl kannte Reielõeulõaelõ persšulieh bereits seit 1920, dann beide uahuõeu an der NaturŸõsc1õeõ'veõ'sauuuluug iu Bad Nauheinõ, 23. Septeuõlmr 1920 und der Plõysiker-'1`aguug in .lena inõ Septeuõber 1921 teil. Das VeflõŠltuis der beiden darf als sehr di- 
In spite of the polemical remarks of Weyl's theories, Reichenbach sent a copy of his \emph{Relativitätstheorie und Erkenntnis Apriori} (Reichenbach 1920). \cop{After some weeks Weyl replied appreciate the sincerity of Reichenbach's remarks. He pointed out that he believed that due their divergent basic orientations in philosophy an agreement would have been difficult; however,  and decided to answer  \q{two remarks, which concern less the philosophical than the physical}. In particular for our goals it is interesting to consider the following remark}

\q{It is certainly not true, as you say on p. 73, that, for me, mathematics (!!, e.g. theory of the $\zeta$-function?) and physics are growing together into a single discipline. I have claimed only that the concepts in geometry and field physics have come to coincide \textelp{} Was nıeine erweiterte Relativitätstlıeorie betrifft. so kann ich nicht zugeben, daß da erkenntnislogisch die Sache irgendwie anders liegt wie bei Einstein. \textelp{}  Der Erfahrung wird durch die Annahme jener allgemeiner Metrik in keiner Weise vorgegriffen; dass die Naturgesetze, an welche die Wirkungsausbreituug in Äther gebunden ist, können ja von solcher Art sein, daß sie keine Streıfkenkrümmung zulassen. \textelp{} Wofür ich allein eintrete, ist dies: Die Integrabilität der Strekken\"ubertraguug (wenn sie besteht, ich glaubs uielıt. denn ich sehe nicht den geriugsteu zwiugeudeu Grund dafür) liegt uielıt im Wesen des metrischen Mediumsm, sondern kann nur auf einem besonderen Wirkungsgesetz berulıen. Ware die historische Entwieklung anders verlaufien, so seheint nıir wåire uienıand darauf verfallen. von vorn- herein gerade nur den Rieuıauuselıeu Fall iu Erwåígımg zu zielıeu. - *Nas die berüchtigte “Abhšingigkeit von der Vorgeschichte" betrifft, so habe ieh darüber wohl nıeine Ansicht deutlieh gemıg in Naulıeinı ausgesprochen. An der 4. Aufl. wird Sie wahrscheinlich vor al- lenı nıeine veränderte Stellımgnalnue zum Problem der Materie i}. The adjustment, and not of persistence. That is like to always vertical and not as the is always vertical like a roly-poly toy, not like a spinning top, get always back on the same.

%Geuau, wie "Einstein zeigen nuõ§, da§ aus der Dyuauõik des starren Kšrpers heraus eiu solches Verhalten folgt, da§ der Ma§stab innner dieselbe LŠnge hat. gemessen iu seinem ds, so uuõ§ ielõ zeigen, da§ er iumõeõ' das gleiche dnrelõ R = eonst uoruõieõ't.e ds hat.. Wie etwa "Einstein so gut als ielõ das zu nõaelõen lõŠ.tten, habe ieh am Selõlu§ uõeiner Arbeit " Feld uud Materie". Ann. d. l'lõys.'2" augedeutet.

%Lieber Herr Doktor,156 warum sind Sie so gegen die Besprechung von Neuauflagen? In ^Wey1s 4. Aufi. liegen allerhand wesentliche €nderungen vor, die bei der Bedeutung dieses Buches eine erneute Besprechung gerecht- fertigt hŠtten. Nuu wage ich nõich gar nicht mit der Bitte hervor,

%\footnote{For Eddington \scare{natural geometry} is exactly Riemannian and \scare{world geometry} was nothing but a conceptual scheme; On the opposite  Weyl insisted on the  the real \scare{aether geometry} was non-Riemannian and the \scare{body geometry} distorted by the mechanism of adjustment} presented his own affine theory, embracing  and the same strategy, was embraced even if with a different turn. This is a necessary condition for what is called affine geometry. It appears to express the condition that; the world is " fiat in its smallest parts" or that it possesses a definite tangent. However, beside that indeed, what he introduced metricity tensor: $\mathbf{K}_{\mu \nu, \sigma}$, defined as 

 %What we have sought ill not the geometry of actual space and time. but the geometry of the world-structure. which is the common basis of space and time and things. Thus Eddington that the real geometry was exactly Riemannian, and was nothing more than a graphical representation, like phase space.  And in March Einstein suggested a sort an intermediate way, a conformal theory, in which \rac are not are not measured. Einstein dislike is a measuring rod geometry, and at the same time real measuring rods behave differently from the ideal one \todo{(du pasquier)}, there was no reason to assume that this was dif
 
 Weyl's reaction seems to manifest that in spite of the attacks his theory was receivn frm this diferent sides, his attitude believed he still had some cards to play. had his how Weyl had introduced \scare{doubling the geometry}. As it turned out there was two different ways to proceed, to different way to save the the new geometry, and in general to to a more general affine connection.  A few days later on February 17 Weyl published the English description on Nature, that is that real ether non-Riemannian geometry was to the measuring rod geometry, that appears in reality. This fundamental imbalance, between in which however, the instrument that serve to measure lengths, differently, in which $k_\sigma$ is exactly Riemannian. Thus the \rac do not conform, that the real geometry of \st is non-Riemannian. A second approach, was pursued Arthur Stanley Eddington, that preferred to that only on the affine connection, without any metric. Is exactly Riemannian, in which $k_\sigma$, is exactly Riemannian, is only a graphical representation, comparable to the representation of with, to adaptation to the radius of curvature of the world, that enter into the dynamic equations determining the behavior of \rac. Introduce, a tensor split into two parts, and fiannly impose. Probably, unaware of both approaches Einstein presented in March, to avoid the \rac at all. Indeed, from one can define the Riemann, tensor contracted into the Ricci tensor, into two tensor, third option, that is a theory that completely renounced to the very existence of transportable measuring rods, on 17th March. In this way there was no concern about doubling the geometry, which seems to feel as unpleasent. In two published in  May (Weyl, 1921c) and July (Weyl, 1921d) return for different audiences. What is most important is that in the July paper Weyl also address in public Weyl and Erwin Freundlich criticisms\footnote{The reference is to \citealp{Reichenbach1920a} and \citealp{Freundlich1920}. The latter however does not mention Weyl}  

\q{From different sides, \footnote{The reference is to \citealp{Reichenbach1920a} and \citealp{Freundlich1920}} it has been argued against my theory, that it would attempt to demonstrate in a purely speculative way something \emph{a priori} about matters on which only experience can actually decide. This is a misunderstanding. Of course from the epistemological principle [aus dem erkenntnistheoretischen Prinzip] of the relativity of magnitude does not follow that the \textquotedblleft{}tract\textquotedblright{} displacement [Streckenübertragung] through \textquotedblleft{}congruent displacement\textquotedblright{} [durch kongruente Verpflanzung] is not integrable; from the principle that no \emph{fact} can be derived. The principle only teaches that the integrability \emph{per se} must not be retained, but, if it is realized, it must be understood as the \emph{outflow} [Ausfluß]\emph{of a law of nature }\citep[475; last emphasis mine]{Weyl1921b}}

At the end of the theory, e.g. electrons withc a certain cine. This can be used. However, that this hoeve this beaavior does not imply. However, this solution was unstatiscy having conpratioan, that real rods and clocks, contradict the real tranport of lenght inaccesble to especoe.

In September 1921, Pauli's encyclopedia article on relativity theory (which was finished in December, but underwent some improvements in April and May) was finally published, as part of the fifth volume of the \textit{Enzyklopädie der Mathematischen Wissenschaften}, and later as a book with an introduction by Pauli's mentor, Arnold Sommerfeld (Pauli, 1921). Pauli of two theories. The first one in which  ... Paragraph 65 amounts to a devastating epistemological \scare{deconstruction} of Weyl's theory in particular, a critique that was hard to dismiss.  In this layed town a fundamental objections. If Weyl's theory seeks to make predictions that are closely linked with the behavior of measuring rods and clocks, just like Einstein's theory, then the theory is clearly wrong. If one renounces this interpretation, as Weyl later suggests, then the theory loses its \scare{convincing power}, becoming just a mathematical scheme that furnishes only \scare{formal, and not physical, evidence for a connection between [the] world metric and electricity} (Pauli, 1921, 763; tr., 1958, 196).  Indeed, soon Pauli would embrace Eddington in which the true geometry of \spti, prefer to goemetrical represntatio, nd that ral geometru of ]rac.

\subsection{The Weyl-Reichenbach Appeasement}

%Sie liefert natürlich eine Bestätigung de Konventionalismus16 aber sie deckt auch jene Tatsachen auf, an denen auch der Konventionalismus nichts interpretieren kann. Besonders merkwürdig ist es, daß es möglich war, die starren Maßstäbe u. Uhren völlig zu eliminieren. 17 Ich konnte allein durch Benutzung von Lichtsignalen die ganze Metrik definieren. 1922

%18 Reichenbach 1922b, "Relativitätstheorie und absolute Transportzeit"


Im Sept. 1921 trug ich schon den ersten Bericht Ÿber die Axiomatik auf dem Physikertag in .lena vor.3 Ich hatte damals gro§en Erfolg; aber niemand ist damals auf den Gedanken gekommen, mich in eine angemessene Stelle zu berufen. Ich blieb in Stuttgart sitzen. Niederschrift und Ausbau im Winter 1921 /22. However the relationship were still excellent. In the meantime on 24 March 1922 Erwin Freundluch sent to Einstein "die Druckbogen einer kritischen Untersuchung von Reichenbach auf dessen Wunsch".  Coudl not take into account just after having presented also. The long Review on Relativity, was read and commented by Einstein, presentation of a very balanced of Weyl theory. Reichenahc Genauere Behandlung der allg. Th. inõ Aug.-Spt. 1922. Vortrag darŸber iu Leipzig, Sept. 1922. FrŸhjahr u. Sommer 1922 entstanden Logos-Aufsatz u. Aufsatz iu revue philoslophiquel de la France, and Logos Aufsatz für Logos. This was a long details. A more balanced. Weyl took exception to the fact, that Einstein has simply condoned [einfach hingenommen] the univocal transportability of natural measuring-rods [eindeutigen Uebertragbarkeit natürlicher Maßstäbe]. He does not want to dispute the axiom of the Riemannian class for natural measuring rods; he wants only to urge that the validity of this axiom, being not logically necessary, \emph{``is understood as an outflow }[Ausfluß]\emph{ of a law of nature}''. That this was a fundamental contribution Admittedly, Weyl was able to explain the univocal transportability of natural measuring-rods only in a very incomplete way}. But the only fact that he had tried to follow this path, regardless of its empirical correctness, was a genial advance [genialer Vorstoß] in the philosophical foundation of physics (\citealp[367f.]{Reichenbach1921}).

Reichenbach now distinguishes two ways to approach Weyl's theory, a more moderate, however, he also fundamentally incorporate Pauli's two theory interpretation. One way was to are measured by rods and clocks, thus that change by transport. This is however, not true. The second approach that was true the do not follow the Weyl connection, but behave in a Riemannian way. \q{er definiert einen ideellen Prozeß der Maßstabs-Uebertragung, der jedoch mit dem Verhalten realer Maßstäbe nichts zu tun hat}. 

(a) Damit verliert jedoch die Theorie ihren überzeugenden Charakter und kommt einem mathematischen Formalismus bedenklich nahe, der um eleganter mathematischer Prinzipien willen die Physik unnötig kompliziert; und aus diesem Gedanken heraus wird die Weylsche Theorie von Physikern (besonders auch von Einstein) sehr zurückhaltend betrachtet. The is very similar to Pauli statement which very probably. Althoug, that Pauli was a measure influece on Reichenbach.

(b) Weyl had then to explain \q{Seine Erklärung, nach der die eindeutige Uebertragbarkeit durch Einstellung der Maßstäbe auf den Krümmungsradius der Welt« erfolgt, ist im wesentlichen nur ein andrer sprachlicher Ausdruck für den vorliegenden Tatbestand, keine Zurückführung auf ein allgemeineres Gesetz. Insbesondere hat diese \scare{Einstellung} nichts zu tun mit seiner kongruenten Verpflanzung, so dass diese physikalisch leer bleibt}. However, appreciate the very distinction, between will play a central role. Indeed, the that need an explanation why have the same lenght. HOwever, Wel explaintion is more.

Thus, to indeed, would, however, there is possibly,

\q{Die phi 1 o so phi s c h e Bedeutung der Weylschen Entdeckung besteht deshalb darin, daß sie bewiesen hat, daß ein Abschluß des Raumproblems auch mit dem Riemannschen Raumbegriff nicht gegeben ist. Wollte also die Erkenntnistheorie heutP- die Behauptung der transzendentalen Aesthetik Kants dahin erweitern, daß die Geometrie der Erfahrung auf jeden Fall wenigstens von Riemannscher Struktur sein muß, so wird sie durch die Weylsche Theorie daran zurückgehalten. Denn daß der Weylsche Raum wenigstens für die Wirklichkeit m ö g 1 i c h ist, läßt sich nicht bestreiten. Man darf nicht einmal glauben, daß mit der Weylschen Theorie nun die höchste Stufe der Allgemeinheit erklommen sei. Einstein hat gezeigt (14), daß man die Weylsche Forderung der Relativität der Größe auch befriedigen kann, ohne von dem Weylschen Meßverfahren Gebrauch zu machen. Danach wurde von Eddington (15) wieder eine Verallgemeinerung entwickelt, von der die Weylsche Raumklasse nur ein Spezialfall ist, und die Eddingtonsche Raumklasse ist wieder als Spezialfall in eine allgemeinere eingegangen, die von Schouten (63) gefunden wurde. Der Vorzug der Schoutenschen Theorie besteht darin, daß hier die Bedingungen angegeben werden, unter welchen die entwickelte Raumklasse die allgemeinste ist; es sind sehr allgemeine Bedingungen, wie Differenzierbarkeit und ähnliches. Aber eine schlechthin allgemeinste Raumklasse gibt es natürlich nicht; und die Geschichte des mathematischen Raumproblems mag der Erkenntnistheorie eine Lehre sein, niemals schlechthin allgemeine Behauptungen aufzustellen. Es gibt keine allgemeinsten Begriffe}.

That Weyl did not agree, with Some technical details, however concering that did not in the attacks \q{Den Plan, starre Maßstäbe mit meiner Verpflanzımg zu identifizieren, lıabe ich aufgegeben. weil ich ilm nie gehabt lıabe: sondern ich war überrascht, als ich salı. daß Plıysiker das in nıeine Worte hineininterpretiert hatten. Genau, wie Einstein zeigen muß, daß aus der Dynamik des starren Körpers lıeraus ein solelıes Verlıalteu folgt, daß der Maßstab iumıer dieselbe Länge hat, gemessen iu seinenı (Is, so muß ich zeigen. daß er inmıer das gleiche durch $R = const$. normierte ds lıat. Wir etwa Einstein so gut als ich das zu maclıen hätten, habe ich am Schluß meiner Arbeit " Feld und Materie”. Ann. d. Plıysik angedeutet.}. Thus, neither, in Einstein nor in the identification is legitimate. 

Two critique, one that epistemological, better than other geometries (there is no reaso, eliminat the asumm), that thus that one should give ... however the theory, formalism, that iintrisc superiority of Weyl geometry is not even true, an there are 27 different connections among which one can chose.  A second aspect is that Weyl requires an, there is than as at least a good overview of some differential geometry, not Weyl but also Eddington and even Schouten. is even more surprising the work of Schouten 1921 classifies 18 possible linear affine connections (\scare{Übertragungen}) numbered as I,..., VI a?c. In Schouten 1922 he consider by he improved he classification trhee tensor 27 possible connection. All the further less general cases of linear connections are obtained by introducing restrictions to such quantities (for more details see Vizgin 1994, 184, Goenner 2004). This classification was Schouten point of view not only Schouten that Weyl was not but one could think in which. The third tensor which never plaed the role, the pwer can be. In which the tensor of asymmetry vanish symmetric, but the tensor of non metricity does not vanish. That more general even non-symmetrical connections (IVc). In which the tensor of non metricity vanish, but the tensor of asymetry does not, which will be the geometry used by Reichenbach  That even Schouten most general linera, lieanr But ieven the condition that a connection is linear was not necessary. Start from a general affine connection, and then restrict the possibilities. However, there was no particular. 

1922 French article, that was empirical fact that \rac behave in a Riemannian way. Thus this resctriv that to Riemannian geometry. However, which Riemannian geomtry remains a question of choince. Idneed, one can imm  \scare{Darrigol classes}. However, was on a fact, that in nature there are. The choice between geometry is conventional; indeed there might be didferent. By the there are no differential forces. 











%lists 18 different linear connections and classifies them invariantly. The most general connection is characterized by two fields of third degree, one tensor field of second degree, and a vector field. These fields are the symmetry tensor $S_{\lambda \mu}^{v}$ the tensor of $Q_{\lambda \mu}^{v}$. One,

%. and a vector C „_ which follows from Cfμ = C μ ôff while C ,f μ = I`§μ+l`í';1, if l" stands for the connection for tangent vectors and 1" ' for the con- nection for linear forms. Torsion is defined by Sfμ = 1/2(l`§μ - FZÄ), non-uıflicity by Vμg“ = Qt“. Furthermore, on page 57 we find: “The general connection for n = 4 at least theoretically opens the door for an extension of Weyl's theory. For such an extension an invariant affixation of the connection is needed, because a physical phenomenon can correspond only to an invariant expression."



%Moreover, expressed his, howeveer, that privildege respect to Eddington's theory. Weyl probably, Das alte MS wurde vš\"ollig umgestossen. 


%[Berlin,] Mittwoch [20 September 1922]

\section{Geometrization: Reichenbach's Correspondence with Einstein}

By the end of the 1922, Einstein had to exploit that Eddington's theory was more. Semi-metrical theory, that is was the reason why Einstein. Indeed, was semi-metrical and indeed, it was in general more effective to go in where the affine connection has not physical meaning at all 1923. That the theory, the goal was to construct an action principle. And indeed, found in first approximation. The manuscript consists of five pages. The first two pages were written on the back of a fragment of an unidentified typescript. The third, fourth, and fifth pages were written on the back of the beginning of a typescript of Reichenbach’s contribution to the meeting of German physicists in Jena, 18-24 September 1921 (Reichenbach 1921). The reason. Weyl's theory was a semi-metrical theory. Vectors have length at one point, but by transportation. In Eddington's theory was purely affine, letting the connection copletly indetermined. That the latter was indeed, without any physical meaining. Was more appropriate. That one serve search for the field equations. To search for the indeed, and variation with respect to 

Indeed, will make Pauli's famous reaction. In a latter to in 1923. Semte er.  Eddington's attitude was in his 1923 to Pauli, As again Pauli pointed out to \scare{graphische Darstellung}, but not a natural geometry like that of \rac. That soon appear in the 1925 German translation of   Pauli's review indeed, presented the same argument.

In a long letter to Eddington, why this was dangerous. The second si that even if we can define, that this particular geometrical structure. There was not start which is undefined, that the graphical presentation was precisely the reason for rejecting the theory. Again Reichenbach's critique would follow Pauli's ideas. From Hans Reichenbach Stuttgart, 19 April 1923 Asks for help in finding a publisher for Reichenbach 1924. ALS. [20 079]. 50. Ilse Einstein to Hans Reichenbach Berlin, 12 May 1923 Informs him that AE has not seen Reichenbach’s letter since he already left for the Netherlands. She no longer forwards AE’s mail since he brings it all back unopened. She assures Reichenbach that she will give his letter to AE upon his return. AKS. [87 944]. 122. From Hans Reichenbach Stuttgart, 10 July 1923 Thanks AE for Abs. 89. Is sorry that the Academy did not agree to print his manuscript. Asks whether this was due to financial or other reasons. Springer cannot accept his suggestion that the \textit{Notgemeinschaft} cover part of the printing cost. Verlag Witwer, on the other hand, has agreed to publish the work with support from the \textit{Notgemeinschaft}. Enclosed sends the request to the \textit{Notgemeinschaft} and asks to forward it to Fritz Haber and to put in a good word for him personally. Asks to mail back the manuscript. ALS. [20 082]. During 1924 Reichenbach finally managed to published, his Axiomatic in which. the negative review finally convinced him of the debacle, Weyl. In 1925

At about the same time Reichenbach started to work. Of course Einstein had by that time lost this was reason of the critique that required, and only a the theory as whole could serve as a that in 1925, getting back increasingly rationalistic. That Einstein was in 1925, that was already became skeptical again was to 1925 non-symmetric affine connection. In 1924 publishing a review of ... was clear that his past position was wrong. And the appear on several. The publication of Eddington 1925 was probably the starting point. However, Reichenbach also borrowed betweeh graphical representation and natural geometry. A proper geometrical interpreation.

During those same months, Reichenbach, despite the support of Max Planck, was struggling to obtain his \german{Umhabilitation}\footnote{The process of obtaining the \latin{venia legendi} at another university} from Stuttgart to Berlin in order to be appointed to a chair of natural philosophy that had been created there \citep{Hecht1982}. On \datef{16}{3}{1926}, Reichenbach sent a letter to Einstein in which, after discussing his academic misadventures, he remarked on the new \scare{metric-affine} theory \citep{Einstein1925a}:

\qt{I have read your last work on the extended Rel. Th\footnote{\cite{Einstein1925a}} more closely, but I still can't get rid of a sense of artificiality which characterizes all these attempts since Weyl. The idea, in itself very deep, to ground the affine connection independently of the metric on the $\Gamma^{i}_{kl}$ alone, serves only as a calculation crutch here in order to obtain differential equations for the $g_{ik}$ and the $\varphi_{ik}$ and the modifications of the Maxwell equations which allow the electron as a solution. If it worked, it would of course be a great success; have you achieved something along these lines with Grommer? However, the whole thing does not have the beautiful convincing power \origins{Ueberzeugungskraft} of the connection between gravitation and the metric based on the equivalence principle of the previous theory}{Ich habe jetzt Ihre letzte Arbeit zur erweiterten Rel. Th. genauer gelesen, aber ich kann auch da das Gefühl des Künstlichen nicht los werden, das allen diesen Versuchen seit Weyl anhaftet. Die an sich doch sehr tiefe Idee, den affinen Zusammenhang unabhängig von dem metrischen zu begründen allein auf die $\Gamma^{i}_kl$ dient doch schließlich nur als Rechenknüppel, um Differentialgleichungen für die $g_{ik}$ und $\varphi_ik$ zu bekommen und solche Abänderungen der Maxwellschen Gleichungen zu bekommen, die das Elektron als Lösung zulassen. Wenn das geht, ist es natürlich ein großer Erfolg; haben Sie eigentlich mit Herrn Grommer etwas in dieser Richtung errichtet? Aber die ganze Sache hat doch nicht die schöne Ueberzeugungskraft, wie die auf das Aequivalenz-Prinzip gestützte Verknüpfung von Gravitation und Metrik in der früheren Theorie}[\letter{Reichenbach}{Einstein}{16}{3}{1926}][20-83][EA]

Reichenbach expressed skepticism early on towards Weyl's theory \citep[73]{Reichenbach1920a}. Even if he partly retracted some of his concerns \citep[367--368]{Reichenbach1921}, he still felt that the theory did not have the same \scare{convincing power} (\german{Überzeugungskraft}) of general relativity \citep[367]{Reichenbach1921}, in which the identification of the $g_{ik}$ with the gravitational potentials was solidly anchored in the principle of equivalence. Perhaps it is not a coincidence that Reichenbach uses the very same turn of phrase in this letter. Einstein's theory introduces the affine connection independently of the metric. However, it does not attribute any physical meaning to the former; the separate variation of the metric and connection was nothing more than a \scare{calculation device} to find the desired field equations. Reichenbach, however, was ready to revise his negative judgment if Einstein's theory delivered the \scare{electron}. At the end of the paper \citep{Einstein1925a}, Einstein had in fact claimed that he was working with his assistant Jakob Grommer on the problem of establishing whether the theory allows for \qt{the existence of singularity-free, centrally symmetric electric masses}{die Existenz singularitätsfreier zentralsymmetrischer elektrischer Massen} \citep[419]{Einstein1925a}. For Einstein this was a fundamental criterion for the viability of a unified field theory \citep[cf.~e.g.,][]{Einstein1923g}.

On \datef{20}{3}{1926} Einstein replied that he warm-heartedly agreed with Reichenbach's \scare{$\Gamma$-Kritik}: \qt{I have absolutely lost hope of going any further using these formal ways}{Ich habe jetzt jede Hoffnung aufgegeben, auf diesem formalen Wege weiter zu kommen}; \qt{without some real new thought}{he continued}{it simply does not work}[Ohne einen wirklich neuen Gedanken geht es nicht][\letter{Einstein}{Reichenbach}{20}{3}{1926}][20-115][EA]. Einstein's reaction reflects his disillusion with the attempts to achieve the sought-for unification of gravitational and electromagnetic field via some generalization of Riemannian geometry. He would have probably been less ready to embrace Reichenbach's critique if he had knew what the latter exactly had in mind (see next section). However, Reichenbach was of course pleased by Einstein's endorsement. On \datef{31}{3}{1926} he revealed that his remarks were not extemporary, but were the fruit of a more thorough consideration of the topic that he had jotted down at the time:

%, after making some comments about his academic situation,

\qt{I'm of course very glad that you agree with my $\Gamma$-critique. I have now made a few reflections on the topic, which seem to me to prove that the Weylean thought, although good mathematically, does not bring about anything new physically. The geometrical interpretation of electricity is only a visualization, which in itself still does not say anything, and can also be realized in the original relativity theory. I have attached the note and would be grateful if you could give it a look \letter{Reichenbach}{Einstein}{24}{3}{26}][20-085][EA]}{Daß Sie meiner $\Gamma$-Kritik zustimmen, hat mich sehr gefreut \textelp{} Ich habe jetzt eine kleine Ueberlegung zu dieser Sache durchgeführt, die mir zu beweisen scheint, warum der Weylsche Gedanke, so gut er mathematisch ist, nichts Physikalisch Neues bringt. Die geometrische Deutung der Elektrizität ist doch nur eine Veranschaulichung, die selbst noch garnichts besagt, und die sich ebenso in der ursprünglichen Relativitätstheorie durchführen läßt. Ich lege Ihnen die Note ein und wäre Ihnen für Durchsicht sehr dankbar}

Reichenbach attached to this letter a typewritten note. As we shall see, far more was at stake in it than a critique of Weyl's theory (which was generally considered a dead horse at the time). Reichenbach intended to call into question the very idea that, since general relativity has \scare{geometrized} the gravitational field, the obvious next move should be to try to \scare{geometrize} the electromagnetic field.

\todo{See Reichenbach}


%\qt{You've run over to the theoretical physicists and moreover at a bad spot. Of course I immediately found some flies in the ointment. First of all, your approach $\varphi^{\tau}_{\mu\nu}=-g_{\mu\sigma}f_\nu^\tau\frac{\partial{f^{\sigma\rho}}}{\partial{x_\rho}}$ is really arbitrary. \textelp{} Your equations of motion do not have any physical meaning, since they describe the behavior of matter only for a value of the relationships between electrical and ponderable density. Finally, your theory does not connect electricity and gravitation, since there are no mathematically unified field equations that provide the field law for gravitation and electromagnetism simultaneously; it does not even provide a connection between electricity and gravitation in the sense that one could infer from the theory which electromagnetic quantities produce the gravitational field. I would not publish it; otherwise, what happened to me will happen to you: you'll have to disown your children}{Sie sind also unter die theoretischen Physiker gegangen, und zwar an einer bösen Stelle. Ich habe natürlich in der Suppe gleich ein paar Haare gefunden. Erstens ist der Ansatz $\varphi^{\tau}_{\mu\nu}=-g_{\mu\sigma}f_\nu^\tau\frac{\partial{f^{\sigma\rho}}}{\partial{x_\rho}}$ recht willkürlich. Zweitens gehört zu Ihren $\Gtmn$ keine Metrik; da ist es unnatürlich, einen Summanden $\gamma$ von $\Gamma$ einer Metrik zuzuordnen. Drittens hat Ihre Bewegungsgleichung deswegen keinen physikalischen Sinn, weil sie ja nur für einen Wert des Verhältnisses zwischen elektrischer und ponderabler Dichte das Verhalten der Materie darstellt. Endlich ist die Theorie insofern keine Verbindung von Elektrizität und Gravitation, als keine mathematisch einheitliche Feldgleichung da ist, welche gleichzeitig das Feldgesetz der Gravitation und das des Elektromagnetismus liefert; sie liefert auch nicht eine Verbindung zwischen Elektrizität und Gravitation in dem Sinne, dass das aus ihr hervorginge von was für elektromagnetischen Grössen das Gravitationsfeld erzeugt wird. Ich würde dies nicht publizieren; sonst wird es Ihnen gehen wie mir, der seine eigenen Kinder verleugnen muss}[\letter{Einstein}{Reichenbach}{31}{3}{1926}][20-116][EA]

Einstein must have immediately read or at least glanced at Reichenbach's attempt at providing a unified field theory, and he replied a few days later on \datef{31}{3}{26}. His initial reaction to the theory was not very encouraging. Einstein deconstructs Reichenbach's theory piece by piece. (1) The definition \eqref{eq:dsd} of the $\varphi\tmn$ does not seem to have any physical motivation. (2) A Riemannian metric determines an affine connection; however, in general this is not true the other way around \citep[cf.\ e.g.,][9]{Einstein1923}. If Reichenbach started from a general displacement space $\Gtmn$ and defined it via parallel transport independently from the metric \gmn, then he should not have reintroduced the $\gmn$ surreptitiously by defining the \gtmn as the negative of the Christoffel symbols of the second kind (which are expressed in terms of the \gmn) (see eq.~\ref{eq:cs}) (3) Reichenbach's equations of motion can be valid only for a certain charge-density-to-mass-density ratio $\rho/\mu$ (or, in the case of particles, a certain charge-to-mass ratio $e/m$). In a given displacement, there is only one straightest line passing through a point in a given direction, but different test particles with different charge-to-mass ratios accelerate differently in the same electric field. Thus they cannot all travel on the same straightest line (see below in \cref{eq}). After all, this is the precise difference between gravitational and non-gravitational forces. Finally, (4) in the note, the \gmn and $f\mn$ are governed respectively by the well-known Einstein and Maxwell equations; thus the theory not only fails to yield a single set of field equations governing both the gravitational and electromagnetic fields, but it does not even bother to supplement the gravitational field equations with electromagnetic terms so that they contain the gravitational effect of the electromagnetic field. That is, the theory does not even yield a geometrization of the Einstein-Maxwell theory. Reichenbach replied by return post on \datef{4}{4}{1926}. There are two aspects to his response, which we will deal with separately for the sake of clarity. First, Reichenbach replied to Einstein's technical objections, insisiting. Concerning 1), the awkward definition of one of the summands of the $\varphi\tmn$ in eq.~\eqref{eq:dsd}, Reichenbach did not hide that his theory was an operation of \scare{reverse engineering}; and for this reason anything goes, even a cheap trick like the one he used in the note.  Objection 2), for Reichenbach, was the consequence of Einstein's hasty reading. Reichenbach did not start from the displacement and then define the tensor $G\mn$ in terms of the latter, as Eddington did, but the other way around. Thus the metric was not obtained as a by-product, but was introduced from the beginning. To make his point, Reichenbach explains to Einstein the geometrical structure he resorted to, a metrical space, in which the symmetry of the lower indices of the \Gtmn is dropped. In Riemannian space the operation of displacement delivers the same comparison of length as the metric. From this the \qt{Riemannian values $\christoffel{\mu}{\nu}{\tau}$ of the \Gtmn}{Rimannschen Werte \christoffel{\mu}{\nu}{\tau}[\letter{Reichenbach}{Einstein}{4}{4}{1926}][20-086][EA] für die \Gtmn} follow, if one assumes that the latter are symmetric in the lower indexes $\mu$ and $\nu$. If one drops this assumption, then \qt{one has at one's disposal a somewhat more general \Gtmn}{Läßt man dies weg, so hat man etwas allgemeinere \Gtmn zur Verfügung}[\letter{Reichenbach}{Einstein}{4}{4}{1926}][20-086][EA]. One can then define the operation of displacement so that charged mass-points move on auto-parallel lines, which in general do not coincide with lines of extremal length: \qt{in this way one obtains a full geometrical visualization of the law of motion}{Auf diese Weise erhält man eine vollkommene geometrische Veranschaulichung des Bewegungsgesetzes}[\letter{Reichenbach}{Einstein}{4}{4}{1926}][20-086][EA]. Thus Reichenbach also makes it clear that he was not concerned with finding the field equations, as Einstein seems to have implied; he was only trying to reinterpret the equations of motion geometrically. Replying to objection 3) Reichenbach insists that these equations are valid for the motion of a body with unit mass \emph{and arbitrary charge}. If one rewrites the tensorial part of the displacement as $\varphi\tmn=-\rho f^\tau_\nu u_\mu$, one sees that the charge density $\rho$ (but not the mass density $\mu$) appears as a parameter. Setting $\mu=1$, Reichenbach argues, is no worse than setting $d s=1$\footnote{See below footnote~\ref{11}}. So Reichenbach seems to have interpreted Einstein's objection as a misunderstanding. However, the validity of Reichenbach's equations of motion for arbitrary charge might have been precisely the severe flaw that Einstein envisaged in his approach, as Reichenbach himself possibly realized later (see below \cref{nv1} and \cref{nv}).\footnote{I thank \hide{Dennis Lehmkuhl} for a discussion on this point}

The second aspect of Reichenbach's defense is even more important for properly understanding his philosophical goals. Einstein misunderstood the spirit of the typescript. Reichenbach makes clear that the physicists should in no way think that he had some \qt{secret physical intention}{eine geheime physikalische Absicht} (\letter{Reichenbach}{Einstein}{4}{4}{1926}[20-086][EA]). Thus, Reichenbach recounted to Einstein why he decided to write the note. He was working on a philosophical presentation of the problem of space (see below in \cref{RZL1926}), and of course he felt compelled to add a chapter about \scare{Weyl space}, or more generally about attempts to \scare{geometrize} the electromagnetic field by using some generalization of Riemannian geometry: \qt{Thereby I wondered what the geometrical presentation of electricity actually means}{Dabei überlegte ich mir, was die geometrische Darstellung der Elektrizität eigentlich bedeutet}[\letter{Reichenbach}{Einstein}{4}{4}{1926}][20-086][EA]. 

Reichenbach concluded that such alleged geometrizations were actually only \scare{graphical representations} (\german{graphische Darstellungen})---an expression he clearly borrowed from \citet[294ff.]{Eddington1925a}. They were comparable to the account of the \qt{Lorentz transformations as rotations in Minkowski space}{Lorentz transformation als Drehung im Minkowskiraum}[\letter{Reichenbach}{Einstein}{4}{4}{1926}][20-086][EA], which is only a formal analogy. To prove his point, Reichenbach decided to construct an \german{Abbildung} or mapping of the Einstein-Maxwell theory onto a non-Riemannian space, \qt{\myemph{without any change of its physical content}}{ohne jede Aenderung ihres Physikalischen Inhaltes}[\letter{Reichenbach}{Einstein}{4}{4}{1926}][20-086\me][EA]. 

Reichenbach was even more ambitious. He aimed to present a geometrical transcription (\german{Umschreibung}) that was in some respects better than the one provided by Weyl and his successors, including Einstein. Reichenbach's geometrical interpretation, he insisted, had \qt{the advantage over other geometrical representations in that \emph{the operation of displacement possesses a physical realization \textins{Realisierung}}}{Diese Umschreibung hat aber vor anderen geometrischen Darstellungen den Vorteil, daß sie für die Verschiebungsoperation eine physikalische Realisierung besitzt}[\letter{Reichenbach}{Einstein}{4}{4}{1926}][20-086\me][EA], namely, the velocity-vector of charged mass particles of unit mass. In Eddington's parlance\footnote{See previous footnote} it is a \scare{natural geometry}.

This point is essential to Reichenbach's argument. It was precisely because his toy-geometrization was not envied by its more titled competitors that Reichenbach believed himself to be in an excellent position to \qt{attack the view that with a geometrical presentation of electricity, one would already gain something}{Ich wollte mich eben mit meiner Darstellung gegen die Auffassung wenden, als ob mit der geometrischen Darstellung der Elektrizität an sich schon etwas gewonnen wäre}[\letter{Reichenbach}{Einstein}{4}{4}{1926}][20-086][EA]. For such a geometrical interpretation of electromagnetism to become a physical theory as successful as general relativity, more was required than a mere geometrization. It called for something \emph{new}:

\qt{If one succeeds in establishing unified field equations that admit the electron as a solution, this would be \myemph{something new}. To this end one should do something more than establish a simple formal pooling \origins{Zusammenfassung} of the Maxwell eq.~and the gravitational equations; these eq.~\myemph{should be changed} in their content. This is the problem on which you are working and of course also what Weyl and Eddington meant. \myemph{But the geometrical representation of electricity in itself does not lead to this goal}. It can at most be an aid \origins{Hilfsmittel} to guessing the right equations; maybe what looks most simple from the standpoint of Weyl geometry, also happens to be correct. But this would be only a coincidence\label{zufall}. \textelp{} Inasmuch, however, as the present theories do not provide the electron as a solution, they also provide nothing more than a simple transcription \origins{Umschreibung} of the old Th.~of Rel}{Würde es gelingen, einheitliche Feldgleichungen aufzustellen, die das Elektron als Lösung zulassen, so wäre das etwas physikalisch Neues. Dazu müßte aber etwas anderes gemacht werden als eine bloß formale Zusammenfassung der Maxwellschen Gl. und der Gravitationsgleichungen, diese Gl. müßten in ihrem Inhalt \emph{geändert} werden. Dies ist ja auch das Problem, an dem Sie arbeiten, und das natürlich auch Weyl und Eddington meinen. Aber die geometrische Darstellung der Elektrizität führt an sich gerade noch nicht zu diesem Ziel. Sie kann höchstens ein Hilfsmittel sein, die richtigen Gleichungen zu raten, vielleicht ist das, was unter dem Gesichtspunkt der Weylschen Geometrie das einfachste erscheint, zufällig auch das physikalisch Richtige. Aber das wäre eben ein Zufall. Ich wollte mich eben mit meiner Darstellung gegen die Auffassung wenden, als ob mit der geometrischen Darstellung der Elektrizität an sich schon etwas gewonnen wäre. Solange aber die vorliegenden Theorien das Elektron nicht als Lösung ergeben, Leisten sie auch nicht mehr als meine einfache Umschreibung der alten Rel. Th}[\letter{Reichenbach}{Einstein}{4}{4}{1926}][20-086\me][EA]

Thus, Reichenbach argued that the geometrical interpretation of a physical field can only be successful if it leads to a \scare{change} in the equations and does not simply rewrite in geometrical terms the equations that are already known. One could object that Weyl, Eddington and Einstein's theories also changed the equations and did not simply rewrite them. However, Reichenbach seems to consider the derivability of solutions that correspond to the electron as a litmus test for a real change in the field equations. Maxwell's field equations are valid in free space and cannot explain why the separate, equally charged parts do not fly apart without introducing a non-electromagnetic cohesion force (the so-called Poincar\'{e} stress). On the other hand, Einstein's field equations, in their original form, do not entail any effect of gravitation on charge and cannot provide the cohesion force. It is only by changing the currently available field equations that it would become  possible to establish a connection between \gmn and $f\mn$, thereby assuring the equilibrium of the electron.

To fully understand Reichenbach's stance on this issue, one must keep in mind that in a paper published in April \citep{Reichenbach1926} he expressed strong skepticism about the possibility of solving the problem of the \scare{grainy} structure of matter and and most all of the \qt{proper quantum-riddle}{eigentlichen Quantenrätsels}[][424][Reichenbach1926] in a field-theoretical/geometrical context. In Reichenbach's view, the \qt{casuality \origins{Zufälligkeit}}{Zufälligkeit}[][424][Reichenbach1926] of the \scare{quantum jumps} (the transition between orbital energy levels in Bohr's atom) suggests that the problem should be tackled from a different angle, by considering whether the very notion of causality in physics should be replaced by that of probability \citep[424]{Reichenbach1926}. After all, one should appreciate Reichenbach's clairvoyance, considering that Max Born's paper \citep{Born1926a} on the statistical interpretation of the wave function appeared only in June\footnote{For Reichenbach's take on the emerging quantum mechanics at this time, see the posthumously published manuscript, \cite{Reichenbach1926e}}.

Reichenbach offered to send Einstein the corresponding epistemological sections of the text on which he was working (possibly \S50 of the \Ap). In a letter from \datef{8}{4}{1926} Einstein did not comment on this offer, but his reaction to Reichenbach took a different tone. Even if Einstein did not reply to Reichenbach's more technical remarks, Reichenbach's philosophical point clearly resonated with him:

\qt{You are completely right. It is incorrect to believe that \scare{geometrization} means something essential. It is instead a mnemonic device \origins{Eselsbrücke} to find numerical laws. If one combines geometrical representations \origins{Vorstellungen} with a theory, it is an inessential, private issue. What is essential in Weyl is that he subjected the formulas, beyond the invariance with respect to \textins{coordinate} transformation, to a new condition (\scare{gauge invariance})\footnote{That is, invariance by the substitution of $g_{ik}$ with $\lambda g_{ik}$ where $\lambda$ is an arbitrary smooth function of position \citep[c\f][468]{Weyl1918}. Weyl introduced the expression \scare{gauge invariance} (\german{Eichinvarianz}) in \cite[114]{Weyl1919a}}. However, this advantage is neutralized again, since one has to go to equations of the 4. order%
%
\footnote{Cf.~\cite[477]{Weyl1918}. Einstein regarded this as one of the major shortcomings of Weyl's theory; see \letter{Einstein}{Besso}{20}{8}{1918}[\VD{8b}{604}][CPAE], \letter{Einstein}{Hilbert}{9}{6}{1919}[\VD{9}{58}][CPAE]},%
%
which means a significant increase of arbitrariness}{Lieber Herr Reichenbach\\ Sie haben vollständig recht. Es ist verkehrt zu glauben, dass die \scare{Geometrisierung} etwas Wesentliches bedeutet Es ist mir eine Art Eselsbrücke zur Auffindung numerischer Gesetze. Ob man dann mit einer Theorie \scare{geometrische} Vorstellungen verbindet, ist unwesentliche Privatsache. Das Wesentliche bei Weyl liegt darin, dass er die Formeln neben der Invarianz bezüglich Transformationen einer neuen Bedingung (\scare{Eichinvarianz}) unterwirft. Dieser Vorteil wird aber wieder neutralisiert dadurch, dass man zu Gleichungen 4. Ordnung übergehen muss, was ein beträchtliches Wachsen der Willkür bedeutet.\\Mit besten Grüssen \\Ihr A. Einstein}[\letter{Reichenbach}{Einstein}{8}{4}{1926}][20-117][EA]

Recently the importance of this letter has been emphasized in Einstein scholarship \citep{Lehmkuhl2014}. It is the first instance where Einstein explicitly claims that general relativity did not geometrize gravitation, thus suggesting a very different interpretation of the achievement of the theory than what we are used to. The geometrization was only a means to the end of finding the field equations, which are \scare{numerical laws}. It is worth noticing that Einstein goes further in claiming that even Weyl's theory should not be seen as an attempt to \scare{geometrize} the electromagnetic field. The core of Weyl's theory consists in the formal requirement of \scare{gauge invariance}%
%
%
 , which, however, led to equations where the choice of the Lagrangian becomes non-unique. For our purposes it is interesting that Einstein not only endorsed Reichenbach's claim that a \scare{geometrization} is not an essential achievement of general relativity, but also questioned the meaning of the notion of \scare{geometrization}, and for that matter the very notion of \scare{geometry} \citep{Lehmkuhl2014}. This latter step was not taken by Reichenbach, who preferred to speak of general relativity as a \scare{geometrical interpretation of the gravitational field}, albeit not a \scare{geometrization} of the latter.

On \datef{26}{5}{1926} Reichenbach might have possibly have presented this improved version of the note in Stuttgart at the Gauvereinstagung of the Deutsche Physikalische Gesellschaft (the regional meeting of the German Physical Society). The abstract of this presentation was published under the title \citetitle{Reichenbach1926d} \citep{Reichenbach1926d}. It is worth quoting at length, since it constitutes a good summary of what Reichenbach's theory looked like after his correspondence with Einstein:

\qt{The meaning of Weyl's extension of the type of space is formulated such that Weyl recognized the independence of the operation of displacement and of the metric. The application of the extended type of space to physics is however characterized by a certain arbitrariness because it remains open to finding certain objects that behave like the operation of displacement. It is shown that these objects are the velocity vectors of electrically charged mass points. With the aid of this coordination, it is possible to interpret gravitational and electrical phenomena as expressions of the geometry of a Weylean space, so that electricity finds a geometrical interpretation \myemph{in the same sense as gravitation}. The remarkable thing here, however, is that this presentation does not change the content of Einstein's theory of gravitation at all; the geometrical interpretation is only a different parlance, which does not entail anything new physically. Of course this geometrical interpretation of electricity cannot solve the problem of the electron, because it cannot achieve anything more than Einstein's theory. The goal of this investigation was only to show the limit of a geometrical interpretation as such. A detailed publication will appear elsewhere}{Die Bedeutung der Weyl'schen Erweiterung des Raumtypus wird dahin formuliert, daß Weyl die Unabhängigkeit der Verschiebungsoperation von der Metrik erkannte. Die Anwendung des erweiterten Raumtypus in der Physik ist jedoch mit einer gewissen Willkür behaftet, weil es freisteht, geeignete Objekte auszusuchen, die sich im Sinne der Verschiebungsoperation verhalten. Es wird nun gezeigt, daß als solche Objekte die Geschwindigkeitsvektoren elektrisch geladener Massenpunkte gewählt werden können. Mit Hilfe dieser Zuordnung gelingt es, die Gravitatons- und elektrischen Erscheinungen als Ausflufi der Geometrie eines Weylschen Raumes zu deuten, so daß die Elektrizität in demselben Sinne eine geometrische Deutung findet wie die Gravitation. Das Merkwürdige ist jedoch, daß durch diese Darstellung der Inhalt der Einsteinschen Gravitationstheorie überhaupt nicht geändert wird; die geometrische Deutung der Elektrizität bedeutet deshalb nur eine andere Sprechweise, ohne etwas physikalisch Neues zu enthalten. Auch das Problem des Elektrons kann die hier gegebene geometrische Deutung der Elektrizität natürlich nicht leisten, weil sie ja nichts anderes leisten kann als die EINSTEINsche Theorie. Die Absicht der Untersuchung besteht allein darin, die Grenzen einer geometrischen Deutung überhaupt aufzuzeigen. Eine ausführliche Veröffentlichung erfolgt an anderer Stelle}[][25; my emphasis][Reichenbach1926d]

However, this abstract registers an aspect not mentioned in either the correspondence with Einstein or in the note. Reichenbach revealed that what he wanted to achieve was a geometrical interpretation of a physical field \scare{in \myemph{the same sense as gravitation}} in Einstein's theory, i.e., one that was \emph{just as good} as that attained by general relativity. The geometrical operation of displacement has a physical interpretation in Reichenbach's toy-theory, just like the $d s$ does in general relativity Thus, Reichenbach claims to have provided not just a successful \scare{geometrical interpretation} of the electromagnetic field, but an interpretation that was of the same \scare{quality} as the one \gr provided for the gravitational field. However, this was Reichenbach's point: the theory was not a successful physical theory like general relativity. Thus, he concluded, providing a geometrical interpretation of a physical field is not in itself a physical achievement.


%https://arxiv.org/pdf/1802.00492.pdf
%http://www.physics.ntua.gr/ModifiedGravity2018/Talks/Iosifidis.pdf
%http://www.weylmann.com/weyltheory.pdf
%https://inis.iaea.org/collection/NCLCollectionStore/_Public/18/010/18010695.pdf?r=1&r=1


\section{Unification: Back to Berlin}

In August 1926, Reichenbach was granted teaching privileges as an \q{unofficial associate professor} (\german{nichtbeamteter au\ss{}erordentlicher Professor}) at the University of Berlin \citep{Hecht1982}. The discussion seminar that he started to hold in October became soon the basis of the so-called \scare{Berlin group}, which, together with Schlick's cognate \scare{Vienna circle} has marked the history of 20th century \scare{scientific philosophy} \citepp{Danneberg1994}{Milkov2013}. By the end of the year, Reichenbach wrote to Schlick, keeping him up to date with the progress of a two-volume book he was writing, which was supposed to bear the title \bt{Philosophie der exakten Naturerkenntnis}. \q{The first volume that deals with space and time,} he wrote, \qt{is finished}{der erste Band der Raum und Zeit behandelt, ist fertig} \letterp{Reichenbach}{Schlick}{6}{12}{1926}[][SN]\label{RZL1926}. Reichenbach hoped to publish the book in the forthcoming Springer series \scare{Schriften zur wissenschaftlichen Weltauffassung} directed by Schlick and Philipp Frank. However, Springer rejected the book as being too long. According to \Reich's later recollections, the manuscript of the first volume was not changed significantly after February 1927\hide{Das MS war seit Febr. 1927 nicht mehr nennenswert geändert worden}. By July Reichenbach could announce to Schlick that he had reached a publication arrangement \letterp{Reichenbach}{Schlick}{2}{7}{1927}[][SN]. The publisher agreed to publish only the first volume under the title \citetitle{Reichenbach1928}. The drafts were finished in September and the preface was dated October 1927.



%review

%Thus the apparent agreement between Reichenbach and Einstein on the geometrization issue actually hides a somewhat complicated dialectic. For Einstein, the very idea of a geometrical interpretation of a physical field was meaningless, and what he wanted to achieve was a unification of two different fields. On the contrary, Reichenbach regarded the geometrical interpretation of a physical field as a meaningful enterprise, which, however, offered no guarantee of physical unification. Moreover according to Reichenbach a good geometrical interpretation implies a \scare{Zurordnung} between the fundamental geometrical structures of the theory and the behavior of suitably chosen probes; on the contrary Einstein had come to realize that this operationalist approach was not only unnecessary, but a detriment to very project of a unified field theory. This dialectic emerges more clearly in Reichenbach's discussion of Einstein's new attempt to develop a unified field theory.





In spring 1928, during a period of rest after a circulatory collapse, Einstein, as he wrote to Besso, \qt{laid a wonderful egg in the area of general relativity}{ein wundervolles Ei gelegt auf dem Gebiete der allgemeinen Relativität}[][40-69][EA]. On \datef{7}{6}{1928} he presented a note to the Prussian Academy on a \scare{Riemannian Geometry, Maintaining the Concept of Distant Parallelism} \citep{Einstein1928}, a flat space-time that is nonetheless non-Euclidean since the connection is non-symmetrical. On \datef{14}{6}{1928} he submitted a second paper in which the field equations are derived from a variational principle \citep{Einstein1928a}. Reichenbach wrote to Einstein with some comments on the theory on \datef{17}{10}{1928}:
 
\qt{Dear Herr Einstein,\\ I did some serious thinking on your work on the field theory and I found that the geometrical construction can be presented better in a different form. I send you the ms. enclosed. Concerning the physical application of your work, frankly speaking, it did not convince me much. \myemph{If geometrical interpretation must be, then I found my approach simply more beautiful, in which the straightest line at least means something.} Or do you have further expectations for your new work?}{Lieber Herr Einstein, \\ Ich habe mir Ihre neuen Arbeiten zur Feldtheorie durch Kopf gehen lassen und gefunden, dass man die geometrische Aufbau besser in anderer Form darstelle kann. Ich schicke Ihnen einliegend des Ms. Was die physikalische Anwendung betrifft, so hat mich Ihre Arbeit, offen gesagt, wenig überzeugt. \myemph{Wenn es nun einmal geometrische Deutung sein muss, so finde ich schlechthin meinen Ansatz schöner, bei dem die geradesten Linien wenigstens etwas bedeuten}. Oder sollten Sie doch noch Aussichten in Ihren neuen Ansatz sehen?}[\letter{Reichenbach}{Einstein}{17}{10}{1928}][20-92\me][EA]

There are two aspects of this passage that should be considered separately. The first part refers to the mathematical-geometrical aspect of Einstein's papers. The manuscript to which Reichenbach refers seems to have been lost. However, from Einstein's reply on \datef{19}{10}{1928} one can easily infer that Reichenbach must have sent him the classification of geometries which would appear in an article Reichenbach submitted in \datemy{1}{2}{1929} \citep[][see below in this section]{Reichenbach1929a}. Einstein agreed that in principle it was possible to proceed as Reichenbach suggested, \qt{starting with displacement law, and to specialize it on the one hand with the introduction of a metric on the other side with the introduction of integrability properties}{Von einem Verschiebungsgesetzt ausgehen und einerseits durch Einführung einer Metrik anderseits durch Einführung Integrabilitätseigenschaften spezializieren}[\letter{Einstein}{Reichenbach}{19}{10}{28}][20-094][EA]. Reichenbach in fact defines a metrical space by imposing the condition $d(l^2)=0$ to the displacement space \Gtmn, which in general is non-symmetrical; he then obtains Einstein space by requiring that the Riemann tensor $R^\tau_{\mu\nu\sigma}(\Gamma)$ vanishes\footnotep{The $\Gamma$ alludes to the fact that this condition can be defined without reference to the $\gmn$}. Einstein, in contrast, preferred the classification he had given in his paper: Weyl's geometry allows for the comparison over finite distances neither of lengths nor of directions; Riemannian geometry allows the comparison of lengths, but not directions; and Einstein's geometry directions but not lengths \citep{Sauer2006}. 

This, however, was only a minor point. Reichenbach's further remark concerning the physical application of Einstein's geometrical setting is, from a philosophical standpoint, more interesting, even if Einstein did not comment on it. Reichenbach claims that, if one really wants to provide a geometrical interpretation of gravitation and electricity, then his own approach was better after all. Reichenbach uses his own toy-theory as a benchmark for a good \scare{geometrical interpretation} (but of course not for a good physical theory). Reichenbach's theory provides a physical meaning to the displacement operation and thus a physical definition of a straightest line. On the contrary, Einstein's theory did not attempt to provide a physical interpretation of the notion of displacement, nor even the field quantities; if the theory has nothing more to offer, Reichenbach claims, (i.e., if the theory does not solve the problem of the electron) it is merely a \scare{graphical representation} (cf.\ also \cite{Eddington1929} for a similar judgment).

In a note added by hand at the bottom of the typewritten letter, Einstein invited Reichenbach and his first wife Elisabeth for a cup of tea on \datef{5}{11}{1928}, mentioning that Erwin Schrödinger\footnote{Schrödinger succeeded Max Planck at the Friedrich Wilhelm University in Berlin in 1927. He held his inaugural lecture on \datef{4}{7}{1929} \citep{Schroedinger1929a}} would also be present (\letter{Reichenbach}{Einstein}{17}{10}{1928}[20-92][EA]). It was probably on that occasion that Einstein told Reichenbach about the physical consequences of the theory he was working on. In the meantime, on \datef{4}{11}{1928}, an article by Paul Miller appeared in \jt{The New York Times} with the sensational title \enquote{Einstein on Verge of Great Discovery; Resents Intrusion}. The paper triggered the curiosity of the press. In the late 1920s Reichenbach was a regular contributor to the \jt{Vossische Zeitung}, at that time Germany's most prestigious newspaper; not surprisingly he was asked for a comment on Einstein's theory. With the advantage of having personally discussed the topic with Einstein, Reichenbach published a brief didactic paper on Einstein's theory on \datef{25}{1}{1929} \citep{Reichenbach1929c}. 

Reichenbach conceded that Einstein's theory provided a unification of gravitation and electricity which had more than just formal significance, since it made \q{new assertions concerning the relation between gravitation and electricity in relatively complicated fields}[][][Reichenbach1929c]. However, he maintained his skepticism by claiming that the theory was \q{only a first draft, lacking the persuasive powers of the original relativity theory because of the \myemph{very formal method by which it is} established}[][\me][Reichenbach1929c]. Reichenbach was clearly not the only one to write about Einstein's new theory in the press. On \datef{12}{01}{1929}---one day after Einstein submitted a third paper on distant parallelism \citep{Einstein1929b} to the Academy---\jt{The New York Times} published an article entitled \scare{Einstein Extends Relativity Theory}. 

It was amid this atmosphere that, at the end of January, Einstein wrote an angry letter to the \jt{Vossische Zeitung} lamenting Reichenbach's \qt{tactless behavior}{taktlose Verhalten} in violating the academic code (\letter{Einstein}{the Vossische Zeitung}{25}{1}{1925}[73-229][EA]). On \datef{26}{1}{1929}, the curator of the literary section, Monty Jakobs \citep[cf.][]{Badenhausen1974}, defended the behavior of the newspaper and forwarded Einstein's letter to Reichenbach (\letter{Jakobs}{Einstein}{26}{1}{1925}[73-230][EA]). Reichenbach wrote to Einstein the next day with feelings ranging from surprise to anger; he complained that Einstein did not write directly to him after all he had done to defend relativity theory \citep{Hentschel1982}, and he denied any wrongdoing (\letter{Reichenbach}{Einstein}{27}{1}{1925}[20-096][EA]). On \datef{30}{1}{1928} Einstein replied that he was somewhat pleased by Reichenbach's annoyance, which was the \qt{fair equivalent}{gerechte Aequivalent} of the annoyance he had caused by feeding the press private information (\letter{Einstein}{Reichenbach}{30}{1}{1920}[20-099][EA]). However, Einstein quickly settled the dispute to Reichenbach's relief (\letter{Reichenbach}{Einstein}{31}{1}{1920}[20-101][EA]). 


On \datef{30}{1}{1929} Einstein's paper was finally published in the proceedings of the Academy with the vague title \scare{On the Unified Field Theory} \citep{Einstein1929b}. On \datef{2}{2}{1929} another semi-popular paper by Reichenbach was published in the \citejournal{Reichenbach1929b} \citep{Reichenbach1929b} without any reaction from Einstein. Einstein's anger at Reichenbach (which might at first seem rather exaggerated) is understandable if one keeps in mind the attention that the theory was attracting among the public; Einstein might have been upset that a colleague and friend would also contribute to the craze. At the beginning of February the \jt{New York Herald Tribune} (\datedm{1}{2}{1929}) printed a translation of the entire paper. Several days later \jt{The New York Times} (\datedm{3}{2}{1929}) and the \jt{London Times} (\datedm{4}{2}{1929}) published Einstein's own popular account. The \scare{irrational exuberance} towards the theory is well attested to by a letter Eddington sent to Einstein a few days later, recounting that Selfridges---a British chain of high-end department stores---had pasted all six pages of Einstein's papers in its window (\letter{Eddington}{Einstein}{11}{2}{1929}[9-292][EA]).


In the meantime, on \datef{22}{1}{1929}, Reichenbach had already submitted a second and more technical paper, which only appeared in the \citejournal{Reichenbach1929a} in September \citep{Reichenbach1929a}. The paper offers a readable presentation of Einstein's new theory; Reichenbach again presented his own take on the relationship between displacement and metrical space, and located Einstein space as an alternative to Riemannian space, rather than a generalization of it \citep[684--687]{Reichenbach1929a}. 

After this semi-popular presentation of Einstein's geometry and its physical application, Reichenbach added some remarks that are interesting from a philosophical point of view. He pointed out that there are two ways to unify two different physical theories. The first is a \emph{formal unification}, comparable to the relationship between Lagrangian and Hamiltonian formalism in classical mechanics (the first can be Legendre transformed into the other without adding any new physical knowledge); the second is an \emph{inductive unification}, exemplified by the relationship between Kepler and Newton's laws (something new is of course added by moving from Kepler's laws to Newton's theory of gravitation). 

The first approach was the one used by Reichenbach himself in his own \scare{unified field theory}:

\qt{The author \textins{Reichenbach} has shown that the first way can be realized in the sense of a combination of gravitation and electricity to one field, which determines the geometry of an extended Riemannian space; it is remarkable that thereby \myemph{the operation of displacement receives an immediate geometrical interpretation, via the law of motion of electrically charged mass-points}. The straightest line is identified with the path of electrically charged mass-points, whereas the shortest line remains that of uncharged mass points. In this way one achieves \myemph{a certain parallelism to Einstein's equivalence principle}. By the way [the theory introduces] a space which is
cognate to the one used by Einstein, i.e., a metrical space with non-symmetrical \Gtmn. The aim was to show that the geometrical interpretation of electricity does not mean a physical value of knowledge per se}{Daß der erste Weg durchführbar ist im Sinne einer Zusammenfassung yon Gravitation und Elektrizität zu einem Feld, welches die Geometrie in einem erweiterten Riemannschen Raum bestimmt, ist vom Verfasser gezeigt worden; es ist bemerkenswert, daß dabei die Verschiebungsoperation eine unmittelbare geometrische Deutung finden kann, nämlich durch das Bewegungsgesetz elektrisch geladener Massenpunkte. Es wird dort die geradeste Linie mit der Bahn des elektrisch geladenen Massenpunkts identifiziert, während die kürzeste Linie die des ungeladenen Massenpunkts bleibt. Hierdurch wird eine gewisse ParallelRat zu dem Einsteinschen-Aquivalenzprinzip erreicht. ?brigens wird dort ein dem Einstein'schen Raum verwandter Raum, nämlich ein metrischer Raum mit unsymmetrischen \Gtmn zugrunde gelegt. Absicht nämlich, zu zeigen, daß geometrische Deutung der Elektrizität an sich noch keinen physikalischen Erkenntniswert bedeutet}[][688\me][Reichenbach1929a]

Notice that, according to Reichenbach, the advantage of his own approach consists in the fact that it provides a physical realization of the displacement operation, and also (Reichenbach insists) an analogon to the equivalence principle (at least for particles of certain charge-to-mass-ratio). The disadvantage is that it is only a \emph{unification of the representations} of two physical fields in a common geometrical setting. The second approach is the one used by Einstein, and it presented  the opposite characteristics:

\qt{On the contrary Einstein's approach of course uses the second way, since it is a matter of increasing physical knowledge; it is the goal of Einstein's new theory to find such a concatenation of gravitation and electricity, that only in first approximation it is split in the different equations of the present theory, while is in higher approximation reveals a reciprocal influence of both fields, which could possibly lead to the understanding of unsolved questions, like the quantum puzzle. However, it seems that this goal can be achieved only \myemph{if one dispences with an immediate interpretation of the displacement, and even of the field quantities themselves}. From a geometrical point of view this approach looks very unsatisfying. Its justification lies only on the fact that the above mentioned concatenation implies more physical facts that those that were needed to establish it }{Der Einsteinsche Ansatz benutzt dagegen natürlich den zweiten Weg, denn ihm ist es ja um Vermehrung des physikalischen Wissens zu tun; es ist als Ziel der neuen Theorie Einsteins, eine derartige Verkettung yon Gravitation und Elektrizität zu finden, daß sie nur in erster Näherung in die getrennten Gleichungen der bisherigen Theorie zerspaltet, während sie in höherer Näherung einen gegenseitigen Einfluss beider Felder lehrt, der möglicherweise zum Verständnis bisher ungelöster Fragen, wie der Quantenrätsel, führt. Aber dieses Ziel scheint nur erreichbar zu sein unter Verzieht auf eine unmittelbare physikalische Interpretation der Verschiebungsoperation, ja sogar der eigentlichen Feldgrössen selbst. Vom geometrischen Standpunkt als deshalb ein solcher Weg sehr unbefriedigend erscheinen; seine Rechtfertigung wird allein dadurch gegeben werden können, daß er durch die genannte Verkettung mehr physikalische Tatsachen umschließt, als zu seiner Aufstellung in ihn hineingelegt wurden}[][688\me][Reichenbach1929a]

Einstein's theory was claimed to be a \emph{unification of the dynamics} of two physical fields, i.e., a unification of the fundamental interactions. However, Reichenbach argues that Einstein could achieve this result only at the cost of dispensing with a physical interpretation of the fundamental quantities. Thus, according to Reichenbach, his own theory had the ambition of being a \scare{\emph{proper geometrical interpretation}} (or, one might say, to provide a \scare{natural geometry}), but it was physically sterile; Einstein's theory sought to be physically fruitful, but it was merely a \scare{\emph{graphical representation}} (see also \cite{Eddington1929}). Clearly, for Reichenbach, only general relativity was able to combine both virtues: it was a proper geometrical interpretation (the $d s$, and thus the \gmn are measured using rods and clocks) that leads to new physical results. Reichenbach did not seem to realize (or at least does not explicitly point out) that this epistemological standard had become hard to comply with in precisely the context of the field-theoretical explanation of the electron that he was calling for. 

\subsection{A Parting of The Ways. Positivists and Metaphysicians}
\label{positivistsmetaphysicians}

Meyerson:

After 1919 Einstein benefited from a universal acclaim among the general public; however his positions among the physics community became progressively more isolated. Till 1925-1926 the \uftp was pursued by scholars of the stature by Weyl and Eddington, but was also regarded as a viable options by leading quantum theoreticians \citep[209]{Vizgin1994}. Even, after the 1925-1927 rapid advance in \qm, were made of relate unified theories to quantum theory \citep{Klein1926a}. However, most leading physicists  soon started to perceive the program as obsolete. Einstein was fully aware of the marginality of his position, but, throughout 1929, continued express his confidence in \FP program. In the second paper of this year finished in \datem{19}{8}{1929}---the fourth in the series in the Berlin Academy---which reflects the priority dispute with Élie Cartan \citep{Debever1979}, Einstein returned to the Hamiltonian principle after objections raised by his collaborators Lanczos and Müntz \citep{Einstein1930c}.  In spite of the many doubts, Einstein was finally convinced that he had \q{found the simplest legitimate characterization of a Riemannian metric with distant parallelism that can occur in physics} \letterp{Einstein}{Cartan}{25}{8}{1929}[\D{V}][Debever1979].

However, like Reichenbach, fellow physicists were not impressed, in particular given the growing success of \qm-program. Weyl, who had always been scolded by Einstein for his speculative style of doing physics could relaunch the accusation in a paper \citep{Weyl1929c} in which he had uncovered the gauge symmetry of the \Dirac theory of the electron \citepp{Dirac1928}{Dirac1928b}. \q{The hour of your revenge has come}, Pauli wrote to Weyl in August: \qt{Einstein has dropped the ball of distant parallelism, which is also pure mathematics and has nothing to do with physics and \emph{you} can scold him}{jetzt hat Einstein den Bock des Fernparallelismus geschossenf , der auch nur reine Mathematik ist und nichts mit Physik zu tun hat, und Sie konnen schimpfen} \letterpaulip{Pauli}{Weyl}{26}{8}{1929}[235].  \cop{Although Einstein's papers had been discussed widely especially among mathematicians, Einstein was aware of the poor reception that his work had especially among the colleagues that he probably felt has his peers} \citep{Goldstein2003}. As Pauli complained, writing to Einstein's close friend Paul \Ehr, \q{God seems to have left Einstein completely!} \letterpaulip{Pauli}{\Ehr}{29}{9}{1929}[237].

%An invitation to the 1930 Rouse Ball lecture at Cambridge gave Weyl the opportunity to review the Whole development of matter concepts which had taken place during the long decade just coming to an end.





%\footnoteh{Einstein agreed and gave a talk on the Problem of Space, Field, and Ether in Physics on December 11, 1929, Essentially the same talk was delivered to a large audience on the opening day of the Second World Power Conference which took place in Berlin from 16–25 June, 1930.  The text of this lecture was then published in the conference’s Transactions [Einstein 1930d]. A similar popular account of Space, Ether and the Field in Physics was published in Forum Philosophicum [Einstein 1930c] together with an English translation. Indeed, the text of the two penultimate paragraphs of this version and [Einstein 1930d] that characterize the distant parallelism are identical. A two-page abbreviated version of [Einstein 1930c] also mentions the distant parallelism approach [Einstein 1930e]}



Nevertheless, Einstein continued to defend the theory in public (in talks given in October and December) \citep{Einstein1930,Einstein1930a,Einstein1930b}, as well as in as well in private correspondence. However, Pauli did not hesitate to describe Einstein's presentation at the Berlin Colloquium as a \qt{terrible rubbish}{schrecklichen Quatsch} \letterpaulip{Pauli}{Jordan}{30}{11}{1929}[238]. When he received the drafts of Einstein's \jt{Annalen} paper, he wrote only slightly more politely \cop{that he no longer believed that the quantum theory might be an argument for the distant parallelism after Weyl's work on Dirac theory had shown that Dirac’s electron theory could be incorporated into a relativistic gravitation theory if the \vbein are introduced but the equations remain invariant if the \vbein at distant points are rotated in arbitrary manner}. Pauli also wrote that he did not find the derivation of the field equations convincing; they show \qt{no similarities with the usual facts confirmed by experience}{kaum eine Ahnlichkeit mit den gewohnlichen durch die Erfahrung gesicherten physikalischen Sachverhalten zu haben scheinen} \letterpaulip{Pauli}{Einstein}{19}{12}{1929}[239]. In particular, Pauli missed the validity of the classical tests of general relativity, perihelion motion and gravitational light bending: \qt{These results seem to be lost in your sweeping dismantling of the general theory of relativity. However, I hold on to this beautiful theory, even if it is betrayed by you!}{Die scheint doch bei Ihrem weitgehenden Abbau der allgemeinen Relativitatstheorie verloren zu gehen. Ich halte jedoch an dieser schonen Theorie fest, selbst wenn sie von Ihnen verraten wird!} \letterpaulip{Pauli}{Einstein}{19}{12}{1929}[239]. When Einstein expressed caution towards the definitive validity of his equations, he, \qt{so to speak, took the words right out of my mouth of criticism-loving physicists}{haben Sie den Kritik libenden Physikern sozusagen das Wort abgeschnitten} \letterpaulip{Pauli}{Einstein}{19}{12}{1929}[239]. Pauli knew that Einstein would not have changed his mind, but he was ready to \q{make any bet} that \q{after a year at the latest you will have given up all the distant parallelism, just as you had given up the affine theory before} \letterpaulip{Pauli}{Einstein}{19}{12}{1929}[239].

%\q{dann sagen Sie erst etwas dariiber, wenn mindestens ein Vierteljahr vergangen ist} 

Einstein complained that Pauli's remarks were superficial and asked him to return on the issue after some months \letterpaulip{Einstein}{Pauli}{19}{12}{1929}[140]. Although the \uftp was disavowed by its own initiators \citep{Weyl1931}, Einstein insisted in the pursuit of \FP discussing with Mayer two solutions of his last field equations \citep{Einstein1930g}\todo{field equations admitted at least one unphysical solution, namely, a static configuration of uncharged, gravitating bodies.}. However, Pauli would have clearly won the bet. Only a few months later Einstein and Walther Mayer presented a new approach \citep{Einstein1931} that, by generalizing the \nbein formalism to five dimensions, may have appeared more promising. This approach was ideally connected with that of Kaluza, but the shortcoming of that theory \qt{by sticking to the four-dimensional continuum, but with vectors with five components}{werden bei der im folgenden dargelegten Theorie dadurch vermieden, daß man zwar bei dem vierdimensionalen Kontinuum bleibt, aber in diesem Vektoren mit fünf Komponenten \textelp{} einführt} at each point of four-dimensional space-time \citep[377]{Einstein1931}. The optimism once again faded away quickly, since the theory was unable to solve the problem of matter. In a popular talk given in Vienna towards mid-\datemy{14}{10}{1931}, Einstein could only describe his field-theoretical work since \gr as a \qt{cemetery of buried hopes}{Friedhof von Begrabener Hoffnungen} \citep[441]{Einstein1932b}.

%\citep{Lanczos1929}
A few days later, Lanczos wrote to Einstein from the United States \letteraeap{Lanczos}{Einstein}{20}{10}{1931}[15-243] where he had just taken a position at Purdue University. Lanczos told Einstein that, at Arnold Berliner's suggestion, the influential editor of the \jt{Die Naturwissenschaften}, he had prepared a semi-popular presentation of \FP approach for the \jt{Ergebnisse der Exakten Wissenschaften}, a series sponsored by Berliner's journal \citep{Lanczos1931}. Lanczos had worked on the topic during his time as Einstein's assistant. The Lanczos/Einstein relation had become somehow strained \citep{Stachel1994}, and Lanczos was not fully convinced by Einstein's approach. However, he was confident to have found \q{a tone that should correspond to your conviction as well. I think that, deep down, we have something in common} \letteraeap{Lanczos}{Einstein}{20}{10}{1931}[15-243]. Lanczos presented \FP as a completion rather than an generalization of Riemannian geometry; nevertheless he also recognized the correctness of Reichenbach's approach \citep[118]{Lanczos1931}. What is more important, he opened the paper with some general considerations which give a glimpse in the philosophical atmosphere which pervaded the physics community. Lanczos distinguished between two \qt{spiritual attitudes}{geistige Haltung} towards relativity: 

\begin{enumerate}
\item\label{p} a \emph{positivist-subjectivist} insistence that physics has to do with observable quantities, and what cannot be observed is not part of physics. This \q{rigorous and therefore more intolerant form of positivism} \citep[104]{Lanczos1931}, defended in particular by quantum theoreticians, lead to the rejection of the \uft program as such.\ Since a field is nothing but a tool to describe the behavior of test particles, \rac and so, it is vain to search for solutions of the field equations that correspond to protons and electrons. In fact, the fields inside of elementary particles \q{could never in their details become the object of observation} \citep[104]{Lanczos1931}, since there are no test particles or measuring scales smaller than the electron itself. 

\item\label{m} A \emph{metaphysical-realistic} perspective, based on the conviction that physical reality exists independently of the possibility of measuring or observing it. If \sr seemed to be close to the positivistic/operationalistic ideals, with \Mink the theory underwent a \qt{\scare{metaphysical} turn}{metaphysische Wandlung} in favor of a \qt{logical-constructive understanding \origins{Verstehen}}{zugunsten eines logisch-konstruktiven Verstehen} \citep[103]{Lanczos1931}. \Gr had finally brought \qt{the logical-deductive exploration into the depths of nature, under the presupposition of its universality and understandability, and with faith in the laws of mathematics}{das logisch-deduktive Eindringen in die Natur, unter Voraussetzung ihrer Universalitiit und Verstehbarkeit, und im Verlrauen au/ das mathematische Gesetz} \citep[102]{Lanczos1931}. 
\end{enumerate} 
%
The positivist described by Lanczos could be easily identified with Pauli, who had indeed raised similar objections against Weyl's theory early on \citepp{Pauli1919}[see][13]{Hendry1984}. However, Pauli,  by reviewing Lanczos article, did not fully recognize himself in the portrait of the \scare{positivist} \citep{Pauli1932-3-11}. Such labels, he argued, \q{are highly subjective and arbitrary}; it is obvious that in order to gain new scientific insights one does not only requires inductive generalizations, but also logical-constructive imagination. Pauli mocked the \emph{Naturwissenschaften} for having published the paper in series entitled \scare{Results in exact sciences} (\german{Ergebnisse der Exakten Wissenschaften}). Indeed, Einstein published this sort of theories at rhythm of one each year and in every case he claims that it is the definitive solution: \qt{Einstein's new field theory is dead, long live Einstein's new theory!}{Die neue Feldtheorie ~EINSTEINS ist tot. ES Iebe die neue Feldtheorie EINSTEINS !}. 


%However, it was undeniable that many supporters of \qm had used a a positivistic rhetorics, and it was against this rhetorics that Einstein, somehow tongue in cheek, was not ashamed to define himself as a metaphysician \citep{Einstein1932a}. It was this attitude that caught many of Einstein's philosophical allies by surprise.\todo{improve}


%Lanczos was meant probably the Pauli-Einstein debate, a more general the reaction of Einstein's philosophical\todo{improve}. 

However, if many readers might have easily recognized someone like Pauli in Lanczos's \scare{positivist}, other were baffled to find out Einstein located among the \scare{metaphysicians}. At the beginning of 1932 the introduction of Lanczos's 1931 paper was published at Berliner's suggestion as a \latin{seperatum} in the \citejournal{Lanczos1932} \q{to make it available to a larger public} \citep[113\fn{1}]{Lanczos1932}. It is probably this article of Lanczos that Frank read with some bewilderment, as he reports in his Einstein's biography \citep{Frank1947}. Frank was \q{quite astonished} to find the theory of relativity characterized as the expression of a realist program \q{since I had been accustomed to regarding it as a realization of \Mach's program} \citep[215]{Frank1947}. However, when Frank met Einstein in Berlin at around the same time, he found out that Lanczos had indeed well characterized Einstein's point of view \citep[215f.]{Frank1947}. According to his recollection, Einstein complained that \q{\textins{a} new fashion} had arisen in physics according to which quantities that in principle cannot be measured do not exist, and that to \q{to speak about them is pure metaphysics} \citep[216]{Frank1947}. Frank objected that this was the very same philosophical attitude that led to relativity theory. By contrast, Einstein insisted, the essential point of relativity theory is to \q{regard an electromagnetic or gravitational field as a physical reality, in the same sense that matter had formerly been considered so}  \citep[216]{Frank1947}. The theory of relativity teaches us the connection between different descriptions of one and the same reality. Was not a theory about the behavior of \rac, but a unification of two fields.


%http://philsci-archive.pitt.edu/18196/1/Coming%20to%20America-%20Carnap%2C%20Reichenbach%2C%20and%20the%20Great%20Intellectual%20Migration.%20Part%20II%2C%20Hans%20Reichenbach.pdf

%
Lanczos' reconstruction is too broadly stroked to be fully accurate; nevertheless it undeniably grasps something of the intellectual mood (\german{geistige Einstellung}) of the time. Reichenbach, like Pauli, would not have been entirely pleased of having been cast among the \scare{positivists}, with whom he was in conflict for quite some time\footnoteh{The  philosophical squabbles between Reichenbach's Berlin Group and the Vienna Circle lead by Schlick started to emerge and the turn of 1930s and soon grew to an open conflict}; however, like Frank, he would have been puzzled, if not appalled, by seeing Einstein categorized among the \scare{metaphysicians}.
\end{comment}

\section{Conclusion}

1938

Althougth this was non-Euclidea geometire, that Einstein something in mind. The non-positivist, that it was not necessary to affine connection to give a physicla meaning. For that time 1949. By that time clearly talikg at cross purpuses.

That coordination... however, according to Einstein


It is quite possible had his in mind. Indeed, that by that time Einstien was ultimately a non-symmetric connection. Of ocurse did not have any direct physical meaning. Indeed, to .... real. That was mathematical simplicty ... had to become a metaphysical, that simple ...this was the only criterion. 

1920 the separation between mathematics and physics by 1949, precisley this semparation had been by Einstein.





\lipsum*[1-4]

\printshorthands
\printbibliography

\end{document}