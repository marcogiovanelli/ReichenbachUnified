
%Einstein was aware that, without an analogon of the equivalence principle, the choice of the basic field structure to represent the combined electromagnetic/gravitational field could not be empirically motivated from the outset, as in the case of his theory of gravitation. One had to proceed tentatively to search the right field quantities that would allow deriving, usually from a variational principle, the desired set field equations---that is to recover Maxwell and Einstein field equations in the first approximation. Once a set of field-equations has tentatively been established, the field components that they connect---\cop{(either the basic field-quantities or vectors and tensors built from them)}---have to be identified, one by one, with the components the known fields, the gravitational and electromagnetic field. Since, the basic field quantities did not have a physical interpretation, e.g. were defined in terms of test particles, the theory could be compared with experience only indirectly by integrating the field equations in the hope of finding static spherically symmetric solutions representing electron and proton and their law of motion. %In Einstein's view, the principle of general covariance considerably reduced the number of possible sets of field equations. However, it did not determine the field equations uniquely. Einstein seemed to have progressively come to accept that only the criterion mathematical simplicity could further limit theory choice. Untitled

\footnoteh{In July, he was still convinced that he had \qt{really found the relationship between gravitation and electricity}{Ich glaube nun, die Beziehung zwischen Gravitation und Elektrizit\"at wirklich gefunden zu haben} \lettercpaep{Einstein}{Millikan}{13}{7}{1925}[15][20]. However, during the summer, Einstein had already started to nurture some skepticism (\lettercpae{Einstein}{Ehrenfest}{18}{8}{1925}[15][49]; \lettercpae{Einstein}{Millikan}{13}{7}{1925}[15][20]; \lettercpae{Einstein}{Ehrenfest}{18}{9}{1925}[15][71]). The paper was published at the beginning of September, and by that time, Einstein probably already moved on (\lettercpae{Einstein}{Rainich}{13}{9}{1925}[15][106]; see \cite{Einstein1927c})}

\footnoteh{These doubts became a certainty when Einstein returned to Europe. \qit{On \datedm{1}{6}{1925}, I got back from South America}{Einstein wrote to Besso}{I am firmly convinced that the whole line of thought Weyl-Eddington-Schouten does not lead to anything useful from a physical point of view and I found a better trail that is physically more grounded}{Am 1. Juni bin ich von S\"udamerika wiedergekommen ... Ich bin fest \"uberzeugt, dass die ganze Gedanken-Reihe Weyl-Eddington-Schouten zu nichts physikalisch brauchbarem f\"uhrt und habe jetzt eine andere Spur gefunden, die mehr physikalisch fundiert ist} \lettercpaep{Einstein}{Besso}{5}{6}{1925}[15][2]}

\footnoteh{In \gr, the \gmn had a physical meaning \emph{ex ante}; they were supposed to be measurable with \rac with respect to a given coordinate system. In his recent theory, Einstein introduced the 40 coefficients \Gtmn as fundamental field variables without giving this quantity any physical meaning. If using the \Gtmn leads to the right set of field equations, then the initial choice would turn out to be justified \emph{post facto} \citep[see][]{Einstein1924}. In this sense, only geometry (\gmn, \Gtmn\etc) and physics (field equations) together could be compared with experience (e.g., by finding appropriate exact solutions corresponding to electrons) \qt{From this point of view is the whole content of geometry conventional. Which geometry one should chose depends on how \scare{simple} the physics can be brought in harmony with experience}{Von diesem Standpunkte aus ist der gesamte Inhalt der Geometrie in konventioneller; welche Geometrie zu bevorzugen sei, bängt davon ab, cine wie \scare{einfache} Physik sich bei ihrer Benutzung im Einklang mit der Erfahrung aufstellen läßt} \citep[19]{Einstein1926}}

\footnoteh{\q{In particular, I would like to mention that criticism was rightly aimed against one statement by the reviewer: that a concept should only be permissible in physics when it can be established whether or not it applies in concrete cases of observation; it is objected that, in general, it is not to an individual concept that possible experiences must correspond but to the system as a whole} \citep[1691]{Einstein1924}}

\footnoteh{I assume that this talk would have again been suspicious since the local is intrinsically mathematical reason. Indeed, in Reuchenbach view there was no reason\todoi{Correspondence with }}

\footnoteh{Einstein used a similar wording by commenting on the manuscript of Cassirer's \scare{Kantian} booklet on relativity \citep{Cassirer1921}. \q{Conceptual systems appear empty to me, if the manner in which they are to be referred to experience is not established} \lettercpaep{\Einstein}{\Cassirer}{6}{6}{1920}[10][44]. In particular, \q{[w]ith the interpretation of the $ds$ as a result of measurement, which is obtainable by means of measuring rods and clocks the general theory of relativity as a physical theory stands or falls} \lettercpaep{\Einstein}{\Cassirer}{6}{6}{1920}[10][44]. The gravitational redshift can be taken as an empirical confirmation of general relativity only because different atoms of the same substance can be regarded as identically constructed clocks reproducing the identical unit of time. For this reason, it is possible to \scare{normalize} the absolute value of $ds$ by counting an atom's wave crests. Accordingly, \WT deprived the $ds$ of any physical meaning. However, real \rac behave differently than predicted by \WT, forcing Weyl to assume an inconsistent position. According to Einstein, like \gr, \WT \q{is based on a measuring rods geometry}, that is, it presupposes the comparability of lengths. However, it pertains only \q{thought measuring rods \origins{nur gedachte Massstäbe}} that behave differently from the real ones. \q{This is repugnant} \lettercpaep{\Einstein}{\Besso}{26}{8}{1920}[10][85\me]}.

\footnoteh{Philipp Frank read \citets{Lanczos1931} article with some bewilderment, as he reports in his Einstein's biography \citep{Frank1947}. Frank was \q{quite astonished} to find the theory of relativity characterized as the expression of a realist program \q{since I had been accustomed to regarding it as a realization of \Mach's program} \citep[215]{Frank1947}. However, when Frank met Einstein in Berlin at around the same time, he found out that Lanczos had indeed characterized Einstein's point of view accurately \citep[215f.]{Frank1947}. According to his recollection, Einstein complained that \q{\textins{a} new fashion} had arisen in physics according to which quantities that in principle cannot be measured does not exist, and that to \q{to speak about them is pure metaphysics} \citep[216]{Frank1947}. Frank objected that this was the same philosophical attitude that led to relativity theory. By contrast, Einstein insisted, the essential point of relativity theory is to \q{regard an electromagnetic or gravitational field as a physical reality, in the same sense that matter had formerly been considered so} \citep[216]{Frank1947}. The theory of relativity teaches us the connection between different descriptions of one and the same reality}

\footnoteh{At the end of 1931, Lanczos wrote a semi-popular presentation of the distant parallelism approach \citep{Lanczos1931}, in which he effectively described the positions on the field. He distinguished between (a )a \emph{positivist-subjectivist} interpretation of \rt as a theory that makes predictions about the behavior of \rac (b) \emph{metaphysical-realistic} interpretation of relativity as a theory that aims to uncover the mathematical structure of the field. Many readers might have easily recognized someone like Pauli in Lanczos's \scare{positivist}; however, others were baffled to find out that Einstein was classified among the \scare{metaphysicians}. However, Pauli, by reviewing Lanczos's article, did not fully recognize himself in the portrait of the \scare{positivist} \citep{Pauli1932-3-11}. Such labels, he argued, \q{are highly subjective and arbitrary}. Pauli mocked the \emph{Naturwissenschaften} for having published Lanczos' paper in series entitled \scare{Results in exact sciences} (\german{Ergebnisse der Exakten Wissenschaften}). Indeed, Einstein published this sort of theory at a rhythm of one each year, and in every case, he claims that it is the definitive solution: \qt{Einstein's new field theory is dead, long live Einstein's new theory!}{Die neue Feldtheorie ~EINSTEINS ist tot. ES Iebe die neue Feldtheorie EINSTEINS !} \citep{Pauli1932-3-11}}

\footnoteh{Pauli did not hesitate to describe Einstein's presentation at the Berlin Colloquium as a \qt{terrible rubbish}{schrecklichen Quatsch} \letterpaulip{Pauli}{Jordan}{30}{11}{1929}[238]. When he received the drafts of Einstein's \jt{Annalen} paper, he wrote only slightly more politely that he did not find the derivation of the field equations convincing \letterpaulip{Pauli}{Einstein}{19}{12}{1929}[239]. Pauli knew that Einstein would not have changed his mind, but he was ready to \q{make any bet} that \q{after a year at the latest you will have given up all the distant parallelism, just as you had given up the affine theory before} \letterpaulip{Pauli}{Einstein}{19}{12}{1929}[239]}


\footnoteh{On \datef{14}{1}{1921}, Einstein, while in Vienna, released the following declaration: \qt{A theoretical system can only claim completeness if the relationships of the concepts to the facts that can be experienced are clearly established. It is not enough, for example, to base the theory of relativity on a mathematical fundamental invariant [$ds$]. It must also be clear how this invariant is related to the observable facts, as [it happenes] for the fundamental concepts of Maxwell's theory by Heinrich Hertz. If one disregards this point of view, one can only arrive at unrealistic systems}{Wenn ich die gegenwärtige Lage der theoretischen Physik überschaue, so scheint mir ein Punkt von großer Wichtigkeit nicht hinlänglich beachtet zu werden. Ein theoretisches System kann erst dann Vollständigkeit beanspruchen, wenn die Beziehungen der Begriffe zu den erlebbaren Tatsachen eindeutig festgelegt sind. Es genügt zum Beispiel nicht, die Relativitätstheorie an eine mathematische auf eine mathematische Fundamentalinvariante zu gründen. Es muß auch klar sein, wie diese Invariante mit den beobachtbaren Tatsachen zusammenhängt wie [das](2] für die Fundamentalbegriffe der Maxwellschen Theorie durch Heinrich Hertz [geschehen ist.] Läßt man diesen Gesichtspunkt außer acht, SO kann man nur zu wirklichkeitsfremden Systemen gelangen} \CPAEp{7\textins{13}}{50a}. A few days later, on \datef{27}{1}{1921}, Einstein held his celebrated \citetitle{Einstein1921} in Berlin. The lecture ultimately meant to address precisely this issue, although in a popular form. (a) the invariant $ds$ is measured by ideal \rac, like the electric field strengths are measured by charged test particles (b) ideal \rac do not exist in nature (as pointed out by Poincaré). Conclusion: \emph{sub specie aeterni} geometry cannot be tested separately from the rest of physics (the famous $G+P$ formula). The choice of a particular geometry is ultimately justified by its success in leading to a good physical theory. The connection of the $ds$ with \rach measurements is only provisional. Ultimately it assumed that solutions to theory's dynamical equations exist that at least arbitrarily closely approximated the postulated behavior of \rac that measure the $ds$. In \WT, however, real \rac that appear as solutions were supposed to behave differently from the ideal \rac described by \WG. This inconsistency was unbearable for Einstein. Thus, in March 1921, he preferred to suggest a theory in which there were no transportable ideal rods at all and only $ds=0$ has physical meaning \citep{Einstein1921c}}

\footnoteh{Commenting on Weyl's talk, he pointed out once again that in the \q{arrangement of \textins{a} conceptual system,} \q{it has become decisive \origins{massgebend} to bring elementary experiences into the language of signs \origins{Zeichensprache}} \citep[650]{Einstein1920c}. For Einstein, \q{temporal-spatial intervals are physically defined with the help of measuring rods and clocks}, under the assumption that \q{their equality is empirically independent of their prehistory} \citep[650]{Einstein1920c}. Einstein insisted that precisely upon this assumption rests \q{the possibility of coordinating \origins{zuzuordnen} a number $ds$ to two neighboring world points}; if this were impossible, general relativity would be robbed of \q{its most solid empirical support and possibilities of confirmation} \citep[650]{Einstein1920c}.
}

\footnoteh{Einstein showed a more flexible attitude replying to Pauli's remarks during the same discussion. Pauli reiterated his objection based on his \scare{observability} criterion. Just as the field strength in the interior of the electron is meaningless because there is no smaller test particle than the electron, \q{one could claim something similar concerning spatial measurements, \myemph{since there are no infinitely small measuring-rods}} \citep[650]{Einstein1920c}. Einstein replied to Pauli that \q{with the increasing refinement of the system of scientific concepts, the manner and procedure of associating the concepts with experiences becomes increasingly more complicated} \citep[650]{Einstein1920c}. In particular, he recognized that in cases such as that of the continuum theories, \q{one finds that a definite experience cannot be associated any longer with a concept} \citep[650]{Einstein1920c}. According to Einstein, there is an alternative: one can abandon \scare{continuum theories} for the sake of Pauli's observability criterion, or replace such a \q{system of associating concepts \textins{with experiences} with a more complicated one} \citep[650]{Einstein1920c}. Einstein in his contributions to the discussion that followed Max von \citets{Laue1920}'s Bad Nauheim paper. Einstein, however, in the very same sentence, did not hesitate to admit that \q{[it] is a logical shortcoming of the theory of relativity in its present form to be forced to introduce measuring rods and clocks \myemph{separately instead of being able to construct them as solutions to differential equations}} \citep[Einstein's reply to][662\me]{Laue1920}}