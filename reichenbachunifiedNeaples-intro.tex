
%As in general relativity one uses an atoms as standard of measure, since all atoms of the same time are identical}. Thus, Weyl's theory should predict that would are not, also that to not tick at the same rate in the presence of the electromagnetic field. However, this is not the case. Indeed, have the same spectral lines. Any physical content. However, they theory has moreover further shortcomings. The trajectories of free uncharged particles are not geodesics (see, for instance, Doc. 579) in the theory because of the presence \phin in the definition of the connection. The field equations are of the fourth order which increased the degree of arbitrariness of such equations. \cop{Moreover, to find the laws of pure gravitation and of electromagnetism, it was necessary to define the Langrangian $\mathcal{H}$, composed of two absolutely independent parts. We thus had a dualism; to avoid it, he arrived at the solution mentioned}. The theory to find the electron a question that Einstein felt was essential for the realization of the unified field theory. 

%216. To Arthur S. Eddington Berlin. d. 15. Dezember 19. %Die Weylsche Theorie (metrische Deutung des elektro-magnetische[n] Potenti- als) halte ich nicht für zutreffend. Sie scheint mir unvereinbar mit der Tatsache, dass Maassstäbe und Uhren ein von ihrer Vorgeschichte unabhängiges metrisches Verhalten zeigen.f

%Weyl hat in der neuen Auflage seines Lehrbuchs nun seine elektromagnetische Theorie leider angefügt, sodass dieser allerdings sehr geistreiche Unfug seinen Weg in die Gehirne nehmen wird.!11) Aber ich tröste mich damit, dass das Sieb der Zeit seine Arbeit auch an dieser Stelle thun wird. In der Relat. Theorie ist mir seither nichts mehr gelungen. Das elektromagnetische Feld trotzt allen Bemühungen

%Hoffentlich geht es Weyl wieder gut. Er ist ein sehr bedeutender Kopf. aber et- was thatsachenfremd. In der neuen Auflage seines Buches hat er mir die Relativität ganzverhunzt-Gottverzeihe es ihm.l 121 Vielleicht wird er doch noch einmal ein- sehen, dass er da bei allem Scharfsinn daneben geschossen hat.

%332. To Heinrich Zangger
%[Berlin, 27 February 1920}(1)

%189. To Paul Ehrenfest
%[Berlin,] 4. XII 19.
%Chronfectl

%Es ist mir unbegreiflich, dass Weyl selbst und alle andern das Erfahrungswid- rige des Grundgedankens der Theorie nicht ohne weiteres spüren

%Pauli works


%In this new theory, the gravitational Lagrangian remained the square of the curvature. The Maxwellian part of it, however, a quas1-Einsteinian form. This led him to a variational principle that included the Einsteinian component (with the cosmological term), the in Einstein's cosmological model.) By following this path, Weyl arrived at unit of charge, the gravitational radius of which, has the same order of magnitude as the radius of the universe. Noting that, in his work, the unit of electricity and the unit of action both have "cosmic values," Weyl emphasized: '"'The 'cosmological' term that Einstein first added to his theory is a natural consequence of our original principles" (Weyl 1919,p. 124, italic in the original).


%a quasi-Einsteinian form. This led him to a variational principle that included the Einsteinian component (with the cosmological term), the Maxwellian component, and a non-Maxwellian component, the square of square of the vector potential p,p' ("the simplest expression found in the Mie theory" (Weyl 1919, p. 122), a theory that claimed to provide a unified field description of the electron and the electromagnetic field).equations. adopted, as his Lagrangian, the square of the curvature} 


%In his 1919 paper, Weyl suggested a new variant of unified theory a that generalized GR and was based on a geometry that he introduced (i.e., the Weyl geometry). In this new theory, the gravitational Lagrangian remained the square of the curvature. The Maxwellian part of it, however, was no longer emerging in a natural way but was added "by hand." Through artful manipulations, Weyl managed to bring this Lagrangian into a quasi-Einsteinian form. This led him to a variational principle that included the Einsteinian component (with the cosmological term), the Maxwellian component, and a non-Maxwellian component, the square of the vector potential p,p' ("the simplest expression found in the Mie theory" (Weyl 1919, p. 122), a theory that claimed to provide a unified field description of the electron and the electromagnetic field).

%If one makes the first assumption in Weyl’s theory, Einstein argued, the rate of a clock will be dependent on its prehistory; if one makes the second, uncharged particles will be affected by the electromagnetic four-vector potential. In Doc. 661, Einstein reiterated both charges, that of the fourth-order and not of the second order}.  Einstein held the opinion that the world-lines of an uncharged particle should be described by a geodesic equation that does not explicitly depend on the electromagnetic four-potential, as it does in Weyl's theory.  \cop{This led him to separate the theory into two distinct parts: 1) the "pure, infinitesimal geometry," which provides a set of geometric structures \gmn and \phin. The form gravitational field.  Thus, the correspond to the presence of the gravitational field, and the to the presence of the electromagnetic field. (2) Is to find the field equations, via the \scare{action principle}. \cop{One constructs a scalar quantity (the action) from the dynamical quantities \gmn and \phin then finds the conditions needed to restrict the scalar to an extremum (a maximum or minimum) with respect to variations in those dynamical quantities. The problem the right action and the right dynamical quantities to produce the desired equations}. 


\subsection{Reichenbach's Notes on Einstein's Lecture and the Notion of Parallel Transport of Vectors}


%Nach seinem Einsatz im Ersten Weltkrieg hörte Reichenbach in Berlin Einsteins Vorlesungen zur Speziellen und Allgemeinen Relativitätstheorie. Seine Mitschriften dieser Veranstaltungen sind erhalten geblieben: Klassische und Statistische Mechanik (SS 1918, HR-028-01-02), Allgemeine Relativitätstheorie (Teil 1 HR-018-0104, Teil 2 HR-028-01-03, Teil 3 HR-028-01-01, alle Tile undatiert). In Teil 1 der Mitschrift zur Allgemeinen Relativitätstheorie, behan- 1917--1918 and 1918-1919. His transcripts of these events have been preserved: Classical and Statistical Mechanics (SS 1918, HR-028-01-02), General Theory of Relativity (Part 1 HR-018-0104, Part 2 HR-028-01-03, Part 3 HR- 028-01-01, all tiles undated). 


%The introduction of the Riemann curvature tensor given in these two papers can be found in these lecture notes as well (see [p. 11]), but it is clear from entries on [p. 10] and [p. 25] and from Reichenbach's notes (the last part of the second notebook) that}. From the metric, Christoffel symbos, and Riemann tensor is flat.

After serving in World War I, Reichenbach attended Einstein's lectures on special and general relativity in Berlin. We posses three sets of Reichenbach;s notes (HR-028-01-04, HR-028-01-03, HR-028-01-01 all undated). The last set of notes seems to corresponds to the Einstein's lecture on spring term 1919 \citep{Einstein1919c}. {The notes taken by Reichenbach very similar to Einstein's own notes\footnote{Further information about Einstein as an academic teacher, see Vol. 3, the editorial note, "Einstein's Lecture Notes,"pp. 3-10, and for a survey of Einstein's academic courses, see Vol. 3, Appendix B.}. \cop{The first sections of the lectures follow the corresponding sections Einstein's previous published presentations \citepp{Einstein1916}{Einstein1914a}.} \cop{The general theory of relativity rests formally on the geometry of Riemann, which bases all its conceptions on that of the interval} 

\begin{equation}\label{eq:lineelement}
ds^2=\gmn dx_\mu dx_\nu\,.
\end{equation}
%
between points indefinitely near together, in accordance with the formula, these magnitudes $g_{\mu \nu}$ determine the behavior of \rac with reference to the coordinate system, as well as the gravitational field \citep[028-01-04, 46]{HR}. From the \gmn the so called Christoffel symbols \christoffel{\mu}{\nu}{\tau}, functions of the \gmn and their first derivatives, linear in the latter. which enter into the geodesic equation \citep[028-01-04, 46, 2ff]{HR}. It is possible to chose a coordinate system Christoffel symbols vanish everywhere, by the vanishing of Riemann tensor \rite, which is function of the Christoffel symbols and their first derivatives  After this standard presentation in both Reichenbach and Einstein's lecture notes show how Einstein also used for the first new interpretation of the curvature in terms of the parallel displacement given by Tullio \citet{Levi-Civita1916} and Hermann \citet{Weyl1918}, who are mentioned explicitly. It is useful to sketch the fundamental ideas in order as premise to the of the paper.



%\begin{equation*}
%\christoffel{\mu}{\nu}{\tau} = \frac{1}{2} g^{\tau \sigma}\left(\frac{\partial x_{\mu \sigma}}{\partial x_{\nu}}+\frac{\partial g_{\nu \sigma}}{\partial x_{\mu}}-\frac{\partial g_{\mu v}}{\partial x_{\sigma}}\right)
%\end{equation*}.

%The latter which enter in the of the geodesics equations, and thus when move on straight line. We know from special relativity that a free particle moves along a timelike geodesic in a non-accelerated coordinate system:
%
%\begin{equation}\label{eq:geodesicchristoffel}
%\frac{d^{2} x_{\tau}}{\dap^{2}} + \christoffel{\mu}{\nu}{\tau} \frac{d x_{\mu}}{\dap} \frac{dx_{\nu}}{\dap} = 0\,,
%\end{equation}
%%
%where $\ap$ is the affine parameter of the geodesic. In unaccelerated rectangular coordinates $K$, the second term vanishes of \cref{eq:geodesicchristoffel}, and the equations reduce to, which are the equations of a straight line. When we switch to other coordinates $K'$ un uniform acceleration with respect to $K$, the path is no longer straight, that is, it is no longer given by a linear relation between the coordinates and right-hand side of \cref{eq:geodesicchristoffel} does not vanish. Again since uniform acceleration from being at rest in a uniform gravitational field, the presence of gravitation is characterized by the non-vanishing of the Christoffel symbols. The necessary and sufficient condition that we can find a coordinate system which is everywhere inertial, is that 
%
%\begin{equation}\label{eq:riemanntensor}
%\riteg =-\frac{\partial}{\partial x_{\tau}}\christoffel{\mu}{\sigma}{\rho}+\frac{\partial}{\partial x_{\sigma}}\christoffel{\mu}{\tau}{\rho} - \\ \christoffel{\mu}{\sigma}{\alpha}\christoffel{\alpha}{\tau}{\rho}+\christoffel{\mu}{\tau}{\alpha} \christoffel{\alpha}{\sigma}{\rho}\,.
%\end{equation}
%%
%It is possible to chose a coordinate system Chirstoffel symbols vanish everywhere if the Riemann tensor vanish. 
%
%\begin{equation*}
%\riteg = 0
%\end{equation*}The key conceptual is that to have extended equations also in the case in which $\riteg \neq = 0$;


In Euclidean geomtry it is always possible to introduce a Cartesian coordinate system in which two vectors are equal and parallel when they have the same components. However, this relation does not hold if we introduce curvilinear coordinates, \eg polar coordinates \citep[028-01-03, 35]{HR}. The coefficient are different from point to point, $\vartheta$. Thus, vectors at different points can no longer be directly compared. If one displaces a vector to a neighboring point $dx_\nu$, one does not know whether the vector has remained the \scare{same}, that is equal and parallel to itself, by simply looking at its components. To reestablished  (\german{Zusammenhang}), one requires to introduce a rule for comparing vectors at infinitesimally separated points. Given a vector $A^\tau$ at \xn in an arbitrary coordinate system, we need to determine the components of the vector $A^{\prime\tau}$ at $\xn+d\xn$ that is to be considered the \scare{same vector} as the given vector $A^\tau$. The vector $A^{\tau}$ at the point $P\left(x^{\nu}\right)$ and the vector $A^{\tau}+d A^{\tau}$ at the point $P^{\prime}\left(x^{\nu}+d x^{\nu}\right)$ are the \scare{same vector}, if they satisfy the condition:

\begin{equation}\label{eq:affine}
dA^\tau = \Gamma^\tau_{\mu\nu}A^{\mu} dx_\nu\,.
\end{equation}
%
the  quantity $\Gamma^\tau_{\mu\nu}$ is known as the \scare{affine connection} or displacement\footnote{The affine geometry is the study of parallel lines. \citet{Weyl1918b}. However, because it is a relation of \scare{sameness} rather than parallelism that is relevant in this context, others prefer to speak of the operation of \scare{displacement} (\german{Verschiebung}), where the latter indicates the small coordinate difference $d\xn$ along which the vector is transferred}. It has three indices, \ie, entails $\tau$ possible combinations of $\mu \times \nu$ coefficients, which can vary arbitrarily from point to point. Because in general $\Gtmn \neq \Gtnm$, the \Gtmn has $n \times n^2$ coefficients. If a vector $A^\tau$ is given at the point $P$ with coordinates \xn, \cref{eq:affine} yields the unknown components of the vector $A^{\prime\tau}$ at $P'$ with coordinates $\xn+dx_\nu$.

Levi-Civita assumed affine connection is symmetric $\Gtmn=\Gtnm$, which assure that the manifold \scare{flat in its smallest parts}. The affine connection, however, the length and angles of vector the inner product\todo{?}. He further imposed that condition that, in arbitrary coordinate system, vectors have length

\begin{equation*}\label{eq:metric}
l^2 = \gmn A^\mu A^\nu\,,
\end{equation*}
%
which does not change under parallel transport. If $A^\tau$ is displacement-vector $dx_\nu$, then thus the \cref{eq:metric} include \cref{eq:linelement} as a special case. this process. Thus, one can parallel-transport the vector $dx^\nu$ from any point to some other point, its length does not change.

%Wie also die Charakterisierung eines Vektors in P durch ein System von Zahlen (seine Komponenten) von der Wahl eines Koordi- natensystems abhängt,

%Thus the implication is that the length if two $ds$ every line is comparable with any other line. If these two conditions are imposed, one finds that the coefficients of the the affine connection \Gtmn are precisely the three-index Christoffel symbols that are obtained from the metric:


%define the law of parallel displacement, that is, the affine structure or space. If we have a space of given metric, we find that the T's are determined by the g's. It is sufficient to stipulate the condition that the modulus of the vector not vary, natural thing since it involves a repeated Euclidean process; that is, we must write that A2 = gIA'A* is invariant, which gives the conditions that the I's must satisfy, and one finds that they are precisely the three-index Christoffel symbols. It is easy to see that Riemann's fundamental tensor can be obtained with the parallel displacement.

\begin{equation*}\label{eq:levicivita}
\Gtmn=-\christoffel{\mu}{\nu}{\tau} = \frac{1}{2} g^{\tau \sigma}\left(\frac{\partial x_{\mu \sigma}}{\partial x_{\nu}}+\frac{\partial g_{\nu \sigma}}{\partial x_{\mu}}-\frac{\partial g_{\mu v}}{\partial x_{\sigma}}\right)
\end{equation*}

With this notion includes the notion of a parallel displacement. alone, without the aid of a metric, we may construct geodesics in the following manner. Continuing this process $\xn+d\xn+d^{\prime} \xn+d^{\prime \prime} \xn \ldots$, we can parallel displace a vector from any given point to any other distant point. As the size of each displacement goes to zero, this broken line becomes a continuous curve. Thus curve starts from P with a well-defined direction. If we use a parameter $\ap$ to designate points along the curve $dx\nu$, then $dx_\nu/d\ap=u^\tau$ is a vector, which The vector $u^\tau$ indicates the direction of the curve $x_\nu(\ap)$ at each point if its components are proportional to the increments $dx_\nu$ along the curve. By parallel-displacing a vector $u^\tau$ indicating the direction of a curve $\xn(\ap)$ at any of its points, one can define a special class of curves, the straightest lines among two points By parallel trasportin the vector $d x^{\mu}(P) / d s$ displaced to a point $P^{\prime}$ are identical with those of the vector $d x^{\mu} / d s$ at $P^{\prime}$, it can be shown:

%Now we use a parameter to designate points along the curve. Let $s$ be chosen to transform as a scaler and to have an invariant value at $P$ and $P^{\prime} .$ Then $d x^{\mu} / d s$ is a vector. The condition for a geodesic is that the components of the vector $d x^{\mu}(P) / d s$ displaced to a point $P^{\prime}$ are identical with those of the vector $d x^{\mu} / d s$ at $P^{\prime}$ :

 
\begin{equation*}\label{eq:geodesicequation}
\frac{d {u^\tau}}{d\ap} = \Gtmn u^{\mu} u^\nu
\end{equation*}
%
It is interesting that we have obtained this equation without recourse to the notion metric With this construction of a geodesic, as the straigthest line, rather that the shortest line. If \ap is the so-called \scare{proper time}, $u^\tau$ as the velocity four-vector of a particle, and $\frac{d {u^\tau}}{d\ap}$ its acceleration. The length of this vector is per defintion $=1$\todo{proper time}. 

As Einstein had remarked on several occasions, equation \cref{eq:geodesicequation} the conceptual core of \tr. According to special relativity, a freely movable body not subjected to external forces moves in a straight line and uniformly with respect to an inertial coordinate system $K$, in which the $\Gtmn=0$. Then, if we introduce new space-time co-ordinates $K'$ uniformly accelerated with respect to $K$, the $\Gtmn\neq 0$ and the trajectory of the particle is not straight. According to the equivalence principle $K'$ is however indistinguishable from $K$ in a gravitational field. Thus the non-vanishing of \Gtmn can be interpreted as the presence of the gravitational field. The \Gtmn are the components of the gravitational field, and the \gmn its potentials. The fact that \Gtmn are not a tensor represented form Einstein the \emph{unification} of inertial and gravitation. The fact that they are \german{wesengleich}. The same motion on a geodesic can be interpreted as inertial motion for $\Gtmn=0$ or as motion under the influence of a gravitational field for $\Gtmn \neq 0$.

%Indeed, the same hough a particle moves on a geodesic both in the absence and in the presence and in absence of 

%He believed that even though a particle moves on a geodesic both in the absence and in the presence of a gravi- tational field, a coordinate system can be chosen such that the connection components Γνμσ vanish or appear, and thus a gravitational field appears or disappears given a certain choice of coordinates

The crucial assumption which gives physical content to the \scare{trick} is that the new field is nontrivial, that is, in which one ca cannot set $\Gtmn=0$ everywhere in a finite region by a simple coordinate transformation. The notion of parallel transport provides a criterion to determine if it is possible to construct a coordinate system, with the $\Gamma$'s are zero. If any vector $A^{\mu}$ transported parallel around any closed curve must return to its initial point will be the same vector:

$$
\oint \Gtmn \frac{d A^{\tau}}{d \ap} d \ap=0
$$

The notion idea captures of \scare{curvature}. A connection is curved, if one parallel displaces $A^\tau$ along different paths, one gets, in general, a different vector $A^{\prime \tau}$ at a distant point \citep[028-01-03, 37]{HR}. \cop{Thus, the notion of \scare{sameness} of vectors refers only to neighboring points and can not in general be extended by continuous transfer along curve, to world points finitely apart from one another} \cop{A more convenient criterion for flat space without involving integration uses the tensor}, is to construct the following tensor:

\begin{equation}\label{eq:riemanntensorgamma}
R_{\mu \nu \sigma}^{\tau}(\Gamma)=\frac{\partial \Gamma_{\mu \nu}^{\tau}}{\partial x^{\sigma}}-\frac{\partial \Gamma_{\mu \sigma}^{\tau}}{\partial x^{\nu}}+\Gamma_{\alpha \nu}^{\tau} \Gamma_{\mu \sigma}^{\alpha}-\Gamma_{\alpha \sigma}^{\tau} \Gamma_{\mu\sigma}^{\alpha}\,.	
\end{equation}
%
This tensor Riemann-Christofel tensor \riteg. Levi-Civita and Weyl have had that can be obtained from a geometrical consideration based solely on the law of the affine connection. The manner in which the \Gtmn are expressible in terms of the \gmn plays no role in the derivation. This tensor the basis, of the general theory. From it the Ricci tensor and Ricci scalar. Finally the field equations can be derived by an action principle by setting the Lagrangian $ \sqrt{g}R$.

\begin{figure}
\begin{center}
\includegraphics[scale=0.3, trim = 0mm 0mm 0mm 0mm, clip]{parallelverschiebungtr.png}
\caption{Reichenbach's depiction of the parallel transport a vectors along different curves}
\end{center}
\end{figure}


%If the affine connection is flat the Riemann tensor. In this way one recover the apparatus of the absolute differential calculus, but with a more geometric approach. After these mathematical preliminaries, Reichenbach's-borrowing elements from Wey~1918b, bute~sentiallyfollowing Einstein 1916o (Vol. 6, Doe. 41)-uses variational methods to using $R$ as a scalar quantity.


%These results naturally lead to a generalisation of Riemann's geometry. Instead of starting off from the metrical relation (r) and deriving from this the coefficients I of the affine relation characterised by (2), we proceed from a general affine relation of the type (2) without postulating (1). The search for the



%and the law of energy-momentum conservation (see [pp. 13-17]). For the a?pro~ImatiVe mtegrat10n of the field equations, Einstein follows (with some small impr~vements)El~stem191Sa (Doe. 1), pp. 17-19. The derivation of the exact solution of the field equatiOns for a pomt mass and the perihelion advance of an orbit is taken from Weyl1918b (see (p. 21] and, for somewh~t more detail, [p. 24]). The final pages of these notes, (pp. 21-24], deal with cosmology and combme ele- ments from Einstein 1917b (Vol. 6 Doe. 43) and Einstein 1919a (Doe. 17). . . These lectures also cover the g~neralizationof electrodynamics from special to general relatiVIty ([p.I2]), the equations of motion for frictionless fluids ([p. 14]), and the behavior of rods and clocks Weak gravitational fields ([p. 20]). Instead of starting off from the metrical relation (r) and deriving from this the coefficients of the affine relation characterized by (2), we proceed from a general affine relation of the type (2) without postulating (1). 

\subsection{Weyl Theory}

The Levi-Civita-Weyl approach has a key conceptual advantage. As Einstein will repeatedly point out, it showed that the central concept was the \scare{displacement field} and not the metric. However, the approach had another implication. If one starts with a symmetric metric \gmn the road is marked. The Christoffel symbols are, so to speak, the only possible destination. However, if one defines the displacement \Gtmn independently from the metric \gmn, the Riemannian connection the connection \cref{eg:levicivita} appears only as a special case that has been achieved by introducing a series of conditions\todo{citations}. These conditions are at first sight natural, but by no means necessary. In 1919 Weyl had included the presentation of this development in the new editions of \citetile{Weyl1919} \citep{Weyl1919}. Weyl conceded that the connection must be symmetric. However, as it is well-known, Weyl was bothered by asymmetry comparison of direction of vectors which is path-dependent could not be the comparison of length is not. \qt{in any case in a pure \scare{neargeometrical} drop inadmissible assumption of the possibility of \scare{comparison at distance}: only distances that are located at the same place are}{ohnehin in einer reinen ,,Nahegeometrio" unzulässige Annahme der Möglichkeit des „Fernvergleichs" fallen lassen: nur Strecken, die sich an der gleichen Stele befinden}, that is in vector at the same point in different direction \citep[102]{Weyl1919a}.

%zieht und sich anheischig macht, aus der Weltgeometrie nicht nur die Gravitations-, sondern auch die elektromagnetischen Erscheinungen abzuleiten. Steckt diese Theorie auch gegen wärtig noch in den Kinderschuhen,

\cop{In order to obtain a truly nearby geometry, Weyl introduced metric \scare{connection} or displacement that make the congruent transport (of length) just as path-dependent as parallel transport}. If a vector of length $l$ is displaced from $x_\nu$ to $x_\nu$, it will in general have a new length $l+dl$, so that $dl/l=\phin dx_\nu$. In this way, in addition to the \scare{metric tensor} \gmn, a \scare{metric vector} $\phin$ of the same importance is introduced. As consequence, Weyl obtained a  symmetric affine connection which is more general than \cref{eq:levicivita} \citep[112]{Weyl1919} is then expressed in terms of the metric tensor and a four vector $\varphi_{\mu}$:

\begin{equation*}\label{eq:weylaffine}
\Gtmn = - \christoffel{\mu}{\nu}{\tau} +\frac{1}{2} g_{\mu}^{\tau} x_{\nu}+\frac{1}{2} g_{v}^{\tau} x_{\mu}-\frac{1}{2} g_{\mu \nu} x_\tau
\end{equation*}
%
By pluggin \cref{eq:weylaffine} into \cref{eq:geodesicequation} on btains the geosic equations. \cop{This makes it possible to determine geodesic null lines. The property of geodesic lines. that they are also shortest lines, is dropped in Weyl's geometry, because- the concept of a curve length becomes ~eaningless here.} An equivalent of the Riemann tensor \riteg can be introduced that can be split invariantly into two parts:

$$\bar{R}_{\tau \mu \nu}^{\sigma}=R_{\tau \mu \nu}^{\sigma}-\frac{1}{2} \delta_{\tau}^{\sigma} F_{\mu \nu}$$
%
where \rite is the curvature tensor in Weyl's theory and $\rite$ is the curvature tensor of Riemannian geometry (i.e., the Riemann tensor) add $F\mn$ is related which is the curl of the \phin. Thus, Weyl connections posses two curvatures. The Riemannian direction curvature, responsible for the change of direction under transport around a closed loop. and the \scare{length curvature} for the change of length. 

%where 4\lambda (x)$ is an arbitrary function of the space-time coordinates. where 1(x) is an arbitrary function of the space-time coordinates. This transformation was interpreted as a change of scale or the measuring standard (which, according to this theory, should be chosen at each point of space-time), since ds'2 = \ ds?.


%\cop{Furthermore, because by definition it measures the nonintegrability of length---the amount that the metric changes under transport around a closed loop---its vanishing is then necessary and sufficient condition for the recovery of Riemannian geometry}.


%This assumption is equivalent to postulating the non-vanishing of the homothetic curvature and the conservation of elements a and b. Analytically, this assumption introduces gauge transformation in addition to coordinate transformations. The affine connection is then expressed in terms of the metric tensor and a four vector $\varphi_{\mu}$ tied to the gauge transformations. Thus, to determine the structure of the universe and the field equations, it is necessary to use the ensemble $\left(\mathrm{g}_{\mu \nu}, \varphi_{\mu}\right)$ and not simply $\mathrm{ds}^{2}$.

%where Pijk/ is anti-symmetric in the indices i and i j as well as k and l. Whereas the equations Fik = 0 characterize the absence of an electromagnetic field i.e. a space in which the transfer of magnitude is integrable, one sees from (13) that pi, 0 are the invariant conditions for the absence of a gravitational field i.e. jkl for the parallel transfer of directions to be integrable. Only in Euclidean space is there neither electromagnetism nor gravitation.

%two fundamental metric forms: the quadratic Gia da' dick (gravitation) and the linear $; dri (electricity). The linking of these two ranges of pheno-

This geometrical setting could be used as the starting  The search for such theory usually implies different steps, of which Weyl had introduced a new presentation of the second edition of his texstbook:

\begin{itemize}
\item the first step is to the geometrical field-structure in this case the fundamental variables are \gmn and \phin. That the \gmn are identified with the potentials of the gravitational field because of a \emph{physical fact} the equivalence principles. The \phin could be identified with the potentials electromagnetic field  because of the \emph{mathematical fact} that the $F\mn$ is the curl of the \phin, like in the first two Maxwell equations. 

\item the second step is find the field equations, via the \scare{action principle}. \cop{One constructs a scalar quantity (the action) from the dynamical quantities \gmn and \phin then finds the conditions needed to restrict the scalar to an extremum (a maximum or minimum) with respect to variations in those dynamical quantities. The problem the right action and the right dynamical quantities to produce the desired equations}, that is to recover Einstein and Maxwell field equations.

\item The final step is the comparison with experiment; in particular to see, if in addition if they imply the existence of the electron and other unexplained atomic phenomena. \co{On the other hand the stable solutions of the equations for the "problem of matter". satisfying adequate regularity conditions should lead to a discrete set of solutions depending on some parameter 3. This expectation had a (formal) similarity to a set of "discrete eigenvalues" of an operator, although here the operator was not linear.} \todo{Weyl1919}
\end{itemize}

Point to construct a geometrization of gravitational phenomena, his new theory represented a unified geometrization of both gravitational and electromagnetic phenomena, which were, at that point, the only kind known\footnote{\q{We have realized that physics and geometry coincide with each other and that the world metrics is one, and even the only one, physical reality}}. \q{Untersuchung herausstellen, daß diese Unterscheidung zwischen Geometrie und Physik ein Irrtum ist, da die Physik gar nicht über die Geometrie} \citep{Weyl116}

%If charges are present at all, this constant cannot vanish. If, in addition, it is assumed to be positive, it follows automatically that the curvature of space is positive and that the universe is finite, so that it is unnecessary to. add a special A-term to the gravit~tional equations.

%\cop{As for the gravitational equations· themselves, finally, these are not identical with Einstein's equations, even in the absence of an electromagnetic field (r/>, = 0), as might have been expected from earlier arguments, and they are of higher order than the second.}


%However, the rethorical declaration the goal was far more complicated. 

%to utter contempt. He admired Weyl's theory "*as a chain of ideas" (Doc. 59), but as a theory of physical reality it was to him "fanciful nonsense" (Doc. 294). By including general relativity into the third edition of his textbook on "Space-TimeMatter," Weyl had, according to Einstein, "messed it up" (Doc. 332).

%gemeine werde also leider nicht in G[ottingen] sein konnen. * Weyls Theorie bewundere ich sehr als Ideen-Folge.>| Aber ich glaube nicht, dass sie der Wahrheit näher führt. Das Aufgeben der metrischen Bedeutung des ds scheint mir nicht begründet, zumal man gezwungen wird die Feldgleichungen als Gleichungen vierter Ordnung anzusetzen. 6)

%59. To David Hilbert Berlin 11. VI. 19

%Es ist eine Grundeigenschaft der Naturvorgänge, von der Vorgeschich- te unabhängige Massstäbe und Uhren zuzulassen. Diese aber erlauben es, zwei benachbarten Weltpunkten eine experimentell bestimmbare Zahl ds zuzuordnen, während Weyls Theorie die Nichtexistenz eines derartigen ds zur Voraussetzung hat. Ich bin ganz überzeugt, dass diese Theorie den Thatsachen gegenüber versagen wird.[111 Was die Linienverschiebung (Sonne) anbelangt, so ist vom Experiment aus das letzte Wort sicherlich noch nicht gesprochen. Man muss zuerst die Bogen- lampe durch eine physikalisch einwandfreiere irdische Lichtquelle ersetzen. Sie werden schon sehen, dass die Theorie endlich vollkommene Bestätigung finden wird. Bei den Fixsternen haben sich neuerdings eklatante Bestätigungen (qualita- tiv) ergeben.[ 121Ein Versagen der Verschiebung der Sonnenlinien würde nach mei- ner Überzeugung die ganze Theorie umwerfen. Was Weyls Theorie anbelangt, so sind ja nach ihr nicht die q>v sondern allenfalls das Integral Jq>vdxv für das metri- sche Verhalten der Uhren massgebend, ohne dass dies mit Sicherheit behauptet werden könnte. Denn so weit ist die Theorie nicht durchgeführt, dass das Verhalten eines als "Uhr" auffassbaren Vorganges aus der Theorie deduziert wäre.[ 131 Sie sprechen von Bohrsehen Bahnen, die sich "nicht ändern" sollen. Aber diese Aus- sage hat eben, von der Weylschen Theorie aus gesehen, keinen Sinn, weil der Be- griff der "natürlich gemessenen Längen" aufgegeben ist.-

%78. To Adriaan D. Fokker Luzern 30. VII [1919} 1] Lieber Herr Fokker!




\subsection{The Bad Nauheim meeting of September 1920}

%Im Nov. 1920 wollte ich einen populären Aufsatz über Rel. th. f. d. Umschau schreiben. I Ich kam auf den Gedanken,

Reichenbach met Weyl for the first time at the 86th Assembly of the \german{Versammlung der Gesellschaft Deutscher Naturforscher und Ärzte} in Bad Nauheim in September 1920. In his talk \citet{Weyl1920a} introduced the distinction between \german{Einstellung} and \german{Beharrung} to explain away the discrepancy between the non-Riemannian behavior of the \scare{ideal} time-like vectors implied by his theory and the Riemannian behavior of the \scare{real} clocks that are actually observed. He suggested that atomic clocks might not \emph{preserve} their Bohr\todo{Laue} radius if transported, but \emph{adjust} it every time to some constant field quantity, which he could identify with the constant radius of the spherical curvature of every three-dimensional slice of the world, furnishing a natural unit of length. The fact that all atoms of the same type are exactly identical clearly cannot depend on an initial agreement established in the past, which has been \scare{preserved} since then, even though the atoms had encountered very different physical circumstances. It was more plausible to argue that they \scare{adjust} anew each time to a certain equilibrium value. Thus one might surmise that vectors behave in a non-Riemannian way, whereas atoms used as clocks, which are after all physical systems like any other, appear to have a Riemannian behavior.


%\footnoteh{He suggested that atomic clocks might not \emph{preserve} their Bohr radius if transported, but \emph{adjust} it every time to some constant field quantity. Weyl suggested that the atoms we use as clocks might not preserve their size if transported, but adjust it every time to some constant field quantity, which he could identify with the constant radius of the spherical curvature of every three-dimensional slice of the world, furnishing a natural unit of length. The geometry read off from the behavior of material bodies would appear different from the actual geometry of space-time, because of the \scare{distortion} due to the mechanism of the adjustment. Two identical \scare{classic} atomic systems with different prehistories would probably differ in some small detail due to their interaction with the environment, and their spectral lines would be slightly shifted, so that classically, a spiraling charge should emit light of all colors. Emerging quantum theory had already made clear that the spectral identity of atoms revealed by experience cannot be explained in this framework. The fact that all atoms of the same type are exactly identical clearly cannot depend on an initial agreement established in the past, which has been \scare{preserved} since then, even though the atoms had encountered very different physical circumstances. It was more plausible to argue that they \scare{adjust} anew each time to a certain equilibrium value}. 

%The size of an electron is determined by adjustment in proportion to the radius of curvature of the world, and not by persistence of anything in its past history. This is the view taken in § 66. and we have seen that it has great value in affording

In the discussion followed, commenting on Weyl's talk, Einstein reiterated his critiques. He pointed out once again that the \q{arrangement of \textins{his} conceptual system,} \q{it has become decisive \origins{massgebend} to bring elementary experiences into the language of signs \origins{Zeichensprache}} \citep[650]{Einstein1920c}. For Einstein, \q{temporal-spatial intervals are physically defined with the help of measuring rods and clocks}, under the assumption that \q{their equality is empirically independent of their prehistory} \citep[650]{Einstein1920c}. Einstein insisted that precisely upon this assumption rests \q{the possibility of coordinating \origins{zuzuordnen} a number $ds$ to two neighboring world points}; if this were impossible, general relativity would be robbed of \q{its most solid empirical support and possibilities of confirmation} \citep[650]{Einstein1920c}.

% The time or space intervals have a physical basis only if there is some actual or possible physical process that has a length or a duration shorter or equal to the space or time interval in question. A distance smaller than the electron would be physically meaningless since there is no physical process that could realize such an interval. The attempt to define the electromagnetic field or gravitational field in the interior of elementary particle to account for their stability should be rejected on epistemological grounds.


However, Einstein was immediately forced to open the possibility of a different stance by replying to a comment of Pauli's. The goal of Weyl theory was to construct a field theory of matter, in which the electron is a region of the field in which the field straight are enormously concentrated. However, the field strength in the interior of the electron is meaningless because there is no smaller test particle than the electron; \q{one could claim something similar concerning spatial measurements, \myemph{since there are no infinitely small measuring-rods}} \citep[650]{Einstein1920c}. If e.g. the theory claims that geometry of is non-Euclidean within elementary particle, there is no way to check this prediction just like we can check that the geometry around the sun is non-Euclidean. Einstein replied to Pauli that \q{with the increasing refinement of the system of scientific concepts, the manner and procedure of associating the concepts with experiences becomes increasingly more complicated} \citep[650]{Einstein1920c}. In particular, he recognized that in cases such as that of the continuum theories, \q{one finds that a definite experience cannot be associated any longer with a concept} \citep[650]{Einstein1920c}. According to Einstein, there the physicists is at a crossroads: one can abandon \scare{continuum theories} for the sake of Pauli's observability criterion, or replace such a \q{system of associating concepts \textins{with experiences} with a more complicated one} \citep[650]{Einstein1920c}. A decision as to which alternative is more suitable, Einstein pointed out, can only be given on the basis of pragmatic reasons \citep[650]{Einstein1920c}. Indeed, it would soon become clear that Einstein ultimately opted for the second choice.  One can already glimpse the main lines of Einstein's in his contributions to the the discussion which followed Max von \citets{Laue1920}'s Bad Nauheim paper on the gravitational redshift\footnote{Laue showed that the coordinate interval $d\vartheta$ measured by an atom on the sun is transmitted unchanged by light signals (at least in a static gravitational field\footnote{In the general case, the number of vibrations of an atom transmitted by light signals is coordinate dependent}), so that the redshift emerges by confronting the frequency of such signals with those of an atom of the same type at rest measuring the proper time $d\tau$}, in which Einstein did not hesitate to admit that \q{[it] is a logical shortcoming of the theory of relativity in its present form to be forced to introduce measuring rods and clocks \myemph{separately instead of being able to construct them as solutions to differential equations}} \citep[Einstein's reply to][662\me]{Laue1920}. Thus, Einstein now openly admitted that it would have been logically or epistemologically preferable if the field equations of the theory had suitable solutions corresponding to particles, from which in principle the stability of a more complicated, bulky configuration of matter could be reconstructed, including rod- and clock-like structures. In this way the necessity of coordinating the geometrical/kinematical structure of the theory separately from the rest in terms of rods-and-clocks behavior would fall and with it also Pauli's objection that such definition is impossible within elementary particles.

Thus Einstein vacillated between to different epistemological stances. On the one hand he considered that \textit{sub specie temporis}, thatthe  \q{invariant $ds$ is connected with observable facts [measured using rods and clocks], just like it happened to the fundamental concepts of Maxwell theories through Heinrich Hertz} \CPAE{??}{??}. However, he continued, he conceded that this empistemological models fails in the infisimaly small and in the finitesimaly large, where no such \rac might be at our disposale. Thus \emph{sub specie aeterni} \rac should do not play the part of irreducible elements, but that of composite [atomic] structures, that comes of out of the theory at the end. Thus, provisionally that geometry can be tested empirically separately from the rest of physics, indeed \gr seems to this the \gmn as measured with \rac with respect to a given corodiante syste, However, in the general case only geoemtry and physocs togehter. One would chose a certain geometrical strucutre, that leed to right field equaitons. The latter would have soltuons atomic struc, wthat would serve as rac.  geometry cannot be tested separately from ultimately only the theory as a whole $G+P$. 

Reichennach and most logical seems to have missed that adress Pauli's and Weyl's crossfire. It was also to open the possibility of his own \uftp, read the lecture simply of between geometry and physsics, whereas that this separation for the sake of \uftp. Indeed, in the following months Einstein will soon first unified field theory conformal. Weyl has accepted the existence of transportable \rac but denied that they preseved their lengths. That \q{mit WEYL auf die Voraussetzung 11, sondern auch auf die Voraussetzung H von der Existenz übertragbarer Maßstäbe (bzw. Uhren) von vornherein verzichtet. Im folgenden soll null gezeigt werden, daß}. In the new theory only the $ds=0$ has a physical meaning. This approach an important stepping stone to will soon abandon the very idea that the notion of parallel transport of vectors, which however have no physica meaning at all, to Reichenbach's dismay.

%That Weyl's theory has the pretence, but . Then was to renouce complelty as Einstien  sugested in March 1921, to avoid the use of \rac alltogeher. That this approach the separation between geometry could be tested separately from physics, then search for the field equations ... then will be considered a proper geometrical strucutre. This was probably the of the famous $G+P$ formlula. In the special and as in general to a  physocal emaing to the geometrical strucutre of $\gmn$ indepednen of the field equaitons. Indeed, \rac just like in the where measure by particel after made some predicitons. The that this model might have be abandoend. The ... garatnede of the theoyr as whole. In partiuclar. In March 1927 Einstien will prefere to set up a theory in which \rac where not part of the fundamental strucutre

%This complicated set of is reflected, which were to interpret. Indeed, there Einstein could precisley, gometry is Riemannain since behave  and the question whether in cosmology and infinte particels this assumption be dropped. Then the choicen becuase of its role in elementary particles, neither the could be defined in this way. The coordination a potential of a sever. Einstien will turned to the second method whe it will be conformatal. 