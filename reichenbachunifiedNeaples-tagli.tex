%The theory, must just in the case of Einstein \hide{Genau, wie Einstein zeigen muß, daß aus der Dynamik des starren Körpers heraus ein solches Verhalten folgt, daß der Maßstab iumıer dieselbe Länge hat, gemessen iu seinenı, Is, so muß ich zeigen. daß er inmıer das gleiche durch $R = const$. normierte $ds$ lıat. Wir etwa Einstein so gut als ich das zu maclıen hätten, habe ich am Schluß meiner Arbeit " Feld und Materie”. Ann. d. Plıysik angedeutet.} . However, Reichenbach would have not been impressed by this argument. Indeed, Weyl seems to believed purely infinitesimal, was to chose a more special one. But Weyl geometry was not the most general one, that choice of such geometry was completely and would have  only at the end when would be able to integrate the equations and obtained \rac as solutions.




%Two critique, one that epistemological, better than other geometries (there is no reaso, eliminat the asumm), that thus that one should give ... however the theory, formalism, that iintrisc superiority of Weyl geometry is not even true, an there are 27 different connections among which one can chose.  A second aspect is that Weyl requires an, there is than as at least a good overview of some differential geometry, not Weyl but also Eddington and even Schouten. is even more surprising the work of Schouten 1921 classifies 18 possible linear affine connections (\scare{Übertragungen}) numbered as I,..., VI a?c. In Schouten 1922 he consider by he improved he classification trhee tensor 27 possible connection. All the further less general cases of linear connections are obtained by introducing restrictions to such quantities (for more details see Vizgin 1994, 184, Goenner 2004). This classification was Schouten point of view not only Schouten that Weyl was not but one could think in which. The third tensor which never plaed the role, the pwer can be. In which the tensor of asymmetry vanish symmetric, but the tensor of non metricity does not vanish. That more general even non-symmetrical connections (IVc). In which the tensor of non metricity vanish, but the tensor of asymetry does not, which will be the geometry used by Reichenbach  That even Schouten most general linera, lieanr But ieven the condition that a connection is linear was not necessary. Start from a general affine connection, and then restrict the possibilities. However, there was no particular. 

%1922 French article, that was empirical fact that \rac behave in a Riemannian way. Thus this resctriv that to Riemannian geometry. However, which Riemannian geomtry remains a question of choince. Idneed, one can imm  \scare{Darrigol classes}. However, was on a fact, that in nature there are. The choice between geometry is conventional; indeed there might be didferent. By the there are no differential forces. 











%lists 18 different linear connections and classifies them invariantly. The most general connection is characterized by two fields of third degree, one tensor field of second degree, and a vector field. These fields are the symmetry tensor $S_{\lambda \mu}^{v}$ the tensor of $Q_{\lambda \mu}^{v}$. One,

%. and a vector C „_ which follows from Cfμ = C μ ôff while C ,f μ = I`§μ+l`í';1, if l" stands for the connection for tangent vectors and 1" ' for the con- nection for linear forms. Torsion is defined by Sfμ = 1/2(l`§μ - FZÄ), non-uıflicity by Vμg“ = Qt“. Furthermore, on page 57 we find: “The general connection for n = 4 at least theoretically opens the door for an extension of Weyl's theory. For such an extension an invariant affixation of the connection is needed, because a physical phenomenon can correspond only to an invariant expression."



%Moreover, expressed his, howeveer, that privildege respect to Eddington's theory. Weyl probably, Das alte MS wurde vš\"ollig umgestossen. 


%[Berlin,] Mittwoch [20 September 1922] %The second si that even if we can define, that this particular geometrical structure. There was not start which is undefined, that the graphical presentation was precisely the reason for rejecting the theory. Again Reichenbach's critique would follow Pauli's ideas. In the meanintime Reichenbach had finished, which will however, for most of Weyl's, which interrupted 
%
%%From Hans Reichenbach Stuttgart, 19 April 1923 Asks for help in finding a publisher for Reichenbach 1924. ALS. [20 079]. 50. Ilse Einstein to Hans Reichenbach Berlin, 12 May 1923 Informs him that AE has not seen Reichenbach’s letter since he already left for the Netherlands. She no longer forwards AE’s mail since he brings it all back unopened. She assures Reichenbach that she will give his letter to AE upon his return. AKS. [87 944]. 122. From Hans Reichenbach Stuttgart, 10 July 1923 Thanks AE for Abs. 89. Is sorry that the Academy did not agree to print his manuscript. Asks whether this was due to financial or other reasons. Springer cannot accept his suggestion that the \textit{Notgemeinschaft} cover part of the printing cost. Verlag Witwer, on the other hand, has agreed to publish the work with support from the \textit{Notgemeinschaft}. Enclosed sends the request to the \textit{Notgemeinschaft} and asks to forward it to Fritz Haber and to put in a good word for him personally. Asks to mail back the manuscript. ALS. [20 082]. During 1924 Reichenbach finally managed to published, his Axiomatic in which. the negative review finally convinced him of the debacle, Weyl. In 1925 with possibly.


%Begriff der Parallelverschiebung entstammt wie alle Begriffe der euklidischen Geometrie der Betrachtung der Lagerungs-Gesetze bezw. der Gesetze der relativen Verschiebung starrer Körper. Daher erhält die (Behauptung) Festsetzung ihre Evi-

%kann ein Gesetz (2) des affinen Zusammenhanges one physikalische Interpretation mittelst des starren Körpers einführen. Aber es ist dann ziemlich willkürlich, von diesem (affinen) Gesetz zu fordern, dass es das Verhältnis der Beträge zweier Vektoren bei der Verschiebung ungeändert lasse (wenn man die Interpretation von (2)).18]

%\cop{In Weyl's theory, a gauge-system is partly physical and partly conventional; lengths in different directions but at the same point are supposed to be compared by experimental (optical) methods; but lengths at different points are not supposed to be comparable by physical methods (transfer of clocks and rods) and the unit of length at each point is laid down by a convention. I think that this hybrid definition of length is undesirable,}. 




One can think of $d\xn$ as the components of a (contravariant) vector $A^\tau$, $n$ numbers $A^\mu$ ($A^1, A^2, A^3, A^4) that we associate with some point $P$ and transform as per certain rules by the change of coordinates. In Euclidean geometry, it is always possible to introduce a Cartesian coordinate system in which two vectors are equal and parallel when they have the same components. However, this relation does not hold if we introduce curvilinear coordinates, \eg polar coordinates. Although parallel vectors are still parallel in the new coordinate system, the equality of the components of two parallel vectors attached to two different points in space is not preserved. \Eg consider two unit vectors $A^\tau$  and $A^{*\tau}$ on a plane pointing along the $x$ direction: one at the point at $(0,1)$ and another at $(1,0)$ in Cartesian coordinates. In this coordinate system, $A^\tau$  and $A^{*\tau}$ have the same components, \ie they are equal and parallel. However, in polar coordinates $r,\vartheta$ (where $r$ represents distance from the origin, and $\vartheta$ represents the angle that the point makes with the origin and the positive $x$-axis), $A^\tau$ has only a $r$ component, whereas $A^{*\tau}$ has only a $\vartheta$ component. Nevertheless, they are still equal and parallel. Indeed, the vector $A^*\tau$ can be obtained by displacing $A^\tau$ parallel to itself along a circle. In polar coordinates, the components $A^\tau$ change at each point even though its length and direction remain the same

In Euclidean geometry, it is always possible to introduce a Cartesian coordinate system in which two vectors are equal and parallel when they have the same components. However, this relation does not hold if we introduce curvilinear coordinates, \eg polar coordinates. Consequently, vectors at different points can no longer be directly compared. If one displaces a vector to a neighboring point $dx_\nu$, one does not know whether the vector has remained the \scare{same} by simply examining its components. The \scare{connection} (\german{Zusammenhang}) from a point to another is lost. Because the affine geometry is the study of parallel lines, \citet{Weyl1918b} used to speak of the necessity of establishing an \scare{affine connection} (\german{affiner Zusammenhang}). However, because it is a relation of \scare{sameness} rather than parallelism that is relevant in this context, others, such as Reichenbach, prefer to speak of the operation of \scare{displacement} (\german{Verschiebung}), where the latter indicates the small coordinate difference $d\xn$ along which the vector is transferred. 


To reinstate the \scare{connection} one requires to introduce a rule for comparing vectors at infinitesimally separated points. Given a vector $A^\tau$ at \xn in an arbitrary coordinate system, we need to determine the components of the vector $A^{\ast\tau}$ at $\xn+d\xn$ that is to be considered the \scare{same vector} as the given vector $A^\tau$. The vector $A^{\tau}$ at the point $P\left(x^{\nu}\right)$ and the vector $A^{\tau}+d A^{\tau}$ at the point $P^{\ast}\left(x^{\nu}+d x^{\nu}\right)$ are the \scare{same vector}, if they satisfy the condition:

\begin{equation}\label{eq:affine}
dA^\tau = \Gamma^\tau_{\mu\nu}A^{\mu} dx_\nu\,.
\end{equation}
%
The quantity $\Gamma^\tau_{\mu\nu}$ is known as the affine connection or displacement. It has three indices, i.e., entails $\tau$ possible combinations of $\mu \times \nu$ coefficients, which can vary arbitrarily from point to point, i.e., in the general case, are functions of $x_\nu$. Because in general $\Gtmn \neq \Gtnm$, the \Gtmn has $n \times n^2$ coefficients. If a vector $A^\tau$ is given at the point $P$ with coordinates \xn, \cref{eq:affine} yields the unknown components of the vector $A^{\ast\tau}$ at $P^*$ with coordinates $\xn+dx_\nu$. Continuing this process $
\xn+d\xn+d^{\ast} \xn+d^{\ast \ast} \xn \ldots$, we can parallel displace a vector from any given point to any other distant point. As is well known, the most characteristic feature of the operation of displacement is that if one parallel displaces $A^\tau$ along different paths, one gets, in general, a different vector $A^{\ast \tau}$ at a distant point:

It is naturally to assume that is symmetric. The displacement allows to establish whether two vectors are the \scare{same}, i.e., having the same length and the same direction. However, it does not provide a measure of the length of differently directed vectors. For this purpose, the notion of dot product of two vectors must be introduced, which, taking the components of the two vectors, returns a single number. In particular, the squared length $l$ of a vector is given by the dot product of the vector with itself $l^2$. In an arbitrary coordinate system, the latter takes the form:

\begin{equation}\label{eq:3}
l^2=\gmn A^\mu A^\nu\,,
\end{equation}
%
where the \gmn is the metric. If $A^\tau$ is considered to correspond to $dx_\nu$, \cref{eq:3} is nothing but \cref{eq:lineelement} and $l$ corresponds to $ds$. However, this notation is more general. One can take $A^\tau$ to be $dx^\nu/ds$, (where $ds$ is the \til interval, which is an element of the four-dimensional trajectory of a moving point), $l$ is the length of the four-velocity vector $u^\nu$. How much a vector varies, this is simple in the case that, covariant differentiation, a second term is necessary, that is that the covariant of the metric vanish


By imposing this condition, one obtains 

\begin{equation}\label{eq:riemannconnection}
\Gtmn=-\christoffel{\mu}{\nu}{\tau} =
\end{equation}
%

The components of \Gtmn have the same numerical values of the so-called Christoffel symbols of the second kind (up to a sign) because they are calculated from the metric \gmn and its first derivatives.




%\cop{Yet it is incorrect to conclude, like Weyl and Haas, that mathematics and physics are but one discipline. The question concerning the validity of axioms for the physical world must be distinguished from that concerning possible axiomatic systems. It is the merit of the theory of relativity that it removed the question of the truth of geometry from mathematics and relegated it to physics. If now, from a general geometry, theorems are derived and asserted to be a necessary foundation of physics, the old mistake is repeated. This objection must be made to Weyl's generalization of the theory of relativity which abandons altogether}. 
%
%
%\q{What is the length of a physical rod? It is defined by a large number of physical equations that are interpreted as length with the help of readings on geodetic instrumets. The definition results from a coordination of things to equations. 1--hus we are faced with the strange fact that in the reahn of cognition two sets are coordinated, one of which not only attains its order through this coordination, but whose elements are defined by 1neans of this coordination}
%
%\cop{Schlick is therefore right when he defines truth in terms of unique coördination. ${ }^{11}$ We always call a theory true when all chains of reasoning lead to the same number for the same phenomenon. This is the only criterion of truth; it is that criterion which, since the}. \cop{By means of which principles will a coördination of equations to physical reality become unique?}. \cop{There exist systems of coördinating principles which make the uniqueness of the coördination impossible; that is, there exist implicitly inconsistent systems.}.
%
%
%\cop{When ever have discovered a coordinating principle used in physics, we can indicate a more general one of which the first is only a special case. We might now n1ake the attetnpt to call the more general principle a priori in the traditional sense and to ascribe eternal validity at least to this principle. But such a procedure fails because for the more general principle an even general one can be indicated; this hierarchy has no upper limit}.
%



%





%%21 (p. 76). Hermann Weyl, Raurn-Zeit-Materie (Berlin: Springer, 1918), p. 227; Arthur Haas, " Die Physik als geometrische Notwendigkeit,'' NatltTwissenschaften} VIII, 7, pp. 121-140.
%
%%If, for instance, Weyl's generalization should turn out to be correct, a new subjective element \vill have appeared in the 1netric. Then the comparison of nvo s1nall n1easuring rods at two different space points also no longer contains the objective relation that it contains in Einstein's theory in spite of the dependence of the 1neasured relation upon the choice of the coordinates, but is only a subjective form of description, con1parable to the position of the coordinates.



%Der erste Punkt betrifft Herrn Dr. Reichenbach, dem ich möglichst schnell zu Habilitation verhelfen möchte.[2] Er hat dazu eine Arbeit: „Die Bedeutung der Re- lativitätstheorie für den physikalischen Erkenntnisbegriff“ eingereicht. Er sagte mir, dass Sie die Arbeit schon gelesen hätten, sodass ich sie Ihnen nicht zu schicken brauchte.[3] Ich muss nun sagen, dass meine eigene Erfahrung nicht hinreicht, um die Arbeit genügend zu würdigen und ich glaube ferner, dass Ihr Urteil über die Ar- beit bei der Hochschule hier sehr viel wiegen wird, sodass ich für den glatten Ver- lauf der Habilitation Günstiges erhoffe. Es wäre das deswegen besonders wertvoll, weil die Arbeit doch zum grossen Teil philosophischen Charakters ist, Herr Rei- chenbach sich aber für Physik habilitieren soll. Nun glaube ich, dass auch Sie mit mir übereinstimmen werden darin, dass gerade solche Arbeiten auch für den Phy- siker bei der augenblicklichen Entwicklung in der Physik sehr nützlich sind, sodass seine Arbeit auch dem Herrn Reichenbach zur Habilitation inder Physik dienen kann.

%The thesis was published after 15 June 1920 as Reichenbach 1920 (see Doc. 57). Stuttgart, Wiederholdstr. 13. d. 15. Juni 1920. Widmung Reichenbach attended Einstein's Ieerures in Berlin (see Doc. 57, note 2). That most of the discussion of the book, relativized constitutive \apr, however, one first by a philosophers of Weyl theory, and in particular was that is itself superior. \q{Philosophen eine Ahnung davon haben, dass mit Ihrer Theorie eine philosophische Tat getan ist, und dass in Ihren physikalischen Begriffsbildungen mehr Philosophie enthalten ist, als in allen vielbändigen Werken der Epigonen des grossen Kant.} \q{die tiefe Einsicht der Kanti- schen Philosophie von ihrem zeitgenössischen Beiwerk zu befreien} \cop{Der Wert der Rel.Th. für die Philosophie scheint mir der zu sein, dass sie die Zweifelhaftigkeit gewisser Begriffe dargethan hat, die auch in der Philosophie als Scheidemünzen anerkannt waren}. 

%Weyl's presentation was Pauli and Einstein, was included the 4th edition of Raum, Zeit, Materie (Weyl, 1921a) finished in November 1920, to explain ?the discrepancy between the idea of congruent transfer and the behavior of measuring-rods and clocks and atoms? (Weyl, 1921a, 280; tr. 1922a, 308).



%The mechanism of adjustment is the only way to explain the surprising fact that electrons and hydrogen nuclei always have the same mass and charge, and thus the very existence of identical atoms, along with the possibly that these atoms, under given external conditions, settle into identical crystalline structures, that we can use as rods.  Weyl thus had good reasons to claim that the only explanation for the fact that electrons always have the same charge, whatever their prehistory, might have been to assume that there is some field quantity of the dimension of a length (i.e., it is simply a number) to which the charge of the electrons ?adjust? themselves in a certain proportion. This was, after all, the general framework that, e.g. Mie (who also gave a talk at Bad Nauheim, Mie, 1920) had tried to realize without any success (cf. also footnote 5). A certain state of equilibrium of the negative (or positive) electricity would always be ?reestablished? whatever disturbance it may have experienced in the past, just as the 
%magnetic needle always points north, despite what may have happened to it previously. 


%Reichenbach's criticism, Erwin Freundlich at about the same time. There was indeed, that the theory could was either refuted, by assuming the physical measuing rod, behave differently from ideal ones, which follow the geometry of \st. 

%Indeed, a few days later \cop{Einstein wrote to Cassirer a few days later, that the meaning of the $ds$}, how the coordination of his objection against Weyl. That was meaningless without rods and clocks 


%Weyl's presentation was Pauli and Einstein, was included the 4th edition of Raum, Zeit, Materie (Weyl, 1921a) finished in November 1920, to explain ?the discrepancy between the idea of congruent transfer and the behavior of measuring-rods and clocks and atoms? (Weyl, 1921a, 280; tr. 1922a, 308).



%The mechanism of adjustment is the only way to explain the surprising fact that electrons and hydrogen nuclei always have the same mass and charge, and thus the very existence of identical atoms, along with the possibly that these atoms, under given external conditions, settle into identical crystalline structures, that we can use as rods.  Weyl thus had good reasons to claim that the only explanation for the fact that electrons always have the same charge, whatever their prehistory, might have been to assume that there is some field quantity of the dimension of a length (i.e., it is simply a number) to which the charge of the electrons ?adjust? themselves in a certain proportion. This was, after all, the general framework that, e.g. Mie (who also gave a talk at Bad Nauheim, Mie, 1920) had tried to realize without any success (cf. also footnote 5). A certain state of equilibrium of the negative (or positive) electricity would always be ?reestablished? whatever disturbance it may have experienced in the past, just as the 
%magnetic needle always points north, despite what may have happened to it previously. 


%Reichenbach's criticism, Erwin Freundlich at about the same time. There was indeed, that the theory could was either refuted, by assuming the physical measuing rod, behave differently from ideal ones, which follow the geometry of \st. 

The importance of this formalism for \gr was to which was the key content of the theory. Let's conisder the four-velocity of a particle $u^\tau=d\xn/d\ap$. By parallel-displacing a vector $u^\tau$ indicating the direction of a curve $\xn(\ap)$ at any of its points, one can define a special class of curves, the straightest lines. The operation of parallel transporting a vector $u^\tau$ along a curve $x(\ap)$ is expressed by the condition that the covariant derivatives of $u^\tau$ with respect to the parameter \ap vanish along the curve: $\frac{d u^{\tau}}{d \ap}-\Gtmn u^{\mu} \frac{d x^\nu}{d\ap}=0$. The vector $u^\tau$ indicates the direction of the curve $x_\nu(\ap)$ at each point if its components are proportional to the increments $dx_\nu$ along the curve, \ie if $u^\tau=d\xn/d\ap$. The curve traced by the parallel displacement of $u^\tau$ along its own direction $d\xn/d\ap$ is the straightest curve. According to special relativity, a freely movable body not subjected to external forces moves, according to the special theory of relativity, in a straight line and uniformly with respect to an inertial coordinate system $K$:

\begin{equation}\label{eq:geodesicelectro}
\frac{d^{2} x_{\tau}}{d s^{2}} = 0
\end{equation}

If \ap is the so-called \scare{proper time}, $u^\tau$ as the velocity four-vector of a particle, and $\frac{d {u^\tau}}{d\ap}$ its acceleration. If now consider from the of a we can, that this is ultimately. If chose a suitable coordinate system that the right had sight of the equation vanish. Then if we introduce new space-time co-ordinates $x_{1}, x_{2}, x_{3}, x_{4}$, by means of any substitution we choose, the go in this new system will no longer be constants, but functions of space and time.

\begin{equation*}
\frac{d {u^\tau}}{d\ap} - \Gtmn u^{\mu} u^\nu =0 
\end{equation*}
 
According to the equivalence principle $K$ and $K'$ is however, indistinguisblae from $K$. All fall with the same velocity, thus the non vanisihig of the \gmn. If the $\Gamma_{\mu \nu}^{\tau}$ vanish, then the point moves uniformly in a straight line. These quantities therefore condition the deviation of the motion from uniformity. They are the components of the gravitational field.


\begin{equation}\label{eq:riemanntensorgamma}
R_{\mu \nu \sigma}^{\tau}(\Gamma)=\frac{\partial \Gamma_{\mu \nu}^{\tau}}{\partial x^{\sigma}}-\frac{\partial \Gamma_{\mu \sigma}^{\tau}}{\partial x^{\nu}}+\Gamma_{\alpha \nu}^{\tau} \Gamma_{\mu \sigma}^{\alpha}-\Gamma_{\alpha \sigma}^{\tau} \Gamma_{\mu\sigma}^{\alpha}\,, = 0
\end{equation}
%
In the general case how, so that without changing its components. The physical hypothesis on which general relativity is based, on the fundamental idea that valid also in the case in which does not vanish. This difference, curvature of space. The vector would not return with the same components.

This point is essential to Reichenbach's argument. It was precisely because his toy-geometrization was not envied by its more titled competitors that Reichenbach believed himself to be in an excellent position to \qt{attack the view that with a geometrical presentation of electricity, one would already gain something}{Ich wollte mich eben mit meiner Darstellung gegen die Auffassung wenden, als ob mit der geometrischen Darstellung der Elektrizität an sich schon etwas gewonnen wäre}[\letter{Reichenbach}{Einstein}{4}{4}{1926}][20-086][EA]. For such a geometrical interpretation of electromagnetism to become a physical theory as successful as general relativity, more was required than a mere geometrization. It called for something \emph{new}. Howeve,r in this cas eht egometrizajo did provide any sucecs. Of course if one could to reight field equatios, im particular than wuld ha bene trasitwryt.


%%Thus, Reichenbach argued that the geometrical interpretation of a physical field can only be successful if it leads to a \scare{change} in the equations and does not simply rewrite in geometrical terms the equations that are already known. One could object that Weyl, Eddington and Einstein's theories also changed the equations and did not simply rewrite them. However, Reichenbach seems to consider the derivability of solutions that correspond to the electron as a litmus test for a real change in the field equations. Maxwell's field equations are valid in free space and cannot explain why the separate, equally charged parts do not fly apart without introducing a non-electromagnetic cohesion force (the so-called Poincar\'{e} stress). On the other hand, Einstein's field equations, in their original form, do not entail any effect of gravitation on charge and cannot provide the cohesion force. It is only by changing the currently available field equations that it would become  possible to establish a connection between \gmn and $f\mn$, thereby assuring the equilibrium of the electron.
%
%To fully understand Reichenbach's stance on this issue, one must keep in mind that in a paper published in April \citep{Reichenbach1926} he expressed strong skepticism about the possibility of solving the problem of the \scare{grainy} structure of matter and and most all of the \qt{proper quantum-riddle}{eigentlichen Quantenrätsels}[][424][Reichenbach1926] in a field-theoretical/geometrical context. In Reichenbach's view, the \qt{casuality \origins{Zufälligkeit}}{Zufälligkeit}[][424][Reichenbach1926] of the \scare{quantum jumps} (the transition between orbital energy levels in Bohr's atom) suggests that the problem should be tackled from a different angle, by considering whether the very notion of causality in physics should be replaced by that of probability \citep[424]{Reichenbach1926}. After all, one should appreciate Reichenbach's clairvoyance, considering that Max Born's paper \citep{Born1926a} on the statistical interpretation of the wave function appeared only in June\footnote{For Reichenbach's take on the emerging quantum mechanics at this time, see the posthumously published manuscript, \cite{Reichenbach1926e}}.




%Weyl explained to Reichenbach in details the strategy he had laid down in Bad Nuhei,
%
%\q{Der Erfahrung wird durch die Annahme jener allgemeineren Metrik in keiner Weise vorgegriffen;  denn die Naturgesetze, an welche die Wirkungsausbreitung im Äther gebunden ist, können ja von solcher Art sein, daß sie keine Streckenkrümmung zulassen. Diese Möglichkeit liegt sogar nicht einmal ferne. Nehmen Sie als Wirkungsgröße das mittels der von mir in Nauheim benutzten "Normaleichung" $F=$ const. gemessene Volumen (Bezeichnungen nach Raum Zeit Materie, 3. oder 4. Aufl.!), so liefert das Wirkungsprinzip \textelp{} Wofür ich allein eintrete, ist dies: Die Integrabilität der Strekkenübertragung (wenn sie besteht, ich glaubs nicht, denn ich sehe nicht den geringsten zwingenden Grund dafür) liegt nicht im Wesen des metrischen Mediums, sondern kann nur auf einem besonderen Wirkungsgesetz beruhen. Wäre die historische Entwicklung anders verlaufen, So scheint mir, wäre niemand darauf verfallen, von vornherein gerade nur den Riemannschen Fall in Erwägung zu ziehen. Was die berüchtigte "Abhängigkeit von der Vorgeschichte" betrifft, so habe ich darüber wohl meine Ansicht deutlich genug in Nauheim ausgesprochen}\hide{\letterp{2}{2}{1921}}} 


% An der 4. Aufl. wird Sie wahrscheinlich vor allem meine veränderte Stellungnahme zum Problem der Materie interessieren; von der Universalität der Feldphysik bin ich gründlich zurückgekommen 




%Geuau, wie "Einstein zeigen nuõ§, da§ aus der Dyuauõik des starren Kšrpers heraus eiu solches Verhalten folgt, da§ der Ma§stab innner dieselbe LŠnge hat. gemessen iu seinem ds, so uuõ§ ielõ zeigen, da§ er iumõeõ' das gleiche dnrelõ R = eonst uoruõieõ't.e ds hat.. Wie etwa "Einstein so gut als ielõ das zu nõaelõen lõŠ.tten, habe ieh am Selõlu§ uõeiner Arbeit " Feld uud Materie". Ann. d. l'lõys.'2" augedeutet.

%Lieber Herr Doktor,156 warum sind Sie so gegen die Besprechung von Neuauflagen? In ^Wey1s 4. Aufi. liegen allerhand wesentliche €nderungen vor, die bei der Bedeutung dieses Buches eine erneute Besprechung gerecht- fertigt hŠtten. Nuu wage ich nõich gar nicht mit der Bitte hervor,

%\footnote{For Eddington \scare{natural geometry} is exactly Riemannian and \scare{world geometry} was nothing but a conceptual scheme; On the opposite  Weyl insisted on the  the real \scare{aether geometry} was non-Riemannian and the \scare{body geometry} distorted by the mechanism of adjustment} presented his own affine theory, embracing  and the same strategy, was embraced even if with a different turn. This is a necessary condition for what is called affine geometry. It appears to express the condition that; the world is " fiat in its smallest parts" or that it possesses a definite tangent. However, beside that indeed, what he introduced metricity tensor: $\mathbf{K}_{\mu \nu, \sigma}$, defined as 

 %What we have sought ill not the geometry of actual space and time. but the geometry of the world-structure. which is the common basis of space and time and things. Thus Eddington that the real geometry was exactly Riemannian, and was nothing more than a graphical representation, like phase space.  And in March Einstein suggested a sort an intermediate way, a conformal theory, in which \rac are not are not measured. Einstein dislike is a measuring rod geometry, and at the same time real measuring rods behave differently from the ideal one \todo{(du pasquier)}, there was no reason to assume that this was dif
 
%This fundamental imbalance, between in which however, the instrument that serve to measure lengths, differently, in which $k_\sigma$ is exactly Riemannian. Thus the \rac do not conform, that the real geometry of \st is non-Riemannian. A second approach, was pursued Arthur Stanley Eddington, that preferred to that only on the affine connection, without any metric. Is exactly Riemannian, in which $k_\sigma$, is exactly Riemannian, is only a graphical representation, comparable to the representation of with, to adaptation to the radius of curvature of the world, that enter into the dynamic equations determining the behavior of \rac. Introduce, a tensor split into two parts, and fiannly impose. Probably, unaware of both approaches Einstein presented in March, to avoid the \rac at all. Indeed, from one can define the Riemann, tensor contracted into the Ricci tensor, into two tensor, third option, that is a theory that completely renounced to the very existence of transportable measuring rods, on 17th March. In this way there was no concern about doubling the geometry, which seems to feel as unpleasent. 




%
%However, Einstein skepticsm grew rapidly, the epistemologicla lode, that the there was. no reason to conenct a phiosclao interpeation from the outset. Ideed, geometry and phsics were  Einstein had already become somewhat disillusioned with the affine approach, he nevertheless delivered a special lecture in Gothenburg on 10 July 1923 on Weyl's and Eddington's generalizations of general relativity, alongside his Nobel Prize lecture. At around that time he also agreed to write an appendix to the German translation of Eddington's "Mathematical Theory of Relativity" (Einstein 1925a [Doc. 282]).}, in which he had expressed. Meyerson's book on relativyt, by 1925 Einstein had moved on.


 %Weyl's reaction seems to manifest that in spite of the attacks his theory was receivn frm this diferent sides, his attitude believed he still had some cards to play. had his how Weyl had introduced \scare{doubling the geometry}. As it turned out there was two different ways to proceed, to different way to save the the new geometry, and in general to to a more general affine connection.  A few days later on February 17 Weyl published the English description on Nature, that is that real ether non-Riemannian geometry was to the measuring rod geometry, that appears in reality.
 
 %\qt{I'm of course very glad that you agree with my $\Gamma$-critique. I have now made a few reflections on the topic, which seem to me to prove that the Weylean thought, although good mathematically, does not bring about anything new physically. The geometrical interpretation of electricity is only a visualization, which in itself still does not say anything, and can also be realized in the original relativity theory. I have attached the note and would be grateful if you could give it a look \letter{Reichenbach}{Einstein}{24}{3}{26}][20-085][EA]}{Daß Sie meiner $\Gamma$-Kritik zustimmen, hat mich sehr gefreut \textelp{} Ich habe jetzt eine kleine Ueberlegung zu dieser Sache durchgeführt, die mir zu beweisen scheint, warum der Weylsche Gedanke, so gut er mathematisch ist, nichts Physikalisch Neues bringt. Die geometrische Deutung der Elektrizität ist doch nur eine Veranschaulichung, die selbst noch garnichts besagt, und die sich ebenso in der ursprünglichen Relativitätstheorie durchführen läßt. Ich lege Ihnen die Note ein und wäre Ihnen für Durchsicht sehr dankbar}

%, after making some comments about his academic situation,



%Reichenbach attached to this letter a typewritten note. As we shall see, far more was at stake in it than a critique of Weyl's theory (which was generally considered a dead horse at the time). Reichenbach intended to call into question the very idea that, since general relativity has \scare{geometrized} the gravitational field, the obvious next move should be to try to \scare{geometrize} the electromagnetic field. The importance of this formalism for \gr was to which was the key content of the theory. 