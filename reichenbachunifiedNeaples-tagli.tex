


\begin{comment}
\subsection{A Parting of The Ways. Positivists and Metaphysicians}
\label{positivistsmetaphysicians}

On \datef{30}{1}{1929}, Einstein's rumored new derivation of the \FP-field equations was published in the Proceedings of the Berlin Academy with the ambitious tile \citetitle{Einstein1929b} \citep{Einstein1929b}. Despite his anger toward Reichenbach's \scare{leaks}, Einstein did not hesitate to feed the hopes of the general public by popularizing his new theory in the daily press. On \datef{2}{2}{1929}, in its section \citet{Nature1929}, \emph{Nature} reported an interview of Einstein published in the \jt{Daily Chronicle}, on \datef{26}{1}{1929}, a day after the publication of Reichenbach's infamous article in the \VZ. Einstein's quarrel with Reichenbach had deeper philosophical roots that went way beyond questions of academic etiquette. A few days later, Einstein wrote a popular account of the new theory \citep{Einstein1929-02-03}. Its English translation was published on the first page of their Sunday supplement of the \jt{New York Times} on \datedm{3}{2}{1929} and in \jt{The Times} of London in two installments on \datedm{4}{2}{1929} and \dated{5}{2}{1929} \citepp{Einstein1929-2-3}{Einstein1929-2-4}[also published as][]{Einstein1930h}. Einstein insisted on the highly speculative nature of \uftp, pointed out that this \q{speculative method} was the same that lead to to success of \gr. Start from a suitable field-structure and search for the most simple equations that govern this structure. In trying to defend this epistemological stance, Einstein was not afraid endorsing even his somewhat outrageous comparison with Hegel.

%In the paper after providing an overview of his work on relativity, he described the method that guided him to his last field theory. 
%Einstein insisted on \qt{the degree of formal speculation, the slender empirical basis, the boldness in theoretical construction, and finally the fundamental reliance on the uniformity of the secrets of natural law and their accessibility to the speculative intellect}{spekulativ-formalistische Zug, die Schmalheit der Erfahrungsbasis[,] die Kühnheit der theoretischen Konstruktion, das ihr zugrunde liegende Vertrauen in die Einheitlichkeit und die Durchdringbarkeit der Geheimnisse der Naturgesetzlichkeit durch die spekulative Vernunft} \citep[114]{Einstein1930h}. This \q{speculative method}, Einstein claimed, was the same that lead to to success of \gr: \qt{Which are the simplest formal structures that can be attributed to a four-dimensional continuum, and which are the simplest laws that may be conceived to govern these structures?}{Welches sind die einfachsten und natürlichsten Bedingungen, welchen ein Kontinuum der skizzierten Art unterworfen werden kann? Die Beantwortung dieser Frage, welche ich in einer neuen Arbeit [7] versucht habe, liefert einheitliche Feldgesetze für Gravitation und Elektromagnetismus} \citep[115]{Einstein1930h}. In trying to defend this epistemological stance, Einstein was not afraid to side with \qt{Meyerson in his brilliant studies on the theory of knowledge}, who had emphasized the \scare{Hegelian} nature of such enterprise, \qt{without thereby implying the censure which a physicist would read into this}{geistreiche Erkenntnisstheoretiker Meyerson die geistige Einstellung der Relativitats-Theoretiker mit derjenigen Descartes und sogar Hegels verglichen ohne indes mit jenem Vergleich jenen Tadel zu verbinden, den das Ohr eines Physikers haturgemass heraushoren wird} \citep[115]{Einstein1930h}. 

The fact Einstein chose to mention Meyerson rather than Reichenbach as a philosophical reference in a popular presentation of his last theory for a major newspaper has of course a quiete symbolic significance. After a decade of personal friendship and intellectual exchange that had shaped the history of 20th-century philosophy of science and Einstein seems to have put into question the very core of his philosophical alliance with Reichenbach. Whereas Reichenbach considered the separation between mathematics and physics the great achiement of the theory, Einstein to consider mathematical simplicity the guide to the unification of the two fields. Einstein was fully aware of the marginality of his position, but, throughout 1929, continued express his confidence in \FP program \citep{Einstein1930c}.

%After 1919 Einstein benefited from a universal acclaim among the general public; however his positions among the physics community became progressively more isolated. Till 1925-1926 the \uftp was pursued by scholars of the stature by Weyl and Eddington, but was also regarded as a viable options by leading quantum theoreticians \citep[209]{Vizgin1994}. Even, after the 1925--1927 rapid advance in \qm, were made of relate unified theories to quantum theory \citep{Klein1926a}. However, most leading physicists  soon started to perceive the program as obsolete. 

%Einstein was fully aware of the marginality of his position, but, throughout 1929, continued express his confidence in \FP program. In the second paper of this year finished in \datem{19}{8}{1929}---the fourth in the series in the Berlin Academy---which reflects the priority dispute with Élie Cartan \citep{Debever1979}, Einstein returned to the Hamiltonian principle after objections raised by his collaborators Lanczos and Müntz \citep{Einstein1930c}.  In spite of the many doubts, Einstein was finally convinced that he had \q{found the simplest legitimate characterization of a Riemannian metric with distant parallelism that can occur in physics} \letterp{Einstein}{Cartan}{25}{8}{1929}[\D{V}][Debever1979].

However, like Reichenbach, fellow physicists were not impressed, in particular given the growing success of \qm-program. 

Weyl, who had always been scolded by Einstein for his speculative style of doing physics could relaunch the accusation in a paper \citep{Weyl1929c} in which he had uncovered the gauge symmetry of the Dirac theory of the electron \citepp{Dirac1928}{Dirac1928b}. \q{The hour of your revenge has come}, Pauli wrote to Weyl in August: \qt{Einstein has dropped the ball of distant parallelism, which is also pure mathematics and has nothing to do with physics and \emph{you} can scold him}{jetzt hat Einstein den Bock des Fernparallelismus geschossenf , der auch nur reine Mathematik ist und nichts mit Physik zu tun hat, und Sie konnen schimpfen} \letterpaulip{Pauli}{Weyl}{26}{8}{1929}[235].  \cop{Although Einstein's papers had been discussed widely especially among mathematicians, Einstein was aware of the poor reception that his work had especially among the colleagues that he probably felt has his peers} \citep{Goldstein2003}. As Pauli complained, writing to Einstein's close friend Paul \Ehr, \q{God seems to have left Einstein completely!} \letterpaulip{Pauli}{\Ehr}{29}{9}{1929}[237].

%An invitation to the 1930 Rouse Ball lecture at Cambridge gave Weyl the opportunity to review the Whole development of matter concepts which had taken place during the long decade just coming to an end.





%\footnoteh{Einstein agreed and gave a talk on the Problem of Space, Field, and Ether in Physics on December 11, 1929, Essentially the same talk was delivered to a large audience on the opening day of the Second World Power Conference which took place in Berlin from 16–25 June, 1930.  The text of this lecture was then published in the conference’s Transactions [Einstein 1930d]. A similar popular account of Space, Ether and the Field in Physics was published in Forum Philosophicum [Einstein 1930c] together with an English translation. Indeed, the text of the two penultimate paragraphs of this version and [Einstein 1930d] that characterize the distant parallelism are identical. A two-page abbreviated version of [Einstein 1930c] also mentions the distant parallelism approach [Einstein 1930e]}

Nevertheless, Einstein continued to defend the theory in public (in talks given in October and December) \citep{Einstein1930,Einstein1930a,Einstein1930b}, as well as in as well in private correspondence. However, Pauli did not hesitate to describe Einstein's presentation at the Berlin Colloquium as a \qt{terrible rubbish}{schrecklichen Quatsch} \letterpaulip{Pauli}{Jordan}{30}{11}{1929}[238]. When he received the drafts of Einstein's \jt{Annalen} paper, he wrote only slightly more politely \cop{that he no longer believed that the quantum theory might be an argument for the distant parallelism after Weyl's work on Dirac theory had shown that Dirac’s electron theory could be incorporated into a relativistic gravitation theory if the \vbein are introduced but the equations remain invariant if the \vbein at distant points are rotated in arbitrary manner}. Pauli also wrote that he did not find the derivation of the field equations convincing; they show \qt{no similarities with the usual facts confirmed by experience}{kaum eine Ahnlichkeit mit den gewohnlichen durch die Erfahrung gesicherten physikalischen Sachverhalten zu haben scheinen} \letterpaulip{Pauli}{Einstein}{19}{12}{1929}[239]. In particular, Pauli missed the validity of the classical tests of general relativity, perihelion motion and gravitational light bending: \qt{These results seem to be lost in your sweeping dismantling of the general theory of relativity. However, I hold on to this beautiful theory, even if it is betrayed by you!}{Die scheint doch bei Ihrem weitgehenden Abbau der allgemeinen Relativitatstheorie verloren zu gehen. Ich halte jedoch an dieser schonen Theorie fest, selbst wenn sie von Ihnen verraten wird!} \letterpaulip{Pauli}{Einstein}{19}{12}{1929}[239]. When Einstein expressed caution towards the definitive validity of his equations, he, \qt{so to speak, took the words right out of my mouth of criticism-loving physicists}{haben Sie den Kritik libenden Physikern sozusagen das Wort abgeschnitten} \letterpaulip{Pauli}{Einstein}{19}{12}{1929}[239]. Pauli knew that Einstein would not have changed his mind, but he was ready to \q{make any bet} that \q{after a year at the latest you will have given up all the distant parallelism, just as you had given up the affine theory before} \letterpaulip{Pauli}{Einstein}{19}{12}{1929}[239].

%\q{dann sagen Sie erst etwas dariiber, wenn mindestens ein Vierteljahr vergangen ist} 

Einstein complained that Pauli's remarks were superficial and asked him to return on the issue after some months \letterpaulip{Einstein}{Pauli}{19}{12}{1929}[140]. Although the \uftp was disavowed by its own initiators \citep{Weyl1931}, Einstein insisted in the pursuit of \FP discussing with Mayer two solutions of his last field equations \citep{Einstein1930g}\todo{field equations admitted at least one unphysical solution, namely, a static configuration of uncharged, gravitating bodies.}. However, Pauli would have clearly won the bet. Only a few months later Einstein and Walther Mayer presented a new approach \citep{Einstein1931} that, by generalizing the \nbein formalism to five dimensions, may have appeared more promising. This approach was ideally connected with that of Kaluza, but the shortcoming of that theory \qt{by sticking to the four-dimensional continuum, but with vectors with five components}{werden bei der im folgenden dargelegten Theorie dadurch vermieden, daß man zwar bei dem vierdimensionalen Kontinuum bleibt, aber in diesem Vektoren mit fünf Komponenten \textelp{} einführt} at each point of four-dimensional space-time \citep[377]{Einstein1931}. The optimism once again faded away quickly, since the theory was unable to solve the problem of matter. In a popular talk given in Vienna towards mid-\datemy{14}{10}{1931}, Einstein could only describe his field-theoretical work since \gr as a \qt{cemetery of buried hopes}{Friedhof von Begrabener Hoffnungen} \citep[441]{Einstein1932b}.

A few days later, Lanczos wrote to Einstein from the United States \letteraeap{Lanczos}{Einstein}{20}{10}{1931}[15-243] where he had just taken a position at Purdue University. Lanczos told Einstein that, at Arnold Berliner's suggestion, the influential editor of the \jt{Die Naturwissenschaften}, he had prepared a semi-popular presentation of \FP approach for the \jt{Ergebnisse der Exakten Wissenschaften}, a series sponsored by Berliner's journal \citep{Lanczos1931}. Lanczos had worked on the topic during his time as Einstein's assistant. The Lanczos/Einstein relation had become somehow strained \citep{Stachel1994}, and Lanczos was not fully convinced by Einstein's approach. However, he was confident to have found \q{a tone that should correspond to your conviction as well. I think that, deep down, we have something in common} \letteraeap{Lanczos}{Einstein}{20}{10}{1931}[15-243]. Lanczos presented \FP as a completion rather than an generalization of Riemannian geometry; nevertheless he also recognized the correctness of Reichenbach's approach \citep[118]{Lanczos1931}. What is more important, he opened the paper with some general considerations which give a glimpse in the philosophical atmosphere which pervaded the physics community. Lanczos distinguished between two \qt{spiritual attitudes}{geistige Haltung} towards relativity: 

\begin{enumerate}
\item\label{p} a \emph{positivist-subjectivist} insistence that physics has to do with observable quantities, and what cannot be observed is not part of physics. This \q{rigorous and therefore more intolerant form of positivism} \citep[104]{Lanczos1931}, defended in particular by quantum theoreticians, lead to the rejection of the \uft program as such.\ Since a field is nothing but a tool to describe the behavior of test particles, \rac and so, it is vain to search for solutions of the field equations that correspond to protons and electrons. In fact, the fields inside of elementary particles \q{could never in their details become the object of observation} \citep[104]{Lanczos1931}, since there are no test particles or measuring scales smaller than the electron itself. 

\item\label{m} A \emph{metaphysical-realistic} perspective, based on the conviction that physical reality exists independently of the possibility of measuring or observing it. If \sr seemed to be close to the positivistic/operationalistic ideals, with \Mink the theory underwent a \qt{\scare{metaphysical} turn}{metaphysische Wandlung} in favor of a \qt{logical-constructive understanding \origins{Verstehen}}{zugunsten eines logisch-konstruktiven Verstehen} \citep[103]{Lanczos1931}. \Gr had finally brought \qt{the logical-deductive exploration into the depths of nature, under the presupposition of its universality and understandability, and with faith in the laws of mathematics}{das logisch-deduktive Eindringen in die Natur, unter Voraussetzung ihrer Universalitiit und Verstehbarkeit, und im Verlrauen au/ das mathematische Gesetz} \citep[102]{Lanczos1931}. 
\end{enumerate} 
%
The positivist described by Lanczos could be easily identified with Pauli, who had indeed raised similar objections against Weyl's theory early on \citepp{Pauli1919}[see][13]{Hendry1984}. However, Pauli,  by reviewing Lanczos article, did not fully recognize himself in the portrait of the \scare{positivist} \citep{Pauli1932-3-11}. Such labels, he argued, \q{are highly subjective and arbitrary}; it is obvious that in order to gain new scientific insights one does not only requires inductive generalizations, but also logical-constructive imagination. Pauli mocked the \emph{Naturwissenschaften} for having published the paper in series entitled \scare{Results in exact sciences} (\german{Ergebnisse der Exakten Wissenschaften}). Indeed, Einstein published this sort of theories at rhythm of one each year and in every case he claims that it is the definitive solution: \qt{Einstein's new field theory is dead, long live Einstein's new theory!}{Die neue Feldtheorie ~EINSTEINS ist tot. ES Iebe die neue Feldtheorie EINSTEINS !}. 


%However, it was undeniable that many supporters of \qm had used a a positivistic rhetorics, and it was against this rhetorics that Einstein, somehow tongue in cheek, was not ashamed to define himself as a metaphysician \citep{Einstein1932a}. It was this attitude that caught many of Einstein's philosophical allies by surprise.\todo{improve}


%Lanczos was meant probably the Pauli-Einstein debate, a more general the reaction of Einstein's philosophical\todo{improve}. 

However, if many readers might have easily recognized someone like Pauli in Lanczos's \scare{positivist}, other were baffled to find out Einstein located among the \scare{metaphysicians}. At the beginning of 1932 the introduction of Lanczos's 1931 paper was published at Berliner's suggestion as a \latin{seperatum} in the \citejournal{Lanczos1932} \q{to make it available to a larger public} \citep[113\fn{1}]{Lanczos1932}. It is probably this article of Lanczos that Frank read with some bewilderment, as he reports in his Einstein's biography \citep{Frank1947}. Frank was \q{quite astonished} to find the theory of relativity characterized as the expression of a realist program \q{since I had been accustomed to regarding it as a realization of \Mach's program} \citep[215]{Frank1947}. However, when Frank met Einstein in Berlin at around the same time, he found out that Lanczos had indeed well characterized Einstein's point of view \citep[215f.]{Frank1947}. According to his recollection, Einstein complained that \q{\textins{a} new fashion} had arisen in physics according to which quantities that in principle cannot be measured do not exist, and that to \q{to speak about them is pure metaphysics} \citep[216]{Frank1947}. Frank objected that this was the very same philosophical attitude that led to relativity theory. By contrast, Einstein insisted, the essential point of relativity theory is to \q{regard an electromagnetic or gravitational field as a physical reality, in the same sense that matter had formerly been considered so}  \citep[216]{Frank1947}. The theory of relativity teaches us the connection between different descriptions of one and the same reality. Was not a theory about the behavior of \rac, but a unification of two fields.


\q{In this connection we are likely to think of the development of the theory of relativity into a world geometry, yet it would be quite erroneous to interpret this development as signifying a fusion of physics and mathematics. The general theory of relativity by no means turns physics into mathematics. Quite the opposite: it brings about the recognition of a physical problem of geometry. \citep Although both mathematics and physics are sciences, the difference between them is fundamental, and we must put it down as quite impossible that it will ever disappear. For mathematics is not a natural science at all.}


%http://philsci-archive.pitt.edu/18196/1/Coming%20to%20America-%20Carnap%2C%20Reichenbach%2C%20and%20the%20Great%20Intellectual%20Migration.%20Part%20II%2C%20Hans%20Reichenbach.pdf

%
\end{comment}






%\footnote{Einstein used a similar wording by commenting on the manuscript of Cassirer's \scare{Kantian} booklet on relativity. \q{Conceptual systems appear empty to me, if the manner in which they are to be referred to experience is not established} \lettercpaep{\Einstein}{\Cassirer}{6}{6}{1920}[10][44]. In particular, \q{[w]ith the interpretation of the $ds$ as a result of measurement, which is obtainable by means of measuring rods and clocks the general theory of relativity as a physical theory stands or falls} \lettercpaep{\Einstein}{\Cassirer}{6}{6}{1920}[10][44]. The gravitational redshift, can be taken as an empirical confirmation of general relativity only because different atoms of the same substance can be regarded as identically constructed clocks reproducing the identical unit of time. For this reason it is possible to \scare{normalize} the absolute value of $ds$ by counting the wave crests on atom. According, Weyl's theory deprived the $ds$ of any physical meaning. However, real \rac behave differently than predicted by Weyl theory forcing Weyl to assume an inconsistent position. According to Einstein, line \gr, Weyl's  \q{theory is based on a measuring rods geometry}, that is it presupposes the comparability of lengths. However, it entains only \q{thought measuring rods \origins{nur gedachte Massstäbe}} that behave differently from the real ones. \q{This is repugnant} \lettercpaep{\Einstein}{\Besso}{26}{8}{1920}[10][85\me]}. 


%The subsequent months will be of fundamental importance in Reichenbach's biography. A correspondence with Schlick in October 1920 seems to have convinced Reichenbach that us axioms of coordination were simply \scare{convetions}. Reichenbach seemed to have been concerned that this solution would have made deprived on any physical content. \cop{Reichenbach explains that the principle of simplicity – used by Poincaré and Schlick to choose the \scare{right} geometry – does not seem “univoque” to him. As a consequence, as far as the principle of simplicity is not involved, the “coordination principles” cannot be conventions, because there must be a “synthetic” criterion to decide which}. ust after the passage Reichenbach adds : “the alternative between the two conventions appears as synthetic knowledge”. It is possible that Einstein's endorsme moght have convied. That to resctrit the coordiantes, ultimately found tin the $G+P$ formula. However, that one has to seprate mathematics from physics will as constantn of Reichenbach's philosopjy. It is this point that not only his critisim of Weyl but also of the \uftp attempts\todp{.


%\cop{that the arbitrariness of the principles is limited as well as princ

%\cop{The nature of physical space cannot be determined by such definitions. It is a thing of the real world, not an object of the logical world of mathematics. In such a system is a statement regarding the significance of physics, the assertion that the system of equations is true for reality. This relation is totally different from the internal simplicity of mathematics}. 

%\cop{The \emph{axioms of connection} are the empirical laws of physics, the fundamental equations of a theory. The \emph{axioms of coordination} determine the rules of the application of the axioms of connection to reality, that is, they determine the rules of the connection.

%Reichenbach will soon abandoned his Kantianism under the influence of Schlick. 

%\cop{Das MS wurde dann abgetippt. Ich zeigte es Einstein u. Lewin. Durch Berliners Vermittl. kam es Zu Springer. Erschienen ist es im Sept. 1920, zum Naturforschertag}.
iples are combined}

%Indeed, to correspond of doubling the geometries, that Reichenbach, like Einstein, found questionable. The real achievement of Weyl's geometry, is mathematical, it was showed that something that was consider as \apr, was punt in question. 

%Mathematics teaches all this quite inde- pendent of the facts; this is preci~ely its function as a science of possibilit %The achievements of mathematics in this respect go in two directions. On the one hand, it shows physics to what extent the observed case is only a special case; it teaches the physicist how to work out more general possibilities an
 


%erfolgt, ist im wesentlichen nur ein andrer sprachlicher Ausdruck für den vorliegenden Tatbestand, keine Zurückführung auf in allgemeineres Gesetz. Insbesondere hat diese »Einstellung« nichts zu tun mit seiner kongruenten Verpflanzung, so dass diese physikalisch leer bleibt. An anderer Stelle (70, S. 133) findet sich eine


% Weyl's explanation according to which the unequivocal transferability takes place by adjusting the standards to the radius of curvature of the world, \q{is essentially just another linguistic expression for the facts at hand, not a reduction to a more general law}. 
 





%As we have seen, the gravitational redshift, just like the transverse Doppler effect in special relativity, can be taken as an empirical confirmation of general relativity only because different atoms of the same substance can be regarded as identically constructed clocks reproducing the identical unit of time. 


%\cop{Ich freue mich wirklich sehr darüber, dass Sie mir Ihre ausgezeichnete Broschüre widmen wollen, noch mehr aber darüber, dass Sie mir als Dozent und Grübler ein so gutes Zeugnis ausstellen. Der Wert der Rel. Th. für die Philosophie scheint mir der zu sein, das sie die Zweifelhaftigkeit gewisser Begriffe dargethan hat, die auch in der Philosophie als Scheidemünze anerkannt waren. Begriffe sind eben leer, wenn sie aufhören, mit Erlebnissen fest verkettet zu sein. Sie gleichen Emporkömmlingen, die sich ihrer Abstammung schämen und sie verleugnen wollen}. Einstein made a similar claims by writing to Cassirer in the very same days.



%Reichenbach attended Einstein's Ieerures in Berlin (see Doc. 57, note 2).

%The thesis was published after 15 June 1920 as Reichenbach 1920 (see Doc. 57). Stuttgart, Wiederholdstr. 13. d. 15. Juni 1920. 
%
%
%Widmung .That most of the discussion of the book, relativized constitutive \apr, however, one first by a philosophers of Weyl theory, and in particular was that is itself superior. 

%\footnote{In the discussions at the Bad Nauheim meeting, Einstein maintained his characteristically ambiguous attitude towards the role of rods and clocks in relativity theory. Commenting on \Weyl's talk, he pointed out once again that the \q{arrangement of \textins{his} conceptual system,} \q{it has become decisive \origins{massgebend} to bring elementary experiences into the language of signs \origins{Zeichensprache}} \citep[650]{Einstein1920c}. For Einstein, \q{temporal-spatial intervals are physically defined with the help of measuring rods and clocks}, under the assumption that \q{their equality is empirically independent of their prehistory} \citep[650]{Einstein1920c}. Einstein insisted that precisely upon this assumption rests \q{the possibility of coordinating \origins{zuzuordnen} a number $ds$ to two neighboring world points}; if this were impossible, general relativity would be robbed of \q{its most solid empirical support and possibilities of confirmation} \citep[650]{Einstein1920c}.
%
%However, Einstein showed a more flexible attitude replying to Pauli's remarks during the same discussion. \Pauli reiterated his objection based on his \scare{observability} criterion. Just as the field strength in the interior of the electron is meaningless because there is no smaller test particle than the electron, \q{one could claim something similar concerning spatial measurements, \myemph{since there are no infinitely small measuring-rods}} \citep[650]{Einstein1920c}. Einstein replied to \Pauli that \q{with the increasing refinement of the system of scientific concepts, the manner and procedure of associating the concepts with experiences becomes increasingly more complicated} \citep[650]{Einstein1920c}. In particular, he recognized that in cases such as that of the continuum theories, \q{one finds that a definite experience cannot be associated any longer with a concept} \citep[650]{Einstein1920c}. According to Einstein, there is an alternative: one can abandon \scare{continuum theories} for the sake of \Pauli's observability criterion, or replace such a \q{system of associating concepts \textins{with experiences} with a more complicated one} \citep[650]{Einstein1920c}.  Einstein's in his contributions to the the discussion which followed Max von \citets{Laue1920}'s Bad Nauheim paper. Einstein, however, in the very same sentence, did not hesitate to admit that \q{[it] is a logical shortcoming of the theory of relativity in its present form to be forced to introduce measuring rods and clocks \myemph{separately instead of being able to construct them as solutions to differential equations}} \citep[Einstein's reply to][662\me]{Laue1920}}. 




%Without an analogon of equivalence principle, the choice of the affine connection of as fundamental variable was not even physically motivated nor it had any physical meaning. However, if using the $\Gamma\tmn$ leads to the right set of field equations, then the use of $\Gamma\tmn$ as a fundamental variable is justified that lead to new results this choice can be justified \emph{post facto}. Only geometry and physics together could be compared with experience.

%\q{Eine geometrische Interpretation des elektromagnetischen Feldes scheint überhaupt zunächst nicht sehr aussichtsvoll}. 


%Ein Analogon zum Aquivalenzprinzip bei der Gravitation besteht hier nicht. (Deswegen haben auch die geodatischen Linien in Ihrer und Weyls Theorie keine unmittelbare physikalische Bedeutung.)

%ede physikalische Theorie, die eine sinngemäße Antwort auf diese Frage zu geben beansprucht, mit einer Definition der in hr verwendeten Feldgrößen beginnen muß, die angibt, wie diese Größen gemessen werden können. Sie muß ferner Beziehungen zwischen elektromagnetischen und auf andere Weise gemessenen Größen aufdekken. (Die schönste Leistung der Relativitätstheorie war ja, die Meßergebnisse von

%In particular that \Gtmn could not be measured, 


%Eine geometrische Interpretation des elektromagnetischen Feldes scheint überhaupt zunächst nicht sehr aussichtsvoll: in Analogon zum Aquivalenzprinzip bei der Gravitation besteht hier nicht. (Deswegen haben auch die geodätischen Linien in Ihrer und Weyls Theorie keine unmittelbare physikalische Bedeutung.)

%vermißt.) Diese Theorien genügen nicht den angegebenen Postulaten. Die Größen va r # können nicht direkt gemessen, sondern müssen aus den direkt gemessenen Größen erst durch komplizierte Rechenoperationen gewonnen werden. Niemand kann empirisch einen affine Zusammenhang zwischen Vektoren in benachbarten Punkten feststellen, wen er nicht vorher berets das Linienelement ermittelt hat. Deswegen halte ich im Gegensatz zu Ihnen und Einstein die Erfindung der Mathematiker, daß man auch ohne Linienelement auf einen affinen Zusammenhang eine Geometrie gründen könne, zunächst für die Physik bedeutungslos. Wenn die direkt aus den Messungen entnommenen Größen (g it Fix) in der Theorie als abgeleitete Größen erscheinen und umgekehrt die in der Theorie als Gundbegrif-


%However, if using the $\Gamma\tmn$ leads to the right set of field equations, then the use of $\Gamma\tmn$ as a fundamental variable is justified that lead to new results this choice can be justified \emph{post facto}

%The beuty of \rt requirement that an abstract concept, like the $\Gamma\tmn$ and the to deduce the \Fmn from it that are observable. Ultimately, it should only be permissible in physics when it can be established whether it applies in concrete cases of observation. This requirement does not seem far from the view that Einstein often defended in the past. However, Einstein realized that this requirement was too severe. In several writings of those years Einstein had repeatedly insisted geometry cannot be tested separately from physics \citep{Einstein1923}. Possible experiences, he claimed, must correspond not to an individual concept but to the system as a whole (\cite[1692]{Einstein1924}; \cf\cite{Giovanelli2014a}). Without an analogon of equivalence principle, the choice of the affine connection of as fundamental variable was not even physically motivated nor it had any physical meaning. However, if using the $\Gamma\tmn$ leads to the right set of field equations, then the use of $\Gamma\tmn$ as a fundamental variable is justified that lead to new results this choice can be justified \emph{post facto}. Only geometry and physics together could be compared with experience.


%\qit{The mathematics is enormously difficult}{he wrote to Besso}{the link with what can be experienced is unfortunately becoming increasingly indirect}{Das Mathematische ist enorm schwierig, der Zusammenhang mit dem Erfahrbaren wird leider immer indirekter} \lettercpaep{Einstein}{Besso}{5}{1}{1924}[14][190]. However, Einstein was still convinced that a field theory that might offer the solution to the quantum problem \citep{Einstein1923f} was at least \qt{a logical \emph{possibility}, to do justice to reality without \emph{sacrificium intellectus}}{\latin{sacrificium intellectus} der Wirklichkeit gerecht zu werden} 	\lettercpaep{Einstein}{Besso}{5}{1}{1924}[14][190], that is, without retreating to a positivist-phenomenalist agnosticism.  Einstein distanced himself from the logical positivists' insistence on the need to coordinate every fundamental concept of a theory to a \q{piece \originsg{Ding} of reality} \citeptra[5]{Reichenbach1924}[8]{Reichenbach1969}, although they did not seem to have taken notice. However, the more rationalistic attitude of neo-Kantian philosophers like Ernst \citet{Cassirer1921} and his followers \citepp{Winternitz1923}{Elsbach1924} did not seem to offer a valid alternative \citepp{Einstein1924a}{Einstein1924}. According to Einstein, science extensively uses non-empirical \scare{ideal} conceptual constructions (say the \gmn, the $\Gamma\tmn$, etc.) \citep[1691]{Einstein1924}. But non wasy there were \apr by their success. 

%When the second edition of this book was translated into German, Einstein an "Appendix," which recapitulated his method of arriving at Eddington's theory, but ended with an even more pessimistic estimate of its physical significance than in his earlier papers. As he put it in a letter to Besso:41 "T am firmly con- Einstein still found Meyerson's account \qt{unfair}{ungerecht} \qt{as the escapades by Weyl and \Eddington are considered to be essential parts of the theory of relativity}{als die Eskapaden von Weyl und \Eddington zum Wesen der Relat.~Theorie gerechnet} (\CPAE{14}{455, 6; \datedm{12}{3}{1925}}). However, he will soon embrance ... The interest soon fadered. \qt{Night, sweating properly \textelp{} the conviction of the impossibility of the field theory in the current sense becomes stronger}{Nachts geh\"{o}rig schwitzen \"uberzeugung von Unm\'ogl. von Feldtheorie im bisherigen Sinne verst\"arkt sich} (\CPAE{14}{455, 9; \datedm{17}{3}{1925}}). 




%The theory, must just in the case of Einstein \hide{Genau, wie Einstein zeigen muß, daß aus der Dynamik des starren Körpers heraus ein solches Verhalten folgt, daß der Maßstab iumıer dieselbe Länge hat, gemessen iu seinenı, Is, so muß ich zeigen. daß er inmıer das gleiche durch $R = const$. normierte $ds$ lıat. Wir etwa Einstein so gut als ich das zu maclıen hätten, habe ich am Schluß meiner Arbeit " Feld und Materie”. Ann. d. Plıysik angedeutet.} . However, Reichenbach would have not been impressed by this argument. Indeed, Weyl seems to believed purely infinitesimal, was to chose a more special one. But Weyl geometry was not the most general one, that choice of such geometry was completely and would have  only at the end when would be able to integrate the equations and obtained \rac as solutions.




%Two critique, one that epistemological, better than other geometries (there is no reaso, eliminat the asumm), that thus that one should give ... however the theory, formalism, that iintrisc superiority of Weyl geometry is not even true, an there are 27 different connections among which one can chose.  A second aspect is that Weyl requires an, there is than as at least a good overview of some differential geometry, not Weyl but also Eddington and even Schouten. is even more surprising the work of Schouten 1921 classifies 18 possible linear affine connections (\scare{Übertragungen}) numbered as I,..., VI a?c. In Schouten 1922 he consider by he improved he classification trhee tensor 27 possible connection. All the further less general cases of linear connections are obtained by introducing restrictions to such quantities (for more details see Vizgin 1994, 184, Goenner 2004). This classification was Schouten point of view not only Schouten that Weyl was not but one could think in which. The third tensor which never plaed the role, the pwer can be. In which the tensor of asymmetry vanish symmetric, but the tensor of non metricity does not vanish. That more general even non-symmetrical connections (IVc). In which the tensor of non metricity vanish, but the tensor of asymetry does not, which will be the geometry used by Reichenbach  That even Schouten most general linera, lieanr But ieven the condition that a connection is linear was not necessary. Start from a general affine connection, and then restrict the possibilities. However, there was no particular. 

%1922 French article, that was empirical fact that \rac behave in a Riemannian way. Thus this resctriv that to Riemannian geometry. However, which Riemannian geomtry remains a question of choince. Idneed, one can imm  \scare{Darrigol classes}. However, was on a fact, that in nature there are. The choice between geometry is conventional; indeed there might be didferent. By the there are no differential forces. 











%lists 18 different linear connections and classifies them invariantly. The most general connection is characterized by two fields of third degree, one tensor field of second degree, and a vector field. These fields are the symmetry tensor $S_{\lambda \mu}^{v}$ the tensor of $Q_{\lambda \mu}^{v}$. One,

%. and a vector C „_ which follows from Cfμ = C μ ôff while C ,f μ = I`§μ+l`í';1, if l" stands for the connection for tangent vectors and 1" ' for the con- nection for linear forms. Torsion is defined by Sfμ = 1/2(l`§μ - FZÄ), non-uıflicity by Vμg“ = Qt“. Furthermore, on page 57 we find: “The general connection for n = 4 at least theoretically opens the door for an extension of Weyl's theory. For such an extension an invariant affixation of the connection is needed, because a physical phenomenon can correspond only to an invariant expression."



%Moreover, expressed his, howeveer, that privildege respect to Eddington's theory. Weyl probably, Das alte MS wurde vš\"ollig umgestossen. 


%[Berlin,] Mittwoch [20 September 1922] %The second si that even if we can define, that this particular geometrical structure. There was not start which is undefined, that the graphical presentation was precisely the reason for rejecting the theory. Again Reichenbach's critique would follow Pauli's ideas. In the meanintime Reichenbach had finished, which will however, for most of Weyl's, which interrupted 
%
%%From Hans Reichenbach Stuttgart, 19 April 1923 Asks for help in finding a publisher for Reichenbach 1924. ALS. [20 079]. 50. Ilse Einstein to Hans Reichenbach Berlin, 12 May 1923 Informs him that AE has not seen Reichenbach’s letter since he already left for the Netherlands. She no longer forwards AE’s mail since he brings it all back unopened. She assures Reichenbach that she will give his letter to AE upon his return. AKS. [87 944]. 122. From Hans Reichenbach Stuttgart, 10 July 1923 Thanks AE for Abs. 89. Is sorry that the Academy did not agree to print his manuscript. Asks whether this was due to financial or other reasons. Springer cannot accept his suggestion that the \textit{Notgemeinschaft} cover part of the printing cost. Verlag Witwer, on the other hand, has agreed to publish the work with support from the \textit{Notgemeinschaft}. Enclosed sends the request to the \textit{Notgemeinschaft} and asks to forward it to Fritz Haber and to put in a good word for him personally. Asks to mail back the manuscript. ALS. [20 082]. During 1924 Reichenbach finally managed to published, his Axiomatic in which. the negative review finally convinced him of the debacle, Weyl. In 1925 with possibly.


%Begriff der Parallelverschiebung entstammt wie alle Begriffe der euklidischen Geometrie der Betrachtung der Lagerungs-Gesetze bezw. der Gesetze der relativen Verschiebung starrer Körper. Daher erhält die (Behauptung) Festsetzung ihre Evi-

%kann ein Gesetz (2) des affinen Zusammenhanges one physikalische Interpretation mittelst des starren Körpers einführen. Aber es ist dann ziemlich willkürlich, von diesem (affinen) Gesetz zu fordern, dass es das Verhältnis der Beträge zweier Vektoren bei der Verschiebung ungeändert lasse (wenn man die Interpretation von (2)).18]

%\cop{In Weyl's theory, a gauge-system is partly physical and partly conventional; lengths in different directions but at the same point are supposed to be compared by experimental (optical) methods; but lengths at different points are not supposed to be comparable by physical methods (transfer of clocks and rods) and the unit of length at each point is laid down by a convention. I think that this hybrid definition of length is undesirable,}. 


%In the original presentation of \rt, Einstein started the metric $\gmn$ a coordinate indent criterion of the equality distance $ds$ of two nearby points  with coordinates $x_\nu$ and $x_\nu+dx_\nu$, the famous formula for the line element $ds^2=$. In the lectures, Einstein showed how one treat $dx_\nu$ as a special case of contravariant vector $A^\tau$. The so called affine connection $$dA^\tau = \Gtmn A^\mu dx_\nu$$ provides a coordinate-independent criterion for the parallelism of two vectors at neighboring points $x_\nu$ and $x+dx_\nu$ are equal and parallel. It can be show, that the notion of \scare{sameness} cannot be in general be extended to distant points by continuous transfer along curve. If one parallel displaces along different paths, one gets, in general, a different vector at a distant point \citep[028-01-03, 37]{HR}. In this way one could recover the notion of \scare{curvature} without any reference to the metric. The metric could be introduced at later stage by associating with any contravariant vector $A^\tau$ with a length $l^2=\gmn A^\mu A^\nu$. It was natural to assume that the length of vectors does not change under parallel transport. By imposing this condition Levi-Civita was able to recover the content Riemannian geometry. The line element $ds$ is nothing but the length $l$ of the contra-variant vector $dx^\nu$. 

Although not mention of this point is made in the notes, from discussions with Einstein, Reichenbach might have become immediately aware that \citep{Weyl1918a,Weyl1919a} was bothered by asymmetry comparison of direction of vectors which is path-dependent could not be the comparison their lengths was distant-geometrical. To overcome this \scare{mathematical injustice}, Weyl a introduced \scare{metric connection} alongside the \scare{affine connection}. If a vector of length $l$ is displaced from $x_\nu$ to $x_\nu+dx_\nu$, it will in general have a new length $l+dl$, so that $dl/l=\phin dx_\nu$. In this way, in addition to the \scare{metric tensor} \gmn, a \scare{metric vector} $\phin$ is introduced:


The \gmn are identified with the potentials of the gravitational field because of a \emph{physical fact} the equivalence principles. Weyl found natural to intepret \phin as the four-potential of electromagnetic field  because of the \emph{mathematical fact} that the tensor $F\mn$ is the curl of the \phin, and in turn constraint equivalent to Maxwell equations. Via the action principles from \gmn and \phin Weyl hoped was able to recover Maxwell and a set gravitational field equations. Weyl was initially confident  the stable solutions of the equations corresponding to elementary particles, allowing the ultimate comparison with experience.



Weyl could then conclude that just like general relativity represented a geometrization of gravitational phenomena, Weyl's theory represented a unified geometrization of both gravitational and electromagnetic phenomena, which were, at that time, the only kind known. Ultimately, matter itself would have become epiphenomenon of the \scare{world metrics}. With some rhetorical exaggeration, Weyl did not hesitate to declare \q{Der Traum des Descartes von einer rein geometrischen Physik} had be fulfilled. Concluding the 1919 edition of the book, Weyl could declare that \q{physics and geometry coincide with each other}. The tendency of physicalizing geometry that have prevailed leading protagonists of the 19th century from Gauss to Helmholtz seemed to superseded have of geometrizing physics that run from Riemann to Einstein: \q{geometry has not been physics but physics has become geometry} \citep[263]{Weyl1919}. 


%The functions $g_{\mu \nu}$ are to be determined in such a way that with the chosen coordinates $d s^{2}=\pm 1$ or 0, respectively, for all mesh points \citeptra[5]{Reichenbach1924}[8]{Reichenbach1969}. If it is impossible to find a coordinate systems. The we have two alternatives the rods are deformed by force that affect all bodies in the sawy way; the gometry of space is non-Euclidean. Of course.



%Reichenbach, as others, probably found the entire procedure questionable. From later by Reichenbach, was again not dissimilar to that of Pauli had confessed in along to Pauli. A good theory should  should first how the field quantities could be measured, an the derived other field quantities. On the contrary, he starts from the \Gtmn that non physical meaning and cannot be measured if not inprinciples, could only  and the measurable derfivanle from the \Gtmn. 




%Without an analogon of equivalence principle, the choice of the affine connection of as fundamental variable was not even physically motivated nor it had any physical meaning. They serve only to construct a suitable Lagrangian, from which the field equations could be derived. The correctness of the theory hat is to observe would lead to solutions corresponding to known elementary particles and their behavior.  

%Einstein, why he was skeptical of Weyl's approach. Weyl's theory was semi-metrical. It assumed the comparability of lengths, although only at neighboriung. Howevr, to claim that measuring instrumetns that serve measure behave differently that the geometry would predict. This inconsistency was for Einstein unacceptable. Einstein considered preferable to adopt Eddington purely affine approach in which the \Gtmn iwhtout introduing any relation between the affine connection and the metric. From the affine connection alone one can derive an in general non-symmetric Ricci tensor, the symmetric of which would be identified with metric tensor up to a constant.  


%This allowed him to construt Ricci tensor Wir führen aber von Anfang an gar keine Voraussetzung ein über einen Zusammenhang zwischen affinem Zusammenhang und Metrik. Dem entspricht es, In th sammenhang zwischen affinem Zusammenhang und Metrik. Dem entspricht es, dass die guv und die Täp unabhängig voneinander variert werden dürfen.(14)


%From the affine connection alone one can derive an in general non-symmetric Ricci tensor, the symmetric of which would be identified with metric tensor up to a constant. The metric would have come up as consequence of the theory. Eddingto does deny that natural geoemtry is Riemannian. But he wants to derive as a consquence of the they. \q{wird nötig sein, an dem zentral-symmetrischen Falle ihre Vereinbarkeit mit der Erfahrung: zu prüfen."12)Ich glaube aber nicht, dass man auf diesem Wege zu ei-}




%One could compllty that only, or Man kann nun zwar aus den Elementen der Theorie gewiss die Begriffe eliminieren, welche den starren Körper und die Uhr betreffen. Man kann auch davon ausgehen, dass nur der Gleichung ds2 = 0 reale Bedeutung zukomme. Man kann ein Gesetz (2) des affinen Zusammenhanges ohne physikalische Interpretation mittelst des starren Körpers einführen. However, the semi metrical approach was incoerent. 


%Man kann nun zwar aus den Elementen der Theorie gewiss die Begriffe eliminieren, welche den starren Körper und die Uhr betreffen. Man kann auch davon ausgehen, dass nur der Gleichung ds2 = 0 reale Bedeutung zukomme. Man kann ein Gesetz (2) des affinen Zusammenhanges one physikalische Interpretation mittelst des starren Körpers einführen. Aber es ist dann ziemlich willkürlich, von diesem (affinen) Gesetz zu fordern, dass es das Verhältnis der Beträge zweier Vektoren bei der Verschiebung ungeändert lasse (wenn man die Interpretation von (2)). [8]





%The metric would have come up as consequence of the theory. From the affine connection alone one can derive an in general non-symmetric Ricci tensor, the symmetric of which would be identified with metric tensor up to a constant. 

%believe that I have finally understood the connection between electricity and gravitation. Eddington has come closer to the truth than Weyl.

%spirit. Eddington evidently gave Einstein a copy of his 1921 paper during the atter's visit to England in that year.29 Einstein's initial response was similar to his view of Weyl's theory. In a letter to Weyl, he called it "beautiful but physically meaningless.»30

% the forgo the very idea of comparability of lengths. Eddington a very general affine connection in which the length of vectors is not defined. That to start from the a general affine connection \Gtmn. Any physical interpretation. The purely affine theory discussed So far involves no metric, and thus provide no basis for a comparison of lengths at different points, even neighboring ones.



%From this one Differently the tensor is not further if is symmetric. From one can obtained the Riemann and Ricci tensor, that can be split it into two parts that, could be  The latter can be split into parts den Komponenten dr. des Linienelementes die Invariante $R dx.dx$, liefert.. that behave as elementary particles. 

 

%is semi-metrical and indeed, it was in general more effective to go in where the affine connection has not physical meaning at all 1923. Weyl's theory assume that there transportable lenghts, but simply denies that is path independent. Howevr, tha such lengths are measured with \rac. Weyl ulitmately must however, that do not behavte. Eddington's theory since the theory, statrs from a generla affine connection of the form



%The choice of   \cop{The procedure is, in any case, arbitrary, because we have taken the I's to have l physical meaning, and then we take the simple expression as a tensor to deduce the laws of gravitation and electromagnetism by variation. Nevertheless, we avoid Weyl's weak point. Up to now, the calculations I have made with respect to gravitational and electromagnetic fields have given results already known. As to the structure of electrons, the calculations I are so complicated that I have not been able to obtain anything definitive up to the present.} Ultimately, sicne ultimatley onlyt combination geometry and physics and testable, that is of gemetry and the field equations. As Einstein's eplianed in Madrid \cop{The procedure is, in any case, arbitrary, because we have taken the I's to have l physical meaning, and then we take the simple expression as a tensor to deduce the laws of gravitation and electromagnetism by variation. Nevertheless, we avoid Weyl's weak point. Up to now, the calculations I have made with respect to gravitational and electromagnetic fields have given results already known. As to the structure of electrons, the calculations I are so complicated that I have not been able to obtain anything definitive up to the present.}, Einstein delivered his lectures in French. 
%\q{an dem zentral-symmetrischen Falle ihre Vereinbarkeit mit der Erfahrung zu prüfen}, t




%
%
%
%It is against this background that Einstein, during a trip to South America in 1925 became interested in in the rationalistic reading of relativity suggested by \citetp{Meyerson1925} \CPAE{14}{455, 6; \datedm{12}{3}{1925}} of positivism and kantianism.  During that trip however, Einstein starting to nuruirng doupts about the entrity of this appraoch (\CPAE{14}{455, 9; \datedm{17}{3}{1925}}). These doubts became certainties when Einstein returned to Europe. \qit{On \datedm{1}{6}{1925}, I got back from South America}{Einstein wrote to Besso}{I am firmly convinced that the whole line of thought Weyl-Eddington-Schouten%
%%
%\footnote{\citet{Schouten1924} claimed that it was possible to overcome a shortcoming of Einstein-Eddington's affine theory (in which no electromagnetic field can exist in a place with vanishing electric current density) by dropping the assumption of the symmetry of the affine connection}% 
%%
%does not lead to anything useful from a physical point of view and I found a better trail that is physically more grounded}{Am 1. Juni bin ich von S\"udamerika wiedergekommen ... Ich bin fest \"uberzeugt, dass die ganze Gedanken-Reihe Weyl-Eddington-Schouten zu nichts physikalisch brauchbarem f\"uhrt und habe jetzt eine andere Spur gefunden, die mehr physikalisch fundiert ist} \lettercpaep{Einstein}{Besso}{5}{6}{1925}[15][2]. As he explaiend to besso to use both \Gtmn and \gmn but were not symemtric\todo{metric affine theory}. At the beginning of \datemy{9}{7}{1925}, Einstein presented at the Academy of Science the new trail he anticipated to Besso, a further attempt at a \uft \citep{Einstein1925a}, in which both the affine connection and the metric were considered as fundamental variables without assuming their symmetry. Einstein commented to Millikan with the usual initial enthusiasm. The paper was published at the beginning of September, and by that time, Einstein probably already moved on (\lettercpae{Einstein}{Rainich}{13}{9}{1925}[15][106]; see \cite{Einstein1927c}).  




%Ihre Gesetze werden ebensowenig in der Wirklichkeit jemals verletzt, wie es Wahrheiten gibt, die mit der Logik nicht im Einklang sind; aber über das inhaltlich-Wesen- hafte dieser Wirklichkeit machen sie nichts aus, der Grund der Wirklich- keit wird in ihnen nicht erfaßt

%ir hatten erkannt, daß Physik und Geometrie schließlich zusammenfallen, daß die Weltmetrik eine, ja viel- mehr die physikalische Realität ist. Aber letzten Endes erscheint so diese ganze physikalische Realität doch als eine bloße Form ; nicht die Geo- metrie ist zur Physik, sondern die Physik zur Geometrie geworden. Wir haben nicht mehr wie nach alter Anschauung einen leeren Raum als die Form, in deren Rahmen sich eine Materie von gediegener Wirklichkeit konstituiert, und als den Schauplatz, auf welchem sich die wirklichen Geschehnisse, das sind dieser Materie Veränderungen abspielen; sondern die gesamte physische Welt ist zur Form geworden, der aus ganz andern Bezirken als denen der Physis ihr Inhalt zuwächst. 

%In his eyes, physics seemed to be transformed to a purely formal status and was absorbed by geometry. Matter had seemingly become an epiphenomenon of the "world metrics" which started to acquire a slightly mystical flavour





One can think of $d\xn$ as the components of a (contravariant) vector $A^\tau$, $n$ numbers $A^\mu$ ($A^1, A^2, A^3, A^4) that we associate with some point $P$ and transform as per certain rules by the change of coordinates. In Euclidean geometry, it is always possible to introduce a Cartesian coordinate system in which two vectors are equal and parallel when they have the same components. However, this relation does not hold if we introduce curvilinear coordinates, \eg polar coordinates. Although parallel vectors are still parallel in the new coordinate system, the equality of the components of two parallel vectors attached to two different points in space is not preserved. \Eg consider two unit vectors $A^\tau$  and $A^{*\tau}$ on a plane pointing along the $x$ direction: one at the point at $(0,1)$ and another at $(1,0)$ in Cartesian coordinates. In this coordinate system, $A^\tau$  and $A^{*\tau}$ have the same components, \ie they are equal and parallel. However, in polar coordinates $r,\vartheta$ (where $r$ represents distance from the origin, and $\vartheta$ represents the angle that the point makes with the origin and the positive $x$-axis), $A^\tau$ has only a $r$ component, whereas $A^{*\tau}$ has only a $\vartheta$ component. Nevertheless, they are still equal and parallel. Indeed, the vector $A^*\tau$ can be obtained by displacing $A^\tau$ parallel to itself along a circle. In polar coordinates, the components $A^\tau$ change at each point even though its length and direction remain the same

In Euclidean geometry, it is always possible to introduce a Cartesian coordinate system in which two vectors are equal and parallel when they have the same components. However, this relation does not hold if we introduce curvilinear coordinates, \eg polar coordinates. Consequently, vectors at different points can no longer be directly compared. If one displaces a vector to a neighboring point $dx_\nu$, one does not know whether the vector has remained the \scare{same} by simply examining its components. The \scare{connection} (\german{Zusammenhang}) from a point to another is lost. Because the affine geometry is the study of parallel lines, \citet{Weyl1918b} used to speak of the necessity of establishing an \scare{affine connection} (\german{affiner Zusammenhang}). However, because it is a relation of \scare{sameness} rather than parallelism that is relevant in this context, others, such as Reichenbach, prefer to speak of the operation of \scare{displacement} (\german{Verschiebung}), where the latter indicates the small coordinate difference $d\xn$ along which the vector is transferred. 


To reinstate the \scare{connection} one requires to introduce a rule for comparing vectors at infinitesimally separated points. Given a vector $A^\tau$ at \xn in an arbitrary coordinate system, we need to determine the components of the vector $A^{\ast\tau}$ at $\xn+d\xn$ that is to be considered the \scare{same vector} as the given vector $A^\tau$. The vector $A^{\tau}$ at the point $P\left(x^{\nu}\right)$ and the vector $A^{\tau}+d A^{\tau}$ at the point $P^{\ast}\left(x^{\nu}+d x^{\nu}\right)$ are the \scare{same vector}, if they satisfy the condition:

\begin{equation}\label{eq:affine}
dA^\tau = \Gamma^\tau_{\mu\nu}A^{\mu} dx_\nu\,.
\end{equation}
%
The quantity $\Gamma^\tau_{\mu\nu}$ is known as the affine connection or displacement. It has three indices, i.e., entails $\tau$ possible combinations of $\mu \times \nu$ coefficients, which can vary arbitrarily from point to point, i.e., in the general case, are functions of $x_\nu$. Because in general $\Gtmn \neq \Gtnm$, the \Gtmn has $n \times n^2$ coefficients. If a vector $A^\tau$ is given at the point $P$ with coordinates \xn, \cref{eq:affine} yields the unknown components of the vector $A^{\ast\tau}$ at $P^*$ with coordinates $\xn+dx_\nu$. Continuing this process $
\xn+d\xn+d^{\ast} \xn+d^{\ast \ast} \xn \ldots$, we can parallel displace a vector from any given point to any other distant point. As is well known, the most characteristic feature of the operation of displacement is that if one parallel displaces $A^\tau$ along different paths, one gets, in general, a different vector $A^{\ast \tau}$ at a distant point:

It is naturally to assume that is symmetric. The displacement allows to establish whether two vectors are the \scare{same}, i.e., having the same length and the same direction. However, it does not provide a measure of the length of differently directed vectors. For this purpose, the notion of dot product of two vectors must be introduced, which, taking the components of the two vectors, returns a single number. In particular, the squared length $l$ of a vector is given by the dot product of the vector with itself $l^2$. In an arbitrary coordinate system, the latter takes the form:

\begin{equation}\label{eq:3}
l^2=\gmn A^\mu A^\nu\,,
\end{equation}
%
where the \gmn is the metric. If $A^\tau$ is considered to correspond to $dx_\nu$, \cref{eq:3} is nothing but \cref{eq:lineelement} and $l$ corresponds to $ds$. However, this notation is more general. One can take $A^\tau$ to be $dx^\nu/ds$, (where $ds$ is the \til interval, which is an element of the four-dimensional trajectory of a moving point), $l$ is the length of the four-velocity vector $u^\nu$. How much a vector varies, this is simple in the case that, covariant differentiation, a second term is necessary, that is that the covariant of the metric vanish


By imposing this condition, one obtains 

\begin{equation}\label{eq:riemannconnection}
\Gtmn=-\christoffel{\mu}{\nu}{\tau} =
\end{equation}
%

The components of \Gtmn have the same numerical values of the so-called Christoffel symbols of the second kind (up to a sign) because they are calculated from the metric \gmn and its first derivatives.




%\cop{Yet it is incorrect to conclude, like Weyl and Haas, that mathematics and physics are but one discipline. The question concerning the validity of axioms for the physical world must be distinguished from that concerning possible axiomatic systems. It is the merit of the theory of relativity that it removed the question of the truth of geometry from mathematics and relegated it to physics. If now, from a general geometry, theorems are derived and asserted to be a necessary foundation of physics, the old mistake is repeated. This objection must be made to Weyl's generalization of the theory of relativity which abandons altogether}. 
%
%
%\q{What is the length of a physical rod? It is defined by a large number of physical equations that are interpreted as length with the help of readings on geodetic instrumets. The definition results from a coordination of things to equations. 1--hus we are faced with the strange fact that in the reahn of cognition two sets are coordinated, one of which not only attains its order through this coordination, but whose elements are defined by 1neans of this coordination}
%
%\cop{Schlick is therefore right when he defines truth in terms of unique coördination. ${ }^{11}$ We always call a theory true when all chains of reasoning lead to the same number for the same phenomenon. This is the only criterion of truth; it is that criterion which, since the}. \cop{By means of which principles will a coördination of equations to physical reality become unique?}. \cop{There exist systems of coördinating principles which make the uniqueness of the coördination impossible; that is, there exist implicitly inconsistent systems.}.
%
%
%\cop{When ever have discovered a coordinating principle used in physics, we can indicate a more general one of which the first is only a special case. We might now n1ake the attetnpt to call the more general principle a priori in the traditional sense and to ascribe eternal validity at least to this principle. But such a procedure fails because for the more general principle an even general one can be indicated; this hierarchy has no upper limit}.
%



%





%%21 (p. 76). Hermann Weyl, Raurn-Zeit-Materie (Berlin: Springer, 1918), p. 227; Arthur Haas, " Die Physik als geometrische Notwendigkeit,'' NatltTwissenschaften} VIII, 7, pp. 121-140.
%
%%If, for instance, Weyl's generalization should turn out to be correct, a new subjective element \vill have appeared in the 1netric. Then the comparison of nvo s1nall n1easuring rods at two different space points also no longer contains the objective relation that it contains in Einstein's theory in spite of the dependence of the 1neasured relation upon the choice of the coordinates, but is only a subjective form of description, con1parable to the position of the coordinates.



%Der erste Punkt betrifft Herrn Dr. Reichenbach, dem ich möglichst schnell zu Habilitation verhelfen möchte.[2] Er hat dazu eine Arbeit: „Die Bedeutung der Re- lativitätstheorie für den physikalischen Erkenntnisbegriff“ eingereicht. Er sagte mir, dass Sie die Arbeit schon gelesen hätten, sodass ich sie Ihnen nicht zu schicken brauchte.[3] Ich muss nun sagen, dass meine eigene Erfahrung nicht hinreicht, um die Arbeit genügend zu würdigen und ich glaube ferner, dass Ihr Urteil über die Ar- beit bei der Hochschule hier sehr viel wiegen wird, sodass ich für den glatten Ver- lauf der Habilitation Günstiges erhoffe. Es wäre das deswegen besonders wertvoll, weil die Arbeit doch zum grossen Teil philosophischen Charakters ist, Herr Rei- chenbach sich aber für Physik habilitieren soll. Nun glaube ich, dass auch Sie mit mir übereinstimmen werden darin, dass gerade solche Arbeiten auch für den Phy- siker bei der augenblicklichen Entwicklung in der Physik sehr nützlich sind, sodass seine Arbeit auch dem Herrn Reichenbach zur Habilitation inder Physik dienen kann.

%The thesis was published after 15 June 1920 as Reichenbach 1920 (see Doc. 57). Stuttgart, Wiederholdstr. 13. d. 15. Juni 1920. Widmung Reichenbach attended Einstein's Ieerures in Berlin (see Doc. 57, note 2). That most of the discussion of the book, relativized constitutive \apr, however, one first by a philosophers of Weyl theory, and in particular was that is itself superior. \q{Philosophen eine Ahnung davon haben, dass mit Ihrer Theorie eine philosophische Tat getan ist, und dass in Ihren physikalischen Begriffsbildungen mehr Philosophie enthalten ist, als in allen vielbändigen Werken der Epigonen des grossen Kant.} \q{die tiefe Einsicht der Kanti- schen Philosophie von ihrem zeitgenössischen Beiwerk zu befreien} \cop{Der Wert der Rel.Th. für die Philosophie scheint mir der zu sein, dass sie die Zweifelhaftigkeit gewisser Begriffe dargethan hat, die auch in der Philosophie als Scheidemünzen anerkannt waren}. 

%Weyl's presentation was Pauli and Einstein, was included the 4th edition of Raum, Zeit, Materie (Weyl, 1921a) finished in November 1920, to explain ?the discrepancy between the idea of congruent transfer and the behavior of measuring-rods and clocks and atoms? (Weyl, 1921a, 280; tr. 1922a, 308).



%The mechanism of adjustment is the only way to explain the surprising fact that electrons and hydrogen nuclei always have the same mass and charge, and thus the very existence of identical atoms, along with the possibly that these atoms, under given external conditions, settle into identical crystalline structures, that we can use as rods.  Weyl thus had good reasons to claim that the only explanation for the fact that electrons always have the same charge, whatever their prehistory, might have been to assume that there is some field quantity of the dimension of a length (i.e., it is simply a number) to which the charge of the electrons ?adjust? themselves in a certain proportion. This was, after all, the general framework that, e.g. Mie (who also gave a talk at Bad Nauheim, Mie, 1920) had tried to realize without any success (cf. also footnote 5). A certain state of equilibrium of the negative (or positive) electricity would always be ?reestablished? whatever disturbance it may have experienced in the past, just as the 
%magnetic needle always points north, despite what may have happened to it previously. 


%Reichenbach's criticism, Erwin Freundlich at about the same time. There was indeed, that the theory could was either refuted, by assuming the physical measuing rod, behave differently from ideal ones, which follow the geometry of \st. 

%Indeed, a few days later \cop{Einstein wrote to Cassirer a few days later, that the meaning of the $ds$}, how the coordination of his objection against Weyl. That was meaningless without rods and clocks 


%Weyl's presentation was Pauli and Einstein, was included the 4th edition of Raum, Zeit, Materie (Weyl, 1921a) finished in November 1920, to explain ?the discrepancy between the idea of congruent transfer and the behavior of measuring-rods and clocks and atoms? (Weyl, 1921a, 280; tr. 1922a, 308).



%The mechanism of adjustment is the only way to explain the surprising fact that electrons and hydrogen nuclei always have the same mass and charge, and thus the very existence of identical atoms, along with the possibly that these atoms, under given external conditions, settle into identical crystalline structures, that we can use as rods.  Weyl thus had good reasons to claim that the only explanation for the fact that electrons always have the same charge, whatever their prehistory, might have been to assume that there is some field quantity of the dimension of a length (i.e., it is simply a number) to which the charge of the electrons ?adjust? themselves in a certain proportion. This was, after all, the general framework that, e.g. Mie (who also gave a talk at Bad Nauheim, Mie, 1920) had tried to realize without any success (cf. also footnote 5). A certain state of equilibrium of the negative (or positive) electricity would always be ?reestablished? whatever disturbance it may have experienced in the past, just as the 
%magnetic needle always points north, despite what may have happened to it previously. 


%Reichenbach's criticism, Erwin Freundlich at about the same time. There was indeed, that the theory could was either refuted, by assuming the physical measuing rod, behave differently from ideal ones, which follow the geometry of \st. 

The importance of this formalism for \gr was to which was the key content of the theory. Let's conisder the four-velocity of a particle $u^\tau=d\xn/d\ap$. By parallel-displacing a vector $u^\tau$ indicating the direction of a curve $\xn(\ap)$ at any of its points, one can define a special class of curves, the straightest lines. The operation of parallel transporting a vector $u^\tau$ along a curve $x(\ap)$ is expressed by the condition that the covariant derivatives of $u^\tau$ with respect to the parameter \ap vanish along the curve: $\frac{d u^{\tau}}{d \ap}-\Gtmn u^{\mu} \frac{d x^\nu}{d\ap}=0$. The vector $u^\tau$ indicates the direction of the curve $x_\nu(\ap)$ at each point if its components are proportional to the increments $dx_\nu$ along the curve, \ie if $u^\tau=d\xn/d\ap$. The curve traced by the parallel displacement of $u^\tau$ along its own direction $d\xn/d\ap$ is the straightest curve. According to special relativity, a freely movable body not subjected to external forces moves, according to the special theory of relativity, in a straight line and uniformly with respect to an inertial coordinate system $K$:

\begin{equation}\label{eq:geodesicelectro}
\frac{d^{2} x_{\tau}}{d s^{2}} = 0
\end{equation}

If \ap is the so-called \scare{proper time}, $u^\tau$ as the velocity four-vector of a particle, and $\frac{d {u^\tau}}{d\ap}$ its acceleration. If now consider from the of a we can, that this is ultimately. If chose a suitable coordinate system that the right had sight of the equation vanish. Then if we introduce new space-time co-ordinates $x_{1}, x_{2}, x_{3}, x_{4}$, by means of any substitution we choose, the go in this new system will no longer be constants, but functions of space and time.

\begin{equation*}
\frac{d {u^\tau}}{d\ap} - \Gtmn u^{\mu} u^\nu =0 
\end{equation*}
 
According to the equivalence principle $K$ and $K'$ is however, indistinguisblae from $K$. All fall with the same velocity, thus the non vanisihig of the \gmn. If the $\Gamma_{\mu \nu}^{\tau}$ vanish, then the point moves uniformly in a straight line. These quantities therefore condition the deviation of the motion from uniformity. They are the components of the gravitational field.


\begin{equation}\label{eq:riemanntensorgamma}
R_{\mu \nu \sigma}^{\tau}(\Gamma)=\frac{\partial \Gamma_{\mu \nu}^{\tau}}{\partial x^{\sigma}}-\frac{\partial \Gamma_{\mu \sigma}^{\tau}}{\partial x^{\nu}}+\Gamma_{\alpha \nu}^{\tau} \Gamma_{\mu \sigma}^{\alpha}-\Gamma_{\alpha \sigma}^{\tau} \Gamma_{\mu\sigma}^{\alpha}\,, = 0
\end{equation}
%
In the general case how, so that without changing its components. The physical hypothesis on which general relativity is based, on the fundamental idea that valid also in the case in which does not vanish. This difference, curvature of space. The vector would not return with the same components.

This point is essential to Reichenbach's argument. It was precisely because his toy-geometrization was not envied by its more titled competitors that Reichenbach believed himself to be in an excellent position to \qt{attack the view that with a geometrical presentation of electricity, one would already gain something}{Ich wollte mich eben mit meiner Darstellung gegen die Auffassung wenden, als ob mit der geometrischen Darstellung der Elektrizität an sich schon etwas gewonnen wäre}[\letter{Reichenbach}{Einstein}{4}{4}{1926}][20-086][EA]. For such a geometrical interpretation of electromagnetism to become a physical theory as successful as general relativity, more was required than a mere geometrization. It called for something \emph{new}. Howeve,r in this cas eht egometrizajo did provide any sucecs. Of course if one could to reight field equatios, im particular than wuld ha bene trasitwryt.


%%Thus, Reichenbach argued that the geometrical interpretation of a physical field can only be successful if it leads to a \scare{change} in the equations and does not simply rewrite in geometrical terms the equations that are already known. One could object that Weyl, Eddington and Einstein's theories also changed the equations and did not simply rewrite them. However, Reichenbach seems to consider the derivability of solutions that correspond to the electron as a litmus test for a real change in the field equations. Maxwell's field equations are valid in free space and cannot explain why the separate, equally charged parts do not fly apart without introducing a non-electromagnetic cohesion force (the so-called Poincar\'{e} stress). On the other hand, Einstein's field equations, in their original form, do not entail any effect of gravitation on charge and cannot provide the cohesion force. It is only by changing the currently available field equations that it would become  possible to establish a connection between \gmn and $f\mn$, thereby assuring the equilibrium of the electron.
%
%To fully understand Reichenbach's stance on this issue, one must keep in mind that in a paper published in April \citep{Reichenbach1926} he expressed strong skepticism about the possibility of solving the problem of the \scare{grainy} structure of matter and and most all of the \qt{proper quantum-riddle}{eigentlichen Quantenrätsels}[][424][Reichenbach1926] in a field-theoretical/geometrical context. In Reichenbach's view, the \qt{casuality \origins{Zufälligkeit}}{Zufälligkeit}[][424][Reichenbach1926] of the \scare{quantum jumps} (the transition between orbital energy levels in Bohr's atom) suggests that the problem should be tackled from a different angle, by considering whether the very notion of causality in physics should be replaced by that of probability \citep[424]{Reichenbach1926}. After all, one should appreciate Reichenbach's clairvoyance, considering that Max Born's paper \citep{Born1926a} on the statistical interpretation of the wave function appeared only in June\footnote{For Reichenbach's take on the emerging quantum mechanics at this time, see the posthumously published manuscript, \cite{Reichenbach1926e}}.




%Weyl explained to Reichenbach in details the strategy he had laid down in Bad Nuhei,
%
%\q{Der Erfahrung wird durch die Annahme jener allgemeineren Metrik in keiner Weise vorgegriffen;  denn die Naturgesetze, an welche die Wirkungsausbreitung im Äther gebunden ist, können ja von solcher Art sein, daß sie keine Streckenkrümmung zulassen. Diese Möglichkeit liegt sogar nicht einmal ferne. Nehmen Sie als Wirkungsgröße das mittels der von mir in Nauheim benutzten "Normaleichung" $F=$ const. gemessene Volumen (Bezeichnungen nach Raum Zeit Materie, 3. oder 4. Aufl.!), so liefert das Wirkungsprinzip \textelp{} Wofür ich allein eintrete, ist dies: Die Integrabilität der Strekkenübertragung (wenn sie besteht, ich glaubs nicht, denn ich sehe nicht den geringsten zwingenden Grund dafür) liegt nicht im Wesen des metrischen Mediums, sondern kann nur auf einem besonderen Wirkungsgesetz beruhen. Wäre die historische Entwicklung anders verlaufen, So scheint mir, wäre niemand darauf verfallen, von vornherein gerade nur den Riemannschen Fall in Erwägung zu ziehen. Was die berüchtigte "Abhängigkeit von der Vorgeschichte" betrifft, so habe ich darüber wohl meine Ansicht deutlich genug in Nauheim ausgesprochen}\hide{\letterp{2}{2}{1921}}} 


% An der 4. Aufl. wird Sie wahrscheinlich vor allem meine veränderte Stellungnahme zum Problem der Materie interessieren; von der Universalität der Feldphysik bin ich gründlich zurückgekommen 




%Geuau, wie "Einstein zeigen nuõ§, da§ aus der Dyuauõik des starren Kšrpers heraus eiu solches Verhalten folgt, da§ der Ma§stab innner dieselbe LŠnge hat. gemessen iu seinem ds, so uuõ§ ielõ zeigen, da§ er iumõeõ' das gleiche dnrelõ R = eonst uoruõieõ't.e ds hat.. Wie etwa "Einstein so gut als ielõ das zu nõaelõen lõŠ.tten, habe ieh am Selõlu§ uõeiner Arbeit " Feld uud Materie". Ann. d. l'lõys.'2" augedeutet.

%Lieber Herr Doktor,156 warum sind Sie so gegen die Besprechung von Neuauflagen? In ^Wey1s 4. Aufi. liegen allerhand wesentliche €nderungen vor, die bei der Bedeutung dieses Buches eine erneute Besprechung gerecht- fertigt hŠtten. Nuu wage ich nõich gar nicht mit der Bitte hervor,

%\footnote{For Eddington \scare{natural geometry} is exactly Riemannian and \scare{world geometry} was nothing but a conceptual scheme; On the opposite  Weyl insisted on the  the real \scare{aether geometry} was non-Riemannian and the \scare{body geometry} distorted by the mechanism of adjustment} presented his own affine theory, embracing  and the same strategy, was embraced even if with a different turn. This is a necessary condition for what is called affine geometry. It appears to express the condition that; the world is " fiat in its smallest parts" or that it possesses a definite tangent. However, beside that indeed, what he introduced metricity tensor: $\mathbf{K}_{\mu \nu, \sigma}$, defined as 

 %What we have sought ill not the geometry of actual space and time. but the geometry of the world-structure. which is the common basis of space and time and things. Thus Eddington that the real geometry was exactly Riemannian, and was nothing more than a graphical representation, like phase space.  And in March Einstein suggested a sort an intermediate way, a conformal theory, in which \rac are not are not measured. Einstein dislike is a measuring rod geometry, and at the same time real measuring rods behave differently from the ideal one \todo{(du pasquier)}, there was no reason to assume that this was dif
 
%This fundamental imbalance, between in which however, the instrument that serve to measure lengths, differently, in which $k_\sigma$ is exactly Riemannian. Thus the \rac do not conform, that the real geometry of \st is non-Riemannian. A second approach, was pursued Arthur Stanley Eddington, that preferred to that only on the affine connection, without any metric. Is exactly Riemannian, in which $k_\sigma$, is exactly Riemannian, is only a graphical representation, comparable to the representation of with, to adaptation to the radius of curvature of the world, that enter into the dynamic equations determining the behavior of \rac. Introduce, a tensor split into two parts, and fiannly impose. Probably, unaware of both approaches Einstein presented in March, to avoid the \rac at all. Indeed, from one can define the Riemann, tensor contracted into the Ricci tensor, into two tensor, third option, that is a theory that completely renounced to the very existence of transportable measuring rods, on 17th March. In this way there was no concern about doubling the geometry, which seems to feel as unpleasent. 




%
%However, Einstein skepticsm grew rapidly, the epistemologicla lode, that the there was. no reason to conenct a phiosclao interpeation from the outset. Ideed, geometry and phsics were  Einstein had already become somewhat disillusioned with the affine approach, he nevertheless delivered a special lecture in Gothenburg on 10 July 1923 on Weyl's and Eddington's generalizations of general relativity, alongside his Nobel Prize lecture. At around that time he also agreed to write an appendix to the German translation of Eddington's "Mathematical Theory of Relativity" (Einstein 1925a [Doc. 282]).}, in which he had expressed. Meyerson's book on relativyt, by 1925 Einstein had moved on.


 %Weyl's reaction seems to manifest that in spite of the attacks his theory was receivn frm this diferent sides, his attitude believed he still had some cards to play. had his how Weyl had introduced \scare{doubling the geometry}. As it turned out there was two different ways to proceed, to different way to save the the new geometry, and in general to to a more general affine connection.  A few days later on February 17 Weyl published the English description on Nature, that is that real ether non-Riemannian geometry was to the measuring rod geometry, that appears in reality.
 
 %\qt{I'm of course very glad that you agree with my $\Gamma$-critique. I have now made a few reflections on the topic, which seem to me to prove that the Weylean thought, although good mathematically, does not bring about anything new physically. The geometrical interpretation of electricity is only a visualization, which in itself still does not say anything, and can also be realized in the original relativity theory. I have attached the note and would be grateful if you could give it a look \letter{Reichenbach}{Einstein}{24}{3}{26}][20-085][EA]}{Daß Sie meiner $\Gamma$-Kritik zustimmen, hat mich sehr gefreut \textelp{} Ich habe jetzt eine kleine Ueberlegung zu dieser Sache durchgeführt, die mir zu beweisen scheint, warum der Weylsche Gedanke, so gut er mathematisch ist, nichts Physikalisch Neues bringt. Die geometrische Deutung der Elektrizität ist doch nur eine Veranschaulichung, die selbst noch garnichts besagt, und die sich ebenso in der ursprünglichen Relativitätstheorie durchführen läßt. Ich lege Ihnen die Note ein und wäre Ihnen für Durchsicht sehr dankbar}

%, after making some comments about his academic situation,



%Reichenbach attached to this letter a typewritten note. As we shall see, far more was at stake in it than a critique of Weyl's theory (which was generally considered a dead horse at the time). Reichenbach intended to call into question the very idea that, since general relativity has \scare{geometrized} the gravitational field, the obvious next move should be to try to \scare{geometrize} the electromagnetic field. The importance of this formalism for \gr was to which was the key content of the theory. 

%%%%%%%%%%%%%%%%%%%INTRO%%%%%%%%%%%%%%%%%%



%As in general relativity one uses an atoms as standard of measure, since all atoms of the same time are identical}. Thus, Weyl's theory should predict that would are not, also that to not tick at the same rate in the presence of the electromagnetic field. However, this is not the case. Indeed, have the same spectral lines. Any physical content. However, they theory has moreover further shortcomings. The trajectories of free uncharged particles are not geodesics (see, for instance, Doc. 579) in the theory because of the presence \phin in the definition of the connection. The field equations are of the fourth order which increased the degree of arbitrariness of such equations. \cop{Moreover, to find the laws of pure gravitation and of electromagnetism, it was necessary to define the Langrangian $\mathcal{H}$, composed of two absolutely independent parts. We thus had a dualism; to avoid it, he arrived at the solution mentioned}. The theory to find the electron a question that Einstein felt was essential for the realization of the unified field theory. 

%216. To Arthur S. Eddington Berlin. d. 15. Dezember 19. %Die Weylsche Theorie (metrische Deutung des elektro-magnetische[n] Potenti- als) halte ich nicht für zutreffend. Sie scheint mir unvereinbar mit der Tatsache, dass Maassstäbe und Uhren ein von ihrer Vorgeschichte unabhängiges metrisches Verhalten zeigen.f

%Weyl hat in der neuen Auflage seines Lehrbuchs nun seine elektromagnetische Theorie leider angefügt, sodass dieser allerdings sehr geistreiche Unfug seinen Weg in die Gehirne nehmen wird.!11) Aber ich tröste mich damit, dass das Sieb der Zeit seine Arbeit auch an dieser Stelle thun wird. In der Relat. Theorie ist mir seither nichts mehr gelungen. Das elektromagnetische Feld trotzt allen Bemühungen

%Hoffentlich geht es Weyl wieder gut. Er ist ein sehr bedeutender Kopf. aber et- was thatsachenfremd. In der neuen Auflage seines Buches hat er mir die Relativität ganzverhunzt-Gottverzeihe es ihm.l 121 Vielleicht wird er doch noch einmal ein- sehen, dass er da bei allem Scharfsinn daneben geschossen hat.

%332. To Heinrich Zangger
%[Berlin, 27 February 1920}(1)

%189. To Paul Ehrenfest
%[Berlin,] 4. XII 19.
%Chronfectl

%Es ist mir unbegreiflich, dass Weyl selbst und alle andern das Erfahrungswid- rige des Grundgedankens der Theorie nicht ohne weiteres spüren

%Pauli works


%In this new theory, the gravitational Lagrangian remained the square of the curvature. The Maxwellian part of it, however, a quas1-Einsteinian form. This led him to a variational principle that included the Einsteinian component (with the cosmological term), the in Einstein's cosmological model.) By following this path, Weyl arrived at unit of charge, the gravitational radius of which, has the same order of magnitude as the radius of the universe. Noting that, in his work, the unit of electricity and the unit of action both have "cosmic values," Weyl emphasized: '"'The 'cosmological' term that Einstein first added to his theory is a natural consequence of our original principles" (Weyl 1919,p. 124, italic in the original).


%a quasi-Einsteinian form. This led him to a variational principle that included the Einsteinian component (with the cosmological term), the Maxwellian component, and a non-Maxwellian component, the square of square of the vector potential p,p' ("the simplest expression found in the Mie theory" (Weyl 1919, p. 122), a theory that claimed to provide a unified field description of the electron and the electromagnetic field).equations. adopted, as his Lagrangian, the square of the curvature} 


%In his 1919 paper, Weyl suggested a new variant of unified theory a that generalized GR and was based on a geometry that he introduced (i.e., the Weyl geometry). In this new theory, the gravitational Lagrangian remained the square of the curvature. The Maxwellian part of it, however, was no longer emerging in a natural way but was added "by hand." Through artful manipulations, Weyl managed to bring this Lagrangian into a quasi-Einsteinian form. This led him to a variational principle that included the Einsteinian component (with the cosmological term), the Maxwellian component, and a non-Maxwellian component, the square of the vector potential p,p' ("the simplest expression found in the Mie theory" (Weyl 1919, p. 122), a theory that claimed to provide a unified field description of the electron and the electromagnetic field).

%If one makes the first assumption in Weyl’s theory, Einstein argued, the rate of a clock will be dependent on its prehistory; if one makes the second, uncharged particles will be affected by the electromagnetic four-vector potential. In Doc. 661, Einstein reiterated both charges, that of the fourth-order and not of the second order}.  Einstein held the opinion that the world-lines of an uncharged particle should be described by a geodesic equation that does not explicitly depend on the electromagnetic four-potential, as it does in Weyl's theory.  \cop{This led him to separate the theory into two distinct parts: 1) the "pure, infinitesimal geometry," which provides a set of geometric structures \gmn and \phin. The form gravitational field.  Thus, the correspond to the presence of the gravitational field, and the to the presence of the electromagnetic field. (2) Is to find the field equations, via the \scare{action principle}. \cop{One constructs a scalar quantity (the action) from the dynamical quantities \gmn and \phin then finds the conditions needed to restrict the scalar to an extremum (a maximum or minimum) with respect to variations in those dynamical quantities. The problem the right action and the right dynamical quantities to produce the desired equations}. 


\subsection{Reichenbach's Notes on Einstein's Lecture and the Notion of Parallel Transport of Vectors}


%Nach seinem Einsatz im Ersten Weltkrieg hörte Reichenbach in Berlin Einsteins Vorlesungen zur Speziellen und Allgemeinen Relativitätstheorie. Seine Mitschriften dieser Veranstaltungen sind erhalten geblieben: Klassische und Statistische Mechanik (SS 1918, HR-028-01-02), Allgemeine Relativitätstheorie (Teil 1 HR-018-0104, Teil 2 HR-028-01-03, Teil 3 HR-028-01-01, alle Tile undatiert). In Teil 1 der Mitschrift zur Allgemeinen Relativitätstheorie, behan- 1917--1918 and 1918-1919. His transcripts of these events have been preserved: Classical and Statistical Mechanics (SS 1918, HR-028-01-02), General Theory of Relativity (Part 1 HR-018-0104, Part 2 HR-028-01-03, Part 3 HR- 028-01-01, all tiles undated). 


%The introduction of the Riemann curvature tensor given in these two papers can be found in these lecture notes as well (see [p. 11]), but it is clear from entries on [p. 10] and [p. 25] and from Reichenbach's notes (the last part of the second notebook) that}. From the metric, Christoffel symbos, and Riemann tensor is flat.

After serving in World War I, Reichenbach attended Einstein's lectures on special and general relativity in Berlin. We posses three sets of Reichenbach;s notes (HR-028-01-04, HR-028-01-03, HR-028-01-01 all undated). The last set of notes seems to corresponds to the Einstein's lecture on spring term 1919 \citep{Einstein1919c}. {The notes taken by Reichenbach very similar to Einstein's own notes\footnote{Further information about Einstein as an academic teacher, see Vol. 3, the editorial note, "Einstein's Lecture Notes,"pp. 3-10, and for a survey of Einstein's academic courses, see Vol. 3, Appendix B.}. \cop{The first sections of the lectures follow the corresponding sections Einstein's previous published presentations \citepp{Einstein1916}{Einstein1914a}.} \cop{The general theory of relativity rests formally on the geometry of Riemann, which bases all its conceptions on that of the interval} 

\begin{equation}\label{eq:lineelement}
ds^2=\gmn dx_\mu dx_\nu\,.
\end{equation}
%
between points indefinitely near together, in accordance with the formula, these magnitudes $g_{\mu \nu}$ determine the behavior of \rac with reference to the coordinate system, as well as the gravitational field \citep[028-01-04, 46]{HR}. From the \gmn the so called Christoffel symbols \christoffel{\mu}{\nu}{\tau}, functions of the \gmn and their first derivatives, linear in the latter. which enter into the geodesic equation \citep[028-01-04, 46, 2ff]{HR}. It is possible to chose a coordinate system Christoffel symbols vanish everywhere, by the vanishing of Riemann tensor \rite, which is function of the Christoffel symbols and their first derivatives  After this standard presentation in both Reichenbach and Einstein's lecture notes show how Einstein also used for the first new interpretation of the curvature in terms of the parallel displacement given by Tullio \citet{Levi-Civita1916} and Hermann \citet{Weyl1918}, who are mentioned explicitly. It is useful to sketch the fundamental ideas in order as premise to the of the paper.



%\begin{equation*}
%\christoffel{\mu}{\nu}{\tau} = \frac{1}{2} g^{\tau \sigma}\left(\frac{\partial x_{\mu \sigma}}{\partial x_{\nu}}+\frac{\partial g_{\nu \sigma}}{\partial x_{\mu}}-\frac{\partial g_{\mu v}}{\partial x_{\sigma}}\right)
%\end{equation*}.

%The latter which enter in the of the geodesics equations, and thus when move on straight line. We know from special relativity that a free particle moves along a timelike geodesic in a non-accelerated coordinate system:
%
%\begin{equation}\label{eq:geodesicchristoffel}
%\frac{d^{2} x_{\tau}}{\dap^{2}} + \christoffel{\mu}{\nu}{\tau} \frac{d x_{\mu}}{\dap} \frac{dx_{\nu}}{\dap} = 0\,,
%\end{equation}
%%
%where $\ap$ is the affine parameter of the geodesic. In unaccelerated rectangular coordinates $K$, the second term vanishes of \cref{eq:geodesicchristoffel}, and the equations reduce to, which are the equations of a straight line. When we switch to other coordinates $K'$ un uniform acceleration with respect to $K$, the path is no longer straight, that is, it is no longer given by a linear relation between the coordinates and right-hand side of \cref{eq:geodesicchristoffel} does not vanish. Again since uniform acceleration from being at rest in a uniform gravitational field, the presence of gravitation is characterized by the non-vanishing of the Christoffel symbols. The necessary and sufficient condition that we can find a coordinate system which is everywhere inertial, is that 
%
%\begin{equation}\label{eq:riemanntensor}
%\riteg =-\frac{\partial}{\partial x_{\tau}}\christoffel{\mu}{\sigma}{\rho}+\frac{\partial}{\partial x_{\sigma}}\christoffel{\mu}{\tau}{\rho} - \\ \christoffel{\mu}{\sigma}{\alpha}\christoffel{\alpha}{\tau}{\rho}+\christoffel{\mu}{\tau}{\alpha} \christoffel{\alpha}{\sigma}{\rho}\,.
%\end{equation}
%%
%It is possible to chose a coordinate system Chirstoffel symbols vanish everywhere if the Riemann tensor vanish. 
%
%\begin{equation*}
%\riteg = 0
%\end{equation*}The key conceptual is that to have extended equations also in the case in which $\riteg \neq = 0$;


In Euclidean geomtry it is always possible to introduce a Cartesian coordinate system in which two vectors are equal and parallel when they have the same components. However, this relation does not hold if we introduce curvilinear coordinates, \eg polar coordinates \citep[028-01-03, 35]{HR}. The coefficient are different from point to point, $\vartheta$. Thus, vectors at different points can no longer be directly compared. If one displaces a vector to a neighboring point $dx_\nu$, one does not know whether the vector has remained the \scare{same}, that is equal and parallel to itself, by simply looking at its components. To reestablished  (\german{Zusammenhang}), one requires to introduce a rule for comparing vectors at infinitesimally separated points. Given a vector $A^\tau$ at \xn in an arbitrary coordinate system, we need to determine the components of the vector $A^{\prime\tau}$ at $\xn+d\xn$ that is to be considered the \scare{same vector} as the given vector $A^\tau$. The vector $A^{\tau}$ at the point $P\left(x^{\nu}\right)$ and the vector $A^{\tau}+d A^{\tau}$ at the point $P^{\prime}\left(x^{\nu}+d x^{\nu}\right)$ are the \scare{same vector}, if they satisfy the condition:

\begin{equation}\label{eq:affine}
dA^\tau = \Gamma^\tau_{\mu\nu}A^{\mu} dx_\nu\,.
\end{equation}
%
the  quantity $\Gamma^\tau_{\mu\nu}$ is known as the \scare{affine connection} or displacement\footnote{The affine geometry is the study of parallel lines. \citet{Weyl1918b}. However, because it is a relation of \scare{sameness} rather than parallelism that is relevant in this context, others prefer to speak of the operation of \scare{displacement} (\german{Verschiebung}), where the latter indicates the small coordinate difference $d\xn$ along which the vector is transferred}. It has three indices, \ie, entails $\tau$ possible combinations of $\mu \times \nu$ coefficients, which can vary arbitrarily from point to point. Because in general $\Gtmn \neq \Gtnm$, the \Gtmn has $n \times n^2$ coefficients. If a vector $A^\tau$ is given at the point $P$ with coordinates \xn, \cref{eq:affine} yields the unknown components of the vector $A^{\prime\tau}$ at $P'$ with coordinates $\xn+dx_\nu$.

Levi-Civita assumed affine connection is symmetric $\Gtmn=\Gtnm$, which assure that the manifold \scare{flat in its smallest parts}. The affine connection, however, the length and angles of vector the inner product\todo{?}. He further imposed that condition that, in arbitrary coordinate system, vectors have length

\begin{equation*}\label{eq:metric}
l^2 = \gmn A^\mu A^\nu\,,
\end{equation*}
%
which does not change under parallel transport. If $A^\tau$ is displacement-vector $dx_\nu$, then thus the \cref{eq:metric} include \cref{eq:linelement} as a special case. this process. Thus, one can parallel-transport the vector $dx^\nu$ from any point to some other point, its length does not change.

%Wie also die Charakterisierung eines Vektors in P durch ein System von Zahlen (seine Komponenten) von der Wahl eines Koordi- natensystems abhängt,

%Thus the implication is that the length if two $ds$ every line is comparable with any other line. If these two conditions are imposed, one finds that the coefficients of the the affine connection \Gtmn are precisely the three-index Christoffel symbols that are obtained from the metric:


%define the law of parallel displacement, that is, the affine structure or space. If we have a space of given metric, we find that the T's are determined by the g's. It is sufficient to stipulate the condition that the modulus of the vector not vary, natural thing since it involves a repeated Euclidean process; that is, we must write that A2 = gIA'A* is invariant, which gives the conditions that the I's must satisfy, and one finds that they are precisely the three-index Christoffel symbols. It is easy to see that Riemann's fundamental tensor can be obtained with the parallel displacement.

\begin{equation*}\label{eq:levicivita}
\Gtmn=-\christoffel{\mu}{\nu}{\tau} = \frac{1}{2} g^{\tau \sigma}\left(\frac{\partial x_{\mu \sigma}}{\partial x_{\nu}}+\frac{\partial g_{\nu \sigma}}{\partial x_{\mu}}-\frac{\partial g_{\mu v}}{\partial x_{\sigma}}\right)
\end{equation*}

With this notion includes the notion of a parallel displacement. alone, without the aid of a metric, we may construct geodesics in the following manner. Continuing this process $\xn+d\xn+d^{\prime} \xn+d^{\prime \prime} \xn \ldots$, we can parallel displace a vector from any given point to any other distant point. As the size of each displacement goes to zero, this broken line becomes a continuous curve. Thus curve starts from P with a well-defined direction. If we use a parameter $\ap$ to designate points along the curve $dx\nu$, then $dx_\nu/d\ap=u^\tau$ is a vector, which The vector $u^\tau$ indicates the direction of the curve $x_\nu(\ap)$ at each point if its components are proportional to the increments $dx_\nu$ along the curve. By parallel-displacing a vector $u^\tau$ indicating the direction of a curve $\xn(\ap)$ at any of its points, one can define a special class of curves, the straightest lines among two points By parallel trasportin the vector $d x^{\mu}(P) / d s$ displaced to a point $P^{\prime}$ are identical with those of the vector $d x^{\mu} / d s$ at $P^{\prime}$, it can be shown:

%Now we use a parameter to designate points along the curve. Let $s$ be chosen to transform as a scaler and to have an invariant value at $P$ and $P^{\prime} .$ Then $d x^{\mu} / d s$ is a vector. The condition for a geodesic is that the components of the vector $d x^{\mu}(P) / d s$ displaced to a point $P^{\prime}$ are identical with those of the vector $d x^{\mu} / d s$ at $P^{\prime}$ :

 
\begin{equation*}\label{eq:geodesicequation}
\frac{d {u^\tau}}{d\ap} = \Gtmn u^{\mu} u^\nu
\end{equation*}
%
It is interesting that we have obtained this equation without recourse to the notion metric With this construction of a geodesic, as the straigthest line, rather that the shortest line. If \ap is the so-called \scare{proper time}, $u^\tau$ as the velocity four-vector of a particle, and $\frac{d {u^\tau}}{d\ap}$ its acceleration. The length of this vector is per defintion $=1$\todo{proper time}. 

As Einstein had remarked on several occasions, equation \cref{eq:geodesicequation} the conceptual core of \tr. According to special relativity, a freely movable body not subjected to external forces moves in a straight line and uniformly with respect to an inertial coordinate system $K$, in which the $\Gtmn=0$. Then, if we introduce new space-time co-ordinates $K'$ uniformly accelerated with respect to $K$, the $\Gtmn\neq 0$ and the trajectory of the particle is not straight. According to the equivalence principle $K'$ is however indistinguishable from $K$ in a gravitational field. Thus the non-vanishing of \Gtmn can be interpreted as the presence of the gravitational field. The \Gtmn are the components of the gravitational field, and the \gmn its potentials. The fact that \Gtmn are not a tensor represented form Einstein the \emph{unification} of inertial and gravitation. The fact that they are \german{wesengleich}. The same motion on a geodesic can be interpreted as inertial motion for $\Gtmn=0$ or as motion under the influence of a gravitational field for $\Gtmn \neq 0$.

%Indeed, the same hough a particle moves on a geodesic both in the absence and in the presence and in absence of 

%He believed that even though a particle moves on a geodesic both in the absence and in the presence of a gravi- tational field, a coordinate system can be chosen such that the connection components Γνμσ vanish or appear, and thus a gravitational field appears or disappears given a certain choice of coordinates

The crucial assumption which gives physical content to the \scare{trick} is that the new field is nontrivial, that is, in which one ca cannot set $\Gtmn=0$ everywhere in a finite region by a simple coordinate transformation. The notion of parallel transport provides a criterion to determine if it is possible to construct a coordinate system, with the $\Gamma$'s are zero. If any vector $A^{\mu}$ transported parallel around any closed curve must return to its initial point will be the same vector:

$$
\oint \Gtmn \frac{d A^{\tau}}{d \ap} d \ap=0
$$

The notion idea captures of \scare{curvature}. A connection is curved, if one parallel displaces $A^\tau$ along different paths, one gets, in general, a different vector $A^{\prime \tau}$ at a distant point \citep[028-01-03, 37]{HR}. \cop{Thus, the notion of \scare{sameness} of vectors refers only to neighboring points and can not in general be extended by continuous transfer along curve, to world points finitely apart from one another} \cop{A more convenient criterion for flat space without involving integration uses the tensor}, is to construct the following tensor:

\begin{equation}\label{eq:riemanntensorgamma}
R_{\mu \nu \sigma}^{\tau}(\Gamma)=\frac{\partial \Gamma_{\mu \nu}^{\tau}}{\partial x^{\sigma}}-\frac{\partial \Gamma_{\mu \sigma}^{\tau}}{\partial x^{\nu}}+\Gamma_{\alpha \nu}^{\tau} \Gamma_{\mu \sigma}^{\alpha}-\Gamma_{\alpha \sigma}^{\tau} \Gamma_{\mu\sigma}^{\alpha}\,.	
\end{equation}
%
This tensor Riemann-Christofel tensor \riteg. Levi-Civita and Weyl have had that can be obtained from a geometrical consideration based solely on the law of the affine connection. The manner in which the \Gtmn are expressible in terms of the \gmn plays no role in the derivation. This tensor the basis, of the general theory. From it the Ricci tensor and Ricci scalar. Finally the field equations can be derived by an action principle by setting the Lagrangian $ \sqrt{g}R$.

\begin{figure}
\begin{center}
\includegraphics[scale=0.3, trim = 0mm 0mm 0mm 0mm, clip]{parallelverschiebungtr.png}
\caption{Reichenbach's depiction of the parallel transport a vectors along different curves}
\end{center}
\end{figure}


%If the affine connection is flat the Riemann tensor. In this way one recover the apparatus of the absolute differential calculus, but with a more geometric approach. After these mathematical preliminaries, Reichenbach's-borrowing elements from Wey~1918b, bute~sentiallyfollowing Einstein 1916o (Vol. 6, Doe. 41)-uses variational methods to using $R$ as a scalar quantity.


%These results naturally lead to a generalisation of Riemann's geometry. Instead of starting off from the metrical relation (r) and deriving from this the coefficients I of the affine relation characterised by (2), we proceed from a general affine relation of the type (2) without postulating (1). The search for the



%and the law of energy-momentum conservation (see [pp. 13-17]). For the a?pro~ImatiVe mtegrat10n of the field equations, Einstein follows (with some small impr~vements)El~stem191Sa (Doe. 1), pp. 17-19. The derivation of the exact solution of the field equatiOns for a pomt mass and the perihelion advance of an orbit is taken from Weyl1918b (see (p. 21] and, for somewh~t more detail, [p. 24]). The final pages of these notes, (pp. 21-24], deal with cosmology and combme ele- ments from Einstein 1917b (Vol. 6 Doe. 43) and Einstein 1919a (Doe. 17). . . These lectures also cover the g~neralizationof electrodynamics from special to general relatiVIty ([p.I2]), the equations of motion for frictionless fluids ([p. 14]), and the behavior of rods and clocks Weak gravitational fields ([p. 20]). Instead of starting off from the metrical relation (r) and deriving from this the coefficients of the affine relation characterized by (2), we proceed from a general affine relation of the type (2) without postulating (1). 

\subsection{Weyl Theory}

The Levi-Civita-Weyl approach has a key conceptual advantage. As Einstein will repeatedly point out, it showed that the central concept was the \scare{displacement field} and not the metric. However, the approach had another implication. If one starts with a symmetric metric \gmn the road is marked. The Christoffel symbols are, so to speak, the only possible destination. However, if one defines the displacement \Gtmn independently from the metric \gmn, the Riemannian connection the connection \cref{eg:levicivita} appears only as a special case that has been achieved by introducing a series of conditions\todo{citations}. These conditions are at first sight natural, but by no means necessary. In 1919 Weyl had included the presentation of this development in the new editions of \citetile{Weyl1919} \citep{Weyl1919}. Weyl conceded that the connection must be symmetric. However, as it is well-known, Weyl was bothered by asymmetry comparison of direction of vectors which is path-dependent could not be the comparison of length is not. \qt{in any case in a pure \scare{neargeometrical} drop inadmissible assumption of the possibility of \scare{comparison at distance}: only distances that are located at the same place are}{ohnehin in einer reinen ,,Nahegeometrio" unzulässige Annahme der Möglichkeit des „Fernvergleichs" fallen lassen: nur Strecken, die sich an der gleichen Stele befinden}, that is in vector at the same point in different direction \citep[102]{Weyl1919a}.

%zieht und sich anheischig macht, aus der Weltgeometrie nicht nur die Gravitations-, sondern auch die elektromagnetischen Erscheinungen abzuleiten. Steckt diese Theorie auch gegen wärtig noch in den Kinderschuhen,

\cop{In order to obtain a truly nearby geometry, Weyl introduced metric \scare{connection} or displacement that make the congruent transport (of length) just as path-dependent as parallel transport}. If a vector of length $l$ is displaced from $x_\nu$ to $x_\nu$, it will in general have a new length $l+dl$, so that $dl/l=\phin dx_\nu$. In this way, in addition to the \scare{metric tensor} \gmn, a \scare{metric vector} $\phin$ of the same importance is introduced. As consequence, Weyl obtained a  symmetric affine connection which is more general than \cref{eq:levicivita} \citep[112]{Weyl1919} is then expressed in terms of the metric tensor and a four vector $\varphi_{\mu}$:

\begin{equation*}\label{eq:weylaffine}
\Gtmn = - \christoffel{\mu}{\nu}{\tau} +\frac{1}{2} g_{\mu}^{\tau} x_{\nu}+\frac{1}{2} g_{v}^{\tau} x_{\mu}-\frac{1}{2} g_{\mu \nu} x_\tau
\end{equation*}
%
By pluggin \cref{eq:weylaffine} into \cref{eq:geodesicequation} on btains the geosic equations. \cop{This makes it possible to determine geodesic null lines. The property of geodesic lines. that they are also shortest lines, is dropped in Weyl's geometry, because- the concept of a curve length becomes ~eaningless here.} An equivalent of the Riemann tensor \riteg can be introduced that can be split invariantly into two parts:

$$\bar{R}_{\tau \mu \nu}^{\sigma}=R_{\tau \mu \nu}^{\sigma}-\frac{1}{2} \delta_{\tau}^{\sigma} F_{\mu \nu}$$
%
where \rite is the curvature tensor in Weyl's theory and $\rite$ is the curvature tensor of Riemannian geometry (i.e., the Riemann tensor) add $F\mn$ is related which is the curl of the \phin. Thus, Weyl connections posses two curvatures. The Riemannian direction curvature, responsible for the change of direction under transport around a closed loop. and the \scare{length curvature} for the change of length. 

%where 4\lambda (x)$ is an arbitrary function of the space-time coordinates. where 1(x) is an arbitrary function of the space-time coordinates. This transformation was interpreted as a change of scale or the measuring standard (which, according to this theory, should be chosen at each point of space-time), since ds'2 = \ ds?.


%\cop{Furthermore, because by definition it measures the nonintegrability of length---the amount that the metric changes under transport around a closed loop---its vanishing is then necessary and sufficient condition for the recovery of Riemannian geometry}.


%This assumption is equivalent to postulating the non-vanishing of the homothetic curvature and the conservation of elements a and b. Analytically, this assumption introduces gauge transformation in addition to coordinate transformations. The affine connection is then expressed in terms of the metric tensor and a four vector $\varphi_{\mu}$ tied to the gauge transformations. Thus, to determine the structure of the universe and the field equations, it is necessary to use the ensemble $\left(\mathrm{g}_{\mu \nu}, \varphi_{\mu}\right)$ and not simply $\mathrm{ds}^{2}$.

%where Pijk/ is anti-symmetric in the indices i and i j as well as k and l. Whereas the equations Fik = 0 characterize the absence of an electromagnetic field i.e. a space in which the transfer of magnitude is integrable, one sees from (13) that pi, 0 are the invariant conditions for the absence of a gravitational field i.e. jkl for the parallel transfer of directions to be integrable. Only in Euclidean space is there neither electromagnetism nor gravitation.

%two fundamental metric forms: the quadratic Gia da' dick (gravitation) and the linear $; dri (electricity). The linking of these two ranges of pheno-

This geometrical setting could be used as the starting  The search for such theory usually implies different steps, of which Weyl had introduced a new presentation of the second edition of his texstbook:

\begin{itemize}
\item the first step is to the geometrical field-structure in this case the fundamental variables are \gmn and \phin. That the \gmn are identified with the potentials of the gravitational field because of a \emph{physical fact} the equivalence principles. The \phin could be identified with the potentials electromagnetic field  because of the \emph{mathematical fact} that the $F\mn$ is the curl of the \phin, like in the first two Maxwell equations. 

\item the second step is find the field equations, via the \scare{action principle}. \cop{One constructs a scalar quantity (the action) from the dynamical quantities \gmn and \phin then finds the conditions needed to restrict the scalar to an extremum (a maximum or minimum) with respect to variations in those dynamical quantities. The problem the right action and the right dynamical quantities to produce the desired equations}, that is to recover Einstein and Maxwell field equations.

\item The final step is the comparison with experiment; in particular to see, if in addition if they imply the existence of the electron and other unexplained atomic phenomena. \co{On the other hand the stable solutions of the equations for the "problem of matter". satisfying adequate regularity conditions should lead to a discrete set of solutions depending on some parameter 3. This expectation had a (formal) similarity to a set of "discrete eigenvalues" of an operator, although here the operator was not linear.} \todo{Weyl1919}
\end{itemize}

Point to construct a geometrization of gravitational phenomena, his new theory represented a unified geometrization of both gravitational and electromagnetic phenomena, which were, at that point, the only kind known\footnote{\q{We have realized that physics and geometry coincide with each other and that the world metrics is one, and even the only one, physical reality}}. \q{Untersuchung herausstellen, daß diese Unterscheidung zwischen Geometrie und Physik ein Irrtum ist, da die Physik gar nicht über die Geometrie} \citep{Weyl116}

%If charges are present at all, this constant cannot vanish. If, in addition, it is assumed to be positive, it follows automatically that the curvature of space is positive and that the universe is finite, so that it is unnecessary to. add a special A-term to the gravit~tional equations.

%\cop{As for the gravitational equations· themselves, finally, these are not identical with Einstein's equations, even in the absence of an electromagnetic field (r/>, = 0), as might have been expected from earlier arguments, and they are of higher order than the second.}


%However, the rethorical declaration the goal was far more complicated. 

%to utter contempt. He admired Weyl's theory "*as a chain of ideas" (Doc. 59), but as a theory of physical reality it was to him "fanciful nonsense" (Doc. 294). By including general relativity into the third edition of his textbook on "Space-TimeMatter," Weyl had, according to Einstein, "messed it up" (Doc. 332).

%gemeine werde also leider nicht in G[ottingen] sein konnen. * Weyls Theorie bewundere ich sehr als Ideen-Folge.>| Aber ich glaube nicht, dass sie der Wahrheit näher führt. Das Aufgeben der metrischen Bedeutung des ds scheint mir nicht begründet, zumal man gezwungen wird die Feldgleichungen als Gleichungen vierter Ordnung anzusetzen. 6)

%59. To David Hilbert Berlin 11. VI. 19

%Es ist eine Grundeigenschaft der Naturvorgänge, von der Vorgeschich- te unabhängige Massstäbe und Uhren zuzulassen. Diese aber erlauben es, zwei benachbarten Weltpunkten eine experimentell bestimmbare Zahl ds zuzuordnen, während Weyls Theorie die Nichtexistenz eines derartigen ds zur Voraussetzung hat. Ich bin ganz überzeugt, dass diese Theorie den Thatsachen gegenüber versagen wird.[111 Was die Linienverschiebung (Sonne) anbelangt, so ist vom Experiment aus das letzte Wort sicherlich noch nicht gesprochen. Man muss zuerst die Bogen- lampe durch eine physikalisch einwandfreiere irdische Lichtquelle ersetzen. Sie werden schon sehen, dass die Theorie endlich vollkommene Bestätigung finden wird. Bei den Fixsternen haben sich neuerdings eklatante Bestätigungen (qualita- tiv) ergeben.[ 121Ein Versagen der Verschiebung der Sonnenlinien würde nach mei- ner Überzeugung die ganze Theorie umwerfen. Was Weyls Theorie anbelangt, so sind ja nach ihr nicht die q>v sondern allenfalls das Integral Jq>vdxv für das metri- sche Verhalten der Uhren massgebend, ohne dass dies mit Sicherheit behauptet werden könnte. Denn so weit ist die Theorie nicht durchgeführt, dass das Verhalten eines als "Uhr" auffassbaren Vorganges aus der Theorie deduziert wäre.[ 131 Sie sprechen von Bohrsehen Bahnen, die sich "nicht ändern" sollen. Aber diese Aus- sage hat eben, von der Weylschen Theorie aus gesehen, keinen Sinn, weil der Be- griff der "natürlich gemessenen Längen" aufgegeben ist.-

%78. To Adriaan D. Fokker Luzern 30. VII [1919} 1] Lieber Herr Fokker!




\subsection{The Bad Nauheim meeting of September 1920}

%Im Nov. 1920 wollte ich einen populären Aufsatz über Rel. th. f. d. Umschau schreiben. I Ich kam auf den Gedanken,

Reichenbach met Weyl for the first time at the 86th Assembly of the \german{Versammlung der Gesellschaft Deutscher Naturforscher und Ärzte} in Bad Nauheim in September 1920. In his talk \citet{Weyl1920a} introduced the distinction between \german{Einstellung} and \german{Beharrung} to explain away the discrepancy between the non-Riemannian behavior of the \scare{ideal} time-like vectors implied by his theory and the Riemannian behavior of the \scare{real} clocks that are actually observed. He suggested that atomic clocks might not \emph{preserve} their Bohr\todo{Laue} radius if transported, but \emph{adjust} it every time to some constant field quantity, which he could identify with the constant radius of the spherical curvature of every three-dimensional slice of the world, furnishing a natural unit of length. The fact that all atoms of the same type are exactly identical clearly cannot depend on an initial agreement established in the past, which has been \scare{preserved} since then, even though the atoms had encountered very different physical circumstances. It was more plausible to argue that they \scare{adjust} anew each time to a certain equilibrium value. Thus one might surmise that vectors behave in a non-Riemannian way, whereas atoms used as clocks, which are after all physical systems like any other, appear to have a Riemannian behavior.


%\footnoteh{He suggested that atomic clocks might not \emph{preserve} their Bohr radius if transported, but \emph{adjust} it every time to some constant field quantity. Weyl suggested that the atoms we use as clocks might not preserve their size if transported, but adjust it every time to some constant field quantity, which he could identify with the constant radius of the spherical curvature of every three-dimensional slice of the world, furnishing a natural unit of length. The geometry read off from the behavior of material bodies would appear different from the actual geometry of space-time, because of the \scare{distortion} due to the mechanism of the adjustment. Two identical \scare{classic} atomic systems with different prehistories would probably differ in some small detail due to their interaction with the environment, and their spectral lines would be slightly shifted, so that classically, a spiraling charge should emit light of all colors. Emerging quantum theory had already made clear that the spectral identity of atoms revealed by experience cannot be explained in this framework. The fact that all atoms of the same type are exactly identical clearly cannot depend on an initial agreement established in the past, which has been \scare{preserved} since then, even though the atoms had encountered very different physical circumstances. It was more plausible to argue that they \scare{adjust} anew each time to a certain equilibrium value}. 

%The size of an electron is determined by adjustment in proportion to the radius of curvature of the world, and not by persistence of anything in its past history. This is the view taken in § 66. and we have seen that it has great value in affording

In the discussion followed, commenting on Weyl's talk, Einstein reiterated his critiques. He pointed out once again that the \q{arrangement of \textins{his} conceptual system,} \q{it has become decisive \origins{massgebend} to bring elementary experiences into the language of signs \origins{Zeichensprache}} \citep[650]{Einstein1920c}. For Einstein, \q{temporal-spatial intervals are physically defined with the help of measuring rods and clocks}, under the assumption that \q{their equality is empirically independent of their prehistory} \citep[650]{Einstein1920c}. Einstein insisted that precisely upon this assumption rests \q{the possibility of coordinating \origins{zuzuordnen} a number $ds$ to two neighboring world points}; if this were impossible, general relativity would be robbed of \q{its most solid empirical support and possibilities of confirmation} \citep[650]{Einstein1920c}.

% The time or space intervals have a physical basis only if there is some actual or possible physical process that has a length or a duration shorter or equal to the space or time interval in question. A distance smaller than the electron would be physically meaningless since there is no physical process that could realize such an interval. The attempt to define the electromagnetic field or gravitational field in the interior of elementary particle to account for their stability should be rejected on epistemological grounds.


However, Einstein was immediately forced to open the possibility of a different stance by replying to a comment of Pauli's. The goal of Weyl theory was to construct a field theory of matter, in which the electron is a region of the field in which the field straight are enormously concentrated. However, the field strength in the interior of the electron is meaningless because there is no smaller test particle than the electron; \q{one could claim something similar concerning spatial measurements, \myemph{since there are no infinitely small measuring-rods}} \citep[650]{Einstein1920c}. If e.g. the theory claims that geometry of is non-Euclidean within elementary particle, there is no way to check this prediction just like we can check that the geometry around the sun is non-Euclidean. Einstein replied to Pauli that \q{with the increasing refinement of the system of scientific concepts, the manner and procedure of associating the concepts with experiences becomes increasingly more complicated} \citep[650]{Einstein1920c}. In particular, he recognized that in cases such as that of the continuum theories, \q{one finds that a definite experience cannot be associated any longer with a concept} \citep[650]{Einstein1920c}. According to Einstein, there the physicists is at a crossroads: one can abandon \scare{continuum theories} for the sake of Pauli's observability criterion, or replace such a \q{system of associating concepts \textins{with experiences} with a more complicated one} \citep[650]{Einstein1920c}. A decision as to which alternative is more suitable, Einstein pointed out, can only be given on the basis of pragmatic reasons \citep[650]{Einstein1920c}. Indeed, it would soon become clear that Einstein ultimately opted for the second choice.  One can already glimpse the main lines of Einstein's in his contributions to the the discussion which followed Max von \citets{Laue1920}'s Bad Nauheim paper on the gravitational redshift\footnote{Laue showed that the coordinate interval $d\vartheta$ measured by an atom on the sun is transmitted unchanged by light signals (at least in a static gravitational field\footnote{In the general case, the number of vibrations of an atom transmitted by light signals is coordinate dependent}), so that the redshift emerges by confronting the frequency of such signals with those of an atom of the same type at rest measuring the proper time $d\tau$}, in which Einstein did not hesitate to admit that \q{[it] is a logical shortcoming of the theory of relativity in its present form to be forced to introduce measuring rods and clocks \myemph{separately instead of being able to construct them as solutions to differential equations}} \citep[Einstein's reply to][662\me]{Laue1920}. Thus, Einstein now openly admitted that it would have been logically or epistemologically preferable if the field equations of the theory had suitable solutions corresponding to particles, from which in principle the stability of a more complicated, bulky configuration of matter could be reconstructed, including rod- and clock-like structures. In this way the necessity of coordinating the geometrical/kinematical structure of the theory separately from the rest in terms of rods-and-clocks behavior would fall and with it also Pauli's objection that such definition is impossible within elementary particles.

Thus Einstein vacillated between to different epistemological stances. On the one hand he considered that \textit{sub specie temporis}, thatthe  \q{invariant $ds$ is connected with observable facts [measured using rods and clocks], just like it happened to the fundamental concepts of Maxwell theories through Heinrich Hertz} \CPAE{??}{??}. However, he continued, he conceded that this empistemological models fails in the infisimaly small and in the finitesimaly large, where no such \rac might be at our disposale. Thus \emph{sub specie aeterni} \rac should do not play the part of irreducible elements, but that of composite [atomic] structures, that comes of out of the theory at the end. Thus, provisionally that geometry can be tested empirically separately from the rest of physics, indeed \gr seems to this the \gmn as measured with \rac with respect to a given corodiante syste, However, in the general case only geoemtry and physocs togehter. One would chose a certain geometrical strucutre, that leed to right field equaitons. The latter would have soltuons atomic struc, wthat would serve as rac.  geometry cannot be tested separately from ultimately only the theory as a whole $G+P$. 

Reichennach and most logical seems to have missed that adress Pauli's and Weyl's crossfire. It was also to open the possibility of his own \uftp, read the lecture simply of between geometry and physsics, whereas that this separation for the sake of \uftp. Indeed, in the following months Einstein will soon first unified field theory conformal. Weyl has accepted the existence of transportable \rac but denied that they preseved their lengths. That \q{mit WEYL auf die Voraussetzung 11, sondern auch auf die Voraussetzung H von der Existenz übertragbarer Maßstäbe (bzw. Uhren) von vornherein verzichtet. Im folgenden soll null gezeigt werden, daß}. In the new theory only the $ds=0$ has a physical meaning. This approach an important stepping stone to will soon abandon the very idea that the notion of parallel transport of vectors, which however have no physica meaning at all, to Reichenbach's dismay.

%That Weyl's theory has the pretence, but . Then was to renouce complelty as Einstien  sugested in March 1921, to avoid the use of \rac alltogeher. That this approach the separation between geometry could be tested separately from physics, then search for the field equations ... then will be considered a proper geometrical strucutre. This was probably the of the famous $G+P$ formlula. In the special and as in general to a  physocal emaing to the geometrical strucutre of $\gmn$ indepednen of the field equaitons. Indeed, \rac just like in the where measure by particel after made some predicitons. The that this model might have be abandoend. The ... garatnede of the theoyr as whole. In partiuclar. In March 1927 Einstien will prefere to set up a theory in which \rac where not part of the fundamental strucutre

%This complicated set of is reflected, which were to interpret. Indeed, there Einstein could precisley, gometry is Riemannain since behave  and the question whether in cosmology and infinte particels this assumption be dropped. Then the choicen becuase of its role in elementary particles, neither the could be defined in this way. The coordination a potential of a sever. Einstien will turned to the second method whe it will be conformatal. 

\label{Coordination}
After serving in World~War~\rom{1}, Reichenbach attended Einstein's lectures on special and general relativity in Berlin. We posses three sets of Reichenbachs of undated notes (HR-028-01-04, HR-028-01-03, HR-028-01-01). A set notes seems to corresponds are very similar to Einstein's own notes to the Einstein's lecture on spring term 1919 \citep{Einstein1919c}\footnote{Further information about Einstein as an academic teacher, see Vol. 3, the editorial note, "Einstein's Lecture Notes,"pp. 3-10, and for a survey of Einstein's academic courses, see Vol. 3, Appendix B.}. \cop{In the lectures follow the corresponding sections Einstein's previous published presentations  of \rt \citepp{Einstein1916}{Einstein1914a}.} However, both Reichenbach's and Einstein's lecture notes show in the 1919 lectures Einstein also used for the first time new interpretation of the curvature in terms of the parallel displacement introduced by Tullio \citet{Levi-Civita1916} and applied to \rt by Hermann \citet{Weyl1918}. Both names are mentioned explicitly. 

In the original presentation of \rt, Einstein started the metric $\gmn$ a coordinate indent criterion of the equality distance $ds$ of two nearby points  with coordinates $x_\nu$ and $x_\nu+dx_\nu$, the famous formula for the line element $ds^2=$. In the lectures, Einstein showed how one treat $dx_\nu$ as a special case of contravariant vector $A^\tau$. The so called affine connection $$dA^\tau = \Gtmn A^\mu dx_\nu$$ provides a coordinate-independent criterion for the parallelism of two vectors at neighboring points $x_\nu$ and $x+dx_\nu$ are equal and parallel. It can be show, that the notion of \scare{sameness} cannot be in general be extended to distant points by continuous transfer along curve. If one parallel displaces along different paths, one gets, in general, a different vector at a distant point \citep[028-01-03, 37]{HR}. In this way one could recover the notion of \scare{curvature} without any reference to the metric. The metric could be introduced at later stage by associating with any contravariant vector $A^\tau$ with a length $l^2=\gmn A^\mu A^\nu$. It was natural to assume that the length of vectors does not change under parallel transport. By imposing this condition Levi-Civita was able to recover the content Riemannian geometry. The line element $ds$ is nothing but the length $l$ of the contra-variant vector $dx^\nu$. 

Although not mention of this point is made in the notes, from discussions with Einstein, Reichenbach might have become immediately aware that \citep{Weyl1918a,Weyl1919a} was bothered by asymmetry comparison of direction of vectors which is path-dependent could not be the comparison their lengths was distant-geometrical. To overcome this \scare{mathematical injustice}, Weyl a introduced \scare{metric connection} alongside the \scare{affine connection}. If a vector of length $l$ is displaced from $x_\nu$ to $x_\nu+dx_\nu$, it will in general have a new length $l+dl$, so that $dl/l=\phin dx_\nu$. In this way, in addition to the \scare{metric tensor} \gmn, a \scare{metric vector} $\phin$ is introduced:

\begin{frame}{\secname}\setcounter{footnote}{0}
ds^2
\end{frame}

The \gmn are identified with the potentials of the gravitational field because of a \emph{physical fact} the equivalence principles. Weyl found natural to intepret \phin as the four-potential of electromagnetic field  because of the \emph{mathematical fact} that the tensor $F\mn$ is the curl of the \phin, and in turn constraint equivalent to Maxwell equations. Via the action principles from \gmn and \phin Weyl hoped was able to recover Maxwell and a set gravitational field equations. Weyl was initially confident  the stable solutions of the equations corresponding to elementary particles, allowing the ultimate comparison with experience.

Weyl could then conclude that just like general relativity represented a geometrization of gravitational phenomena, Weyl's theory represented a unified geometrization of both gravitational and electromagnetic phenomena, which were, at that time, the only kind known. Ultimately, matter itself would have become epiphenomenon of the \scare{world metrics}. With some rhetorical exaggeration, Weyl did not hesitate to declare \q{Der Traum des Descartes von einer rein geometrischen Physik} had be fulfilled. Concluding the 1919 edition of the book, Weyl could declare that \q{physics and geometry coincide with each other}. The tendency of physicalizing geometry that have prevailed leading protagonists of the 19th century from Gauss to Helmholtz seemed to superseded have of geometrizing physics that run from Riemann to Einstein: \q{geometry has not been physics but physics has become geometry} \citep[263]{Weyl1919}. 


%Ihre Gesetze werden ebensowenig in der Wirklichkeit jemals verletzt, wie es Wahrheiten gibt, die mit der Logik nicht im Einklang sind; aber über das inhaltlich-Wesen- hafte dieser Wirklichkeit machen sie nichts aus, der Grund der Wirklich- keit wird in ihnen nicht erfaßt

%ir hatten erkannt, daß Physik und Geometrie schließlich zusammenfallen, daß die Weltmetrik eine, ja viel- mehr die physikalische Realität ist. Aber letzten Endes erscheint so diese ganze physikalische Realität doch als eine bloße Form ; nicht die Geo- metrie ist zur Physik, sondern die Physik zur Geometrie geworden. Wir haben nicht mehr wie nach alter Anschauung einen leeren Raum als die Form, in deren Rahmen sich eine Materie von gediegener Wirklichkeit konstituiert, und als den Schauplatz, auf welchem sich die wirklichen Geschehnisse, das sind dieser Materie Veränderungen abspielen; sondern die gesamte physische Welt ist zur Form geworden, der aus ganz andern Bezirken als denen der Physis ihr Inhalt zuwächst. 

%In his eyes, physics seemed to be transformed to a purely formal status and was absorbed by geometry. Matter had seemingly become an epiphenomenon of the "world metrics" which started to acquire a slightly mystical flavour





%The great success which Einstein had attained with his geometrical interpretation of gravitation, led Weyl to believe that similar success might be obtained from a geometrical interpretation of electricity. Since the tensor of Riemannian space was already appropriated by gravitation, he constructed a wider geometrical frame which contained some unassigned geometrical elements which he could ascribe to electricity. 
%
%Before we \st{criticize} examine this idea, let us first review the actual achievements of the geometrical interpretation of gravitation. The field of force of gravitation affects the behavior of measuring instruments. Besides serving in their customary capacity of determining the geometry of space and time, they serve, therefore, also as indicators of the gravitational field. The geometrical interpretation of gravitation is consequently an expression of a real situation; namely, of the actual effect of gravitation on measuring rods and clocks.

%The geometrical interpretation of gravitation is merely the visual cloak in which the factual assertion is dressed. It would be a mistake to confuse the cloak with the body which it covers; rather, we may infer the shape of the body from the shape of the cloak which it wears. After all, only the body is the object of interest in physics. If we want to do the same for electricity, we must search for a similar physical fact which relates the electrical field to the behavior of measuring instruments, thus permitting a geometrical expression of the electrical field. However, the fundamental fact which would correspond to the principle of equivalence is lacking.



%Thus, Weyl's displacement space is not suited to describe the behavior of \rac and charged mass points in a combined electrical and gravitational field. \q{This means that we have found a cloak in which we can dress the new theory, but we do not have the body that this new cloak would fit} \rzlap{353}{493}. What alternatives do we have at our disposal? According to Reichenbach, physicists had tried to \q{forgo \textelp{} such a realization of the process of displacement} \rzlap{371}{519}.
%
%
%In Reichenbach's parlance, it is necessary to introduce a coordinative definition of the operation of displacement. Weyl's geometry is a balanced space, in which the comparison of lengths is defined, although not at a distance. Thus, it is natural to assume that the length of vectors can be measured with \rac. Weyl uses \rac as indicators of the gravitational field and, at the same time, indicators of the electromagnetic field.
%
%
%However, this was not conventional wisdom. \q{The great success, which Einstein had attained with his geometrical interpretation of gravitation, led Weyl to believe that similar success might be obtained from a geometrical interpretation of electricity} \rzlap{352}{491}. Just after \gr was accepted by the physics community, the search for a suitable geometrical cloak that could cover the naked body of the electromagnetic field began. To this end, one needed something analogous to the equivalence principle, a physical fact that relates the electrical field to the behavior of measuring instruments. \q{However, the fundamental fact which would correspond to the principle of equivalence is lacking} \rzlap{354}{493}. Thus, physicists had to proceed more speculatively.
%
%
%
%If reads the conclusion the long appendix has not been translated, but clearly the fundametal message of the book. That onserved into physicsc, but into geometry. The developed on the had ... ... So far, we have always that a simple physical realization must be given for the process of displacement. In our example we ourselves gave such a realization and we obtained, in this way, an actual geometrical interpretation of electricity. Attempts which were made by Weyl, Eddington and Einstein, on the other hand, renounced such a realization of the process of displacement. It is generally believed that such \scare{tangible} realizations does not lead to the desired field equations. 





%This was a battle on what was \gr and what was continue the key insight of general relativity. The argument was mean to counter who declared the \uftp as mislieading without the equivalence that garaneed between a connection with gravitationa field and \rac lighr rays, that is geometrical measuring instruments. 

%and the theory did not turn physics into geometry; Einstien to show that difference between geomery and rest of mathematics; the unificaiton program

%Objections against the very project of geometrization were raised my some at that time. Pauli e.g. has clearly a similar point to Reichenbach only graphical representations, however there was no motivation for geometrizing the electromangetic field wihout the equivalence principle. Einstein, in his willingels, that geometrization was not point. As we shall see for Einstein the point was the unification of the two fields. 


%\q{The general theory of relativity by no means turns physics into mathematics. Quite the opposite: it brings about the recognition of a physical problem of geometry} \q{development of the theory of relativity into a world geometry}. 

%The abstract of this presentation was published under the title \citetitle{Reichenbach1926d} \citep{Reichenbach1926d}. Thus, he concluded, providing a geometrical interpretation of a physical field is not in itself a physical achievement: \q{the geometrical interpretation is only a different parlance, which does not entail anything new physically}. \citep[25; my emphasis]{Reichenbach1926d}. As one can infer that at this point the pages were already in of developmen. The reason for the success is that theory had indeed effects on gravitational field, not that has been reduced to geometry. That the was uneceasry, there was not reason to pursue the proect. .That difference between geometry and physics, that on the very oppsite, the gola that there was non difference between geometry and the rest mathematics. Thus, could be further pursed, of unification and not of geometrya,







%The pages of this, that geometrization was but has an effect on geometry geometry is like cloak. The main goal. Geometry was different from physics, and maintial the difference. Indeed, that has effect on geometry, but it is not geometry. Einstein wanted to that geometry was not different from the rest of mathematics. The goal of introducing was to unify the two fields, into a single Lagrangian. Neither the of these variables was not essential, and only at the end after having integrated the field equations.

%eses, and since the theory of relativity has revealed the physical character of geometry as a science of real space, we can no longer doubt that these, too, are basically divisions of physics and that only the division of labor justifies the existing separation of specialties. We may, then, say that there are only


%Mathematics is the intellectual tool of physics; it teaches what is permissible and what is forbidden, but never what is physically co"ect.


%
%
%
%Reichenbach revealed that what he wanted to achieve was a geometrical interpretation of a physical field \scare{in \myemph{the same sense as gravitation}} in Einstein's theory, i.e., one that was \emph{just as good} as that attained by general relativity. The geometrical operation of displacement has a physical interpretation in Reichenbach's theory, just like the $ds$ does in general relativity Thus, Reichenbach claims to have provided not just a successful \scare{geometrical interpretation} of the electromagnetic field, but an interpretation that was of the same \scare{quality} as the one \gr provided for the gravitational field. However, this was Reichenbach's point: the theory was not a successful physical theory like general relativity. Thus, he concluded, providing a geometrical interpretation of a physical field is not in itself a physical achievement: \q{the geometrical interpretation is only a different parlance, which does not entail anything new physically}. [][25; my emphasis][Reichenbach1926d]. 



%https://arxiv.org/pdf/1802.00492.pdf
%http://www.physics.ntua.gr/ModifiedGravity2018/Talks/Iosifidis.pdf
%http://www.weylmann.com/weyltheory.pdf
%https://inis.iaea.org/collection/NCLCollectionStore/_Public/18/010/18010695.pdf?r=1&r=1



%that is gravitationa as efect on geometrical measureing insturmes \rac, light rays\etc, that could be isolated. On the opposite, to use geometrica withou, mathematical simplicity, itsefl Reichenbach hoped that his critical epistemological reflections could have served, so to speak, to tie physicists to the the mast of empiricism, so that they could resist to \q{the sirens' enchantment \origins{Sirenenzauber} of a unified field theory} \citep[373]{Reichenbach1928}.


%Some weeks later, in \datem{01}{12}{1927}, Reichenbach wrote to Einstein that Paul Hinneberg, the editor of the \citejournal{Einstein1928d} had told him that Einstein intended to write a review of his forthcoming book, \citetitle{Reichenbach1928}. Reichenbach sent him the galley proofs and also added that he would send an \Ap in the coming days \letteraeap{Einstein}{Reichenbach}{1}{12}{1927}[20-090]\todo{abs 295?}. Einstein's review appeared in the first 1928 issue of the \citejournal{Einstein1928d} \citep{Einstein1928d}. 




%The review was published in Spring 1928 in French \citep{Einstein1928b}. 



% Historical reasons aside, there was no real ground to define \gmn, the gravitational field, as a geometrical field, and, say, the \Fmn, the electromagnetic field, as a non-geometrical field. 



%Only its success can decide Its correctness, for it is purely a \emph{matter of experience}, whether the way to a simple and natural geometry also leads to an approximation to reality\footnote{Weyl in particular through his concept of \scare{Gauge-invariant}, has developed a method for narrowing the choice among the available equations. This is, indeed, a rigorously formulated principle which is more than a mere guide to to geometrical feeling. Whether this principle is correct is, of course, a purely empirical question}. It is noteworthy that contemporary discussions of these problems are filled with concepts like \scare{most natural assumption}, \scare{simplest invariant}, etc. ;


%However, the success of Einstein's geometrical interpretation of geometrical field was based, on somewhat expcetional circumstances:
%
%\begin{itemize}
%\item \emph{Physically motivated} It was based the empirical fact of the equality of inertial and gravitational mass implies that free-fall is locally indistinguishable from inertial motion. The equivalence principle is the \emph{physical hypothesis} that this indistinguishability can be extended to all non-mechanical phenomena \rzlp{264}{229f.}. Because of the equivalence principle, gravitation is a \emph{universal force} that cannot be neutralized or shielded. Thus, there is no way to separate the geometrical measuring instruments that are not affected by the field (\rac, light rays, force-free particles) from the dynamical ones that react to the field (charged particles). Thus, the geometrical measuring instruments became at once indicators of the gravitational field.
%
%\item \emph{heuristically powerful} Such a geometrical interpretation accounts for old inconsistencies in Newton's theory concerning the irregularities of Mercury's orbit motion and allows for new predictions like a more pronounced deflection of light by the Sun. Measurements carried out with real physical systems, \rac, light rays, free-falling particles\etc, seem to have confirmed the theory's predictions. Thus, in the case of \gr, the \emph{geometrical interpretation} had indeed been instrumental in delivering new testable resulsts providing a smooth interpolation within a domain of observations. 
%\end{itemize}
%
%
%Neither of this circumstances were replicated in the previous theories. The geometrization was not physically motivated. As Reichenbach's it was ... however, it is not itself heuritically. The project of geometrizing the must should rehotu \q{The many ruins along this road urgently suggest that solutions should be sought in an entirely different direction}.   

%\qt{the degree of formal speculation, the slender empirical basis, the boldness in theoretical construction, and finally the fundamental reliance on the uniformity of the secrets of natural law and their accessibility to the speculative intellect}{spekulativ-formalistische Zug, die Schmalheit der Erfahrungsbasis[,] die Kühnheit der theoretischen Konstruktion, das ihr zugrunde liegende Vertrauen in die Einheitlichkeit und die Durchdringbarkeit der Geheimnisse der Naturgesetzlichkeit durch die spekulative Vernunft} \citep[114]{Einstein1930h}. This \q{speculative method}, Einstein claimed, was the same that lead to to success of \gr: \qt{Which are the simplest formal structures that can be attributed to a four-dimensional continuum, and which are the simplest laws that may be conceived to govern these structures?}{Welches sind die einfachsten und natürlichsten Bedingungen, welchen ein Kontinuum der skizzierten Art unterworfen werden kann? Die Beantwortung dieser Frage, welche ich in einer neuen Arbeit [7] versucht habe, liefert einheitliche Feldgesetze für Gravitation und Elektromagnetismus} \citep[115]{Einstein1930h}. In trying to defend this epistemological stance, Einstein was not afraid to side with \qt{Meyerson in his brilliant studies on the theory of knowledge}, who had emphasized the \scare{Hegelian} nature of such enterprise, \qt{without thereby implying the censure which a physicist would read into this}{geistreiche Erkenntnisstheoretiker Meyerson die geistige Einstellung der Relativitats-Theoretiker mit derjenigen Descartes und sogar Hegels verglichen ohne indes mit jenem Vergleich jenen Tadel zu verbinden, den das Ohr eines Physikers haturgemass heraushoren wird} \citep[115]{Einstein1930h}. 


%Reichenbach's from Kantianism to a sort of convetionalism was completed. Reichenbach had good reason to quote, Reichenbach also mentioned the letter that Einstein to further support this claim\todo{netter}. For our purposes that entails also a long, a more balanced review of Weyl theory\footnote{which was surprisingly excluded from the translations of this writing in the 1970s}. The merit of relativity that Euclidean geometry, was and that the choice among geometries is ultimately conventional and depend on which rigid, although this convetion is limited, by  The convention can be fixed by eliminating what we later will called forces of type $X$, that he will later call \scare{universal forces}.


% \q{he term 'convention' overemphasizes he arbitrary elements in the principles of knowledge}

%\q{not only to detect the arbitrary principles of knowledge, but also to determine the totality of admissible combinations.}. he presentation of this issue in my book (47, pp. 27-8 [1920f]) is not quite correct; I have given a clearer exposition in (52 [1921cl).
%
%\q{Er unterscheidet sodann die rein begriffliche »axiomatische Geometrie« von der »praktischen Geometrie«, d. i. der auf Dinge der Wirklichkeit angewandten Geometri}
%
%\cop{diese Vorschrift entsteht erst nach Festlegung der physikalischen Gesetze (dem P der Einsteinsehen Formel); und man kann auch die Maßbestimmung verändern, wenn man nur die Physik entsprechend ändert. Aber der Zusammenhang dieser Aende- rungen drückt eine invariante Tatsache aus}
%
%\q{Zwar ist nach ihr die Wahl der Geometrie willkürlich; aber sie ist nicht mehr willkürlich, wenn man die Festsetzung getroffen hat, daß die starren Körper die Kongruenz definieren sollen. }


%Reichenbach, Hans, 'Erwiderung auf H. Dinglers Kritik an der Relativitatstheorie'. [1921c).




%Es wurde oben ausgeführt, daß die Kongruenz zweier Strecken durch Transport eines natürlichen Maßstabs definiert werden kann; aber das ist natürlich nur eine D e finit i o n. Man kann auch anders definieren, z. B. einen Maßstab nach zweimaligem Anein- anderlegen so groß nennen, nach dreimaligem Aneinanderlegen 1/ 3 so groß usw. Man erhält dann eine Riemannsche Geometrie von anderer Maßbestimmung. Die »Veränderung« des Maßstabs kann man dabei als Wirkung einer Kraft deuten, die auf diese Weise »hinzudefiniert« wird. Je nach der Wahl des zusätzlichen Kraftfeldes

%Jedoch ist zu beachten, daß diese ganze Klasse von Geome- trien doch wieder nicht willkürlich ist, sondern auf der Geltung eines Axioms beruht, das einen empirischen Tatbestand bezeichnet; es ist die Annahme: zwei natürliche Maßstäbe, die sich einmal zur Deckung bringen lassen, lassen sich auch nach dem Transport auf verschiedenen Wegen wieder zur Deckung bringen. In der Geltung dieses Axioms - wir wollen es das Axiom der Riemannklasse nennen- liegt die invariante Charakterisierung der nach Einstein noch :mög- lichen Geometrien



%%%[6]Tn Reichenbach's review, his own philosophical work is discussed in Reichenbach 1922, pp. 362-366. Reichenbach was working on an axiomatic analysis of relativity theory, a full account of which was published in 1924 but was "begun in the fall of 1920 and essentially completed in March 1923" ("im Herbst 1920 begonnen und im März 1923 im wesentlichen abgeschlossen"*); see Reichen
%1021 Dortios ‡ocellechoft





%By the time. Reichenbach has now abandoned his Kantianism, and to develop a form conventionalism to translate in a form of geometrical empiricism.  Reichenach that this separation was in the wrok as he have leard from personal conservations.

%Man darf eine Darstellung der relativistischen Philosophie nicht abschließen, ohne der wichtigen Erweiterung zu gedenken, die vor 3 Jahren Weyl dem Raumproblem zuteil werden ließ. Denn obgleich es sich hier zunächst um die Aufstellung einer mathemati- schen Theorie handelt, ist sie von ähnlicher philosophischer Bedeu- tung wie die Riemannsche Verallgemeinerung der euklidischen Geometrie, und darum unabhängig von aller Anwendung auf die Physik eine Erweiterung unseres philosophischen Wissens vom Raume. Die grosse Entdeckung Weyls besteht darin, daß er einen allgemeineren Mannigfaltigkeitstypus aufdeckte, von dem auch der Riemannsche Raum nur ein Spezialfall is

%The class (a) of Riemannian geometries is fixed by the axiom that lengths do not depend on their prehistory. \q{zwei natürliche Maßstäbe, die sich einmal zur Deckung bringen lassen, lassen sich auch nach dem Transport auf verschiedenen Wegen wieder zur Deckung bringen}. The (b) choice among geometries is ultimately conventional, that we consider rigid. The convention can be fixed by eliminating what we later will called forces of type $X$. After that the question whether is euclidean or not is empirical question, that can be answered by carefully by shielding from differential forces. Thus the sum, $G+P$ ultimately fixes the convention of (b). it may well also have Euclidean relations with the field $X = 0$, but this is a point which we can never know a priori. The merit of Weyl to have shown that the assumption (a) was not necessary. The first way to show it, in two ways Reichenbach seems to espouses that idea of the two versions of Weyl theory:



%It was however, most of all Einstein's \citetle{Einstein1921} became for Reichenbach of the separation between geometry and physics, and on the other hand to avoid the would. On the one hand, on the other he limti to the formula $G+P$.
%
% \cop{For instance, in a thermostatic universe with heterogeneous temperature, iron rods and copper rods form two distinct classes of approximately rigid rods}. The $G+P$ formula, offer avoid a completely arbitrary. The convention can be fixed by eliminating what we later will called forces of type $X$, that he will later call \scare{universal forces}. However, the of  class if coincidence can be obtained in one place between a pair of points of one rod and a pair of points of the other, this coincidence will be possible at any other place and time, no matter how their prehistory might be.

%Jedoch ist zu beachten, daß diese ganze Klasse von Geome- trien doch wieder nicht willkürlich ist, sondern auf der Geltung eines Axioms beruht, das einen empirischen Tatbestand bezeichnet; es ist die Annahme: zwei natürliche Maßstäbe, die sich einmal zur Deckung bringen lassen, lassen sich auch nach dem Transport auf verschiedenen Wegen wieder zur Deckung bringen. In der Geltung dieses Axioms - wir wollen es das Axiom der Riemannklasse nennen- liegt die invariante Charakterisierung der nach Einstein noch :mög- lichen Geometrien.




%An epistemological analysis can tell us only whether a chosen method is \emph{permissible} or not. Only the physical instinct, whose content lies completely outside the realm of epistemological criticism, can judge, for t the time being, whet\hook{h}er it will lead us to the physical goal we have described. That the message not absorbed into geometry, but lowered geometry into the to physics. 



%The \S49 of the \Ap was meant to show that this was not the case. Even a geometrizaiton in thse $A$ bring anithig. In this way, hoed to save \q{the sirens' enchantment \origins{Sirenenzauber} of a unified field theory} \citep[373]{Reichenbach1928}.





% with the inclusion of the \Ap that the book that to denouce this misunderstaind. \q{It is not the theory of gravitation that becomes geometry, but it is geometry that becomes an expression of the gravitational field}. If this true the entire project \ufpr of geometrizaiton the electromangeti is baed on a misudenrstand.   In this way, hoed to save \q{the sirens' enchantment \origins{Sirenenzauber} of a unified field theory} \citep[373]{Reichenbach1928}.


% \cop{The field of force of gravitation affects the behavior of measuring instruments.  Besides serving in their customary capacity of determining the geometry of space and time, they serve, therefore, also as indicators of the gravitational field. The geometrical interpretation of gravitation is consequently an expression of a real situation; namely, of the actual effect of gravitation on measuring rods and clocks}, and in particula in the same way.  \q{The geometrical interpretation of gravitation is merely the visual cloak in which the factual assertion is dressed. It would be a mistake to confuse the cloak with the body which it covers; rather, we may infer the shape of the body from the shape of the cloak which it wears. After all, only the body is the object of interest in physics}. 





%Because of the equivalence principle, gravitation is a \emph{universal force} that cannot be neutralized or shielded, that uss all also the \rac became indicator of the gravitational field
%As Reichenbach pointed out, according to \gr, the universal effect of gravitation on all kinds of measuring instruments defines a \emph{single geometry}, an, in general, non-flat Riemannian geometry. 



%\q{In this respect, we may say that gravitation is \oemph{geometrized}} \rzlp{294\oe}{256}. However this conclusion would be misleading. \q{It is not the theory of gravitation that becomes geometry, but it is geometry that becomes an expression of the gravitational field}.  ... 




%Thus, Reichenbach insisted that there two different philosophical issues at stake that should not be confused:
%
%\begin{itemize}
%\item Riemann, Helmholtz, and Poincar\'e introduced the problem of the \emph{coordinative definition} in the philosophy of geometry \rzlp{**}{257}. \Gr continued in this tradition. Once one defines $ds\pm =1$, $ds=0$ in terms of \rac and light rays, the geometry of \st can be ascertained empirically; it is a branch of physics that can be true or false.
%
%%Conceptual definition of $ds\pm =1$, $ds=0$ in integrated by by light rays, there is a coordinative definition that \rac.
%
%\item Einstein introduced the problem of a \emph{scientific explanation} of physical geometry, which finds its mathematical solution in the field equations \cref{eq:einsteinfieldequations}. The gravitational field has an effect on \rac and light rays, that is comparable to that of any other field of force, if not for the fact that it is a universal effect \rzlp{**}{256}.
%\end{itemize}

%The fact hat the \Ap has not been included in the final of the paper that a fundamental topic of the book. Was written in therms Riemann, Helmholtz, and Poincar\'e introduced the problem of the \emph{coordinative definition} in the philosophy of geometry \rzlp{**}{257}. \Gr continued in this tradition. Once one defines $ds\pm =1$, $ds=0$ in terms of \rac and light rays, the geometry of \st can be ascertained empirically; it is a branch of physics that can be true or false. Einstein introduced the problem of a \emph{scientific explanation} of physical geometry, which finds its mathematical solution in the field equations \cref{eq:einsteinfieldequations}. The gravitational field has an effect on \rac and light rays, that is comparable to that of any other field of force, if not for the fact that it is a universal effect \rzlp{**}{256}. 

%In this sense the has an effect on gemetry but it is not geometry, howeer that gravitaiona is  One starts for the coordinations, the fact that have a Riemannian behavior. One can start if one fails there are two different intepretaition. For a given temperature distribution over space, the metric depends on the kind of rod used, on different geometries Since there are other things which are not influenced in a similar manner to the little rods (or perhaps not at all) by the temperature of the table, in a thermostatic universe with heterogeneous temperature, iron rods and copper rods form two distinct classes of approximately rigid rods, different \emph{different geometries}. 

%As Reichenbach pointed out, according to \gr, the universal effect of gravitation on all kinds of measuring instruments defines a \emph{single geometry}, an, in general, non-flat Riemannian geometry. We do not speak of deformation of our measuring instruments \q{produced by the gravitational field}, but we regard \q{the measuring instruments as \scare{free from deforming forces} in spite of the gravitational effects} \rzlp{294}{256}.  \q{In this respect, we may say that gravitation is \oemph{geometrized}} \rzlp{294\oe}{256}. However this conclusion would be misleading. \q{It is not the theory of gravitation that becomes geometry, but it is geometry that becomes an expression of the gravitational field}. \cop{The field of force of gravitation affects the behavior of measuring instruments. 
%
%
%The reason of this success have been however misunderstood, however the reason of this success has been misunderstood, then they ahve confusde th clocka with that finding the cloak was the very was sufficient to reval it sehaps:
%
%\begin{itemize}
%\item As have seen, \q{The great success, which Einstein had attained with his geometrical interpretation of gravitation, led Weyl to believe that similar success might be obtained from a geometrical interpretation of electricity} \rzlap{352}{491}. If we want to do the same for electricity, we must search for a similar physical fact which relates the electrical field to the behavior of measuring instruments, thus permitting a geometrical expression of the electrical field \q{However, the fundamental fact which would correspond to the principle of equivalence is lacking}. Thus Weyl had to proced in more speculative. Following the analogy with relativity, it is natural to assume that the length of vectors can be measured with \rac. Weyl uses \rac as indicators of the gravitational field and, at the same time, indicators of the electromagnetic field. In the absence of the electromagnetic field, the \rac behave in a Riemannian way. Thus, \q{the Weylian space now constitutes the natural cloak for the field, which is composed of electricity and gravitation} \rzlap{354}{494}. However, however, it is precisely this behavior which does occur in reality \q{This means that we have found a cloak in which we can dress the new theory, but we do not have the body \hook{which} this new cloak would fit} 
%
%\item According to Reichenbach, physicists had tried follow, \q{Attempts which were made by Weyl, Eddington and Einstein, on the other hand, renounced such a realization of the process of displacement. It is generally believed that such \scare{tangible} realizations does not lead to the desired field equations. Consequently the problem of realization is left open for the time being}. This means that we have found a cloak in which we can dress the new theory, but we cannot even establish whether or not it would fit the body. That one uses the \Gtmn, \gmn or \phin etc. in order to find the right field equations. Indeed, \q{In this \scare{guessing} the geometrical interpretation of electricity is supposed to be the guide}. \q{with concepts like \scare{most natural assumption}, \scare{simplest invariant}, etc. ;}. The point of departure in this approach is \q{the (unwritten) assumption that whatever looks \emph{simple} and \emph{natural} from the viewpoint of the geometrical interpretation will lead to the desired changes in the equations of the field}.  It is this assumption which constitutes \q{the \emph{physical hypothesis} contained in these attempts}. \q{The geometrical interpretation of electricity can be carried through in any case, but it by no means follows that this added hypothesis must also be correct}.
%\end{itemize}
%
%Both strategies were motivated by a fundamental interpretational mistake. That the key of the success of \gr was the geometrization of the gravitational field, and thus that it was possible to replicate this success, by geometrizing the other field. If the proper geometrization was not sufficient, physicists was ready to use a mere graphical representation. That \gr was indeed a geometrical interpeation was by particular circumstances:  (a) \emph{Physically motivated}. Because of the equivalence principle, gravitation is a \emph{universal force} that cannot be neutralized or shielded, that uss all also the \rac became indicator of the gravitational field (b) \emph{heuristically powerful} Such a geometrical interpretation accounts for old inconsistencies in Newton's theory concerning the irregularities of Mercury's orbit motion and allows for new predictions like a more pronounced deflection of light by the Sun. Measurements carried out with real physical systems, \rac, light rays, free-falling particles\etc, seem to have confirmed the theory's predictions. 
%
%


%In the absence of the electromagnetic field, the \rac behave in a Riemannian way, that is the direction would haven under parallel transport. Thus, \q{the Weylian space now constitutes the natural cloak for the field, which is composed of electricity and gravitation} \rzlap{354}{494}. However, however, it is precisely this behavior which does occur in reality \q{This means that we have found a cloak in which we can dress the new theory, but we do not have the body \hook{which} this new cloak would fit}.

% It is this assumption which constitutes \q{the \emph{physical hypothesis} contained in these attempts}. 



%
% With the aid of this coordination, it is possible to interpret gravitational and electrical phenomena as expressions of the geometry of a Weylean space, so that electricity finds a geometrical interpretation \myemph{in the same sense as gravitation}.  The motion of charged particles under the influence of both the gravitational and the electromagnetic field deviate from the straightest lines, and takes the following form.  In the presence of charge, the \Gtmn is non-Riemannian, charged particles move on the straightest lines, and uncharged particles on the shortest lines. In the absence of charge, vanishes, and the connection reduces to that of Riemannian geometry. 





%
%Reichenbach praised Weyl's idea of defining the \Gtmn independently from the \gmn. However, he condemned Weyl's abandoned of the \cref{W2} in favor of the \cref{W1} strategy. For this reason, he proposed a theory in which on the \Gtmn have a physical meaning. The metric \gmn is measured not only by rigid rods but also by ideal clocks, which give physical meaning to the length of the four-dimensional vector $dx^\nu$. A similar coordinative definition should be provided for the displacement \Gtmn. Since we have to maintain the direction of a four-dimensional vector, 

Reichenbach suggested that one can tentatively adopt the velocity four-vector $u^\tau=d\xn/d\ap$ as the physical realization of the displacement \citep[\todo{page}]{Eddington1923} along a curve parametrized by the proper time $\ap$. By parallel-displacing a vector $u^\tau$ indicating the direction of a curve $\xn(\ap)$ at any of its points, one can define a special class of curves, the straightest lines. When the particle is not accelerating that is maintain the same velocity, the direction of the velocity vector does not vary. Thus, the motion of force-free particles can be used to define physically the straightest line between two \spti points. The motion of charged particles under the influence of both the gravitational and the electromagnetic field deviate from the straightest lines, and takes the following form:
%
%\begin{equation*}\label{eq:forceequation}
%\frac{d {u^\tau}}{d\ap} - \notateol{\Gtmn}{2}{\texts{Levi-Civita connection}} u^{\mu} u^\nu = \notateur{\frac{\rho}{\mu} f_{\tau}^{\mu} u^\tau}{2}{\texts{force term}}
%\end{equation*}
%%
%since the gravitational charge is the same to all particles, the mass coupling factor can be eliminated from the geodesic equation\todo{check}. Planets describe a trajectory which itself is assigned because it is attracted by the force of the sun, but we say: the planet moves along the straightest line defined by the \Gtmn. The force term indicates that the force experienced by charged particles is directly proportional to the charge and inversely proportional to the mass. Thus gravitation appears to be geometrized, but  electromagnetism does not. In order to geometrize the latter as well one might tentatively try to absorb the force term into the definition of the affine connection, and thus transform force equation a geodesic equations. To this purpose Reichenbach introduced a non-symmetric affine connection which we label \Gtmnbar. He imposed the in which however the lengths of vectors $l^2 \gmn dx^\nu dx^\nu$  does not change under parallel transport. Indeed, the velocity vector of particles is per definition always equal $1$ and only the change in direction indicates the change of velocity. A non-symmetric connection connection is always the sum of a symmetric connection and a non-symmetric three rank tensor with two lower indexes. With some change in notation, Reichenbach's definition of the connection is the following:
%
%  
%\begin{equation*}\label{eq:reichenbachconnection}
%\Gtmnbar = \notateor{\Gtmn}{2}{\texts{Levi-Civita connection}} + \notateul{\frac{\rho}{\mu} f_{\tau}^{\mu} u^\tau}{2}{\texts{skew-symmetric tensor}}
%\end{equation*}
%
%Put in this form one can see by simple inspection the force term of the geodesic equation has been introduced in the definition of the of the tensorial part of the affine connection. By substituting \Gtmnbar into \cref{force equation} one can turn the force equation into a geodesic equations of the same physical content. In Riemannian geometry the straightest lines are identical with the shortest lines. It his characteristic of a non-Riemannian connection, that the straightest lines do not generally coincide with its geodesics, that is with the shortest lines. In the presence of charge, from every point, in every direction, an auto-parallel and a geodesic emerge, which generally diverge. This divergence is a sign of the presence of charge\todo{better}.





%Weyl had immediately suggested a generalization of this theory in which  Between 1918 and 1919, Einstein criticized Weyl's theory repeatedly. The most famous but by no means only objection, regarded the physical interpretion of this geometrical apparatus\todo{better}. The length $ds$ of the time-like displacement vector $dx_\nu$ is measured atomic clocks; thus, Weyl theory should have predicted that the rate of ticking of atomic clocks should depend on the electromagnetic field \phin\todo{better}they have been through in the past. However, atomic spectroscopic overwhelmingly show that spectral lines of atoms are well-defined. In general, Einstein was rather cautious about the possibility of a unification project that he considered too speculative. In 1919 he submitted a paper in which who gravitational field could play structure of the electron of a modification of the original gravitational field equations of general relativity \citep{Einstein1919d}\todo{better}. However, there was initially no attempt to pursue a full fledged unification of the two fields. In spring 1919, however, Einstein seems to have reoriented his views probably after a correspondence with Theodore Kaluza, who suggested a \uft based on Riemannian geometry with five dimensions \citep{Wuensch2005}. Einstein ultimately decided not publish Kaluza's paper, however his attitude seems to have progressively acquired more interest in the unification project.



%There remains the peculiarity that the defined side does not carry its justification within itself ; its structure is determined from outside. Although there is a coordination to undefined elements, it is restricted, not arbitrary. This restriction is called "the determination of knowledge by experience." We notice the strange fact that it is the defined side that determines the individual things of the undefined side, and that, vice versa, it is the undefined side that prescribes the order of the defined side. The existence of reality is expressed in this mutuality of coordination.

%trikingly this idea of the mutuality of coordinating principles is presented in a very straightforward way in Reichenbach (1920). Here Reichenbach describes how the mathematical framework of a theory (the “defined side”) is related to empirical reality (the “undefined side”) :

%on, but n oretic language, he explains th coordination depends on the d typesofconcept.Theformeri of m athem atics," i.e., it is im p em atical concepts so that it "re definitions."23 Definitions, in f whereas the rules according to latter, however, cannot be dete mathematicalequationswemay afundamentalstatement,that for reality.





%Reichenbach abandoned his \scare{Kantinanism} but will remain faithful to this line of criticism in the following years. \cop{In this work, which deals with the possibility of reconciling Einstein's theory with the Kantian system, the notion of constitutive a priori appears to be tied to a view of the cognitive coordination (Zuordnung) that can be traced back to the influence of Moritz Schlick's General Theory of Knowledge (1918/1974)}. According  \cop{knowledge, according to Reichenbach (1920, chapter 4), 

%While a priori in the constitutive sense, the coordination principles are contingent, process of a construction of a representation is hidden behind the idea of \scare{coordination}. \todo{??}. T
%The novelty is that Reichenbach (1920, chapter 6) also insists on the need for a special class of physical principles-"coordinating principles" or "axioms of coordination" to insure that this correspondence is uniquely well defined. For further discussion see again Friedman (2005).}. The separation was the an essential part of the book. And it is precisely in this context that Weyl cites Weyl theory as a counter example.



%As suggested several times above, however, I now think that my earlier presentations were too closely tied to the problem, first formulated by Reichenbach and Schlick, of establishing a "coordination" (Zuordnung) between abstract mathematical structures and concrete empirical phenom- ena.276 

%The difficulty arises when one accepts the sharp distinction, emphasized by Schlick, between an uninterpreted axiomatic system and intuitive perceptible experience, and one then views the constitutive principles in question (which, following Reichenbach, I called "coordinating principles" or "axioms of coordination") as characterizing an abstract function or mapping associating the former with the latter. This picture is


%Reichenbach \scare{axioms of connection} expressing express the connection between the specific physical magnitudes and \scare{axioms of coordination} that the rules to give particular values to those mangitudes.


%\cop{The \emph{axioms of connection} are the empirical laws of physics, the fundamental equations of a theory. The \emph{axioms of coordination} determine the rules of the application of the axioms of connection to reality, that is, they determine the rules of the connection.



%The specific laws can be combined into a deductive system so that all of them appear as consequences of a few fundamental equations. We will call these equations axioms of connection because they express the connection between the specific physical magnitudes. Opposite to these are the axioms of coordination, which represent the properties of all bodies, reduced to a minimum of propositions. An example of coordinating axioms of old physics are the axioms of geometry; Maxwell's equations are an example of connecting axioms.41 (HR 026-03-01, 56 bis)


%\cop{presented in a very straightforward way in Reichenbach (1920). Here Reichenbach describes how the mathematical framework of a theory (the "defined side") is related to empirical reality (the "undefined side") :} he \cop{“defined side” is the mathematical theory, and the “undefined side” represents the elements of reality. Yet the same idea is already presented }. 

%However, we will consider Reichenbach's early work only in as much as it includes his first critique to the \uftp.

%E.g. he mentions \q{neuen Einsteinschen Auffansung} in which  \q{bei der innerhbalb des Elcktroms wicder die nicht-Euklid. Geometric gilt} \citep[028-01-04, Randbemerkung zu Blatt 18]{HR}\todo{check}.  In particular, Reichenbach had probably become aware of Weyl's attempt to a geometrization of the electormagnetic field alongside with the gravitational field. Moreover, he seemed to have been aware of Einstein's early work on the topic. 

%As long as there was no non-Euclidean geometry, there was only one model of space for mathematics and physics alike. It seemed, therefore, that physics could simply take over from mathematics the concept of space and its laws, while mathematics, in turn, had the task of examining these laws by means of its own methods, thought to be inde- pendent of all experience. 

%Indeed, it is normal that \cop{formal structures could describe physical structures only by way of a certain approximation}.  


%reviously unseen connections. Consequently, Reichenbach considers the axioms connection as empirical laws in the usual sense, involving already sufficiently well efined concepts. Yet, as we have seen, the concepts in such equations require further ualification, viz., the assertion that they are valid for reality. It is only through the xioms of coordination that how these concepts can de facto apply to reality can be hown. And that is precisely their role: providing a "physical" definition of the conepts occurring within the axioms of connection. In that sense, the former determine meaning of the latter and they are therefore constitutive of the concept of the physcal object. Thus, the axioms of coordination determine the rules of the application of he axioms of connection to reality, that is, they determine the rules of the connection.


%The case of geometry \cop{{It is important to notice in this context the difference between physics and mathematics. Mathematics is indifferent with regard to the applicability of its theorems to physical things, and its axioms contain merely a system of rules according to which its concepts can be related to each other. A purely mathematical axiomatization never leads to principles of an empirical theory. Therefore, the axioms of geometry could not assert anything about the epistemological problem of physical space. Only a physical theory could answer the question of the validity of Euclidean space and discover at the same time the epistemological principles holding for the space of physical objects.}. 


%Indeed, Ernst Cassirer an Hans Reichenbach, 7. Juni 1920, ASP, HR 015-50-09.

%Ernst Cassirer an Hans Reichenbach, 2. Juli 1920, ASP, HR 015-50-10. Der Verweis auf Reichenbachs Arbeit findet sich tatsächlich ganz am Ende der Schrift. Aus dem Eintrag geht hervor, dass Reichenbach ursprünglich plante, die Arbeit unter dem Titel Die Bedeutung der Relativitätstheorie für den physikalischen Erkenntnisbegriff zu veröffentlichen (vgl. Cassirer 1921, Zur Einsteinschen Relativitätstheorie, S. 134).









%\cop{that the arbitrariness of the principles is limited as well as principles are combined}





%As we have seen, the gravitational redshift, just like the transverse Doppler effect in special relativity, can be taken as an empirical confirmation of general relativity only because different atoms of the same substance can be regarded as identically constructed clocks reproducing the identical unit of time. 


%\cop{Ich freue mich wirklich sehr darüber, dass Sie mir Ihre ausgezeichnete Broschüre widmen wollen, noch mehr aber darüber, dass Sie mir als Dozent und Grübler ein so gutes Zeugnis ausstellen. Der Wert der Rel. Th. für die Philosophie scheint mir der zu sein, das sie die Zweifelhaftigkeit gewisser Begriffe dargethan hat, die auch in der Philosophie als Scheidemünze anerkannt waren. Begriffe sind eben leer, wenn sie aufhören, mit Erlebnissen fest verkettet zu sein. Sie gleichen Emporkömmlingen, die sich ihrer Abstammung schämen und sie verleugnen wollen}. Einstein made a similar claims by writing to Cassirer in the very same days.

%konstanten rein mathematisch berechnen lassen. Was ist z.B. ein Meter im Sinne einer solchen Theorie? Ein Raumgitter aus Pt-Atomen, deren jedes wieder aus Protonen und anderen Elementarteilchen besteht, die nach einem bestimmten Gesetz angeordnet sind; über alle Einzelheiten dieser Anordnung muss eine bestimmte Lösung der Feldgleichungen Auskunft geben. Und was ist eine Sekunde? Das so-und-so-vielfache der Schwingung in einem H-atom, der wiederum eine Lösung der Feldgleichungen entspricht. Also muss man auch aus der Feldtheorie ermitteln können, in wieviel Sekunden sich ein Lichtsignal vom einen zum

%Einstein sagt: Die Maßstablängen und die Frequenzen der Atomuhren folgen einer kongruenten Verpflanzung; mit ihrer Hilfe wird der Absolutwert des ds normiert (was nur möglich ist wegen der stillschweigend vorausgesetzten oder aus dem Verhalten der materiellen Körper abgelesenen Integrabilität der Streckenübertragung); bei solcher Normierung stellt sich der Krümmungsradius als konstant heraus. Ich sage: Maßstablängenund die Perioden der Atomuhren bestimmen sich durch Einstellungauf den Krümmungsradius; mit Hilfe des Krümmungsradius als Längeneinheit wird das ds normiert (diese Normierung ist stets möglich); als eine Folge der geltenden Naturgesetze kommt dann heraus, daß die kongruente Verpflanzung sich ebenso vollzieht, wie es die Einstellung bedingt und daher integrabel ist. Außerdem führt diese Theorie auf ein-

%Feld als ein Euklidisches ansehen kann. E: ist danach sicher, daß die „Körpergeometrie" welche in der geläufigen Weise das Maßver halten der materiellen Körper und ire Bewegung festlegt, nicht die ,Äthergeometrie" ist, sonderr diejenige Riemannsche Geometrie, in welche sie sich verwandelt, wenn man die kongruente Verpflanzung durch die Einstellung auf der Krümmungsradius ersetzt. In diesem Sinne ha® Einstein vollständig recht. Daß die der Naturgesetzen gemäß verlaufende Bewegung ernes Körpers und die Ubertragung der Uhr perioden nicht dem affinen Zusammenhang des Äthers folgt, geht übrigens schon rein forma

%rangieren. 



%Reichenbach attended Einstein's Ieerures in Berlin (see Doc. 57, note 2).

%The thesis was published after 15 June 1920 as Reichenbach 1920 (see Doc. 57). Stuttgart, Wiederholdstr. 13. d. 15. Juni 1920. 
%
%
%Widmung .That most of the discussion of the book, relativized constitutive \apr, however, one first by a philosophers of Weyl theory, and in particular was that is itself superior. 







%Riemann, Helmholtz, and Poincar\'e introduced the problem of the \emph{coordinative definition} in the philosophy of geometry \rzlp{**}{257}, having shown definitions to be necessary at certain points where, earlier, scientists sought for facts, like examples are the congruence of distant spatial segments. We assume a fact that these rods are Riemannian. If we assemble \cop{twelve identical rods  into a hexagonal pattern with 6 rods joining at the center. If this assembly is carried out in a suitable gravitational field, the 6 rods no longer join at the center}. There are at least two geometrical explanations for this result. (a) The rods are rigid and the geometry is non-Euclidean. (b) The rods are deformed by a force and the geometry is non-Euclidean. 
%
%Forces are \emph{differential} they have different effect on different materials. For a given temperature distribution over space, the metric depends on the kind of rod used,  iron rods and copper rods form two distinct classes of approximately rigid rods, different different geometries. However, we could also imagine a universal force demonstrate that this force has the same effect upon everything (a) and (b) are equivalent descriptions In this case, however, geometrical claims like the earth is a sphere, will be empty. However, if one stipulates that are exluded the question of the geometry of space is completely determined and there of which rods are rigid. To avoid that is true intslef and convetion, and sophustatie geometrical empiricism.
%
%
%Deviation from Euclidean geometry, then the preceding argument means cab be always interpret interpret this deviation as action of a force that deforms the measuring rods. But to admit the existence of such universal forces in physics would be to introduce uncertainty into all practical measurements. 
%\paragraph{coordination}
%Einstein was to realized that because of the equivalence principle, gravitation is a \emph{universal force} that cannot be neutralized or shielded. Thus, there is no way to separate the geometrical measuring instruments that are not affected by the field (\rac, light rays, force-free particles) from the dynamical ones that react to the field (charged particles). Thus, we do not speak of the deformation of our measuring instruments \q{produced by the gravitational field}, but we regard \q{the measuring instruments as \scare{free from deforming forces} in spite of the gravitational effects} \rzlp{294}{256}. Once one defines $ds\pm =1$, $ds=0$ in terms of \rac and light rays, the geometry of \st can be ascertained empirically; it is a branch of physics that can be true or false. Indeed, the \gmn can measured with respect to a certain coordinate system under the assumptions that $ds$ has the same value in the every orientation and in every position. In turn the \gmn are the potentials of the gravitational field
%
%
%Einstein introduced the problem of a \emph{scientific explanation} of physical geometry, which finds its mathematical solution in the field equations \cref{eq:einsteinfieldequations}. The gravitational field has an effect on \rac and light rays, that is comparable to that of any other field of force, if not for the fact that it is a universal effect \rzlp{**}{256}.
%
%\paragraph{geometrizaion}
%
%However, Reichenbach wants to disabused the readers for taking this conclusion.   The presence revealed by, and not defined by the geometry, it is revealed by their behavior of geometrical measuring instruments. Besides serving in their usual capacity of determining the geometry of space and time, they serve, therefore, also as indicators of the gravitational field. The geometrical interpretation of gravitation is consequently an expression of a real situation, the \emph{universal coupling} of gravitation and matter. Measuring instruments made of whatever fields and particles can be used to explore the gravitational field, and the result of such measurements is independent of the device. In principle this physical fact should be explained by a feature theory \q{It is not the theory of gravitation that becomes geometry, but it is geometry that becomes an expression of the gravitational field}. 



%(a) As we ahve sse by itroduce, a more general affine connection, which depedns nto onlu \gmn nut also new field \phin. the latter could be idnetified althotu only for formal reasosn. Following the analogy with Einstein's, Weyl assumed that the length of vectors can be measured with \rac. l uses \rac as indicators of the gravitational field and, at the same time, as indicators of the electromagnetic field. The failure of this project,  ... 



